\begin{figure}[tb]
  \centering
  \small {
  \begin{tabular}{p{2.5in}|p{.01in}p{2.9in}}
\begin{Verbatim}[fontsize=\relsize{-2.5}, numbersep=4pt,numbers=left,
frame=topline, framesep=4mm, label=\fbox{(a) Array search program}]
int ArraySearch(int [] a, int d, int s, int e, int n) {
@pre as { 0 <= s && s <= e && e < n && n <= MAXSIZE; }
  int i = s;          // pc = pc+1
  while(i <= e) {     // pc = (i <= e) ? 5 : 10;
    if (a[i] == d) {  // pc = (a[i] == d) ? 6 : 8;
      break;}         // pc = 10;
    else {
      i = i+1;}       // pc = 4;
  }
  return i;           // pc = 11; rv = i;
@post as {
  ((rv >= s && rv <= e) -> a[rv] == d) xor 
  (rv == -1 -> forall(int i:[s .. e]) { a[i] != d })
} }
\end{Verbatim}
& &
%\begin{Verbatim}[fontsize=\relsize{-2}], numbersep=4pt,numbers=left]
\begin{Verbatim} [fontsize=\relsize{-2.5}, numbersep=4pt,numbers=left,
frame=topline, framesep=4mm, label=\fbox{(b) Program with program counter}]
dotogether {
  preas = 0 <= s && s <= e && e < n && n <= MAXSIZE;
  //initial values
  pc = 0; notdone = true; postas = true; }
while (notdone) { dotogether {   // next state functions
  i = (pc == 3) ? s : (pc == 8) ? i+1 : i;
  notdone = (pc == 11) ? false : true;
  rv = (pc == 10) ? i : rv;
  pc = (pc == 0) ? 3: (pc == 3) ? 4 : 
    (pc == 4) ? ( (i <= e) ? 5 : 10 ) :
    (pc == 5) ? (a[i] == d) ? 6 : 8 ) :
    (pc == 6) ? 10 :
    (pc == 8) ? 4  : 
    (pc == 10) ? 11 : pc; }
  postas = ((rv >= s && rv <= e) -> a[rv] == d) xor
    (rv == -1 -> forall(int i:[s .. e]) { a[i] != d }); }
\end{Verbatim}
\end{tabular}
}
\caption{(a) Array search program, and (b) equivalent array search program with program
counter}
\label{f:arraysearch}
\end{figure}

The Array search program in Figure~\ref{f:arraysearch}(a) takes as input an array $a$, 
a start index $s$, an end index $e$, a data value $d$, and the number of 
elements in the array $n$.
It is annotated with a specification consisting of a precondition and a postcondition. 
The precondition states that the start $s$ and end $e$ indices are within array
bounds and that the array size $n$ is within the bound on array sizes. 
The postcondition states that if $rv$ is valid between $s$ and $e$ inclusive, 
then $a[rv]$ must be equal to $d$, 
otherwise, $rv$ must be invalid (-1) and all entries in $a$ between $s$ 
and $e$ inclusive are not equal to $d$. 

Figure~\ref{f:arraysearch}(b) shows an equivalent encoding of the array search 
program using a program counter execution model. 
The equivalent program introduces Boolean variables \cci{preas}, \cci{postas}, and 
\cci{notdone} to encode the preconditon, the postcondition, and the running
state of the program, respectively. 
The equivalent program also introduces a program counter variable
\cci{pc} which encodes the control flow of the program as indicated in 
the comments of Figure~\ref{f:arraysearch}(a). 
The \cci{rv} variable denotes the return value of the original program. 

The \cci{notdone} variable is initialized to $true$, and \cci{pc} 
program counter is initialized to the first executable line of the program $3$. 
Once \cci{pc} reaches the last executable line of the program $13$, 
the program terminates and thus \cci{notdone} becomes $false$. 
Assignment statements are grouped by target variables, and encoded
into conditional assignment statements that depend on the value of \cci{pc}. 
For example, the iterator $i$ is assigned to $s$ when \cci{pc} is $3$, incremented
when \cci{pc} is $8$, and remains the same otherwise. 

The program in Figure~\ref{f:arraysearch}(b) is semantically equivalent to the original
program in Figure~\ref{f:arraysearch}(a). 
Furthermore, the assignment statements on Lines 1 and 3 assign initial values to the 
target variables. 
The assignment statements inside the \cci{while} loop (Lines 5 to 13) compute 
the next state value of each of the target variables of the program. 

Our method translates the program in Figure~\ref{f:arraysearch}(b) to a sequential 
circuit where an iteration of the \cci{while} loop
is equivalent to a single time step in the sequential circuit. 
The method represents each Boolean variable with one register, and each scalar 
variable with a finite vector (bit-vector) of registers 
with initial value and next state functions. 
The initial state functions of the vector of registers corresponding to
a variable are connected to a vector of gates that represents the right hand 
side initial value assignment statement of the variable. 
For example, 
\cci{pc} ranges from $0$ to $11$ and can be encoded using $4$ registers.
%\todo{Add a figure for the sequential circuits}.
%The sequential circuit in Figure~\ref{f:pccircuit} shows the registers,
%their initial value functions set to $0$, and their next state functions
%set to gates representing multiplexers that choose the next value of the
%\cci{pc} registers based on the current state of the cuicuit. 


Program variables that are not initialized in the code are considered
input variables and the methods connects their initial value functions 
to free primary inputs.
The method connects the next state functions of register vectors corresponding to 
program variables to gates that represent
the right hand side of the next state assignment.
The conditional, arithmetic, and Boolean operations in the right hand side
expressions are encoded as combinational logic circuits in the usual manner. 

Our method takes the resulting sequential circuit, designates a gate therein that 
represents $preas \wedge done \rightarrow postas$ as the output gate, passes
the circuit to ABC, and checks for the validity of the designated gate.
The ABC solver returns a counterexample $a=[0 0 0],s=0,e=1,n=3,d=1,rv=2$ 
where $d$ is not in $a$, and the return value is $e+1$, while the postcondition
requires an invalid index (-1). 

The provided counter example can be used to fix the program. 
A possible fix is to replace Line 6 with \cci{return i;}, and Line 10 with
\cci{return -1;}.
Our method takes the fixed program, transforms it into a sequential circuit,
and passes it to ABC which validates the correctness of the program modulo the 
finite size of the variable vectors using symbolic model checking. 


