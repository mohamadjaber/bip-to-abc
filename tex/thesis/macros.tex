%%%%%%%%%%%%%%%%%%%%%% PACKAGES %%%%%%%%%%%%%%%%%%%%%%%%%%%%%
%\RequirePackage{mathabx}
\RequirePackage{relsize}
\RequirePackage{fancyvrb}
\RequirePackage{ifthen}
\RequirePackage{multirow}

\RequirePackage{amsmath}
\RequirePackage{amsthm}
\RequirePackage{amssymb}
\RequirePackage{latexsym}
\RequirePackage{xspace}
\RequirePackage{url}
\RequirePackage[pdftex]{graphicx}
\RequirePackage{color}
\RequirePackage{mathpartir}

%\usepackage{graphviz}
\usepackage{dot2texi}

\usepackage{tikz}
\usetikzlibrary{shapes,arrows}

\usepackage{pgfplots}
\pgfplotsset{compat=1.8}

\usepackage[colorinlistoftodos]{todonotes}
\usepackage{color,soul}

%----------------------------------------------------------------------------%
\usepackage{tikz}
\usetikzlibrary{calc,trees,positioning,arrows,chains,shapes.geometric,%
    decorations.pathreplacing,decorations.pathmorphing,shapes,%
    matrix,shapes.symbols}
\tikzset{
>=stealth',
  punktchain/.style={
    rectangle, 
    rounded corners, 
    % fill=black!10,
    draw=black, very thick,
    text width=10em, 
    minimum height=3em, 
    text centered, 
    on chain},
  line/.style={draw, thick, <-},
  element/.style={
    tape,
    top color=white,
    bottom color=blue!50!black!60!,
    minimum width=8em,
    draw=blue!40!black!90, very thick,
    text width=10em, 
    minimum height=3.5em, 
    text centered, 
    on chain},
  every join/.style={->, thick,shorten >=1pt},
  decoration={brace},
  tuborg/.style={decorate},
  tubnode/.style={midway, right=2pt},
}
%----------------------------------------------------------------------------%

%---------------------
\usepackage{setspace}
\usepackage{listings}
\usepackage{float}
\usepackage{enumerate}
\usepackage{framed}
\usepackage{tikz}
\usepackage{pifont}
%\usepackage[Conny]{fncychap}
%\bibliographystyle{plain}
\usepackage{wrapfig}
\usepackage{subfig}

%\usepackage[dvips]{graphicx}
\usepackage{latexsym}
%\usepackage{epsf}
%\usepackage{epsfig}
\usepackage{a4}
%
%\usepackage{amsmath} % real numbers, natural numbers, ...
%\usepackage{amssymb}
%\usepackage{amsthm}  % used for theorems and definitions 
%\usepackage{amsxtra}
%\usepackage{amsxtra}
%\usepackage{amsfonts}
%
\usepackage{fancyhdr}
\usepackage{tocbibind} %bibliography in the table of contents
\usepackage{enumerate}
%\usepackage{amsthm}
\usepackage{amsbsy}
%
%\usepackage{tikz}
%
\usepackage{booktabs}
\usepackage{paralist}
%
\usepackage{algorithm}
\usepackage{algpseudocode}
%
%\usepackage{graphics}
%\usepackage{color}
%%\usepackage[usenames,dvipsnames]{xcolor}
%\usepackage{graphicx}
%\usepackage{enumerate}
\usepackage{fancyvrb}
%%\usepackage{alltt}
%\usepackage{url}
%
%\usepackage{colortbl}
%\usepackage{multirow}
\usepackage{chngpage}
\usepackage{tabularx}
%
%\usepackage{amsfonts}
%\newcommand{\Yes}{\checkmark}
%\usepackage{pifont}
%\newcommand{\No}{\hspace{1pt}\ding{55}}
%
%\usepackage{relsize}
%\include{geometry}

\definecolor{darkblue}{rgb}{0.0,0.0,0.6}
\definecolor{darkgreen}{rgb}{0.0,0.6,0.0}

\usepackage{listings}
\lstset{ %
language=C,                % choose the language of the code
basicstyle=\tiny,       % the size of the fonts that are used for the code
numberstyle=\tiny,      % the size of the fonts that are used for the line-numbers
frame=tb,
rulesep=.4pt,
mathescape, % Allows escaping to (La)TeX mode within $$,
stringstyle=\color{redb},
morekeywords={Boolean, dotogether, foreach, endfor, endif, Let},
showstringspaces = false,
basicstyle=\scriptsize\ttfamily,
keywordstyle=\color{darkblue}\bf,
commentstyle=\color{darkgreen},
numbers=left,
stepnumber=1,
%
breaklines=true
}


\input{cmdComment}

%%%%%%%%%%%%%%%%%%%%%%%%%%%%%%%%%%%%%%%%%%%%%%%%%%% SPECCHECK SPECIFIC SYMBOLS
\newcommand{\MATH}[1]{\ensuremath{#1}\xspace}

\newcommand{\aff}{[\ff,\ldots,\ff]} 
\newcommand{\att}{[\tt,\ldots,\tt]}  
\newcommand{\arr}[1]{[#1]}  


%% Algorithm names
\newcommand{\mkAdeq}{\textsf{MakeAdequate}}
\newcommand{\constFor}{\textsf{ConstructFormula}}

\newcommand{\checkAd}{\textsf{CheckAdequacy}}
\newcommand{\checkAdA}[1]{\textsf{CheckAdequacy}$(#1)$}

%% theories
\newcommand{\gth}{\MATH{\gamma}}                              % generic theory
\newcommand{\tth}{\MATH{\tau}}                                     % type theory
\renewcommand{\eth}{\MATH{\varepsilon}}                       % equivalence theory
\newcommand{\eps}{\ensuremath{\varepsilon}\xspace}
\newcommand{\voc}{\MATH{\mathit{\nu}}}                        % vocabulary

\newcommand{\Vt}{\MATH{V_\tau}}                                   % assignment to type theory                            
\newcommand{\Ve}{\MATH{V_\varepsilon}}                        % assignment to equivalence theory                            
\newcommand{\Vv}{\MATH{V_\voc}}                                  % assignment to vocabulary

%% bounded theories
\newcommand{\ethb}[1]{\MATH{\varepsilon_{#1}}}             % bounded equivalence theory
\newcommand{\vocb}[1]{\MATH{\mathit{\nu}_{#1}}}           % bounded vocabulary

\newcommand{\Veb}[1]{\MATH{V_{\varepsilon_{#1}}}}         % assignment to bounded equivalence theory 
%\newcommand{\Veb}[1]{\MATH{V_{\varepsilon}^{#1}}}                   % assignment to bounded equivalence theory 

             
\newcommand{\Vvb}[1]{\MATH{V_{\voc}^{#1}}}                         % assignment to bounded vocabulary

%% metasyntax
\newcommand{\ethm}{\MATH{\hat{\varepsilon}}}
\newcommand{\ethmb}[1]{\MATH{\hat{\varepsilon}_{#1}}}



%% formulae
\newcommand{\Pm}{\ensuremath{\mathcal{S}}\xspace}         % program
\newcommand{\Sp}{\ensuremath{\mathcal{S}}\xspace}         % specification
\newcommand{\cfor}[1]{\MATH{\mathit{cfor}[#1]}}               % correction formula for inadequate vocabulary
\newcommand{\for}[1]{\MATH{\mathit{fm}(#1)}}                     % formula corresponding to an assignment
\newcommand{\Fo}{\ensuremath{\mathcal{F}}\xspace}
\newcommand{\Pre}{\ensuremath{\mathcal{P}}}                                % precondition
\newcommand{\Post}{\ensuremath{\mathcal{Q}}}                              % postcondition

%% tool
\newcommand{\mytool}{\ensuremath{\{\mathcal{P}\}\mathcal{S}\{\mathcal{Q}\}}} % the name of our tool
\newcommand{\thislanguage}{\ensuremath{\mathcal{CAIG}} }
\newcommand{\biptool}{\ensuremath{\mathcal{BIP\{I\}}}}
\newcommand{\psqlanguage}{\ensuremath{\mathcal{T}iny}}
\newcommand{\aigcircuit}{\ensuremath{\mathcal{C}}}

\newcommand{\lft}{\ensuremath{\mathcal{left}}\xspace}
\newcommand{\rgt}{\ensuremath{\mathcal{right}}\xspace}
\newcommand{\eina}{\ensuremath{\mathcal{eina}}\xspace}

%% model theory
\renewcommand{\th}{\MATH{\beta}}                                 % threshold for satisfiability checking
\newcommand{\md}[1]{\MATH{[ #1 ]}}                               % set of points satisfying a formula ``model''
%\newcommand{\md}[1]{\ensuremath{[ #1 ]}\xspace}
\newcommand{\Bool}{\set{\tt,\ff}}                                % the Boolean type
\newcommand{\Part}{\ensuremath{\set{\vtt,\vff}}\xspace}          % the partition of D
\newcommand{\asg}{\ensuremath{\sigma}}                         % assignment, ie state
\newcommand{\beh}{\ensuremath{\sigma}}                         % behavior
\newcommand{\behi}{\ensuremath{\beh_i}}                      % input behavior
\newcommand{\beho}{\ensuremath{\beh_o}}                    % output behavior
\newcommand{\Beh}{\ensuremath{\Sigma}\xspace}             % the set of behaviors
\newcommand{\D}{\ensuremath{\MATH{D}}}
\newcommand{\U}{\ensuremath{\MATH{U}}}



%% logic
\newcommand{\IL}{\ensuremath{L_{\omega_{1},\omega}}\xspace}

%% variables
\newcommand{\sv}{x}   % vars used in specification

%% ranges




%% old?
\newcommand{\E}{\mathcal{E}}
\newcommand{\B}{\mathcal{B}}
\newcommand{\redTo}{\ensuremath{\rhd}}
\newcommand{\FS}{\mathcal{F}}
\newcommand{\PS}{\mathcal{P}}
\newcommand{\M}{\mathcal{M}}
%%%%%%%%%%%%%%%%%%%%%%%%%%%%%%%%%%%%%%%%%%%%%%%%% END SPECCHECK SPECIFIC SYMBOLS





%%%%%%%%%%%%%%%%%%%%%%%%%%% Abbreviations
\newcommand{\assig}{\ensuremath{\sigma}}
\newcommand{\sg}{\ensuremath{\sigma}}
\newcommand{\tf}{\ensuremath{\{\tt, \ff\}}}
\newcommand{\form}{\ensuremath{\mathit{form}}}
\newcommand{\vphi}{\ensuremath{\varphi}}




%%%%%%%%%%%%%%%%%%%%%%%%%%%%%%%%%%%%%%%%%%%%%%%%%%%%%%%%%%%%%%%%%%%%%%%%%%%%%%%%%%%%%%%%%%%%
%%%%%%%%%%%%%%%%%%%%%%%%%%%%%%%%%%%%%%%%%%%%%%%%%%%%%%%%%%%%%%%%%%%%%%%%%%%%%%%%%%%%%%%%%%%%







%%%%%%%%%%%%%%%%%%%%%%%% Lists
\newcommand{\be}{\begin{itemize}}
\newcommand{\ee}{\end{itemize}}
\newcommand{\bdn}{\begin{description}}
\newcommand{\edn}{\end{description}}
\newcommand{\bn}{\begin{enumerate}}
\newcommand{\en}{\end{enumerate}}
\renewcommand{\i}{\item}

\newenvironment{closeitemize}{\begin{list}%
{$\bullet$}%
{\setlength{\itemsep}{-0.2\baselineskip}%
\setlength{\topsep}{0.2\baselineskip}}}%
{\end{list}}




%%%%%%%%%%%%%%%%%%%%%%%%%%%%%% MISC

\newcommand{\remove}[1]{}

%\newcommand{\case}[2]{\vspace{1.5ex} \noindent \textit{Case} #1: \emph{#2}.}
\newcommand{\scase}[2]{\vspace{1.5ex} \noindent \textit{Subcase} #1: #2.}
\newcommand{\sscase}[2]{\vspace{1.0ex} \noindent \textit{Subsubcase} #1: #2.}

%\newcommand{\defn}[1]{\textit{#1}}
\newcommand{\defi}[1]{\textit{#1}:}

\newcommand{\lb}{\linebreak}

%%%%%%%%%%%%%%%%%%%%%%%%%%%% FOOTNOTES AND THEOREMS


\newtheorem{definition}{Definition}
\newtheorem{theorem}{Theorem}
\newtheorem{lemma}{Lemma}
\newtheorem{proposition}{Proposition}
\newtheorem{corollary}{Corollary}
\renewcommand{\thefootnote}{\arabic{footnote}}


%%%%%%%%%%%%%%%%%%%%%%%%%%%%%%%%%%% Text
\newcommand{\smpage}{\noindent \parbox{\textwidth}}
\newcommand{\samep}{\parbox{\textwidth}}   % put onto the same page


\newcommand{\mathid}[1]{\ensuremath{\mathit{#1}}\xspace}

\newcommand{\bc}{\begin{center}}
\newcommand{\ec}{\end{center}}
%\newcommand{\ul}{\underline}
\newcommand{\bs}{\bigskip}
%\newcommand{\ms}{\medskip}
%\renewcommand{\ss}{\smallskip}

\newcommand{\bfg}{\begin{figure}}
\newcommand{\efg}{\end{figure}}

\renewcommand{\ss}{\smallskip}
\newcommand{\Eg}{E.g.,\xspace}
\newcommand{\eg}{e.g.,\xspace}
\newcommand{\ie}{i.e.,\xspace}


\newcommand{\intr}{\empi}
\newcommand{\intrdef}{\emph}

\newcommand{\emp}[1]{\textbf{#1}}
\newcommand{\empp}[1]{\emph{#1}}
\newcommand{\empb}[1]{\textbf{#1}}
\newcommand{\empi}[1]{\textit{#1}}
\newcommand{\empbi}[1]{\textbf{\textit{#1}}}

\newcommand{\ind}{\hspace*{3.0em}}

%%%%%From Nancy's book directory:  This spaces the symbols within a
%%%%%word nicely, in math mode.
\newcommand{\ms}[1]{%
        \relax\ifmmode
                \mathord{\mathcode`\-="702D\it #1\mathcode`\-="2200}%
        \else
                $\mathord{\mathcode`\-="702D\it #1\mathcode`\-="2200}$%
        \fi
}


\newcommand{\cmnt}{\`//}


%%%%%%%%%%%%%%%%%%%%%%% GENERAL MATH SYMBOLS %%%%%%%%%%%%%%%%%%%%%%%%%%%%%%

\newcommand{\Adj}{\mathop{\rm Adj}\nolimits}
\newcommand{\abs}[1]{\left| #1\right|}
\newcommand{\card}[1]{\left| #1\right|}
\newcommand{\ar}{\rightarrow}
%\newcommand{\ar}{\longrightarrow}
\newcommand{\al}{\alpha}
\renewcommand{\b}[1]{\overline{#1}}
\newcommand{\cat}{\mathbin{\frown}}
\newcommand{\choice}{\mbox{$[\hspace*{-1.0pt}]$}}
\newcommand{\ceil}[1]{\left\lceil #1 \right\rceil}
\renewcommand{\d}{\, . \,}      % separator in quantified formulae
\newcommand{\df}{\triangleq}
%\newcommand{\df}{\mbox{$\:\stackrel{\rm df}{=\!\!=}\:$}}
\newcommand{\dn}{\mbox{$\hspace{-0.1em}\downarrow\hspace{-0.1em}$}}
\newcommand{\Ex}{\mathop{\rm Ex}}
\newcommand{\es}{``"}   %empty string
\newcommand{\expect}[1]{{\rm E}\left[ #1 \right]}
\newcommand{\expectsq}[1]{{\rm E}^2\left[ #1 \right]}
\newcommand{\floor}[1]{\left\lfloor #1 \right\rfloor}
\renewcommand{\ge}{\geqslant}
\newcommand{\given}{\mid}
\newcommand{\halfind}{\hspace*{1.5em}}
\newcommand{\ifof}{\Longleftrightarrow} % logical equivalence
\newcommand{\img}{\mathrm{Image}}
\newcommand{\ints}{\cap}
\renewcommand{\l}{\ell}
\newcommand{\la}[1]{\mbox{$\, \stackrel{#1}{\rightarrow} \,$}}
\renewcommand{\le}{\leqslant}
\newcommand{\lra}{\mbox{$\longrightarrow$}}
%\newcommand{\mod}{\ \mathrm{mod}\ }
\newcommand{\oneton}{\{1,\,\ldots, n\}}
\newcommand{\n}{\ensuremath{[1:n]}} 
\newcommand{\paren}[1]{\left( #1 \right)}
\newcommand{\pair}[2]{\ensuremath{(#1, #2)}}
\newcommand{\pind}{\hspace*{3.0em}}
\newcommand{\pj}{\!\upharpoonright\!}
\newcommand{\proj}{\pj}
\newcommand{\preimg}{\mathrm{PreImage}}
\newcommand{\pl}{\!\parallel\!}
\newcommand{\s}{\mbox{$\hspace{-1pt}-\hspace{-2pt}$}}
\newcommand{\set}[1]{\ensuremath{\{ #1 \}}\xspace}
%\newcommand{\set}[1]{\{#1\}}
\newcommand{\seq}{\approx}
\newcommand{\spc}{\mbox{\vspace{-0.25in}}}
\newcommand{\stt}{\ | \ }
%\newcommand{\st}[1]{\ensuremath{[ #1 ]}\xspace}
\newcommand{\sub}{\subseteq}
\newcommand{\twodots}{\mathinner{\ldotp\ldotp}}
\newcommand{\tiff}{\textup{\ iff\ }}
\newcommand{\tl}[1]{\mbox{$\tilde{#1}$}}% abbreviated tilde
\newcommand{\tpl}[1]{\ensuremath{\langle #1 \rangle}}
\newcommand{\un}{\cup}
\newcommand{\union}{\bigcup}
\newcommand{\up}{\mbox{$\hspace{-0.1em}\uparrow\hspace{-0.1em}$}}
\newcommand{\Var}{\mathop{\rm Var}\nolimits}
\newcommand{\variance}[1]{{\rm Var}\left[ #1 \right]}
\newcommand{\w}{\omega}


%%%%%%%%%%%%%%%%%%% Ints, Reals, etc %%%%%%%%%%%%%%%%%%%%%%%%%%%%%

\newcommand{\reals}{{\mathbb R}}
\newcommand{\integers}{{\mathbb Z}}
\newcommand{\naturals}{{\mathbb N}}
\newcommand{\nat}{\naturals}
\newcommand{\nats}{\naturals}
\newcommand{\rationals}{{\mathbb Q}}
\newcommand{\complex}{{\mathbb C}}
\newcommand{\complexes}{{\mathbb C}}



%%%%%%%%%%%%%%%%%%%%%%%%% MATH MACROS AND ENVIRONMENTS %%%%%%%%%%%%%%%%%%%%%%%%%%

%\newcommand{\se}[3]{\ensuremath{#1[#2 .. #3]}}
\newcommand{\se}[3]{\ensuremath{#1[#2 \! : \! #3]}}
%\newcommand{\se}[3]{\ensuremath{#1[#2,\ldots,#3]}}


\newcommand{\ang}[1]{\ifmmode{\left\langle #1 \right\rangle}
   \else{$\left\langle${#1}$\right\rangle$}\fi}
        % the \if allows use outside mathmode,
        % but will swallow following space there!

\newcommand{\struct}[2]{\raisebox{-0.1in}{$\stackrel { \displaystyle
#1} {\scriptstyle #2}\,$}}

\newcommand{\pbx}[2]{\stackrel{\fbox{\begin{Beqnarray*} #1\\.\\.\\.\\#2 \end{Beqnarray*}}}{}}      % proof box

\newcommand{\asrt}[1]{\ensuremath{\{ #1 \}}}

%%%%%%%%%%%%%%%%%%%%%%% Hoare Logic Symbols
\newcommand{\lng}{\langle}
\newcommand{\ra}{\rangle}
\newcommand{\htp}[3]{\ensuremath{\{#1\}\,#2\,\{#3\}}}
\newcommand{\htptc}[3]{\ensuremath{\langle#1\rangle\,#2\,\langle#3\rangle}}
\newcommand{\var}{\ensuremath{\varphi}}

\newcommand{\as}[1]{\ensuremath{\{#1\}}}               % partial correctness asserion
%\newcommand{\ts}[1]{\ensuremath{\langle #1 \rangle}}   % termination correctness asserion

\newcommand{\ch}{\mbox{[\hspace{-0.15ex}]}}

\newcommand{\pa}[1]{\{\ensuremath{#1}\}}
%\newcommand{\ta}[1]{$<$#1$>$}
\newcommand{\ta}[1]{$\langle$\ensuremath{#1}$\rangle$}

\newcommand{\pre}[1]{\textsf{Precondition: #1}}
\newcommand{\post}[1]{\textsf{Postcondition: #1}}

\newcommand{\satt}{\equiv}
\newcommand{\sat}{\models}
\newcommand{\satf}{\mbox{\ensuremath{=\hspace*{-5pt}|}}}   % satisfiable - backward turnstile
%\newcommand{\satf}{\mbox{\ensuremath{=\!\!\!\!|}}}   % satisfiable - backward turnstile


\newcommand{\yld}{\vdash}
\newcommand{\yldd}{\equiv}
%\newcommand{\yldd}{\dashv \vdash}

\newcommand{\prob}{\ensuremath{\Pm \sat \pair{\Pre}{\Post}|_b }}



\newcommand{\pc}[1]{{#1}}    %% font for pseudocode
%\newcommand{\pc}[1]{\codetext{#1}}    %% font for pseudocode



%%%%%%%%%%%%%%%%%%%%%%%% JUSTIFY EXAMPLE MACROS
\newcommand{\valpha}{\ensuremath{\mathit{alpha}}}
\newcommand{\numeric}{\ensuremath{\mathit{numeric}}}
\newcommand{\wspace}{\ensuremath{\mathit{wspace}}}
\newcommand{\vnewline}{\ensuremath{\mathit{newline}}}
\newcommand{\blank}{\ensuremath{\mathit{blank}}}
\newcommand{\word}{\ensuremath{\mathit{word}}}
\newcommand{\lline}{\ensuremath{\mathit{line}}}
\newcommand{\para}{\ensuremath{\mathit{para}}}
\newcommand{\numWords}{\ensuremath{\mathit{numWords}}}
\newcommand{\selectWord}{\ensuremath{\mathit{selectWord}}}
\newcommand{\numLines}{\ensuremath{\mathit{numLines}}}


%\RequirePackage{metre}

\newcommand{\nl}{\ensuremath{\backslash n}}
\newcommand{\bl}{\textvisiblespace}
\newcommand{\tx}{tx}



%%%%%%%%%%%%%% MACROS FOR NUMBERED LINES IN CODE TABBING ENVIRONMENT %%%%%%%%%%%%%%%%

\newcounter{lctr}

\newcommand{\li}{\addtocounter{lctr}{1}\arabic{lctr}.}
\newcommand{\lio}[1]{\addtocounter{lctr}{1}\arabic{lctr}.\>\ensuremath{#1}\\}
\newcommand{\lit}[1]{\addtocounter{lctr}{1}\arabic{lctr}.\>\>\ensuremath{#1}\\}
\newcommand{\lih}[1]{\addtocounter{lctr}{1}\arabic{lctr}.\>\>\>\ensuremath{#1}\\}

%%%% 2'nd argument is for a commment
\newcommand{\lioc}[2]{\addtocounter{lctr}{1}\arabic{lctr}.\>\ensuremath{#1}\`{#2}\\}
\newcommand{\litc}[2]{\addtocounter{lctr}{1}\arabic{lctr}.\>\>\ensuremath{#1}\`{#2}\\}
\newcommand{\lihc}[2]{\addtocounter{lctr}{1}\arabic{lctr}.\>\>\>\ensuremath{#1}\`{#2}\\}




%%%%%%%%%%%%%%%%%%%%%%%%%%% PSEUDOCODE SECTION

\newcommand{\gt}{\ensuremath{:=}}   % assignment
%\newcommand{\gt}{\ensuremath{\leftarrow}}   % assignment
\newcommand{\swap}{\ensuremath{\leftrightarrow}}   % swap two vars

\newcommand{\pseudocode}[1]{\ensuremath{\mathbf{#1}}\xspace}
%\newcommand{\pseudocode}[1]{\ensuremath{\mathbf{#1}\ }}
\newcommand{\pseudocodensp}[1]{\ensuremath{\mathbf{#1}}}
%\newcommand{\pseudocode}[1]{\mbox{${\bf #1}$}\xspace}

\newcommand{\IFC}[1]{\pseudocode{if}\ (\ensuremath{#1})}
\newcommand{\WHILEC}[1]{\pseudocode{while}\ (\ensuremath{#1})}
\newcommand{\RETURNE}[1]{\pseudocodensp{return}(\ensuremath{#1})}

\newcommand{\IF}{\pseudocode{if}}
\newcommand{\FI}{\pseudocode{fi}}
\newcommand{\THEN}{\pseudocode{then}}
\newcommand{\ELSE}{\pseudocode{else}}
\newcommand{\ELSF}{\pseudocode{else\ if}}
\newcommand{\ENDIF}{\pseudocodensp{endif}}

\newcommand{\WHILE}{\pseudocode{while}}
\newcommand{\ENDWHILE}{\pseudocode{endwhile}}
\newcommand{\FOR}{\pseudocode{for}}
\newcommand{\FORALL}{\pseudocode{forall}}
\newcommand{\FOREACH}{\pseudocode{foreach}}
\newcommand{\CONTINUE}{\pseudocode{continue}}
\newcommand{\ENDFOR}{\pseudocodensp{endfor}}
\newcommand{\DO}{\pseudocode{do}}
\newcommand{\OD}{\pseudocode{od}}

\newcommand{\BEGIN}{\pseudocode{begin}}
\newcommand{\END}{\pseudocode{end}}
\newcommand{\PROC}{\pseudocode{procedure}}
\newcommand{\CALL}{\pseudocode{call}}
\newcommand{\VAL}{\pseudocode{value}}
\newcommand{\VALRES}{\pseudocode{value\!-\!result}}
\newcommand{\RES}{\pseudocode{result}}
\newcommand{\RETURN}{\pseudocodensp{return}}
\newcommand{\DOWNTO}{\pseudocode{downto}}
\newcommand{\TO}{\pseudocode{to}}

\newcommand{\function}{\pseudocode{function}}
\newcommand{\operation}{\pseudocode{operation}}

\newcommand{\newo}{\pseudocode{new}}
\renewcommand{\int}{\pseudocode{int}}

\newcommand{\skipp}{\pseudocode{skip}}







%%%%%%%%%%%%%%%%%%%%%%%%%%%% Boolean Constants




\newcommand{\T}{\mbox{\rm T}}
\newcommand{\F}{\mbox{\rm F}}
\newcommand{\Fa}{\mbox{\rm F}}

\newcommand{\true}{\ensuremath{\mathit{true}}}
\renewcommand{\tt}{\ensuremath{\mathit{tt}}}
\newcommand{\vtt}{\ensuremath{\mathit{vtt}}\xspace}
%\newcommand{\true}{\mbox{\textsc{true}}\xspace}

\newcommand{\false}{\ensuremath{\mathit{false}}}
\newcommand{\ff}{\ensuremath{\mathit{ff}}}
\newcommand{\vff}{\ensuremath{\mathit{vff}}\xspace}
%\newcommand{\false}{\mbox{\textsc{false}}\xspace}

%\newcommand{\TT}{\top}
%\newcommand{\FF}{\bot}

\newcommand{\TT}{\mathit{true}}
\newcommand{\FF}{\mathit{false}}

\newcommand{\andw}{\mbox{ and }}

%\newcommand{\ANDW}{\mbox{\bf and \xspace}}
%\newcommand{\OR}{{\bf or} }
%\newcommand{\NOT}{{\bf not} }




%%%%%%%%%%%%%%%%%%%%%%%%% Boolean Connectives

\renewcommand{\iff}{\equiv}
%\renewcommand{\iff}{\Leftrightarrow}
\newcommand{\imp}{\Rightarrow}
\newcommand{\ev}{\equiv}
%\newcommand{\implies}{\Rightarrow}





%%%%%%%%%%%%%%%%%%%%%%%%%%%%%%%%%%% QUANTIFIERS

\newcommand{\qt}[3]{#1\,#2\,#3}
\newcommand{\qtr}[4]{(#1\,#2 : #3 : #4)}

\newcommand{\Q}{\mbox{${\bf Q}$}\xspace}
\newcommand{\q}{\mbox{${\bf q}$}\xspace}

%% quantifiers
\newcommand{\AND}{\bigwedge}
\newcommand{\INTER}{\bigcap}
\newcommand{\OR}{\bigvee}
\newcommand{\UN}{\bigcup}
\newcommand{\SUM}{\Sigma}
%\newcommand{\INT}{\bigcap}

%% Logical Quantifiers
\newcommand{\fa}{\ensuremath{\forall\,}}
%\newcommand{\fa}{\mbox{${\mathbf{\forall}}$}\xspace}
\newcommand{\ex}{\ensuremath{\exists\,}}
%\newcommand{\ex}{\mbox{${\mathbf{\exists}}$}\xspace}

\newcommand{\LQ}{\mbox{${\bf LQ}$}\xspace}


%% Arithmetic Quantifiers
%\renewcommand{\S}{\Sigma\,}
%\renewcommand{\S}{\mbox{${\mathbf{\Sigma}}$}\xspace}
\renewcommand{\P}{\Pi\,}
\newcommand{\N}{\ensuremath{\mathrm{N}\,}}
\newcommand{\AQ}{\ensuremath{\mathrm{AQ}\,}}
\newcommand{\MAX}{\ensuremath{\mathrm{MAX}\,}}
\newcommand{\MIN}{\ensuremath{\mathrm{MIN}\,}}


%%%%%%%%%%%%%%%%% ENVIRONMENTS
\newcommand{\bd}{\begin{definition}}
\newcommand{\ed}{\end{definition}}

%%%%%%%%%%%%%%%%% EQUATION ENVIRONMENTS
\newcommand{\beqn}{\begin{centeqn}}
\newcommand{\eeqn}{\end{centeqn}}
\newcommand{\beqnnbsp}{\begin{centeqn-nbsp}}
\newcommand{\eeqnnbsp}{\end{centeqn-nbsp}}
\newcommand{\bleqn}[1]{\begin{centlabeqn}{#1}}
\newcommand{\eleqn}{\end{centlabeqn}}
\newcommand{\bleqnnbsp}[1]{\begin{centlabeqn-nbsp}{#1}}
\newcommand{\eleqnnbsp}{\end{centlabeqn-nbsp}}

\newenvironment{centeqn}	% centered equation environment
   {{\ss\\ \hspace*{\fill}}}
   {\hspace*{\fill}\ss\\}

\newsavebox{\EqnLabel}
\newenvironment{centlabeqn}[1]	% centered labeled equation environment
   {\sbox{\EqnLabel}{#1}
    {\medskip\\ \hspace*{\fill}}
   }
   {\hfill{\makebox[0in][r]{\usebox{\EqnLabel}}}\medskip\\}

\newenvironment{centeqn-nbsp} % centered equation - no vertical space at the bottom
   {{\medskip\\ \hspace*{\fill}}}
   {\hspace*{\fill}}




%%%%%%%%%%%%%%%% SpecCheck Specific macros

\newcommand{\cd}[1]{{\small\texttt{#1}}}
\newcommand{\cci}[1]{\mbox{\textup{\texttt{#1}}}}
%\newcommand{\cci}[1]{{\small\texttt{#1}}}
%\newcommand{\chci}[1]{`\cci{#1}'}
\newcommand{\chci}[1]{\cci{#1}}

\newcommand{\good}{\cci{good}\xspace}
\newcommand{\bad}{\cci{bad}\xspace}
\newcommand{\dc}{\cci{dontCare}\xspace}
\newcommand{\tgood}{\tiny\texttt{good}\xspace}
\newcommand{\tbad}{\tiny\texttt{bad}\xspace}
\newcommand{\tdc}{\tiny\texttt{dontCare}\xspace}

\newcommand{\ord}{\sqsubseteq}
\newcommand{\ordne}{\sqsubset}

\newcommand{\white}{\ensuremath{\mathit{white}}\xspace}
\newcommand{\black}{\ensuremath{\mathit{black}}\xspace}

\newcommand{\ls}{\cd{ls}}
\newcommand{\af}{f}



\newcommand{\vit}{{\cal T}_I}
\newcommand{\vst}{{\cal T}_S}
\newcommand{\vocrm}{\mathit{voc}_R^M}
\newcommand{\tp}{\mathit{type}}
\newcommand{\tll}{\mathit{tail}}
\newcommand{\hdd}{\mathit{head}}




\newcommand{\Data}{\mathrm{Data}}
\newcommand{\func}{\mathit{func}}
\newcommand{\port}{\mathit{port}}
\newcommand{\export}{\mathit{export}}

\newcommand{\guard}{\mathit{guard}}
\newcommand{\source}{\mathit{src}}
\newcommand{\dest}{\mathit{dest}}
%\DeclareMathOperator{\var}{var}

\newcommand{\ignore}[1]{}
%\newcommand{\ie}{i.e.,}
\newcommand{\secref}[1]{Section~\ref{#1}}
\newcommand{\figref}[1]{Fig.~\ref{#1}}
\newcommand{\defas}{\stackrel{\mathrm{\scriptscriptstyle def}}{=}}

%\newcommand{\true}{\ensuremath{\mathtt{true}}}
%\newcommand{\false}{\ensuremath{\mathtt{false}}}
\newcommand{\goesto}[1][]{\stackrel{#1}{\longrightarrow}} % -->

\newcommand{\valu}[1]{\mathbf{#1}}
