\begin{frame}{Verification of C and BIP using ABC }
\begin{itemize}
 \item 
  We present two techniques with supporting tools to address the C and BIP verification problems
\begin{itemize} \item \mytool{} \item \biptool{}
 \end{itemize}
\end{itemize}
\end{frame}



\begin{frame}{\mytool }
\begin{itemize}
  \item Input
    \begin{itemize}
  \item An imperative program \Pm
  \item A precondition \Pre{} 
  \item A postcondition \Post{} written in (FOL)
  \item A bound $b$ on the data width of \Pm variables 
    \end{itemize}
  \item Translates $\Pm \models \pair{\Pre}{\Post}|_b$ into an equisatisfiable AIG
    \begin{itemize}
      \item Using program counter semantics
    \end{itemize}
  \item Uses ABC reduction and proof algorithms to reduce and check the generated AIG
\end{itemize}
\end{frame}

\begin{frame}{\mytool -- continued  }
\begin{itemize}
  \item In case ABC($\Pm \models \pair{\Pre}{\Post}|_b$) was not conclusive
    \begin{itemize}
  \item Guess or use heuristics to compute a temporal bound $T$
  \item Use BMC within ABC to check whether $\Pm \models \pair{\Pre}{\Post}|_b$ within $T$
  \item Use BMC within ABC to check whether $\Pm$ terminates within $T$
    \end{itemize}
\end{itemize}
\end{frame}


\begin{frame}{\biptool}
\begin{itemize}
 \item Input
 \begin{itemize}
  \item An input BIP system $B$
  \item An invariant $\mathcal{I}$ in FOL
  \item A bound $b$ on the width of data variables
  \end{itemize}
  \item Translates the problem $\left( B \models \mathcal{I}|_b \right)$ into an equisatisfiable AIG
  \item Uses ABC reduction and proof algorithms to reduce and check the generated AIG
\end{itemize}
\end{frame}

\section{Demo}
\begin{frame}{Binary search}
\begin{itemize}
 \item {\bf Inputs.}
 \begin{itemize}
  \item $a$: array of scalars
  \item $n$: number of elements in $a$
  \item $e$: element to search for in $a$
 \end{itemize}
 \item {\bf Output.}
  \begin{itemize}
   \item $idx$: index of $e$ in $a$ if found, $-1$ otherwise
  \end{itemize}
 \item {\bf Specification.}
  \begin{itemize}
   \item 
    $\Pre = n \geq 0 \land \forall{(i \in 0\ldots(n-2))}.~(a[i] \leq a[i+1])$
   \item 
    $\Post = (idx = -1) \implies \forall{(i \in 0\ldots n)}.~(a[i] \neq e) ~\land$ \\
    $\quad\quad(idx \neq -1) \implies (a[idx] = e)$
  \end{itemize}
\end{itemize}
\end{frame}

\begin{frame}[containsverbatim]
\frametitle{Binary search code}
\begin{Verbatim}[fontsize=\relsize{-2.5}, numbersep=4pt,numbers=left]
int binarysearch(int a [], int n, int e) {
 // @pre bs {(n >= 0) && forall(int i:[0..n-2]) {a[i] <= a[i+1]}}
   int low = 0, high = n - 1;
   while (low < high) {
     mid = (low + high) >> 1;
     if (a[mid] < el) {
       low = mid + 1;
     } else {
       high = mid;
     }
   }  
   if ((low == high) && (a[low] == el)) {
     return low;
   } 
   return -1;
   
   // @post bs {((rv == -1) -> (forall(int i:[0..n-1]) {a[i] != e}) &&
   //           ((rv != -1) -> (a[rv] == e)}
}
\end{Verbatim}
\end{frame}


\begin{frame}[c]
 \frametitle{\mytool{} demo}
 \begin{itemize}
  \item 
  \href{run:/home/mohamad/Desktop/Demo/BinarySearch.avi}{\mytool{} in action!} 
  \item 
  \href{run:/home/mohamad/Desktop/Demo/CounterEx.avi}{Check counter example!}
 \end{itemize} 
\end{frame}



%\begin{frame}{We propose \mytool}
%  \begin{itemize}
%  	\item 
%	 Given :
%	 \begin{enumerate}
%	 	\item
%		 $S$ : An imperative program implementing a functional specification.
%		\item 
%		 $P, Q$: pre/post conditions in temporal and first order logic specifying the functionality of $S$.
%		\item 
%		 Semantics: $P$ holds, then $S$ executes and completes, then $Q$ holds after the completion of $S$
%	 \end{enumerate}
%	\item 
%	 \mytool:
%	 \begin{enumerate}
%	 	\item
%		 Assumes a sequential program counter execution model
%	 	\item
%		 Generates $C$ an sequential circuit equivalent to $(P,S,Q)$. 
%		\item 
%		 Uses the synthesis and verification ABC framework~\cite{brayton2010abc} to validate $C$. 
%	 \end{enumerate}
%  \end{itemize}
%\end{frame}
%
%\begin{frame}{The program $S$}
%  \begin{itemize}
%  	\item
%	 \mytool~supports:
%	 \begin{enumerate}
%	 	\item
%		 Data: scalars, arrays, assignments, and arithmetic, and Boolean operations.
%	 	\item
%		 Control: branch, loops, and function calls including recursion.
%	 \end{enumerate}
%	\item
%	 The program counter execution model eliminates the need to unwind loops.
%  \end{itemize}
%\end{frame}
%
%\begin{frame}{The conditions $P$ and $Q$}
%  \begin{itemize}
%  	\item 
%	 The pre/post conditions are expressed in:
%	 \begin{enumerate}
%	 	\item 
%		 First Order Logic (FOL) 
%		 \begin{itemize}
%		 	\item 
%			Boolean formulas over Boolean and arithmetic operations.
%		 	\item 
%			Boolean formulas over existential and universal quantification of scalar variables.
%	     \end{itemize}
%		\item
%		 Temporal Logic : 
%		 \begin{itemize}
%		 	\item
%			 Use Sequential Extended Regular Expressions (SERE), a subset of the property specification language (PSL).
%		 \end{itemize}
%	 \end{enumerate}
%  \end{itemize}
%\end{frame}
