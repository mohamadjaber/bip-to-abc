%% C to ETCircuit transformation
After preprocessing, \mytool~transforms the program $\Pm' = \rho(\Pm)$ into 
an equisatisfiable \thislanguage~program $\mathcal{C}$. Synthesizing 
an AIG from $\mathcal{C}$ is then a direct translation of variables into bit registers
and building their next state function according to the \cci{next} block
definition in $\mathcal{C}$. 
\mytool~uses {\em program counter semantics} to translate \Pm'
into $\mathcal{C}$. The program counter is used to ensure the
correct sequencing of assignments in the program; i.e. to provide the concept of time.

Our method assigns a unique {\em label} for each statement in the
program \Pm'. Let $s$ be any statement in \Pm', 
and let $s_a, s_i, s_w$, and $f_d$ be an assignment statement, 
a conditional statement, a loop statement and a function declaration
in \Pm', respectively. Additionally, let $e$ be an array access expression,
and let $call(f_d)$ be an added \cci{call\_func} statement
that calls the function declaration $f_d$. 
Our method defines the following functions:
\begin{itemize}
 \item \cci{label($s$)}: The unique label identifying the statement $s$
 \item \cci{next($s$)}: The label of the statement that directly follows $s$ in the program order.
 \item \cci{condition($s_i$)}: The Boolean condition of the selection statement $s_i$.
 \item \cci{then($s_i$)}: The label of the first statement in the {\em then} 
 code block of $s_i$.
 \item \cci{else($s_i$)}: The label of the first statement in the {\em else}
 code block of $s_i$.
 \item \cci{condition($s_w$)}: The Boolean condition of the loo statement $s_w$.
 \item \cci{body($s_w$)}: The label of of the first statement in the body of the $s_w$.
 \item \cci{last($s_w$)}: The last statement in the body of the loop $s_w$.
 \item \cci{target($s_a$)}: The target variable of the assignment $s_a$.
 \item \cci{expression($s_a$}: The expression to be assigned to the target of $s_a$.
 \item \cci{body($f_d$)}: The label of the first statement in the body of the 
 function $f_d$.
 \item \cci{return($call(f_d)$)}: The label of the statement to which the function call 
 statement should return after calling $f_d$.
 \item \cci{base($e$)}: The array which the expression $e$ is indexing.
 \item \cci{index($e$)}: The index at which the the base of the expression $e$ is indexed.
\end{itemize}


\begin{algorithm}[tb]
\begin{algorithmic}[1]
\State {\bf Input:} \psqlanguage~ program \Pm', $entry\_statement$
\State {\bf Output:} \thislanguage~ program $\mathcal{C}$
\State 
\State // build the initialization list
\State init\_list $[ ]$
\ForAll { $v \in V$}
 \If{used\_before\_assigned($v$)}
  \State init\_list.insert ($``v = free\_inputs"$)
 \EndIf
\EndFor
\State init\_list.insert ($``pc = label(entry\_statement)"$)
\State
\State // build the next state list for variables
\State next\_state\_list $[ ]$
\ForAll {$v \in V$}
 \State $next_v = ``v = "$
 \ForAll {Assignment statement $s$ such that $target(s) = v$}
  \State $next_v += ``(pc==label(s))? ~expression(s) : "$
 \EndFor
 \State $next_v += ``v"$
 \State next\_state\_list.insert $(next_v)$
\EndFor
\State 
\State // build the next state for the program counter
\State $n_{pc} = ``pc = "$
\ForAll {statement $s$}
 \If {is\_conditional\_statement ($s$)}
   \State $n_{pc} += ``(pc==label(s))?~ (condition(s)?~then(s) : else(s)) :"$
 \ElsIf {is\_loop\_statement ($s$)}
   \State $n_{pc} += ``(pc==label(s))?~ (condition(s)?~body(s):next(s)) :"$ 
 \ElsIf {is\_function\_call(s)}
   \State $f_d :=$ function\_declaration\_of $(s)$
   \State $n_{pc} += ``(pc==label(s))?~ (body(f_d)) :"$
 \ElsIf {is\_return\_statement (s)}
   \State $f_d :=$ function\_declaration\_of $(s)$
   \State $n_{pc} += ``(pc==label(s))?~(return(call(f_d))) :"$
 \Else 
   \State $n_{pc} += ``(pc==label(s))?~(next(s)) :"$
 \EndIf
\EndFor
\State next\_state\_list.insert $(n_{pc})$
\end{algorithmic}
\caption{\psqlanguage~to \thislanguage~transformation algorithm}
\label{algo:psqtoettrans}
\end{algorithm}

Algorithm~\ref{algo:psqtoettrans} shows the procedure used 
in \mytool~to translate a preprocessed \psqlanguage~program \Pm'
with a set of variables $V$ and an initial entry statement, into an \thislanguage~program 
$\mathcal{C}$, to be used for the synthesis of the equisatisfiable 
AIG. Intuitively, the wire declarations, variable definitions and wire definitions are directly moved from 
 \Pm' to their corresponding blocks in \aigcircuit.
Additionally, in order to model time, \mytool~adds a new
scalar variable to the program \Pm', the {\em program
counter} \cci{pc}. This variable is key to avoid 
inlining functions and unrolling loops. It is used to 
ensure the proper sequencing of the statements of \Pm';
the current value of the \cci{pc} variable defines which 
statement is to be execute. For example, reassigning the 
value of \cci{pc} to the starting point of a loop's body
allows \mytool~to make use of the same code block (thus
the same AIG) to execute a single loop infinitely many 
times. 

The algorithm starts by building the initialization list
of \aigcircuit. It assigns the program counter variable
\cci{pc} to the label of the entry statement of \Pm'.
Then for each variable $v \in V$ that is used before
being assigned (i.e. used with an undefined value), it 
assigns $v$ to a set of free inputs (AIG primary inputs)
who's values are to be set by the ABC AIG solver. 
Note that since variables can be initialized at different
points in the program (and not necessarily at the starting
point of the algorithm), we delay their
initialization step 
into the \cci{next} block, where these variables
will get their initial value at the appropriate \cci{pc}
value.

Subsequently, the algorithm moves to building the
\cci{next} state block of \aigcircuit. For each 
variable $v \in V$, and for each assignment statement 
$s$ such that $target(s)=v$, the next state of $v$ will
be $expression(s)$ iff $pc == label(s)$. If at a given 
statement label, $v$ is not being assigned, it retains
its current value. For example, consider the 
following assignment
statements for $v$:
\begin{Verbatim}[fontsize=\relsize{-1.0}, numbersep=4pt,numbers=left]
(l1): x = 1;
(l2): x = 3;
(l3): x = 5;
\end{Verbatim}
The algorithm builds the next state function of $x$ as 
the following statement
\begin{Verbatim}[fontsize=\relsize{-1.0}, numbersep=4pt,numbers=left]
x = (pc == l1)? 1 : (pc == l2)? 3 : (pc == l3)? 5 : x;
\end{Verbatim}

After traversing all of \Pm's variables, the algorithm
then builds the next state assignment for the program counter.
At a given statement $s$, the default behavior of the \cci{pc} is to move from the current label $label (s)$
to the label of the statement that directly follows
$s$, i.e. $next(s)$. 
Only control statements such as conditional statements, 
loop statements, function call statements and return 
statements are allowed to alter the default behavior of 
\cci{pc}.

Conditional statements alter the value of the 
program counter based on the evaluation of their
conditional expression. Consider a conditional statement
$s$ and let $e = condition(s)$. When \cci{pc} is at 
$label(s)$, its next value is either the label 
of the first statement in then \cci{then} branch of $s$
if $e$ evaluates to $true$, or to the first
statement in the \cci{else} branch of $s$ otherwise. 
Therefore at $label(s)$, the next state of the \cci{pc}
is defined as $(e)?~ then(s) : else(s)$.

Similarly, loop statements move the program counter
into their body when their conditional expression evaluates
to $true$, and to their next statement once it is $false$.
Consider a loop statement $s$ and let $e = condition(s)$.
At $label(s)$, the next state of \cci{pc} is defined
as $(e)?~ body(s):next(s)$. Note that according to the 
\psqlanguage, the loop statement block is considered 
as a single statement, and thus the next statement 
to follow $s$ would be the first statement outside its 
body, i.e. its exit point. Additionally, for the last
statement in the body of the loop, its next statement
is defined as the loop statement itself. This ensure that
once the \cci{pc} reaches that last statement in the 
loop, it once again evaluates its condition in order to
determine whether to exit or stay in the loop.  

Finally, function call statements and return statements 
redirect the program counter into the body of the function 
and to the first statement after the call, respectively. 
Note that for a non-recursive function, there is at most
one active function call at a time, therefore it is easy
to track the label at which the program counter 
should return to.
For a recursive function, we maintain a stack of function calls and redirect the control to statement after the 
call that is on top of the stack. 

\subsubsection*{Post-processing}
After constructing the \thislanguage~program \aigcircuit,
\mytool~employs a last post-processing step to resolve 
array accessing. Given an array access expression 
$a[i]$ where $a$ is an array and $i$ is an index, resolving
this access expression is to transform it into an array
access where the index is constant. An array access 
expression with a constant index is handled as a regular 
register variable in \aigcircuit. 

The first part of algorithm~\ref{alg:resolv_target_exp} 
shows the post
processing of a reference to an array access expression 
$e$. The algorithm pushes the variable index $i$ outside
of the array access by creating a set of checks for the 
value of $i$ at each possible index of the array $a$, and
returning the appropriate array element accordingly.

The second part of algorithm~\ref{alg:resolv_target_exp} 
shows the 
post processing of statement $s$ where $target(s)$
is an array access of the form $a[i]$. Since this 
assignment statement can be to any of the elements of the 
array $a$, as dictated by the value of $i$, the algorithm
creates an assignment statement for each possible 
element of $a$. This ensure that only the correct element
get assigned to the value of $expression(e)$ by creating
a check for $i$ at each possible index of the array.  

Figure~\ref{fig:trans:arrays} shows an \thislanguage~program 
\aigcircuit~that contains array access assignments and 
references. Post-processing transforms the assignment
statement on line 14 of the program \aigcircuit into 
assignment statements over all the elements of the
array $a$, with added checks on the value of the index
$i$. Also post-processing replaces the expression $b[k]$ 
in the right hand side of the assignment statements 
with the expression $(k==0)?~b[0] : b[1]$. This expression
checks on the value of $k$ for all possible indices to the 
array $b$ and returns the corresponding
element. 

\begin{figure}[bt]
\centering
\begin{tabular}{p{0.33\textwidth}|p{0.01in}p{0.33\textwidth}}
\begin{Verbatim}[fontsize=\relsize{-2.5}, numbersep=4pt,numbers=left,
frame=topline, framesep=4mm, label=\fbox{Program \aigcircuit}]
int a [2];
int b [2];
int i, k;
@dotogether {
  a[0] = free_inputs;
  a[1] = free_inputs;
  i = free_inputs;
  k = free_inputs;
  b[0] = free_inputs;
  b[1] = free_inputs;
}
 
@dotogether {
  a[i] = (pc == l1)? b[k] : a[i];
}
\end{Verbatim}
& &
\begin{Verbatim} [fontsize=\relsize{-2.5}, numbersep=4pt,numbers=left,
frame=topline, framesep=4mm, label=\fbox{\aigcircuit~after
post-processing}]
int a [2];
int b [2];
int i, k;
@dotogether {
  a[0] = free_inputs;
  a[1] = free_inputs;
  i = free_inputs;
  k = free_inputs;
  b[0] = free_inputs;
  b[1] = free_inputs;
}
 
@dotogether {
  a[0] = (i == 0)? 
           ((pc == l1)? ((k == 0)? b[0] : b[1]) : a[0]) 
           : a[0];
  a[1] = (i == 1)? 
           ((pc == l1)? ((k == 0)? b[0] : b[1]) : a[1]) 
           : a[1];
}
\end{Verbatim}
\end{tabular}
\caption{Post processing of a \thislanguage~\aigcircuit}
\label{fig:trans:arrays}
\end{figure}


\begin{algorithm}[h!]
 \begin{algorithmic}[1]
  \State // array access on right hand side
 \State {\bf Input:} Array access expression $e$
  \State {\bf Output:} Array access expressions $e'$ with 
  constant indexing
  \State $a := base(e)$
  \State $i := index(e)$
  \State $N := size(a) - 1$
  \For{$j \in 0 \ldots N-1$}
   \State $e' += (i == j)?~ a[j] :$
  \EndFor
  \State $e' += a[N]$
  \State
  \State // array access on left hand side
  \State {\bf Input:} Statement of the form $a[i] = e$, 
  $e$ is an expression
  \State {\bf Output:} A list of statements where all array accesses have a constant index 
  \State List\_of\_statements $[ ]$
  \State $N := size(a) - 1$
  \For{$j \in 0 \ldots N-1$}
   \State List\_of\_statements.insert $(a[j] = (i == j)?~ e : a[j])$
  \EndFor
 \end{algorithmic}
 \caption{Array access resolving algorithm}
 \label{alg:resolv_target_exp}
\end{algorithm}

