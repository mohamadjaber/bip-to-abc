%% Introduction section
It is critical for software and hardware developers to design 
correct and reliable systems. In particular, safety critical 
systems such as medical equipment, navigation control and 
targeting devices do not tolerate defects in the their logical 
components. 
%\todo{use the term defects instead of errors across the thesis.}
A logical defect might lead to severe consequences including loss of human life. 
A loss of precision due to a 
conversion from 64-bits into 16-bits of an integer value, 
caused the Ariane-5 missile to crash only 16 seconds after
it has been launched. Additionally, a software bug in the control of a radiation
therapy machine, ``Therac-25'', has lead to the death of six cancer patients between 
1985 and 1987~\cite{baier2008principles}.

%% Use this as intro for BIP
In addition, information technology (IT) systems have made their 
way into several aspects of modern life.
Smart phones, TVs, personal computers, laptops, banking,
money transfer and e-shopping rely on IT systems. 
Logic defects in such systems can lead to loss of service, as well as other damages. 
Intel had to recall its faulty Pentium processors, that had a bug in the floating point unit, 
for a cost of 475 million US dollars~\cite{kropf1999introduction}.

Researchers introduced techniques to 
(1) accurately specify the requirements of logic components, 
(2) validate that the implementation of a logic component respects the specifications for a finite bounded domain, 
and (3) prove that a logic component respects its specifications for unbounded domains. 
%\todo{what is the difference between 2 and 3? bounded and not bounded?} 
{\em Dynamic analysis} techniques exercise the logical component for a given set of inputs, e.g. test cases, and 
check the results against a correctness predicate or an oracle~\cite{baier2008principles}.
Exhaustive testing is impossible in practical scenarios, because it involves exercising a 
practically infinite number of test cases~\cite{ammann2008introduction}.

{\em Static analysis} techniques are used to check and prove correctness 
of logic components with respect to formal specifications. %\todo{you use units/components and design/system ... use one of each terms}
Tools exist that synthesize, optimize, and check sequential circuits. 
ABC~\cite{brayton2010abc} is an industrial strength academic sequential synthesis 
and verification tool. It takes as input an And-Inverter-Graph (AIG) representation 
of a sequential circuit, reduces it using synthesis and reduction algorithms, 
and checks it for correctness using proof algorithms. An AIG is a directed acyclic graph with 
two input AND gates, inverters and memory elements. Since AND gates and inverters
(i.e. NAND gates) are functionally complete, the restriction of logic gates to 
ANDs and inverters does not affect expressiveness.

Software systems are usually designed in high level programming languages 
such as C, C++ and JAVA. 
Software verification tools such as CBMC~\cite{clarke2004tool}
take as input software programs and check them for 
memory safety, array access safety, and user defined assertions (invariants). 
Given bounds on the unwinding depth of the loops, and on range of variables in the program, 
CBMC unfolds the program
into a Boolean {\em Conjunctive Normal Form} (CNF) formula that asserts the specified 
properties. CBMC then uses Boolean {\em satisfiability} (SAT) solvers such as MiniSat~\cite{sorensson2005minisat}
to check for correctness. CBMC verification tasks often fail to scale well 
with the increase in the size of the problem; specifically, with the loop unwinding bound and 
the complexity of the program. 

%% Talk about hardware and embedded systems design. Introduce AIG + ABC and BIP
% CBS
Furthermore, in recent years, 
the application area of embedded systems has witnessed a large expansion, especially with 
the emergence of automotive electronics and mobile and control devices.
Computations in embedded systems are subject to several physical and architectural 
constraints that render the separation between software and hardware design impractical~\cite{henzinger2006embedded}.
%The design of embedded systems involves
%{\em Component-based systems} design 
%and control devices
%Furthermore, {\em component-based system} (CBS) design is gaining more prominence 
%after the widespread of embedded systems, especially for mobile and control devices. 
The Behavior-Interaction-Priority (BIP) framework 
%%%%%%%%%%%%%%%%%%%%%%%%%%%%%%%%%%%%%%%%%%%%%%%%%%%%%%%%%% Double check this todo note %%%%%%%%%%%%%%%%%%%%%%%%%%
%\todo{lots of grammatical errors. plz check thoroughly. plural vs singular, missing noun between a sequence of qualitative adjectives/adverbs and the verbs, parallel structures}
is a {\em Component-Based System} (CBS) design framework that uses a dedicated 
language and tool-set in order to support a rigorous and layered design flow for emebedded systems.  
%\todo{the introduction sentence of BIP is weak and obviously a sales statement repeating design several times.}
It allows
to build complex systems by coordinating the behavior of a set of atomic components~\cite{BasuBBCJNS11}.
In order to check CBS,  the BIP framework uses: (1) DFinder~\cite{dfinder}, a compositional  
and incremental verification tool-set, and (2) the NuSMV~\cite{nusmv} model checker.
However, DFinder does not handle data transfer between individual 
BIP components, and does not support checking for properties other than deadlock freedom. 
Additionally, for complex systems, NuSMV often suffers from the state space explosion 
problem~\cite{sipser2006introduction}, and fails to perform its verification tasks.

In this thesis, we present two techniques with supporting tools 
%\todo{and supporting tools?} 
to address the verification of both software systems and CBSs: 
(1) \mytool and (2) \biptool, respectively.
\mytool~is a tool that takes 
%\todo[inline=true]{tool versus technique? make up your mind.}
an imperative program \Pm with a specification,
a precondition and postcondition pair \pair{\Pre}{\Post},
and checks
whether \Pm satisfies the specification. This check 
is performed  within a 
bound $b$ on the domain of the program and specification
variables (\prob);
i.e. when the bounded inputs of \Pm satisfy \Pre, 
the outputs of \Pm satisfy \Post. 
The program is written in a subset of C++ that 
includes integers, arrays, loops, and recursion,
extended with a \cci{do\_together} construct.
The specification consists of a precondition \Pre and
a postcondition \Post, both written in 
{\em first order logic} (FOL). 
Our method translates the problem \prob~
into an equisatisfiable AIG,
and passes the AIG to ABC for
sequential synthesis and model checking. 
If successful, the check implies that the \Pm satisfies its 
specification pair \pair{\Pre}{\Post}. Otherwise, it returns a counterexample
and enables the developer to visualize it using the GtkWave~\cite{bybell2010gtkwave} waveform viewer for
debugging. 

\biptool~ is a tool that targets two design goals of the BIP framework:
(1) verification and (2) code generation. Similarly to \mytool, it takes an input 
BIP system with a set of optional specifications, 
and translates it in an equisatisfiable AIG. It then uses the ABC model checker 
to perform synthesis and verification. 
Additionally, unlike the BIP code generator used in~\cite{BasuBBCJNS11} that
uses a generic implementation of the BIP engine, 
\biptool~generates an AIG system with a customized engine that is able to 
execute multiple atomic component transitions simultaneously. 


\section{Advantages of sequential circuits}
\label{s:intro:advantages}
We formally define sequential circuits in
Chapter~\ref{chap:back}; for now a sequential
circuit can be viewed as a restricted C++ program,
specifically a multi-threaded program in which all
variables are either integers, whose range is statically
bounded, or Boolean-valued, and dynamic allocation is
forbidden~\cite{edwards2005challenges}.
%
%Given a design with a property and a bound, we
%automatically derive a sequential circuit and a Boolean
%variable therein that serves as an {\em invariant}, i.e.,
%the variable is stuck to $\mbox{true}$ if and only if the
%property of the design is valid within the bound.

There are two key advantages to compiling systems 
into sequential circuits:
\begin{description}
\item[Advantage 1] Sequential encodings are much more 
succinct than pure combinational SAT formulae.
They are imperative and state-holding while CNF formulas, for example, are 
declarative and state-free.  
For example, they can naturally represent 
the execution of quantifiers and loops without 
the need for unrolling them.
Moreover, they can store and reuse 
intermediate results in local variables.
In cases, SAT encoding algorithms produce a
data structure that uses several orders of magnitude more 
memory to represent.

\item[Advantage 2]  Casting the decision problem for a 
property of a system as an invariant check on a sequential 
circuit allows us to make use of a number of powerful 
automated analysis techniques that we discuss in 
Chapter~\ref{chap:back}
and that have no counterpart in CNF analysis. 
\end{description}


\mytool~and \biptool~use ABC~\cite{brayton2010abc} to automatically check
invariants in AIGs. ABC is
%We use ABC~\cite{brayton2010abc}, an open source sequential circuit 
%solver tool, to automatically check invariants in sequential circuits. 
%ABC is 
a transformation-based 
verification (TBV)~\cite{kuehlmann2001transformation} framework that 
encompasses reduction and abstraction techniques such as 
%{\bf TODO: all to those references citations to Mishenko}
retiming~\cite{KuBa01}, redundancy 
removal~\cite{HmBPK05,KuMP01,BjesseC00,aziz-fmsd-00}, logic
rewriting~\cite{BjBo04}, interpolation~\cite{McMillan03}, 
and localization~\cite{Wang03}.
It operates on AIGs; Boolean netlists with
memory elements, and iteratively and synergistically 
calls numerous transformation and abstraction algorithms. 
These algorithms simplify and decompose complex problems 
until they become tractable for decision techniques such 
as symbolic model checking,
bounded model checking, induction, interpolation, 
circuit SAT solving, and target
%{\bf TODO: all to those references citations to Mishenko}
enlargement~\cite{MoGS00,MoMZ01,HoSH00,BaKuAb02,Hari05expert}.

In this thesis we make the following contributions. 
\be
\i We encode an imperative program \Pm with a first order logic 
specification \pair{\Pre}{\Post} into an AIG with an invariant
that is stuck to $\mathit{true}$ iff  \Pm satisfies \Post~ for all inputs that
satisfy \Pre within a given bound on the range of the program variables. 

\i We use a program counter semantics to encode the \Pm~ and
\pair{\Pre}{\Post} into an AIG. 
We use that to encode and compute
a termination guarantee check within a bound on the number of
iterations or recursive calls a program can make. 
We use the termination guarantee bound to prove run time efficiency of given
algorithms.
%We also use the termination guarantee bound to prove conclusive correctness
%using bounded model checking, in case other proof techniques did not return
%conclusive results. 
\i We encode a BIP system into an AIG with an invariant that is 
stuck to $\mathit{true}$ iff the system is deadlock free, or satisfies 
a set of specifications. The generated AIG system will have its own 
customized engine that is able to execute multiple transitions simultaneously.

\i We allow for simulation of the BIP system by generating a C++ code, 
while allowing for optional user control over the interactions to execute. 

\i We use the ABC sequential circuit verification framework to check the
generated AIGs, and we check systems and programs that are orders of magnitude
higher than those possible with the other techniques. 

\i We implement our methods in two tools: \mytool~and \biptool. 

\i We provide our tools and benchmarks online~\footnote{\label{fn:online}\url{
http://webfea.fea.aub.edu.lb/fadi/dkwk/doku.php?id=sa}}. 
\ee


%% organization
The rest of this thesis is orgranized as follows. Chapter~\ref{chap:back} gives an overview
of the preliminary information needed throughout this thesis. Chapters~\ref{chap:c2aig}
and~\ref{chap:bip2aig} present the techniques used in \mytool{} and \biptool{}, respectively. 
Chapter~\ref{chap:implementation} highlights the main implementation details and gives a brief
user manual for each of the two tools. We present our results in chapters~\ref{chap:results}
and~\ref{chap:bipresults}. We discuss relevant related work in chapter~\ref{chap:related}
to finally conclude in chapter~\ref{chap:conclusion}. 




%%%%%%%%%%%%%%%%%%%%%%%%%%%%%%%%%%%%%%%%%%%%%%%%%%%%%%%%%%%%%%%%%%%%%%%%%%%%%%%%%%%%%%%%%%%%%%%%%%%%%%%%%%%%%
%%%%%%%%%%%%%%%%%%%%%%%%%%%%%%%%%%%%%%%%%%%%%%%%%%%%%%%%%%%%%%%%%%%%%%%%%%%%%%%%%%%%%%%%%%%%%%%%%%%%%%%%%%%%%

%%In  this thesis, we present a method that takes 
%%an imperative program \Pm with a specification,
%%a precondition and postcondition pair \pair{\Pre}{\Post},
%%and checks
%%whether \Pm satisfies the specification within a 
%%bound $b$ on the domain of the program and specification
%%variables (\prob);
%%i.e. when the bounded inputs of \Pm satisfy \Pre, 
%%the outputs of \Pm satisfy \Post. 
%%The program is written in a subset of C++ that 
%%includes integers, arrays, loops, and recursion,
%%extended with a \cci{do\_together} construct.
%%The specification consists of a precondition \Pre and
%%a postcondition \Post, both written in 
%%first order logic. 
%%Our method translates the problem \prob~
%%into an equisatisfiable sequential circuit,
%%and passes the circuit to ABC~\cite{brayton2010abc}, 
%%a sequential circuit solver, to decide its validity. 








%% Then move to talking about the presented tools : psq and bipi
%CBMC's check, if valid, is only as good as the bound


%In particular, model checkers such as ABC~\cite{brayton2010abc}, 
%take an And-Inverter-Graph (AIG),
%a directed acyclic graph with two inputs AND gates, inverters and 
%memory elements, 
%reduce it using synthesis algorithms, and check it for correctness
%using proof algorithms. 
%At a higher level, existing techniques transform software programs
%and embedded system design components into Conjunctive Normal Form (CNF)
%formaulae or SMV code, and use satisfiability (SAT) solvers and symbolic
%model checkers to check their validity. These techniques often fail to scale well with 
%the increasing size of today's systems. 
%
%Safety critical systems such as medical equipment, navigation control and targeting devices 
%rely on logical units such as software and hardware programs to provide accurate services.
%Therefore it is critical for software and hardware
%developers to design correct and reliable systems. 
%Static analysis techniques are ones of the most used 
%techniques to achieve such a goal.
%A logical defect might lead to loss of human life. A simple loss of precision due to a 
%conversion of a 64-bit value into a 16-bit integer, caused the Ariane-5 missile to crash only 16 seconds after
%it has been launched. Additionally, a software bug in the control of a radiation
%therapy machine, ``Therac-25'', has lead to the death of six cancer patients between the years of 
%1985 and 1987~\cite{baier2008principles}.
%%
%
%Researchers introduced techniques to (1) accurately specify the requirements of logical units, (2) validate
%that the implementation of a logical unit respects the specifications for a given domain, 
%and (3) prove that a logical unit respects the specifications. 
%{\em Dynamic analysis} techniques exercise the logical unit for a given set of inputs, e.g. test cases, and checks
%the results against a correctness predicate or an oracle~\cite{baier2008principles}.
%Exhaustive testing is impossible in practical scenarios, because it involves exercising an infinite number of test cases~\cite{ammann2008introduction}.







%Static analysis techniques include {\em theorem proving}, 
%{\em model checking} and {\em bounded model checking}. 
%Theorem proving techniques search a computational knowledge tree composed of meta-axioms, deduction 
%rules and a set of assumptions for a valid sequence of statements that lead from the programs to 
%the specifications~\cite{ouimet2007formal}. Model checking uses symbolic computations of a transition system that represents
%the logical unit and symbolically computes reachable states~\cite{clarke1999model,d2008survey}.
%Theorem proving and model checking suffer from the state explosion problem. 
%Bounded model checking techniques such as SPIN~\cite{holzmann1997model}, CBMC~\cite{clarke2004tool} and KodKod~\cite{torlak2007kodkod}, validate that the specifications hold 
%for a finite domain that is bounded by either the number of objects, the number of transitions, 
%or the execution time of the logical unit. 
%
%
%%-- CBMC --%
%CBMC is a bounded model checking tool that checks for pointer safety, within bound
%array access and user defined assertions in ANSI C programs~\cite{clarke2004tool}.
%Given a C program and a bound on the range of variables, CBMC unfolds the program 
%into a Boolean {\em Conjunctive Normal Form} (CNF) formula that asserts the specified properties. 
%It then uses satisfiability (SAT) methods and tools such as MiniSat~\cite{sorensson2005minisat}
%to check the CNF formula for counter examples. 
%CBMC's check, if valid, is only as good as the bound for which 
%it made the check, any bug that occurs beyond this bound is not detected.
%
%
%%CBMC is one of the most famous tools to perform bounded model checking. It allows the checking of a
%%C program against a set of assertions, written in the form of \CodeIn{assert} statements. Additionally,
%%it allows checking the equivalence of a C model, to another model written in VHDL or Verilog~\cite{clarke2004tool}.  
%%CBMC supports loop unwinding, recursion unwinding, and perform single static assignment (SSA) to handle
%%branches, function calls and \CodeIn{goto} statements. After a series of transformation, CBMC will hopefully 
%%generate a formula in conjunctive normal form (CNF) that is equivalent to the original program. The problem 
%%is then reduced to a satisfiability (SAT) problem, and a SAT solver such as MiniSat~\cite{sorensson2005minisat}
%% is used to perform the check.
%%CBMC's check, if successful, is only as good as the bound for which it made the check, any bug that occurs
%%beyond this bound can never be reached.
%
%%-- Alloy --%
%Alloy is a small tool with a language targeted at describing structural properties. It presents an expressive 
%language to formally describe the properties of a certain system, and then check whether such properties 
%are valid, within a finite scope (bound), i.e. Alloy tries to generate a model that satisfies the given 
%properties withing the finite scope~\cite{jackson2002alloy}. 
%
%%-- Kodkod --%
%KodKod is a tool that combines first order logic (FOL) with relational algebra and transitive closure to 
%express properties. Then given some properties and a bound, KodKod tries to find a model (or a partial model)
%that satisfies (or partially satisfies) the properties. KodKod attempts to detect symmetries in a given 
%problem, and predicates that break such a symmetries. It then translates such expressions into CNF 
%and uses a SAT solver to find models for the properties~\cite{torlak2007kodkod}.
%
%%-- Previous work to build on --%
%In~\cite{fz2007relational, fz2007program}, the usage of CNF formulas for checking C programs and performing relational 
%analysis is replaced by the usage of sequential circuits and sequential verification techniques. 
%The authors present SERA and SEBAC,
%two algorithms for encoding Alloy formulas and C programs with assertions as sequential circuits, respectively. 
%Sequential circuits allow for more succinct representations of the given problems than CNF. Additionally, 
%sequential circuits analysis offers a wide range abstraction and verification techniques that can be applied 
%to generate models, and validate systems. Furthermore, word level optimizations can be combined with 
%circuit analysis techniques to further simplify the problem, and scale the verification of software 
%systems into higher bounds.  
%
%In this work we build on~\cite{fz2007relational,fz2007program} and present an encoding of software programs 
%and FOL properties as And Inverter Graphs (AIGs). Given an imperative program with a set of FOL properties, 
%we generate an equivalent AIG with a primary output, such that the primary output is asserted if and only if
%the program violates its properties, i.e. if there exists an execution path of the program that violates
%the properties. We present an encoding of the control flow of the program in the AIG, thus reducing the 
%explosion problem cause by SSA. Additionally, we present a novel synthesis technique of formulas written in 
%Linear Temporal Logic (LTL) into finite state machines (FSM) with number of states linear in the number of 
%propositions in a given formula. We also encode such FSMs as AIG. 
%
%%% TODO: add all the contributions
%We make the following key contributions:
%\begin{enumerate}
% \item We fully synthesize the control flow and data path of a given program into AIGs.
% \item We make use of bit level optimizations to reduce the amount of work needed to perform verification, 
% thus allowing for larger bounds.
% \item We allow the usage of recursive functions and encode them in the control flow of the program, 
% without the need for unwinding. 
% \item We present a novel synthesis technique for LTL formulas into AIGs. 
% %\item We present a technique to compute the diameter of a synthesized circuit, in order to be able to 
% perform BMC. 
% \item We integrate the above techniques with the ABC framework~\cite{brayton2010abc}, and present them in one tool, {\bf \mytool}.
% \item We present a debugging interface that lifts back the counter example of a failed verification to the 
% original source, allowing for the localization of the faulty statement. 
%\end{enumerate}

%% Organization
%The rest of this proposal is organized as follows: preliminary information 
%is presented in Chapter~\ref{chap:prelim}.
%The software verification technique using sequnetial circuit encoding is presented in chapter~\ref{chap:soft_analysis}. 
%We present the novel LTL synthesis technique in chapter~\ref{chap:ltlSynth}.
%Chapter~\ref{chap:related} compares against different other techniques present in the literature for 
%both software verification and LTL synthesis.
%We finally present a conclusion and a future time plan for the thesis in chapter~\ref{chap:conclusion}.
