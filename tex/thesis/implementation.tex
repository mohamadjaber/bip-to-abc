%% discuss the implementation of the tools and give a brief user manual
%% on how to use them
This chapter discusses the implementation of \mytool{} and \biptool{}. 
It highlights the main design options and gives a brief introduction to the usage
of each of the tools.

\section{\mytool{}} \label{chap:implementation:psq}
Figure~\ref{fig:psq_architecture} shows the overall architecture of the \mytool{} tool. 
We implemented a parser for \psqlanguage{} using ANTLR~\cite{parr1995antlr}.
Given a program \Pm and an FOL precondition postcondition pair \pair{\Pre}{\Post},
the parser generates a {\em control flow graph} (CFG) $G_\Pm$  representing the different
control paths that can be taken in \Pm. The code transformation engine in \mytool{}
operates on $G_\Pm$ and builds a CFG $G'_\Pm$ in which all function calls, recursive functions, 
quantifiers and specifications have been resolved. 

\begin{figure}[tb]
\centering
\resizebox{0.8\textwidth}{!}{
 \input{figures/psq_diag.pdftex_t}
}
\caption{Architecture of \mytool{}}
\label{fig:psq_architecture}
\end{figure}

Given a bound $b$ on the data width of variables in \Pm, $G'_\Pm$ is then translated into an equivalent
\thislanguage{} CFG $G''_\Pm$ in accordance with the translation algorithms presented in Section~\ref{chap:c2aig:sec:transformation_et}.
We use the AIG API provided in ABC to translate $G''_\Pm$ into an equisatisfiable AIG \aigcircuit.
We check the consistency and sanity of \aigcircuit{} using the provided ABC checks, and can thus start performing synthesis 
and verification procedures. If the verification is conclusive, either a proof for $\Pm \models \pair{\Pre}{\Post}|_b$ has been
established, or a counterexample has been generated. The user can visualize the generated counterexample using the Gtkwave tool 
for debugging before fixing the code and re-initiating the verification procedure again. 
In the case where ABC fails to provide a conclusive result, the user can retry the verification procedures using higher 
computational and architectural resources if available. We implemented \mytool{} entirely in C++.

\subsection*{A short tutorial}
\mytool{} is equipped with an interactive front-end that uses the GNU readline library~\cite{fox1994gnu}
and aims at providing users with a user friendly command based interface to parse input files, start
synthesis and verification procedures, and explore counter examples. Each command has a set of input arguments
that can be controlled by the user to customize the synthesis and verification engines. 
%Accessing the list 
%for arguments for any command can be done using the \cci{--help} or \cci{-h} arguments. 

In addition, \mytool{} supports a batch execution mode where it takes an input file containing a set of
commands, one per line, and executes them on behalf of the user without the need for going through the interactive
interface.

Figure~\ref{fig:example_code} shows a simple example program that adds to positive integers
\cci{x} and \cci{y} and checks that the result is larger than both. The \cci{class} keyword
is required for future support of object oriented programming, and currently has no effect
on the rest of the source code. The \cci{main} function specifies the entry of the program 
and is the only function that is allowed and required to be of \cci{void} return type,
and thus does not include a \cci{return} statement.
\begin{figure}[bt]
\centering
\begin{Verbatim}[fontsize=\relsize{-2.0}, numbersep=4pt,numbers=left,frame=lines]
class test {
  int x;
  int y;

  void main ()
  {
	  int z;

	  @pre add {
	    x > 0 && y > 0;
	  }

	  z = x + y;

	  @post add {
	    z > x && z > y;
	  }
  }
};
\end{Verbatim}
\caption{Example source code}
\label{fig:example_code}
\end{figure}

Starting \mytool{} in interactive mode can be done using the \cci{--int} argument flag as follows:
\begin{Verbatim}[fontsize=\relsize{-2.0}, frame=lines]
./run --int
### This is {P}S{Q} version 1.1.0 brought to you by
### the Software Analysis and Research Lab (SARLA) at
### the American University of Beirut (AUB).

### This version was compiled on May  2 2014 08:02:00

{P}S{Q}>
\end{Verbatim}
The \cci{help} command lists all of the available \mytool{} commands. The description of each command can be 
viewed by passing it the \cci{--help} or \cci{-h} command line argument. We start by parsing the source code
presented in figure~\ref{fig:example_code} using the \cci{read} command, and we pass it the name of the source file
using \cci{-i} flag:
\begin{Verbatim}[fontsize=\relsize{-2.0}]
{P}S{Q}> read -i add.c 
{P}S{Q}::Parsing file add.c...
{P}S{Q}::Parsing successful...
\end{Verbatim}
Next we start the synthesis and verification engine using the \cci{start\_prove} command, to which we 
pass our customized data width bounds. \mytool{} supports bounds on the data width of variables, 
on the number of elements in the arrays, and on the recursion depth of recursive functions.  
It also allows users to alter some flags that enable and disable array index checks, unsigned
integer overflow checks and unrolling of quantifiers if possible. We start the synthesis and verification
procedures with a data bit width on variables of $4$ and with all checks enabled:
\begin{Verbatim}[fontsize=\relsize{-2.0}]
{P}S{Q}> start_prove -b 4
*** Warning! Using default recursion check depth = 3
{P}S{Q}: Removed 0 dangling nodes from code graph!
+----------------------------------------------------------------------+
Configuration parameters:
-------------------------
Abc Debugging:               Disabled
Index Checking:              Enabled
Overflow Checking:           Enabled
Unroll quantifiers:          Yes
Group conditions:            Yes
Check termination:           No
SD Computation:              No
Latch bit width:             4
Array Bound:                 15
Stack Limit:                 15
Program Counter bit width:   4
+----------------------------------------------------------------------+
\end{Verbatim}
\mytool{} prints the list of configuration parameters that it is going to use for synthesis
and verification. The issued warning draws the user's attention to the fact that the 
default value for the depth of the recursion check is $3$. This value is used to set a bound 
on the maximum depth that \mytool{} will explore in order to determine whether a function 
is indirectly recursive or not. 

Once the \cci{start\_prove} command terminates, \mytool{} has generated the AIG \aigcircuit{}
and has linked it with ABC in order to apply reduction and proof algorithms. \mytool{} redirects
all commands that are prefixed with \cci{abc} to the ABC model checker. Reducing and proving 
\aigcircuit{} is done as follows:
\begin{Verbatim}[fontsize=\relsize{-2.0}]
{P}S{Q}> abc balance
Calling ABC with the following command
   balance 
{P}S{Q}> abc ssweep -r
Calling ABC with the following command
   ssweep -r 
{P}S{Q}> abc print_stats
Calling ABC with the following command
   print_stats 
(null)       : i/o =    8/    1  lat =   18  and =    129  lev = 10
{P}S{Q}> abc dprove 
Calling ABC with the following command
   dprove 
Output 0 of miter "(null)" was asserted in frame 3. Time =     0.00 sec
Networks are not equivalent.
\end{Verbatim}
We have applied the balancing and structural register sweeping reduction algorithms on the 
generated AIG before calling the proof procedure \cci{dprove}. ABC has detected a counterexample
that asserts the output of \aigcircuit{} in $3$ time frames. We can dump the values of the
counterexample in a \cci{vcd} file that can be read by the Gtkwave waveform viewer using 
the $debug -o add.vcd$ command. The counterexample
shows that the values of $x$ and $y$ are $14$ and $6$ respectively. Although this does not violate
the postcondition \Post, it creates an overflow exception since that the largest representable unsigned 
integer with a data bit-width of $4$ is $15$, \ie{} the addition $14 + 6$ overflows. 
We resort to making the precondition \Pre stronger in order to resolve the overflow. Adding the condition 
$x < 15 - y$ to \Pre would be enough to remove the overflow exception. We fix the source code accordingly and
rerun the verification procedure using the interpolation proving algorithm:
\begin{Verbatim}[fontsize=\relsize{-2.0}]
{P}S{Q}> abc balance
Calling ABC with the following command
   balance 
{P}S{Q}> abc ssweep -r
Calling ABC with the following command
   ssweep -r 
{P}S{Q}> abc int
Calling ABC with the following command
   int 
Property proved.  Time =     0.05 sec
\end{Verbatim}

Running the above commands in batch mode can be easily done by saving the commands in a text
file, one per line, and then passing the \cci{--batch} command line argument along with the name 
of the file to \mytool{}.

\subsection*{Command line reference}
Table~\ref{tb:command:reference} provides a reference for the command line interface options in
\mytool{}. The list of command line options that can be passed to each of these commands can be
retrieved by passing the \cci{-h} flag to the command for help. 
%% TODO: include a fill appendix if needed

\begin{table}[tb]
\caption{Summary of \mytool{} commands}
\begin{tabular}{|p{0.2\textwidth}|p{0.7\textwidth}|}
\hline
\multicolumn{1}{|c}{{\bf Command}} & \multicolumn{1}{|c|}{{\bf Description}} \\ \hline
\cci{abc cmd} & Call ABC with the command \cci{cmd} \\ \hline
\cci{echo str} & Print \cci{str} to the standard output \\ \hline
\cci{help} & Display a list of available commands \\ \hline
\cci{read} & Parse an input file and generate a code graph \\ \hline
\cci{print\_time} & Print the elapsed time in the last command and the total time elapsed \\ \hline
\cci{print\_conf} & Print the current configuration of \mytool{} \\ \hline
\cci{prove\_status} & Print the status of the current proving session \\ \hline
\cci{sim\_status} & Print the status of the current simulation session \\ \hline
\cci{start\_prove} & Start a proving session \\ \hline
\cci{start\_sim} & Start a simulation session \\ \hline
\cci{sim\_variable} & Print variable values for a range of simulation frames \\ \hline
\cci{stop\_prove} & Stop the current proving session \\ \hline
\cci{stop\_sim} & Stop the current simulation session \\ \hline
\cci{debug} & Debug the counter example generated from the current proving session \\ \hline
\cci{quit,exit} & Exit the \mytool{} enviroment \\ \hline
\end{tabular}
\label{tb:command:reference}
\end{table}



\section{\biptool{}} \label{chap:implementation:bip}
We implemented \biptool{} in JAVA and used the BIP parser provided 
freely in~\cite{verimagbiponline}. Given an input BIP program \Pm, 
the parser generates a parse graph $G$. \biptool{} traverses $G$ 
and translates \Pm into an equivalent \thislanguage{} program 
\Pm'. We then use the same synthesis engine in \mytool{} to synthesize
an equisatisfiable AIG circuit \aigcircuit{} that is passed to ABC 
for sequential synthesis and verification.

\biptool{} is equipped with a command line interface that accepts a set
of configuration options. It takes the name of the input BIP file, optional 
configuration flags and an option to generate C++ emulation code. After 
parsing the input file, \mytool{} performs synthesis steps and generates 
an output \thislanguage{} file. 
We then pass this file to \mytool{} to synthesize the AIG circuit \aigcircuit{}
and call ABC for reduction and verification. \biptool{} makes use of the same debugging
interface provided by \mytool{} in order to allow users to visualize counterexamples
if any is return by ABC. 

Additionally, \mytool{} supports the generation of C++ emulation code. It allows
the users to guide the execution engine by providing a guide file
that lists the order in which interactions are to take place. In case no such file 
is provided, the emulator uses a random number generator to choose an interaction to execute
in case several candidates  are available. \biptool{} also allows
users to set a bound on the number of clock cycles to execute in the emulation code. 
Furthermore, \mytool{} allows users to specify whether to initialize the variables 
in the atomic components or to leave them as free inputs. Figure~\ref{fig:implementation:bip}
shows \biptool's command line options along with their description and an example usage. 

\begin{figure}[bt]
\centering
\begin{Verbatim}[fontsize=\relsize{-2.0},frame=single]
##Usage
`java -jar bip-to-abc.jar [options] input.bip output.abc [property.txt]`

where:

  input.bip    = input BIP file name (required)
  output.abc   = ABC file name to be generated (required)
  property.txt = Pre and Post condition written in two different lines (optional)

and options are:

  -?                prints usage to stdout; exits (optional)  
  -emulator <s>     Generate emulation code output.abc.cpp (optional)  
                     - guide.txt: indices of interactions assigned to selector 
                     - integer <= 0: infinite exection 
                     - integer > 0: number of cycles to be executed 
  -h                prints usage to stdout; exits (optional) 
  -help             displays verbose help information (optional)
  -initialize-vars  Initialize free variables (optional)
  -version          displays command's version (optional)

##Example

`java -jar bip-to-abc.jar -initialize-vars -emulator=guide input.bip output.abc property.txt'  
`java -jar bip-to-abc.jar -o -i -e=guide input.bip output.abc property.txt'  
`java -jar bip-to-abc.jar -o -i -e=0 input.bip output.abc property.txt'  
`java -jar bip-to-abc.jar -o -i -e=30 input.bip output.abc property.txt'  
`java -jar bip-to-abc.jar -o -e=guide input.bip output.abc'  
\end{Verbatim}
\caption{\biptool's command line options}
\label{fig:implementation:bip}
\end{figure}
