% First Order Logic
\begin{frame}{First Order Logic}
  \begin{itemize}
    \item Terms: \begin{itemize} \item constants (0,1), \item scalar variables (x,y), \item arithmetic and functional operations over terms (x+y, f(x,y))
      \end{itemize}
    \item Atomic propositions: Boolean variables (b), predicate operations over terms (f(x,y) <y-1)
    \item Propositions: 
    \begin{itemize}\item Atomic propositions, 
        \item Boolean operations over Propositions (x< y and b),
        \item Universal and existential quantifiers over scalar and Boolean variables ($\forall x,y,z. (x\le y \wedge y\le z) \implies (x \le z)$)
   \end{itemize}
  \end{itemize}  
%%  
%%  \begin{itemize}
%%  	\item
%%	 Scalars.
%%	\item 
%%	 Boolean formulas over:
%%	 \begin{enumerate}
%%	  \item 
%%	   Binary and unary arithmetic and boolean operators.
%%	  \item
%%	   Universal and Existential quantifiers of scalar variables~\cite{bradley2007calculus}.
%%	 \end{enumerate}
%%	\item e.g. $\forall{i} \in [0\dots10], i < n \imp a[i] == e$
%%  \end{itemize}
\end{frame}

\begin{frame}{AIG and sequential circuits}
\begin{definition}
  A {\em sequential circuit } is a tuple $C = \left((V,E),G,O)\right)$ where
\begin{itemize}
 \item $(V,E)$ is a directed acyclic graph 
 \item $G: V \mapsto types$ maps vertices to their types 
 \begin{itemize} \item {\em primary inputs, registers} and {\em logic gates}\end{itemize}
 \item $O \subseteq V$ primary outputs of $C$
 \item An AIG is sequential circuit such as
 \begin{itemize}
  \item $types = \set{primary~input, register, 2~bit~AND~gate, inverter}$
 \end{itemize}
\end{itemize}
\end{definition}
\end{frame}

\begin{frame}{AIG registers and gates}
\begin{itemize}
 \item An AIG register node has two fan-ins
 \begin{enumerate}
  \item initial value function (can be $true$, $false$, dont-care)
  \item a next state function
 \end{enumerate}
 \item An AIG AND gate has two input fan-ins
 \item An AIG Inverter has one input fan-in
\end{itemize}
\end{frame}

%%\begin{frame}{AIG}
%% \begin{definition}
%%  An AIG is a sequential circuit $C = \left((V,E),G,O)\right)$ such that
%% \end{definition}
%% \begin{itemize}
%%  \item $(V,E)$ is acyclic
%%  \item $types = \set{primary~input, register, 2~bit~AND~gate, inverter}$
%%  \item For simplicity, inverters are labels on the edges 
%% \end{itemize}
%%\end{frame}

\begin{frame}{Advantages of AIG encoding}
\begin{enumerate}
  \item More succinct than CNF
  \item State-holding while CNF is state free
  \begin{itemize}
   \item preserves the structure of the program
   \item removes the necessity to unroll loops
  \end{itemize}
  \item Allows the reuse of subcomponents of the AIG 
  \item There is a wide range of automatic analysis techniques for AIG not applicable to CNF
 \end{enumerate}
\end{frame}

\begin{frame}[containsverbatim]
\frametitle{Software encoding}
\begin{itemize}
 \item \thislanguage{} is a software interpretation of an AIG
\end{itemize}
\begin{Verbatim}[frame=single,fontsize=\relsize{-2.5}]
declarations of variables //registers
declarations of wires //named intermediate results, outputs

definition of wires 
  //assignment statements binding wire values to expressions

initial value definitions 
 //assignment statements binding variables to their initial value gates

while (true) {
    next state function definitions 
      //assignment statements binding variables to their next state 
      //value fucntions
}
\end{Verbatim}
\end{frame}

\begin{frame}[containsverbatim]
\frametitle{Example}
\begin{itemize}
 \item Consider the following example of a two bits counter
\end{itemize}
\begin{Verbatim}[frame=single,fontsize=\relsize{-2.5}]
// Block 1:
int x; // this represents two registers

// Block 2: skipped since no wires

// Block 3: initial value for registers
x = 0; 

// Block 4: next state computation 
while (true) {
   // this statement signals that everything inside it is done
   // at the same clock cycle
   @dotogether {
      x = (x == 0)? 1 :
          (x == 1)? 2 :
          (x == 2)? 3 :
          0;
   }
}
\end{Verbatim}
\end{frame}
