%% The ETCircuit component language
\begin{figure}[tb]
\centering
\begin{Verbatim}[fontsize=\relsize{-2.5}, numbersep=4pt,numbers=left]
component: decl wiredef init `while(true)' `{' next `}'
type: bool | int | bool `['NUM`]' | int `['NUM`]'
declaration: wire type ID `;' | type ID `;'

decl: declaration+
wiredef: (target = expr `;')*

init: `@do_together' `{' (target = expr `;')* `}'
next: `@do_together' `{' (target = expr `;')* `}'
target: ID | ID `['expr`]'
expr: expr? expr : expr
\end{Verbatim}
\caption{\thislanguage component language grammar}
\label{fig:etcircuit}
\end{figure}

The grammar in Figure~\ref{fig:etcircuit} describes \thislanguage, a high level
imperative language that describes a sequential circuit.
An \thislanguage program starts with a list of declarations of wire, register,
and array variables. Wires are defined in a list of assignment statements
in the \cci{wiredef} block.
Each wire can be the target at most one assignment statement. If a wire is not 
assigned, then it is left as a free input to the circuit. 

The \cci{init} list of statements assigns initial values for the register variables.
All assignment statements within the \cci{init} block execute simultaneously as
indicated with the \cci{do\_together} keyword. 
Similary, the \cci{next} list of statements updates the values of the register variables. 

Each assignment statement has a left hand side target term which is either a variable or 
an access operator to an array element. The right hand side of an assignment is a combinational
expression that is either a term (from Definition~\ref{def:term}) or a ternary choice
expression. The ternary choice (\cci{a?b:c}) returns \cci{b} if \cci{a} is \cci{true}
and \cci{c} otherwise. 