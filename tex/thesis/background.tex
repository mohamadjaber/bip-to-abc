%% Background chapter
In this chapter, we present formalisms used throughout this thesis. 
A reader well-formed in logic verification and sequential circuits may
wish to skip this section, using it only as a reference. 

Let $V=\set{v_1,v_2,\ldots,v_m}$ be a set of scalar 
variables and $A=\set{a_1,a_2,\ldots,a_n}$ be a set of 
array variables. 
\begin{definition}[terms] \label{def:term}
  \rm A {\em term} is either
  a variable $v\in V$,
 a constant $c\in \mathbb{Z}$, or 
 an indexed array variable of the form $a[t]$ 
 which denotes the $t^{th}$ element of $a$ where $a\in A$ and 
 $t$ is a term.
 Arithmetic expressions of the form 
 $-t, t_1+t_2, t_1-t_2, t_1*t_2, t_1/t_2, t_1\%t_2$ 
 are all terms where $t,t_1,t_2$ are terms and 
 $-,+,*,/,$ and $\%$ denote the substraction, 
 addition, multiplication, division and remainder operations,
 respectively. 
\end{definition}

\begin{definition}[Boolean term]
  \rm A constant from the set $\mathbb{B}=\set{\true, \false}$
  is a {\em Boolean term}. 
  The expressions 
  $t_1<t_2, t_1\le t2, t_1> t_2, t_1\ge t_2, t_1==t_2$ are
  all Boolean terms where $t_1,t_2$ are terms and $<,\le,>,\ge,$ and $==$ 
  denote smaller, less than or equal, bigger than, 
  bigger than or equal, and equal,
  respectively. 
  The expressions $b_1\&\& b_2, b_1 \|\| b_2, !b_1, ->, ==$ are 
  all Boolean terms where $b_1,b_2$ are Boolean terms and 
  $\&\&, \|\|, !, -> ,$ and $==$ denote logical conjunction, disjunction,
  negation, implication, and equivalence, respectively.
\end{definition}

\begin{definition}[First order logic formula]
  \rm A Boolean term is a {\em first order formula}. 
  A quantified formula of the form $Q q. b(q)$, where
  $Q\in \set {\forall, \exists}$ is either a universal
  or existential quantifier, $q$ is a 
  quantified variable, and $b(q)$ is a first order
  formula with $q$ as a free variable. 
\end{definition}

\section{Sequential circuits}
\label{s:back:crct_semantics}
The ABC solver operates on an sequential circuit 
representation of a program.

\begin{definition}[Sequential circuit]
\rm A {\em sequential circuit} is a tuple $\big( (V, E),G,
O\big)$.  The pair $(V,E)$ represents a directed graph on
vertices $V$ and edges $E \subseteq V\times V$ where $E$
is a totally ordered relation.  The function $G: V \mapsto
{\mathit types}$ maps vertices to ${\mathit types}$.
There are three disjoint types: {\em primary inputs}, {\em
bit-registers} (which we often simply refer to as {\em
registers}), and logical {\em gates}.  Registers have designated
{\em initial values}, as well as {\em next-state
functions}.  Gates describe logical functions such as
the conjunction or disjunction of other vertices. 
A subset $O$ of $V$ is specified as the {\em
primary outputs} of $V$.  
We will denote the set of primary input variables by $I$,
and the set of bit-register variables by $R$.  
\label{def:back:seq_circuit}
\end{definition}

\begin{definition}[Fanins]
\rm We define the direct {\em fanin}s of a gate $u$ to be
$\{v: (v,u)\in E\}$ the set of source vertices connected
to $u$ in $E$.  We call the {\em support} of $u$ $\{v:
(v\in I \vee v \in R) \wedge (v,u) \in \ast E\}$ all
source vertices in $R$ or $I$ that are connected to $u$
with $\ast E$, the transitive closure of $E$.
\label{def:back:fanins} 
\end{definition}

%The ABC solver restricts gates to have 2~fanins, and
%computes the NAND function; since NAND is functionally
%complete, this is not a limitation.  
%For the sequential
%circuit to be syntactically well-formed, vertices in $I$
%should have no fanins, vertices in $R$ should have
%2~fanins (the next-state function and the initial-value
%function of that register), gates should have two fanins,
%and every cycle in the sequential circuit should contain
%at least one vertex from $R$.  The initial-value functions
%of $R$ shall have no registers in their support.  
%All sequential circuits we consider will be well-formed. 

The ABC solver reasons about {\em And-Inverter-Graphs (AIG)}
which are acyclic sequential circuits with only AND gates and inverters.
All AND gates are restricted to have 2 fanins. 
Since AND gates and inverters are functionally complete, 
this is not a limitation. 
For the sequential
circuit to be syntactically well-formed, vertices in $I$
should have no fanins, vertices in $R$ should have
2~fanins (the next-state function and the initial-value
function of that register) and gates should have two fanins.
%and every cycle in the sequential circuit should contain
%at least one vertex from $R$.  
The initial-value functions
of $R$ shall have no registers in their support.  
All sequential circuits we consider will be well-formed. 

\begin{definition}[State]
\rm A {\em state} is a Boolean valuation to vertices in $R$. 
%A {\em concrete input} is a Boolean valuation to vertices 
%in $I$.
\end{definition}

\begin{definition}[Trace]
\rm A {\em trace} is a mapping $t: V \times \mathbb{N} \mapsto
\mathbb{B}$ that assigns a valuation to all vertices in
$V$ across time {\em steps} denoted as indexes from
$\mathbb{N}$.  The mapping must be consistent with $E$ and
$G$ as follows.  Term $u_{j}$ denotes the source vertex of
the $j$-th incoming edge to $v$, implying that
$(u_{j},v)\in E$.  The value of gate $v$ at time $i$ in
trace $t$ is denoted by $t(v,i)$.
\[
t(v,i)=
   \begin{cases}
      s^i_{v}            &:v \in I \ \text{with sampled value $s_{v}^i$}\\
      t(u_2, i-1)        &:v \in R,i>0,u_2:=\ \text{next-state of $v$}\\
      t(u_1, 0)       &:v \in R,i=0,u_1:=\ \text{initial-state of $v$}\\
      G_v\big(t(u_{1},i),...,t(u_{n},i)\big) &: v \ \text{is a combinational gate with function 
$G_v$}
   \end{cases} \newline
\]
\end{definition}

The semantics of a sequential circuit are defined with
respect to semantical traces.  Given an input valuation
sequence and an initial state, the resulting trace is a
sequence of Boolean valuations to all vertices in $V$
which is consistent with the Boolean functions at the
gates.  We will refer to the transition from one valuation
to the next as a {\em step}.  A node in the circuit is
justifiable if there is an input sequence which when
applied to an initial state will result in that node
taking value $\mbox{true}$.  A node in the circuit is
valid if its negation is not justifiable.  We will refer
to targets and invariants in the circuit; these are simply
vertices in the circuit whose justifiability and validity
is of interest respectively.

\section{The \thislanguage component language}
\label{s:back:etc}
%% The ETCircuit component language
\begin{figure}[tb]
\centering
\begin{Verbatim}[fontsize=\relsize{-2.5}, numbersep=4pt,numbers=left]
component: decl wiredef init `while(true)' `{' next `}'
type: bool | int | bool `['NUM`]' | int `['NUM`]'
declaration: wire type ID `;' | type ID `;'

decl: declaration+
wiredef: (target = expr `;')*

init: `@do_together' `{' (target = expr `;')* `}'
next: `@do_together' `{' (target = expr `;')* `}'
target: ID | ID `['expr`]'
expr: expr? expr : expr
\end{Verbatim}
\caption{\thislanguage component language grammar}
\label{fig:etcircuit}
\end{figure}

The grammar in Figure~\ref{fig:etcircuit} describes \thislanguage, a high level
imperative language that describes a sequential circuit.
An \thislanguage program starts with a list of declarations of wire, register,
and array variables. Wires are defined in a list of assignment statements
in the \cci{wiredef} block.
Each wire can be the target at most one assignment statement. If a wire is not 
assigned, then it is left as a free input to the circuit. 

The \cci{init} list of statements assigns initial values for the register variables.
All assignment statements within the \cci{init} block execute simultaneously as
indicated with the \cci{do\_together} keyword. 
Similary, the \cci{next} list of statements updates the values of the register variables. 

Each assignment statement has a left hand side target term which is either a variable or 
an access operator to an array element. The right hand side of an assignment is a combinational
expression that is either a term (from Definition~\ref{def:term}) or a ternary choice
expression. The ternary choice (\cci{a?b:c}) returns \cci{b} if \cci{a} is \cci{true}
and \cci{c} otherwise. 

%\section{Sequential circuits as C++ classes}
%\label{s:back:circuit_as_cpp}
%A sequential circuit can be seen as a C++ class with Boolean variables representing
%its registers, an initialization function, a next state function and an output
%function as shown in Figure~\ref{fig:seq_as_cpp}. 
%%
%The initialization function computes the set of initial values for all 
%the registers in the circuit. The next state function is responsible for computing 
%the next state functions and updating the values of all register variables in the 
%circuit. The output function finally returns the result of the computation of the 
%circuit given the values of the registers and the inputs.
%\cci{done} is an internal variable signaling that the circuit has 
%successfully executed. It is set by the \cci{nextStateComputation} function.
%
%\begin{figure}[bt]
%\begin{Verbatim}[fontsize=\relsize{-2.5}, numbersep=4pt,numbers=left]
%class SequentialCircuit {
%	bool registers [];
%	bool done;
%	
%	void intialiaze (bool inputs[]);
%	void nextStateComputation (bool inputs[]);
%	bool outputFunction (bool inputs[]);
%	
%	bool simulateCircuit (bool inputs[]) 
%	{
%		initialize(inputs);
%		while (!done) {
%			nextStateComputation (inputs);		
%		}
%		return outputFunction(inputs);
%	}	
%}
%\end{Verbatim}
%\caption{Sequential circuit as a C++ class}
%\label{fig:seq_as_cpp}
%\end{figure}


\section{ABC sequential solver}
\label{s:back:abc}
\section{ABC reduction and verification techniques}
\label{app:abc}

The ABC framework provides a set of algorithms that can 
be applied iteratively to (1) reduce the AIG into 
an equivalent AIG and (2) verify that a designated 
output of an AIG is always true. 
In what follows we provide brief descriptions of several 
reduction and verification ABC algorithms. 

% -- Synthesis techniques -- %
\subsection{Structural register sweep (SRS)}
SRS detects registers that are stuck-at-constant and eliminates 
them from a given sequential AIG circuit. The technique starts by zeroing up all 
initial values of registers in the circuit. It then uses the ternary simulation
algorithm in order to detect stuck-at-constant registers. The algorithm starts from 
the initial values of the registers and simulates the circuit using x values for the
circuit's primary inputs. The simulation algorithm stops when a new ternary state is 
equal to a previously computed ternary state. In this case, any register having the 
same constant value at each reachable ternary state will be declared to be 
stuck-at-constant and thus eliminated. The structural sweeping algorithm stop when 
no further reduction in the number of registers is possible~\cite{mishchenko2008scalable}. 

\subsection{Signal correspondence (Scorr)} 
Scorr uses $k$-step induction in order to detect and merge sets of classes of 
sequentially-equivalent nodes~\cite{mishchenko2008scalable}. The base case for this algorithm is that the equivalence
between the classes holds for the first $k$ frames, and the inductive case is that 
given the base case, starting from any state, the equivalence holds in the 
$(k+1)^{st}$ state. Key to the signal correspondence algorithm is the way the candidate
equivalences are assumed for the base case. Abc implements speculative reduction, 
originally presented in~\cite{mony2005exploiting}, which merges, but does not remove, any node of an equivalence 
class onto its representative, in each of the first $k$ time frames. Instead of removing the 
merged node, a constraint is added to assert that the node and its representative are equal. 
This technique is claimed to decrease the number of constraints added to the SAT solved for 
induction. 

\subsection{Rewriting}
Rewriting aims at finding nodes in a Directed Acyclic Graph (DAG) where by replacing subgraphs rooted 
at these nodes by pre-computed subgraphs can introduce important reductions in the DAG size, while 
keeping the functionality of these nodes intact. The algorithm traverses the DAG in depth-first post-order
and gives a score for each root node. The score represents the number of nodes that would result
from performing a rewrite at this node. If a rewrite exists such that the size of the DAG is decreased, such 
a rewrite is performed and scores are recomputed accordingly.  
Rewriting has been proposed initially in~\cite{bjesse2004dag}, targeted for Reduced Boolean Circuits (RBC); 
it was later implemented and improved for ABC in~\cite{mishchenko2006dag}. 

%% Retiming
\subsection{Retiming}
Retiming a sequential circuit is a standard technique used in sequential synthesis, 
aiming at the relocation of the registers in the circuit in order to optimize 
some of the circuit characteristics. Retiming can either targets the minimization of the delay 
in the circuit, or the minimization of the number of registers given a delay constraint, 
or the unconstrained minimization of the number of registers in the circuit. It 
does so while keeping the output functionality of the circuit intact~\cite{hurst2007fast}

% -- Verification techniques -- %
\subsection{Property directed reachability (Pdr)}
The Pdr algorithm aims at proving that no 
violating state is reachable from the initial state of a given AIG network. 
It maintains a trace representing a list of over-approximations of the states
reachable from the initial state, along with a set of {\em proof-obligations}, 
which can be a set of bad states or a set of states from which a bad state is
reachable. Given the trace and the set of obligations, the Pdr algorithm manipulates 
them and keeps on adding facts to the trace until either an inductive invariant 
is reached and the property is proved, or a counter example is found (a bad state
is proven to be reachable). The algorithm was originally developed by Aaron Bradley 
in~\cite{bradley2011sat,bradley2007checking} and was later improved by Een et. al in~\cite{een2011efficient}.

\subsection{Temporal induction}
Temporal induction carries an inductive proof of the property 
over the time steps of a sequential circuit.
Similar to a standard inductive proof, it consists of a base
case and an inductive hypothesis. These steps are typically 
expressed as SAT problems to be solved by traditional SAT solvers.  
$k$-step induction strengthens simple temporal inductive proofs 
by assuming that the property holds for the first $k$ time steps (states), 
i.e. a longer base case needs to be proven~\cite{een2003temporal}. Since the target is
to prove unsatisfiability (proving that the negation of the property 
is unsatisfiable), if the base case is satisfiable, a counter-example 
is returned. Otherwise, the induction step is checked by assuming that
the property holds for all the states except the last one (the $(k+1)$'th 
state)~\cite{biere2009handbook}.   

\subsection{Interpolation}
Given an unsatisfiable formula $A \land B$, an interpolant $I$ is
a formula such that $A \implies I$, $I \land B$ is unsatisfiable and
$I$ contains only common variables to $A$ and $B$. 
Given a system $M$, a property $p$ and a bound $k$, interpolation
based verification starts by attempting bounded model-checking (BMC) with the bound $k$. 
If a counter-example is found, the algorithm returns. Otherwise, it
partitions the problem into a prefix $pre$ and a suffix $suf$, such that the 
problem is the conjunction of the two. 
Then the interpolant $I$ of $\mathit{pre}$ and $\mathit{suf}$ is computed, it represents
an over-approximation of the set of states reachable in one step from the initial state
of the algorithm. If $I$ contains no new states, a fixpoint is reached 
and the property is proved. Otherwise, the algorithm reiterates and replaces
the initial states with new states added by $I$~\cite{amla2005analysis}. 




