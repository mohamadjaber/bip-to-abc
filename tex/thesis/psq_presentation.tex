\subsection*{Methodology}
%% presentation for the psq tool
\begin{frame}{\mytool{}}
 \begin{itemize}
  \item \mytool{} accepts input programs and specifications written in an imperative language
  \psqlanguage
  \item Given a program \Pm and a precondition postcondition pair \pair{\Pre}{\Post},  \mytool{}:
  \begin{enumerate}
   \item resolves function calls and quantifiers
   \item translates $\Pm \models \pair{\Pre}{\Post}|_b $ into an equivalent program \Pm' in \thislanguage
   \item synthesizes \aigcircuit an equisatisfiable AIG for \Pm'
   \item calls ABC and performs synthesis and verification on \aigcircuit
  \end{enumerate}
 \end{itemize}
\end{frame}

\begin{frame}[containsverbatim]
\frametitle{\psqlanguage}
\begin{itemize}
 \item A program written in \psqlanguage~is defined as follows
\end{itemize}
\begin{Verbatim}[fontsize=\relsize{-2.0}]
program := (statement | declaration)+
declaration := variable_declaration | property_declaration 
              | function_declaration
statement := assignment_statement | selection_statement 
              | loop_statement | sync_statement

property_declaration := '@pre' precondition | '@post' postcondition

sync_statement := '@dotogether' { statement+ }

... 
\end{Verbatim}
\end{frame}

\begin{frame}[containsverbatim]
\frametitle{Array search}
  \small {
  \begin{tabular}{p{2.2in}|p{2.4in}}
\begin{Verbatim}[fontsize=\relsize{-3.5}, 
frame=topline, framesep=4mm, label=\fbox{\Pm in \psqlanguage}]
int ArraySearch(int [] a, int d, 
                   int s, int e, int n) {
@pre as { 0 <= s && s <= e && 
              e < n && n <= MAXSIZE; }
 int i = s;          // pc = pc+1
 while(i <= e) {     // pc = (i<=e)? 5:10;
   if (a[i] == d) {  // pc = (a[i]==d)? 6:8;
     break;}         // pc = 10;
   else {
     i = i+1;}       // pc = 4;
 }
 return i;           // pc = 11; rv = i;
@post as {
  ((rv >= s && rv <= e) -> a[rv] == d) xor 
  (rv == -1 -> forall(int i:[s .. e]) 
                           { a[i] != d })
} }
\end{Verbatim}
& 
%\begin{Verbatim}[fontsize=\relsize{-2}], numbersep=4pt,numbers=left]
\begin{Verbatim} [fontsize=\relsize{-3.5},
frame=topline, framesep=4mm, label=\fbox{\Pm' in \thislanguage}]
dotogether {
  preas = 0 <= s && s <= e 
            && e < n && n <= MAXSIZE;
  //initial values
  pc = 0; notdone = true; postas = true; }
// next state functions
while (true) { dotogether {  
 i = (pc == 3) ? s : (pc == 8) ? i+1 : i;
 notdone = (pc == 11) ? false : true;
 rv = (pc == 10) ? i : rv;
 pc = (pc == 0) ? 3: (pc == 3) ? 4 : 
   (pc == 4) ? ( (i <= e) ? 5 : 10 ) :
   (pc == 5) ? (a[i] == d) ? 6 : 8 ) :
   (pc == 6) ? 10 :
   (pc == 8) ? 4  : 
   (pc == 10) ? 11 : pc; }
 postas = ((rv>=s && rv<=e) -> a[rv]==d) xor
   (rv == -1 -> forall(int i:[s .. e]) 
                          { a[i]!=d }); }
\end{Verbatim}
\end{tabular}
}
\end{frame}

\begin{frame}[containsverbatim]
\frametitle{Code transformation}
\begin{itemize}
 \item \mytool{} preprocesses the program with $\rho$ 
 \item $\rho$ resolves function calls
 \begin{itemize} \item assignment statement marshalling arguments, return values, and save program counter 
   \end{itemize}
 \item $\rho$ resolves recursion
 \begin{itemize} \item increases dimensions of local variables, and simulate stack 
   \end{itemize}
 \item $\rho$ resolves quantifiers
 \begin{itemize} \item unrolls quantifiers with finite constant bounds
   \end{itemize}
 \item $\rho$ injects new variables and statements  as needed
 %that simplify the transformation
 %from \psqlanguage{} to \thislanguage
\end{itemize}
\end{frame}

\begin{frame}[containsverbatim]
\frametitle{Handling quantifiers and properties}
%\small {
  \begin{tabular}{p{2.2in}|p{2.4in}}
\begin{Verbatim}[fontsize=\relsize{-3.5}, 
frame=topline, framesep=4mm, label=\fbox{\Pm}]
int a [N], b [N], i;

@pre inv {
  // Q1:
  forall(int i:[0..N-2]) {a[i] <= a[i+1]}; }

i = 0;
while (i < N) {
 @dotogether {
  b[i] = a[(N-1)-i];
  i = i + 1; } }
  
@post inv {
  // Q2:
  forall(int i:[0..N-2]) {b[i] >= b[i+1]} }

\end{Verbatim}
& 
%\begin{Verbatim}[fontsize=\relsize{-2}], numbersep=4pt,numbers=left]
\begin{Verbatim} [fontsize=\relsize{-3.5},
frame=topline, framesep=4mm, label=\fbox{$\rho(\Pm)$}]
int a[N], b[N], i, Q1::i, Q2::i;
bool Q1::rv, Q2::rv, preinv, postinv;

@dotogether {
 Q1::i = 0; Q2::i = 0;
 Q1::rv = true; Q2::rev = true;
}
while (Q1::i <= N-2) {
 @dotogether { 
  Q1::rv = Q1::rv && (a[i] <= a[i+1])
  Q1::i = Q1::i + 1; }}
preinv = Q1::rv;

i = 0;
while (i < N) {
 @dotogether {
  b[i] = a[(N-1)-i];
  i = i + 1; } }
  
while (Q2::i <= N-2) {
 @dotogether { 
  Q2::rv = Q2::rv && (b[i] >= b[i+1])
  Q2::i = Q2::i + 1; }}
postinv = Q2::rv;
assert (preinv -> postinv);

\end{Verbatim}
\end{tabular}
%}
\end{frame}

\begin{frame}{Handling recursion}
\begin{itemize}
 \item $\rho$ simulates recursion using arrays
 \item It increases the dimension of the recursive function's local variables
 \item It injects a stack pointer variable \CodeIn{sp} to address the variables
 \item Stack pointer is incremented before recursive call and decremented 
 after it returns
\end{itemize}
\end{frame}

\begin{frame}[containsverbatim]
\frametitle{Function calls and recursion}
  \begin{tabular}{p{2.2in}|p{2.4in}}
\begin{Verbatim}[fontsize=\relsize{-3.5}, 
frame=topline, framesep=4mm, label=\fbox{\Pm}]
int mult (x, y) {
  int result;
  if (y == 1) {
    result x;
  } else {
    result = mult(x, y-1);
  }
  return result;
}

int a, b;

a = 2; b = 3;
a = mult(a, b);
\end{Verbatim}
& 
%\begin{Verbatim}[fontsize=\relsize{-2}], numbersep=4pt,numbers=left]
\begin{Verbatim} [fontsize=\relsize{-3.5},
frame=topline, framesep=4mm, label=\fbox{$\rho(\Pm)$}]
int sp;
sp = 0;

int mult(int x [], int y []) {
  int result [];
  if (y[sp] == 1) {
    result[sp] = x[sp]; 
  } else {
    x[sp+1] = x[sp];
    y[sp+1] = y[sp] - 1;
    sp = sp + 1;
    call_func(mult);
    sp = sp - 1;
    result[sp] = mult::rv[sp+1];
  }
  mult::rv[sp] = result[sp];
}

int a, b;
a = 2; b = 3;
x[0] = 2;
y[0] = 3;
call_func(mult);
a = mult::rv[0];
\end{Verbatim}
\end{tabular}
\end{frame}

\begin{frame}{Transformation to \thislanguage}
 \begin{itemize}
  \item \mytool{} uses {\em program counter} semantics to 
  synthesize AIGs from \psqlanguage{} programs
  \item Each statement $s$ is assigned a unique label ($label(s)$)
  \item Assignment statements alter the data flow of the program 
  \item Selection, loop and function call statements alter the control
  flow (program counter) of the program
  \item \mytool{} adds a program counter variable (\cci{pc}) to the code
  and uses it for control flow
 \end{itemize}
\end{frame}

\begin{frame}{Transformation algorithm}
\begin{itemize}
 \item To generate \thislanguage{} program \Pm' given a \psqlanguage{} program \Pm
 \begin{enumerate}
  \item The wire and variable declaration block in \Pm' is directly mapped from the 
  declarations in \Pm
  \item The wire definitions block in \Pm' is directly mapped from the definitions 
  in \Pm
  \item The variable initialization block in \Pm' is automatically generated
  for variables used before assignment (i.e. free variables)
 \end{enumerate}
\end{itemize}
\end{frame}

\begin{frame}[containsverbatim]
\frametitle{Data Transformation algorithm}
\begin{lstlisting}
foreach target x // variable (v) or array access (a[exp])
  $next\_list$ += $``x :="$ 
  foreach assignment statement $s = x := exp$
   $next\_list$ += $``(pc == label(s))? exp :"$ 
  endfor
  $next\_list$ += $``x;"$
endfor
\end{lstlisting}
\end{frame}

\begin{frame}[containsverbatim]
\frametitle{Control Transformation}
\begin{lstlisting}
$n_{pc} = ``pc =" $
foreach statement $s$
  if (is_conditional_statement ($s$))
    $n_{pc}$ += $``(pc==label(s))?~ (condition(s)?~then(s) : else(s)) :"$
  elsif (is_loop_statement ($s$))
    $n_{pc}$ += $``(pc==label(s))?~ (condition(s)?~body(s):next(s)) :"$ 
    $n_{pc}$ += $``(pc==last\_statement\_body(s))?~label(s) :"$
  elsif (is_function_call($s$))
    $f_d :=$ function_declaration_of $($s$)$
    $n_{pc}$ += $``(pc==label($s$))?~ (body(f_d)) :"$
  elsif (is_return_statement ($s$))
    $f_d :=$ function_declaration_of $($s$)$
    $n_{pc}$ += $``(pc==label(s))?~(return(call(f_d))) :"$
  else 
    $n_{pc}$ += $``(pc==label(s))?~(next(s)) :"$
  endif
endfor
\end{lstlisting}
\end{frame}

\begin{frame}[containsverbatim]
\frametitle{Post processing}
\begin{itemize}
 \item Post processing step to eliminate non-constant array access terms
\end{itemize}
  \begin{tabular}{p{1.2in}|p{3.4in}}
\begin{Verbatim}[fontsize=\relsize{-3.5}, 
frame=topline, framesep=4mm, label=\fbox{\Pm}]
 ....
 l1: a[i] = x + 1;
 ...
 
 l2: x = a[i] + 2;
\end{Verbatim}
& 
%\begin{Verbatim}[fontsize=\relsize{-2}], numbersep=4pt,numbers=left]
\begin{Verbatim} [fontsize=\relsize{-3.5},
frame=topline, framesep=4mm, label=\fbox{\Pm'}]
a[0] = ((pc == l1) && (i == 0))? (x+1) : a[0];
a[1] = ((pc == l1) && (i == 1))? (x+1) : a[1];
a[2] = ((pc == l1) && (i == 2))? (x+1) : a[2];
...
a[N-1] = ((pc == l1) && (i == N-1))? (x+1) : a[N-1];

x = (pc == l2)? (i == 0)? a[0] :
                (i == 1)? a[1] :
                (i == 2)? a[2] :
                ...
                (i == N-2)? a[N-2] : a[N-1];
\end{Verbatim}
\end{tabular}
\end{frame}

\begin{frame}{Finally }
\begin{itemize}
 \item Translation of variables and operations into bit-vector registers and operations
   \begin{itemize}
     \item \cci{initial value definitions} map to register initial value gates
     \item \cci{next state function definitions} map to register next state gates
\end{itemize}
\end{itemize}
\end{frame}

\subsection{Results}
\begin{frame}{Comparison with CBMC}
\begin{itemize}
 \item $\Pm_1$: linear search algorithm, $\Pm_2$: binary search algorithm
\end{itemize}
\begin{table}
\centering
\footnotesize{
\begin{tabular}{|c|c|c|c|c|c|c|c|c|}
\cline{3-9}
\multicolumn{1}{l}{} & \multicolumn{1}{c}{} &  \multicolumn{2}{|c|}{CBMC formula size} & \multicolumn{3}{l|}{\mytool{} AIG size} & \multicolumn{2}{c|}{Time (s)} \\ \hline
\Pm & size  & Vars  & Clauses & lat & and & lev & \mytool & CBMC \\ \hline
$\Pm_1$ & 3 & 2416 & 6784 & 41 & 313 & 15 & 4.36 & 0.016 \\ \hline
$\Pm_1$ & 7 & 4612 & 15008 & 68 & 568 & 19 & 12.4 & 722.4 \\ \hline
$\Pm_1$ & 15 & 9112 & 34496 & 119 & 1116 & 21 & 16.87 & {\color{red} Time-out} \\ \hline
$\Pm_1$ & 31 & 18332 & 84928 & 226 & 2346 & 24 & 33.67 & {\color{red} Time-out} \\ \hline
$\Pm_1$ & 63 & 37216 & 230208 & 461 & 5100 & 26 & 99.64 & {\color{red} Time-out} \\ \hline
$\Pm_1$ & 127 & 75876 & 695616 & 984 & 11315 & 28 & 396.98 & {\color{red} Time-out} \\ \hline
\hline
$\Pm_2$ & 3 & 6503 & 24533 & 56 & 55 & 19 & 1.04 & 0.085 \\ \hline
$\Pm_2$ & 7 & 16172 & 68130 & 83 & 850 & 17 & 1.47 & 1.91 \\ \hline
$\Pm_2$ & 15 & 42461 & 197223 & 143 & 1943 & 20 & 27.69 & 38.493 \\ \hline
$\Pm_2$ & 63 & 390623 & 2133649 & 529 & 9052 & 25 & 1152.22 & {\color{red} Time-out} \\ \hline
\end{tabular}}
\end{table}
\end{frame}

\begin{frame}{Array algorithms}
\begin{table}
\tiny{
\centering
\begin{tabular}{|c|c|c|c|c|c|c|c|c|c|c|}
\cline{3-10}
\multicolumn{1}{c}{} & \multicolumn{1}{c}{} & \multicolumn{3}{|c|}{Before reduction} & \multicolumn{3}{c|}{After reduction} & \multicolumn{2}{c|}{Time (s)} & \multicolumn{1}{c}{}  \\ \hline
\Pm & $b$ & lat & and & lev & lat & and & lev & Ver. & Total & Check \\ \hline
ls & 2 & 86 & 719 & 24 & 41 & 313 & 15 & 0.33 & 4.36 & \checkmark \\ \hline
ls & 3 & 118 & 1064 & 27 & 68 & 568 & 19 & 3.89 & 12.4 & \checkmark \\ \hline
ls & 4 & 174 & 1781 & 30 & 119 & 1116 & 21 & 2.41 & 16.87 & \checkmark \\ \hline
ls & 5 & 286 & 3362 & 45 & 226 & 2346 & 24 & 1.43 & 33.67 & \checkmark \\ \hline
ls & 6 & 526 & 6895 & 78 & 461 & 5100 & 26 & 4.57 & 99.64 & \checkmark \\ \hline
ls & 7 & 1054 & 14780 & 143 & 984 & 11315 & 28 & 21.32 & 396.981 & \checkmark \\ \hline
ls & 8 & 4798 & 70742 & 529 & 4718 & 55364 & 33 & 682.11 & 8022.11 & \checkmark \\ \hline
\hline
bs & 3 & 94 & 879 & 30 & 56 & 555 & 19 & 0.11 & 1.04 & \checkmark \\ \hline
bs & 4 & 151 & 1832 & 42 & 83 & 850 & 17 & 0.54 & 1.47 & \checkmark \\ \hline
bs & 5 & 268 & 5185 & 62 & 143 & 1943 & 20 & 25.42 & 27.69 & \checkmark \\ \hline
\hline
bsort & 2 & 114 & 1198 & 29 & 44 & 393 & 16 & 0.29 & 5.79 & \checkmark \\ \hline
bsort & 3 & 169 & 2218 & 35 & 68 & 885 & 20 & 17.1 & 31.09 & \checkmark \\ \hline
bsort & 4 & 276 & 5607 & 47 & 117 & 2106 & 22 & 1390.25 & 1426.98 & N/A \\ \hline
\hline
ss & 2 & 112 & 1208 & 27 & 43 & 427 & 15 & 0.53 & 5.81 & \checkmark \\ \hline
ss & 3 & 167 & 2239 & 35 & 69 & 949 & 19 & 209.37 & 223.54 & \checkmark \\ \hline
ss & 4 & 280 & 5676 & 47 & 125 & 2236 & 22 & 1800 & 1852.76 & N/A \\ \hline
\hline
a-p & 2 & 110 & 1896 & 33 & 57 & 689 & 19 & 0.87 & 2.79 & \checkmark \\ \hline
a-p & 3 & 147 & 2509 & 34 & 87 & 1174 & 24 & 93.47 & 97.56 & \checkmark \\ \hline
a-p & 4 & 208 & 3829 & 38 & 141 & 2419 & 29 & 2127 & 2135.7 & N/A \\ \hline
\end{tabular}
}
\end{table}
\end{frame}

\begin{frame}{List algorithms}
\begin{table}
\footnotesize{
\centering
\begin{tabular}{|c|c|c|c|c|c|c|c|c|c|c|}
\cline{3-10}
\multicolumn{1}{c}{} & \multicolumn{1}{c}{} & \multicolumn{3}{|c|}{Before reduction} & \multicolumn{3}{c|}{After reduction} & \multicolumn{2}{c|}{Time (s)} & \multicolumn{1}{c}{}  \\ \hline
\Pm & $b$ & lat & and & lev & lat & and & lev & Ver. & Total & Check \\ \hline
lli & 2 & 237 & 4310 & 38 & 98 & 871 & 19 & 109.63 & 118.89 & \checkmark \\ \hline
lli & 3 & 344 & 6117 & 41 & 179 & 1693 & 26 & 1800 & 1811 & N/A \\ \hline
\hline
llr & 2 & 197 & 2906 & 33 & 84 & 722 & 21 & 47.72 & 71.3829 & \checkmark \\ \hline
llr & 3 & 293 & 4454 & 39 & 157 & 1387 & 25 & 1800.15 & 1830.21 & N/A \\ \hline
\end{tabular}
}
\end{table}
\begin{itemize}
 \item On average $43\%$ reduction in number of latches, $53\%$ in number of AND gates,
 $47\%$ reduction in number of logic levels
\end{itemize}
\end{frame}

\begin{frame}{SVCompetition'13 (ControlFlowInteger)}
\begin{itemize}
 \item Ranked closely behind silver tool
\end{itemize}
\begin{table}
 \centering
 \footnotesize{
\begin{tabular}{|c|c|c|c|c|c|}
\hline
\Pm & status & time (s) & gold  & silver & bronze \\ \hline
locks\_5\_safe & safe & 0.28 & 0.4 & 1.2 & 1.3 \\ \hline
locks\_6\_safe & safe & 0.32 & 0.43 & 1.3 & 1.6 \\ \hline
locks\_7\_safe & safe & 0.52 & 0.4 & 1.3 & 1.9 \\ \hline
locks\_8\_safe & safe & 0.62 & 0.4 & 1.3 & 2.3 \\ \hline
locks\_9\_safe & safe & 1.05 & 0.39 & 1.3 & 3.5 \\ \hline
locks\_10\_safe & safe & 0.88 & 0.4 & 1.3 & 6.4 \\ \hline
locks\_11\_safe & safe & 3.42 & 0.39 & 1.4 & 24 \\ \hline
locks\_12\_safe & safe & 4.21 & 0.4 & 1.4 & 110 \\ \hline
locks\_13\_safe & safe & 4.18 & 0.43 & 1.4 & 100 \\ \hline
locks\_14\_safe & safe & 5.9 & 0.4 & 1.4 & 100 \\ \hline
locks\_15\_safe & safe & 6.83 & 0.4 & 1.4 & 100 \\ \hline
locks\_14\_unsafe & unsafe & 0.99 & 3.2 & 1.6 & 1.8 \\ \hline
locks\_15\_unsafe & unsafe & 0.91 & 4.2 & 1.6 & 1.8 \\ \hline
\hline
Average & N/A & 2.316 & 0.911 & 1.377 & 34.969 \\ \hline
\end{tabular}}
\end{table}
\end{frame}

\begin{frame}{SVCompetition'13 (Loops)}
\begin{itemize}
 \item Solved problems correctly that silver and bronze got wrong
 \item Ranked closely behind gold tool
\end{itemize}
\begin{table}
 \centering
 \footnotesize{
\begin{tabular}{|c|c|c|c|c|c|}
\hline
\Pm & status & time (s) & gold  & silver & bronze \\ \hline
array\_safe & safe & 0.116 & 0.05 & 0.15 & 0.48 \\ \hline
array\_unsafe & unsafe & 0.091 & 0.09 & 0.47 & 0.46 \\ \hline
count\_up\_down\_s & safe & 0.137 & 0.28 & {\bf \color{red}{450}} & 0.39 \\ \hline
count\_up\_down\_u & unsafe & 0.095 & 0.04 & 0.15 & 0.43 \\ \hline
invert\_string\_s & safe & 1.808 & 0.03 & 14 & {\bf \color{red}{0.56}} \\ \hline
invert\_string\_u & unsafe & 0.266 & 0.11 & 10 & 0.68 \\ \hline
\hline
Average & N/A & 0.419 & 0.100 & 79.128 & 0.500 \\ \hline
\end{tabular}}
\end{table}
\end{frame}
