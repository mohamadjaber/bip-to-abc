%% related work 
%% file : Literature.tex
%% Literature review for LTL synthesis

%\chapter{Related Work} \label{chap:related}

%% intro for software verification
%Ever since the success of formal verification methods in the hardware domain, researchers in the 
%software area have been trying to port the success of such techniques from hardware to software.
\section{Verification of software programs}
The need to design correct software and hardware systems has pushed researchers to design verification 
techniques and tools targeted towards software, hardware and embedded systems.
SPIN is a model checking tool targeting the verification of process interactions. It presents 
a front-end with a high level specification language called Promela, aimed at allowing users
to provide descriptions of concurrent systems or distributed algorithms~\cite{holzmann1997model}. 
SPIN accepts {\em Linear Temporal Properties} (LTL) specifications 
as input, and translates them into B\"uchi automaton using simple on-the-fly construction. 
In fact, SPIN considers the input specifications as impossible conditions, i.e. conditions that should 
never be met for correct behavior of a given system. Therefore it aims at checking whether the
language of the design and that of the specification do not intersect. If such an intersection 
exists, SPIN returns a counter example. Otherwise the design is considered to be correct. 
\mytool{} differs from SPIN in that it takes a imperative program \Pm and 
FOL precondition-postcondition pair \pair{\Pre}{\Post}, 
and translates them into an equisatisfiable AIG. On the resulting 
AIG, several reduction and proof algorithms can be applied to perform model checking, most
of which have no counterpart in SPIN.

%% Alloy and CBMC
Alloy~\cite{jackson2002alloy} and CBMC~\cite{clarke2004tool} are  tools 
that perform bounded model checking. CBMC accepts ANSI-C input code and supports
several standard C libraries. It checks for pointer operations and within bound array access, 
and allows for dynamic memory allocation using \cci{new} and \cci{malloc}. 
CBMC takes an input program and 
an unwinding bound, and generates a CNF formula that describes the behavior of the program. CBMC 
uses {\em Single Static Assignment} (SSA) transformations and loop unwinding in order to generate the appropriate CNF formula.   

Similarly, Alloy takes input structural specifications where entities can either be 
sets, functions or relations, and allows operations on these three elementary types. 
It then transforms such specifications into CNF formulae and uses SAT solvers to generate
satisfying models. Alloy differs from CBMC in that it
is aimed at generating systems that satisfy the given specifications, i.e. find models for the 
structural specifications, while CBMC is aimed at the verification of a given design against
a set of specifications. Both Alloy and CBMC transform the given problems into SAT 
problems and use an off the shelf SAT solver~\cite{goldberg2007berkmin, marques1999grasp, sorensson2005minisat} 
to try and find a model for the generated CNF formula. The model will then be a counter example
for CBMC, and a correct system for Alloy. 

Techniques based on loop unwinding and translation to CNF suffer from the rapid increase 
in the size of the generated CNF formulae with respect to the size of the input programs and
specifications, and the unwinding bounds. Additionally, the generated CNF formula may need to be
regenerated in cases where the unwinding bounds were not large enough. This iterative procedure
is costly and might require intractable resources. 

%% start here for BIP
\section{Verification of embedded systems}
The overlap between software and hardware design in embedded systems creates more challenges 
for the verification process. SystemC~\cite{systemc} is a modeling platform based on C++ that provides
design abstractions at the {\em Register Transfer Level} (RTL), behavior, and system levels. 
It aims at providing a common design environment for embedded system design and hardware-software
co-design. SystemC designers write their systems in C++ using SystemC class libraries that 
provide implementations for hardware specific objects such as concurrent modules and clocks.
Therefore the input systems can be compiled using standard C++ compilers to generate binaries
for simulation. SystemC allows for the communication between different components of a system
through the usage of ports, interfaces and channels.  

The BIP framework differs from SystemC in that it presents a dedicated language and supporting
tool-set that describes the behavior of individual system components as symbolic LTS. 
Communication between components in BIP is ensured through ports and interactions.   
BIP operates at a higher level than SystemC and does not provide support for circuit level 
constructs.

Verification techniques for SystemC and BIP make use of symbolic model checking tools. 
NuSMV2~\cite{nusmv} is a symbolic model checker that employs both 
SAT and BDD based model checking techniques. It processes an input 
describing the logical system design as a finite state machine, and a set of specifications
expressed in LTL, Computational Tree Logic (CTL) and Property Specification Language (PSL).
Given a system $\Pm$ and a set of specifications $P$, NuSMV2 first flattens $\Pm$ and $P$ by 
resolving all module instantiations and creating modules and processes, thus generating one 
synchronous design. It then performs a Boolean encoding step to eliminate all scalar variables, 
arithmetic and set operations and thus encode them as Boolean functions.   

In order to avoid the state space explosion problem, NuSMV2 performs a cone of 
influence reduction~\cite{berezin1998compositional} step in order to eliminate
non-needed parts of the flattened model and specifications. The cone of influence
reduction abstraction technique aims at simplifying the model in hand by only 
referring to variables that are of interest to the verification procedure, i.e. variables
that influence the specifications to check~\cite{clarke1999model}.

DFinder~\cite{dfinder} is an automated verification tool for checking invariants
on systems described in the BIP language. Given a BIP system \Pm and 
an invariant $\mathcal{I}$, DFinder operates  compositionally and iteratively
to compute invariants $\mathcal{X}$ of the interactions and the atomic 
components of \Pm. It then uses the Yices {\em Satisfiability Modulo
Theory} (SMT) solver~\cite{dutertre2006fast} to check for the validity 
of the formula $\mathcal{X} \land \lnot \mathcal{I} = false$. 
Additionally, DFinder checks the deadlock freedom of  \Pm by building an invariant 
$\mathcal{I}_d$ that represents the states of of \Pm in which no interactions 
are enabled, \ie{} a deadlock occurs. It then checks the for the formula
$\mathcal{X} \land \mathcal{I}_d = false$, \ie{} none of the deadlock states
are reachable in \Pm.   

Techniques based on symbolic model checking for the verification of 
BIP designs suffer from the state space explosion problem, and often 
fail to scale with the size and the complexity of the systems. 
On the other hand, DFinder does not handle data transfer between 
atomic components, thus limiting the range of practical applications 
on which it can be applied. Our technique handles data transfers and uses the wide range of synthesis 
and reduction algorithms provided by ABC to effectively reduce the size and 
the complexity of the verification problem. Most of these algorithms have no counterpart
in symbolic model checking.  


%The work in\cite{seraICSE07} takes a declarative 
%formula $\phi$ in first order logic
%(FOL) with transitive closure and a bound on the 
%universe of discourse and
%translates it to a %n equisatisfiable 
%sequential circuit expressed in VHDL. 
%It then passes the sequential circuit to a sequential 
%circuit solver and decides
%the validity of $\phi$ within the bound. 
%It scales to bounds larger than what is possible with 
%Kodkod~\cite{kodkodTJ2007}
%which translates $\phi$ into a propositional Boolean 
%formula in conjunctive
%normal form (CNF) and checks its validity with a 
%Boolean satisfiability solver. 
%
%The work in\cite{sebacASE07} translates an imperative 
%C program, with an assertion
%statement therein, and a bound on the input size, 
%into a %n equisatisfiable
%sequential circuit expressed in VHDL. 
%It then passes the sequential circuit to a 
%sequential circuit solver and decides
%the validity of the assertion within the bound. 
%It scales to bounds larger than what is possible 
%with CBMC\cite{cbmcDAC03} which
%translates the program with a bound on the input 
%size and the number of loop
%iterations into a propositional Boolean formula 
%in conjunctive normal form (CNF)
%and checks for correctness using a Boolean 
%satisfiability solver. 
%
%Our method extends the work in \cite{seraICSE07,sebacASE07} in that
%\be
%\i it supports function calls including recursion, and requires a bound on
%recursion depth only if the recursive function uses local variables, 
%\i it enables a termination guarantee check within a bound on execution
%time, it then uses the execution time bound with bounded model checking to
%decide correctness,
%\i it directly translates the program into bit level representation using {\em
%and inverted graphs} (AIG) instead of the VHDL representation that requires a
%VHDL compiler to be translated into bit level,
%\i it uses ABC~\cite{brayton2010abc}, an open source sequential circuit solver, instead of
%SixthSense~\cite{mony2004scalable} an IBM internal sequential circuit solver, 
%\i and it is an open source tool available online
%~\ref{fn:online}
%\ee
%
%%%%%%%%%%%%%%%%%%%%%%%%%%%%%%%%%%%%%%%%%%%%%%%%%%%%%%%%%%%%%%%%%%%%%%%%%%%%%%%%%%%%%%%%%%%%%%%%%%
%%% NuSVM
%NuSMV2~\cite{cimatti2002nusmv} is a symbolic model checking tool that employs both 
%satisfiability (SAT) and BDD based model checking techniques. It processes an input 
%describing the logical system design as a finite state machine, and a set of specifications
%expressed in LTL, Computational Tree Logic and Property Specification Language.
%Given a model $M$ and a set of specifications $P$, NuSMV2 first flattens $M$ and $P$ by 
%resolving all module instantiations and creating modules and processes, thus generating one 
%synchronous design. It then performs a boolean encoding step to eliminate all scalar variables, 
%arithmetic and set operations and thus encode them as boolean functions.   
%
%In order to avoid the state space explosion problem, NuSMV2 performs a cone of 
%influence reduction~\cite{berezin1998compositional} step in order to eliminate
%non-needed parts of the flattened model and specifications. The cone of influence
%reduction abstraction technique aims at simplifying the model in hand by only 
%referring to variables that are of interest to the verification procedure, i.e. variables
%that influence the specifications to check~\cite{clarke1999model}.
%%%%%%%%%%%%%%%%%%%%%%%%%%%%%%%%%%%%%%%%%%%%%%%%%%%%%%%%%%%%%%%%%%%%%%%%%%%%%%%%%%%%%%%%%%%%%%%%%%%%%
%%%%%%%%%%%%%%%%%%%%%%%%%%%%%%%%%%%%%%%%%%%%%%%%%%%%%%%%%%%%%%%%%%%%%%%%%%%%%%%%%%%%%%%%%%%%%%%%%%%%%

%\mytool{} differs from CBMC and Alloy in that it translates an input imperative program with 
%FOL specifications into a sequential circuit, encoded as an AIG. Such a representation is much 
%more succinct than CNF, and allows for much more opportunity for bit level optimizations and
%abstraction techniques that can reduce the size and the complexity of the problem. Additionally,
%it is worthy to note that network in AIG format can be easily translated in CNF formulas, so
%in the case where CNF and SAT solving outperforms sequential verification, \mytool{} can 
%perform as well as CBMC and Alloy. \mytool{} is also equipped with a debugger that allows
%a step by step inspection of the state of the each element in the code in the generated counter
%example. Such a debugger can help developers locate possible problems in their code, and even
%in their specifications.   

%%% Introduction
%Several techniques have been developed in the literature in order to synthesize LTL formulas, 
%usually describing properties that hold over real-life hardware systems and designs. These synthesis
%techniques have different targets, some aim to generate complete and correct systems based on 
%input specifications, while others are targeted at generating monitors to ensure correct
%functionality of systems through assertion checking.
%
%%% FoCs
%FoCs is an industrial tool developed at IBM research labs, targeted at generating simulation 
%checkers from formal specifications~\cite{p:Focs}. The tool's goal is to reduce, or possibly eliminate the 
%amount of human intervention in writing and maintaining functional checkers. FoCs takes input specification
%expressed in RCTL~\cite{p:rctl}, and generates formal checkers written in VHDL. These checkers
%are then linked with the original VHDL and executed on a set of test programs. The role of the formal checkers 
%is to make sure that the original design never goes into any error state. 
%
%The generation of the formal checkers from the RCTL specifications is done in three steps. First, the RCTL is 
%translated into a NFA according to the algorithm described in~\cite{p:rctl}. This NFA will have a set of error
%states, which represent the states that the design should never go into if it meets the required specifications.  
%In order to be able generate the VHDL checkers, the NFA has to be translated into a DFA, which is in turn translated
%into a VDHL process. This process will then be run alongside the original design to check for any violations of the 
%specifications. 
%
%The key drawback of FoCs' approach is that transformation algorithm generates a DFA that can be exponential 
%in the number of states of the NFA, which takes us back to the state-space explosion problem. The authors claim 
%that such a limitation does not exist in their case, since the simulation is rather sensitive to the number 
%of VHDL lines in the generated checker, which is at most quadratic in the size of the property to check. 
%Our approach differs from FoCs in that it aims at generating a DFA with auxiliary variables that is linear in 
%the size of the property, without generating any NFAs. Therefore, it can help rendering the generated VHDL checker 
%even smaller in terms of the lines of code. 
%
%%% Wring
%Daniele et. al presented an explicit state automaton based approach for model checking formulas expressed in 
%LTL~\cite{p:wring}. Explicit state refers to the fact that the automaton explicitly enumerates all of the states 
%in the system, as opposed to symbolic techniques that tend to group several states into Order Binary Decision 
%Diagrams (OBDD).
%
%In this approach,  both the system and the `negation' of the input specification are translated into automata on 
%infinite words, specifically B\"{u}chi automaton. Then, a synchronous product is applied to both of the 
%automata. If the product is empty, then the system can never go into a state that satisfies the negation 
%of the input property, i.e. the property is always satisfied and the system is correct with respect to the 
%given property. Otherwise, there exists a path that can take the system into a state that violates the 
%input specification (by satisfying its negation), and the system does not adhere to the property.  
%
%The presented algorithm, namely \texttt{`LTL2AUT'}, aims at generating a labeled B\"{u}chi automaton 
%that recognizes all of the paths that model a given LTL formula $\psi$. \texttt{LTL2AUT} in fact 
%shares its core structure with two previous algorithms, GPVW and GPVW+~\cite{p:gpvw}. Each 
%algorithm differs from the two others with the implementation of certain routines present in the 
%shared main core. The authors claim that using \texttt{LTL2AUT}, they were able to generated automata 
%with an improvement in both time (time for generation) and space (number of states) complexities, when
%compared to the other two algorithms. 
%
%%%TODO: include the rest of the literature review and compare against it
