
The work in\cite{seraICSE07} takes a declarative 
formula $\phi$ in first order logic
(FOL) with transitive closure and a bound on the 
universe of discourse and
translates it to a %n equisatisfiable 
sequential circuit expressed in VHDL. 
It then passes the sequential circuit to a sequential 
circuit solver and decides
the validity of $\phi$ within the bound. 
It scales to bounds larger than what is possible with 
Kodkod~\cite{kodkodTJ2007}
which translates $\phi$ into a propositional Boolean 
formula in conjunctive
normal form (CNF) and checks its validity with a 
Boolean satisfiability solver. 

The work in\cite{sebacASE07} translates an imperative 
C program, with an assertion
statement therein, and a bound on the input size, 
into a %n equisatisfiable
sequential circuit expressed in VHDL. 
It then passes the sequential circuit to a 
sequential circuit solver and decides
the validity of the assertion within the bound. 
It scales to bounds larger than what is possible 
with CBMC\cite{cbmcDAC03} which
translates the program with a bound on the input 
size and the number of loop
iterations into a propositional Boolean formula 
in conjunctive normal form (CNF)
and checks for correctness using a Boolean 
satisfiability solver. 

Our method extends the work in \cite{seraICSE07,sebacASE07} in that
\be
\i it supports function calls including recursion, and requires a bound on
recursion depth only if the recursive function uses local variables, 
\i it enables a termination guarantee check within a bound on execution
time, it then uses the execution time bound with bounded model checking to
decide correctness,
\i it directly translates the program into bit level representation using {\em
and inverted graphs} (AIG) instead of the VHDL representation that requires a
VHDL compiler to be translated into bit level,
\i it uses ABC~\cite{brayton2010abc}, an open source sequential circuit solver, instead of
SixthSense~\cite{mony2004scalable} an IBM internal sequential circuit solver, 
\i and it is an open source tool available online
~\footnote{\label{fn:online}\url{
http://webfea.fea.aub.edu.lb/fadi/dkwk/doku.php?id=sa}}.
\ee
