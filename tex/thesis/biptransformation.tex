%% bip transformation

%% defintions
\section{BIP - Behavior Interaction Priority}
\label{sec:bip}
%%%%%%%%%%%%%%%%%%%%%%%%%%%%%%%%%%%%%%%%%%%%%%%%%%%%%%%%%%%%%%%%%%%%%%%%%%%%%%%%%%%%%%%%%%%%%%%
%%%%%%%%%%%%%%%%%%%%%%%%%%%%%%%%%%%%%%%%%%%%%%%%%%%%%%%%%%%%%%%%%%%%%%%%%%%%%%%%%%%%%%%%%%%%%%%
%
We recall the necessary concepts of the BIP framework~\cite{bip11}.
BIP allows to construct systems by superposing three layers of modeling: Behavior, Interaction, and Priority.
The \emph{behavior} layer consists of a set of atomic components represented by transition systems. The \emph{interaction} layer models the collaboration between components. Interactions are described using sets of ports. The \emph{priority} layer is used to specify scheduling policies applied to the interaction layer, given by a strict partial order on interactions.
%
% \vspace{-1em}
%%%%%%%%%%%%%%%%%%%%%%%%%%%%%%%%%%%%%%%%%%%%%%%%%%%%%%%%%%%%%%%%%%%%%%%%
%%%%%%%%%%%%%%%%%%%%%%%%%%%%%%%%%%%%%%%%%%%%%%%%%%%%%%%%%%%%%%%%%%%%%%%%
\subsection{Component-based Construction}
%%%%%%%%%%%%%%%%%%%%%%%%%%%%%%%%%%%%%%%%%%%%%%%%%%%%%%%%%%%%%%%%%%%%%%%%
%%%%%%%%%%%%%%%%%%%%%%%%%%%%%%%%%%%%%%%%%%%%%%%%%%%%%%%%%%%%%%%%%%%%%%%%
%
BIP offers primitives and constructs for modeling and composing complex behaviors from atomic components. Atomic components are Labeled Transition Systems (LTS) extended with C functions and data. Transitions are labeled with sets of communication ports. Composite components are obtained from atomic components by specifying connectors and priorities.
%
%
\subsubsection{Atomic Components.}
%
An atomic component is endowed with a finite set of local variables $X$ taking values in a domain $\mathit{\Data}$. Atomic components synchronize and exchange data with each others through \emph{ports}.
%
\begin{definition}[Port]
A port $p[x_p]$, where $x_p\subseteq X$, is defined by a port identifier $p$ and some data variables in a set $x_p$ (referred to as the support set). We denote by $p.X$ the set of variables assigned to the port $p$, that is, $x_p$.
\end{definition}
%
%
\begin{definition}[Atomic component]
An atomic component $B$ is defined as a tuple $(P,L,$ $T,X,\{g_\tau\}_{\tau \in T}, \{f_{ \tau}\}_{\tau \in T})$, 
where:
\begin{itemize}
\item $(P,L,T)$ is an LTS over a set of ports $P$. $L\ignore{=\{l_1,l_2,\ldots,l_k\}}$ is a set of control locations and $T \subseteq L \times P \times L$ is a set of transitions.
\item $X$ is a set of variables.
\item For each transition $\tau \in T$: 
\begin{itemize}
\item $g_{\tau}$ is a Boolean condition over $X$: the guard of $\tau$,
\item $f_\tau\in \{x := f^x(X)\mid x\in X\}^*$: the computation step of $\tau$, a sequence of assignments.
%\item $f_\tau$ is a function that updates the set of variables $X$: the computation step of $\tau$.
\end{itemize}
\end{itemize}
\end{definition}
%
For $\tau = (l,p,l')\in T$ a transition of the internal LTS, $l$ (resp. $l'$) is referred to as the source (resp.
destination) location and $p$ is a port through which an interaction with another component can take place. Moreover, a transition $\tau = (l,p,l')\in T$ in the internal LTS involves a transition in the atomic component of the form $(l,p,g_\tau,f_\tau,l')$ which can be executed only if the guard $g_\tau$ evaluates to $\true$, and $f_\tau$ is a computation step: a set of assignments to local variables in $X$.

In the sequel we use the dot notation.
Given a transition $\tau = (l,p,g_\tau,f_\tau,l')$, $\tau.\source$, $\tau.\port$, $\tau.\guard$, $\tau.\func$, and $\tau.dest$ denote $l$, $p$, $g_\tau$, $f_\tau$, and $l'$, respectively.
Also, the set of variables used in a transition is defined as $\var(f_\tau) = \{x \in X \mid x:= f^x(X) \in f_\tau\}$.
Given an atomic component $B$, $B.P$ denotes the set of ports of the atomic component $B$, $B.L$ denotes its set of locations, etc.

Given a set $X$ of variables, we denote by $\valu{X}$ the set of valuations
defined on $X$. Formally, $\valu{X} = \{ \sigma : X \rightarrow \mathit{\Data}\}$, where
$\mathit{\Data}$ is the set of all values possibly taken by variables in $X$.

%
\paragraph{Semantics of Atomic Components.}
The semantics of an atomic component is an LTS over configurations and ports, formally defined as follows:
%
\begin{definition}[Semantics of Atomic Components]
The semantics of the atomic component $B = (P,L, T, X, \{g_{\tau}\}_{\tau \in T}, \{f_{\tau}\}_{\tau \in T})$ is defined as the labeled transition system  $S_B = (Q_B, P_B, T_B)$, where: 
\begin{itemize}
\item $Q_B = L\times \valu{X}$, where $\valu{X}$ denotes the set of valuations on $X$,
\item $P_B = P \times \valu{X}$ denotes the set of labels, that is, ports augmented with valuations of variables, 
\item $T_B$ is the set of transitions defined as follows. $T_B= \{ ((l',v'),p(v_{p}), (l,v))\in Q_B\times P_B\times Q_B \mid \exists \tau = (l', p[x_{p}], l) \in T: g_{\tau}(v') \wedge v=f_{\tau}(v'/v_{p})\}$, where $v_{p}$ is a valuation of the variables of $p$.  
\end{itemize}
\end{definition}
%
A configuration is a pair $(l,v)\in Q_B$ where $l \in L$ is a control location, $v \in \valu{X}$ is a valuation of the variables in $X$. The evolution of configurations $(l', v')\stackrel{p(v_{p})}{\rightarrow} (l, v)$, where $v_{p}$ is a valuation of the variables attached to the port $p$, is possible if there exists a transition $(l', p[x_p], g_\tau, f_\tau, l)$, such that $g_\tau(v')=\true$. As a result, the valuation $v'$ of variables is modified to $v=f_\tau(v'/v_p)$.
%
\subsubsection{Creating composite components.} Assuming some available atomic components $B_1,$ $\ldots,B_n$, we show how to connect the components in the set $\{B_i\}_{i\in I}$ with $I\subseteq [1,n]$ using an \emph{interaction}. An interaction $a$ is used to specify the sets of ports that have to be jointly executed.


\begin{definition}[Interaction]
\label{def:connector}
An interaction $a$ is a tuple $(P_a, G_a,F_a)$, where:
\begin{itemize}
\item $P_a \subseteq \cup_{i = 1} ^ { n } B_i.P$ is a nonempty set of ports that contains
at most one port of every component, that is, $\forall i: 1 \leq i \leq n: |B_i.P \cap P_a| \leq 1 $. We denote by $X_a = \cup_{p \in P_a} p.X$ the set of variables available to $a$,
\item $G_a : \valu{X_a} \rightarrow \{\true, \false\}$ is a guard, 
\item $F_a : \valu{X_a} \rightarrow \valu{X_a}$ is an update function. 
\end{itemize}
\end{definition}
%
$P_a$ is the set of connected ports called the support set of $a$. For each $i\in I$, $x_i$ is a set of variables associated with the port $p_i$.


%
\begin{definition}[Composite Component]
A composite component is defined from a set of available atomic components $\{B_i\}_{i\in I}$ and a set of interaction $\gamma$.
The connection of the components in $\{B_i\}_{i\in I}$ using the set $\gamma$ of connectors is denoted by $\gamma(\{B_{i}\}_{i\in I})$.
\end{definition}

%
\begin{definition}[Semantics of Composite Components]
\label{def-runtimesemanticscomposite}
A state $q$ of a composite component $\gamma(\{B_1, \ldots, B_n\})$, where $\gamma$ connects the $B_i$'s for $i\in [1,n]$, is an $n$-tuple $q=(q_1,\ldots,q_n)$ where $q_i=(l_i,v_i)$ is a state of $B_i$. Thus, the semantics of $\gamma(\{B_1, \ldots, B_n\})$ is precisely defined as the labeled  transition system $S = (Q,\gamma,\goesto)$, where:
\begin{itemize}
\item $Q= B_1.Q\times \ldots\times B_n.Q$, 
\item $\goesto$ is the least set of transitions satisfying the following rule:
\begin{mathpar}
\inferrule*
{
    a = (\{p_i\}_{i \in I},G_a,F_a) \in \gamma \and G_a(\{v_{p_i}\}_{i \in I}) \hva\\
    \forall i\in I:\ q_i \goesto[p_i(v_i)]_i q'_i \wedge v_i = F^i_{a}(\{v_{p_i}\}_{i \in I}) \and
    \forall i\not\in I:\ q_i = q'_i
}
{
    (q_1,\dots,q_n) \goesto[a] (q'_1,\dots,q'_n)
}
\end{mathpar}
where $v_{p_i}$ denotes the valuation of the variables attached to the port $p_i$ and $F^i_{a}$ is the partial function derived from $F_a$ restricted to the variables associated with $p_i$.
\end{itemize}
\end{definition}
%
The meaning of the above rule is the following: if there exists an interaction $a$ such that all its ports are enabled in the current state and its guard evaluates to \true, then the interaction can be fired. When $a$ is fired, all involved components evolve according to the interaction and uninvolved components remain in the same state. 

Notice that several distinct interactions can be enabled at the same time, thus introducing non-determinism in the product behavior.
One can add priorities to reduce non-determinism. In this case, one of the interactions with the highest priority is chosen non-deterministically.\footnote{The BIP engine implementing this semantics chooses one interaction at random, when faced with several enabled interactions.}
%
\begin{definition}[Priority]
  \label{defn:priority}
  Let $S = (Q,\gamma,\goesto)$ be the behavior of the composite component $\gamma(\{B_1, \ldots, B_n\})$.  A {\em priority model} $\pi$ is a
  strict partial order on the set of interactions $A$. Given a priority model $\pi$, we
  abbreviate $(a,a')\in \pi$ by $a \prec_\pi a'$ or $a \prec a'$ when clear from the context. Adding the priority model $\pi$ over $\gamma(\{B_1, \ldots, B_n\})$ defines a new composite component $\pi\big(\gamma(\{B_1, \ldots, B_n\})\big)$ noted $\pi(S)$ and whose behavior is defined by $(Q, \gamma, \goesto_\pi)$, where $\goesto_\pi$ is the least set of transitions satisfying the following rule:
\begin{mathpar}
\inferrule*
	{
      q \goesto[a] q' \and
      \neg\big(\exists a'\in A,\exists q''\in Q: a \prec a' \wedge q \goesto[a'] q'' \big)
    }
    {
      q \goesto[a]_\pi q'
    }
\end{mathpar}
\end{definition}
%
An interaction $a$ is enabled in $\pi(S)$ whenever $a$ is enabled in $S$ and $a$ is maximal according to $\pi$ among the active interactions in $S$.


%% the algorithm
\section{BIP to \thislanguage}
\label{chap3:bip2aig:bip2etc}
Given a BIP system $\Pm = (B, Init, v)$, \biptool~translates \Pm into 
an \thislanguage~program \aigcircuit~with its own customized execution engine. 
Listing~\ref{alg:bip2aig} shows the algorithm used by \biptool~to implement 
this translation. The algorithm is divided into 4 steps: (1) wire and variable
declarations, (2) wire definitions, (3) \cci{init} block and (4) \cci{next} block construction. 
The lists \cci{decl-list}, \cci{wiredef-list}, \cci{init-list} and \cci{next-list}
represent the list of declarations and lists of statements  used in the 
declaration, wire definition, \cci{init} and \cci{next} blocks of \aigcircuit, respectively. 

% ${\bf Input:}~ (\pi\big(\gamma(\{B_i\}_{i \in I})\big)$, K), where $\gamma = \{a_j\}_{j \in J}$
\begin{lstlisting}[caption=BIP to \thislanguage~transformation algorithm,label=alg:bip2aig,float=bt]
${\bf Input:}~ (\pi\big(\gamma(\{B_i\}_{i \in I})\big)$), where $\gamma = \{a_j\}_{j \in J}$
${\bf Output:}~ \thislanguage~program~ \aigcircuit$

/***************** declarations ******************/
// interaction wires
decl-list += wire bool ie[|J|]; // interaction enablement 
decl-list += wire bool ip[|J|]; // interaction priority
decl-list += wire bool is[|J|]; // interaction selected

decl-list += wire int selector; // non-deterministic priority selector

decl-list += bool cycle; // cycle denotes whether we are executing an interaction or a transition 

foreach $i \in [1..|I|]$
  foreach $j \in [1..|B_i.P|]$ 
    decl-list += wire bool $B_i.p_j.e$; // port enablement
    decl-list += wire bool $B_i.p_j.s$; // port selected
  
  decl-list += int $B_i.\ell$;
  foreach $j \in [1..|B_i.X|]$ 
    decl-list += int $B_i.x_j$; // variable registers

/***************** wire list definitions ******************/
foreach $j \in [1..|J|]$
  wiredef-list += ie[j] := $a_j.guard \wedge \bigwedge_{p\in a_i.P} component(p).p.e$ 
  wiredef-list += ip[j] := ie[j] $\wedge \,(\forall k \neq j: ie[k] \Rightarrow a_k < a_j)$ 
  wiredef-list += is[j] := ip[j] $\wedge \,(selector = j \vee (\lnot ip[selector] \land \forall k > j: \neg ip[k])$   

foreach $i \in [1..|I|]$
  foreach $j \in [1..|B_i.P|]$ // where $B_i.P = \{p_1, \ldots, p_{|B_i.P|}\}$ 
    wiredef-list += $B_i.p_j.e := \bigvee_{\tau \in transitions(B_i.p)} \tau.guard \land B_i.\ell = \tau.src$; 
    wiredef-list += $B_i.p_j.s := \bigvee_{a_k \in interactions(B_i.p_j)} is[k]$; 

/***************** init list definitions ******************/
init-list += $cycle := 0$; 
foreach $i \in [1..|I|]$
  init-list += $B_i.\ell := Init.B_i$; 
  foreach $j \in [1..|B_i.X|]$
    init-list += $B_i.x_j := v(B_i.x_j)$; // v is the initial valuation

/***************** next list definitions ******************/
foreach $i \in [1..|I|]$
  foreach $j \in [1..|B_i.X|]$ 
    st = $B_i.x_j$ := (cycle = 0)?
    foreach $k \in [1..|J|]$
      foreach assignment $\sigma \in a_k.action$
        if($B_i.x_j = \sigma.term$)
          st += $is[k]?\, \sigma.expr:$
    
    foreach $\tau \in B_i.T$ 
      foreach assignment $\sigma \in \tau.action$
        if($B_i.x_j = \sigma.term$)
          st += $B_i.port(\tau).s?\, \sigma.expr:$
    st += $B_i.x_j$; 
    
    st += $B_i.\ell := (cycle = 1)?$
    foreach $\tau \in B_i.T$
      st += $B_i.port(\tau).s?\, \tau.dest:$
    st += $B_i.\ell$    
    
    next-list += st;  
  next-list += $cycle := \neg cycle$; 
\end{lstlisting}


Given a port $p$ from the system \Pm, we define the function
$interaction(p) = \set{a \in \gamma~|~p \in a.P}$ where $\gamma$
is the set of interactions in the composite component $B$ of \Pm.
This function returns the set of interactions in which the port 
$p$ takes part. Additionally, the function 
$component(p) = B_i$ such that $p \in B_i.P$ for $i \in I$ defines
the component to which the port $p$ belongs. Furthermore the function
$transitions(p) = \set{\tau \in component(p)~|~\tau \in B.T \land \tau.port = p}$ defines the 
set of transitions $\tau$ in the component $B = component(p)$ such that $\tau$ 
is labeled by $p$. 

Intuitively, for each component $B_i$ in $B$, \biptool~defined a scalar variable 
$B_i.l$ that represents the control location $l \in B_i.L$ at which the 
component $B_i$ is currently at. Additionally, it declares a scalar variable
$B_i.x_j$ for each variable $x_j \in B_i.X$. 
The algorithm defines the initial values for control locations and data variables 
based on the $Init$ relation of the system \Pm. It thus builds the initialization 
list by adding the assignments $``B_i.\ell := Init.B_i"$ and $``B_i.x_j := v(B_i.x_j)"$
for $j \in [1\ldots|B_i.X|]$ and $i \in [1\ldots|I|]$, where $Init.B_i$ is 
the initial location from which $B_i$ starts and $v(B_i.x_j)$ is the initial valuation of the
variable $x_j \in B_i.X$.


For each atomic component $B_i$ in $B$, and for every port $p \in B_i.P$, 
\biptool~defines two Boolean wires. 
(1) The {\em port enablement} wire ($B_i.p_j.e$) that is set to \true~iff the port $p_j \in B_i$ is 
{\em enabled}. The port $p_j$ is enabled iff there exists a transition $\tau \in transitions(p_j)$
such that the guard of $\tau$ evaluates to \true~and $B_i$ is in the control location $l = \tau.src$.
(2) The {\em port selected} wire ($B_i.p_j.s$) that is set to \true{} iff the port $p_j \in B_i$ is
{\em selected}. The port $p_j$ is selected iff there exists an interaction $a \in interactions(p_j)$
that is selected for execution by the BIP engine.

For each interaction $a_k \in \gamma$, \biptool~defines three Boolean wires. 
(1) The {\em interaction enablement} wire, $ie[k]$, that is set to \true{} iff the interaction
$a_k$ is enabled. $a_k$ is enabled iff the guard of the interaction $a_k.guard$ evaluates
to \true{} and all ports $p \in a_k.P$ are enabled.  
(2) The {\em interaction priority} wire, $ip[k]$, that is set to \true{} iff the interaction 
$a_k$ is enabled and there does not exist any interaction $a_j \in \gamma$ such that 
$a_j$ is enabled and has a higher priority than $a_k$.
(3) The {\em interaction selected} wire, $is[k]$, that is set to \true{} iff the
interaction $a_k$ is selected for execution. Given a set of enabled interactions 
with equal priority, the BIP engine synthesized by \biptool{}
selects interactions for execution based on the following procedure. It makes use of a
non-deterministic selector (the \cci{selector} scalar wire defined on line 10 
of Listing~\ref{alg:bip2aig}) and selects the active interaction $a_j$ having 
an index $j$ equal to the non-deterministic value of the selector wire.
If no such interaction exists, the engine selects the interaction with the largest 
index. 
This is depicted by the assignment $``is[j] = ip[j] \land \left( (selector = j) 
\land (\lnot (ip[selector] \land \forall{k > j}\,:\lnot ip[k])) \right)"$ on line 32 of Listing~\ref{alg:bip2aig}.

%% the bip engine we use
\biptool{} synthesizes a BIP execution engine that is based on a two clock cycles 
approach. In the first cycle, the engine selects an interaction $a \in \gamma$ 
and executes its actions. In the second cycle, the engine checks for possible
transitions $\tau \in B_i$ such that $\tau.port$ is selected, for $i \in [1\ldots|I|]$,
and executes their actions $\tau.actions$. Contrarily to the BIP engine in~\cite{BasuBBCJNS11},
\biptool's engine executes the actions of all possible transitions simultaneously.
Additionally, the engine updates the locations of the atomic components $B_i$ of $B$
in accordance with the executed transitions; \ie{} executing a transition $\tau \in B_i.L$ 
transitions $B_i$ from control location $l_1 = \tau.src$ to location $l_2 = \tau.dest$.

\biptool{} updates the values of the data variables in accordance with the 
actions of the executed interactions and transitions. An assignment statement 
$\sigma \in a_k.actions$ where $a_k \in \gamma$ is an interaction executes 
and updates the value of its target data variable iff the interaction $a_k$ 
is selected for execution, \ie when $is[k]$ evaluate to \true{} and the engine
is in its first cycle. Similarly $\sigma \in \tau.actions$ where $\tau$ is a transition
executes and update the value of its target data variable iff the transitions $\tau$
is allowed to execute, \ie when $\tau.port$ is selected and the engine is in its
second cycle. Updates to data variable are shown on lines 61 and 69 of the 
algorithm in Listing~\ref{alg:bip2aig}.

\biptool{} uses the same AIG synthesis engine used in \mytool{} in order to 
synthesize an AIG from the generated \thislanguage{} program \aigcircuit. 
If the developer wishes the check for deadlock freedom, \biptool{} adds 
an assertion statement to \aigcircuit{} to check that at each cycle, 
there is at least one interaction enabled for execution. Additionally, 
developers can also add custom assertions defined as FOL properties. These assertions
are handled in the same manner that \mytool{} handles them, and \biptool{}
uses the ABC AIG solver to check for the validity of the assertions. 

\section{Illustrative Example}
Table~\ref{tb:bip:exec} shows a sample execution trace of \mytool's BIP engine
on the traffic light example presented in Figure~\ref{fig:traffic:bip}. The variables 
$timer.t$, $timer.n$, $timer.\ell$, $light.\ell$ and $light.m$ represent $t$, $n$
and the control location variable in the timer component, and the control location variable
and $m$ in the light component, respectively. 
We start at cycle $c$ where that state of the BIP system is $$\left(timer.t = 9, timer.n = 10,
timer.\ell = s_0, light.\ell = G, light.m = 3\right)$$
and we assume that the execution engine has picked the interaction involving the 
$timer$ port of the timer component has been selected and executed. 

At cycle $c+1$, the engine executes the internal transition in the timer component
that corresponds to the $timer.timer$ port being selected and the component is in location $s_0$. 
Therefore the variable $timer.t$ is incremented and the component 
remains in the same control location $s_0$. Since the light component
had no ports involved in the executed interaction, it will not change 
state. 

At cycle $c+2$, the guard $t \geq n$ is enabled and thus the $timer.done$ and $light.done$ ports are enabled. 
The engine selects the interaction connecting the two components
for execution. It executes the actions associated with this interaction 
and transfers the value of the variable $done.m$ to the variable $done.n$.

At cycle $c+3$, the engine executes the interaction transitions that 
correspond to the interaction executed at cycle $c+2$. In the timer 
component, the variable $timer.t$ is reset $0$ and the control location 
remains in location $s_0$. In the light component, the variable 
$light.m$ takes the value $5$ and the control location moves to location 
$Y$ corresponding to the yellow light being displayed. 

\begin{table}[bt]
\centering
\caption{Sample of \biptool{} execution}
\begin{tabular}{|c|c|c|c|c|c|}
\hline
Cycle & $timer.t$ & $timer.n$ & $timer.\ell$ & $light.\ell$ & $light.m$ \\ 
\hline
$c$ & $9$ & $10$ & $s_0$ & $G$ & $3$ \\
		\hline
		$c+1$ & $10$ & $10$ & $s_0$ & $G$ & $3$ \\
			\hline
			$c+2$ & $10$ & $3$ & $s_0$ & $G$ & $3$ \\
				\hline
	  $c+3$ & $0$ & $3$ & $s_0$ & $Y$ & $5$ \\
	  \hline
\end{tabular}
\label{tb:bip:exec}
\end{table}

Listing~\ref{sample:code:bip} shows a code snippet from the code that \biptool{} generates
when applied to the traffic light controller BIP system shown in Figure~\ref{fig:traffic:bip}.
Lines 2 to 19 show the wire definitions for the system. The interaction 
labeled by $0$ corresponds to the interaction involving the $timer.timer$
port, while the interaction labeled $1$ correspond to the interaction 
involving the ports $timer.done$ and $light.done$. 
Interaction $0$ is enabled ($ie[0]$) when its guard is enabled ($true$ 
in this case) and the port $timer.timer$ that it involves is enabled, 
\ie{} the wire $timer.timer.e$ is set to $true$. 
Similarly, interaction $1$ is enabled ($ie[1]$) when its guard is
enabled and the ports $timer.done$ and $light.done$ are enabled 
($timer.done.e \land light.done.e$).

The wires $ip[0]$ and $ip[1]$ are asserted when each of the two interactions 
is prioritized. We assume in our case that interaction $1$ has a higher
priority, and thus interaction $0$ is only prioritized when it is enabled
and interaction $1$ is not. In any other case, it is interaction $1$ that
is prioritized by the engine. 

Furthermore, the wires $is[0]$ and $is[1]$ represent the execution of the 
engine's interaction selection procedure. The wire $selector$ is
a non-deterministic wire used by the engine to make non-deterministic
choices when several interactions are prioritized. 
The interaction $0$ is selected
when it is prioritized and either the non-deterministic selector 
selects it ($selector == 0$)
or the interaction $1$ is not prioritized. 
The interaction $1$ is selected when it is prioritized and either 
the selector selects it ($selector == 1$) or the selector wire
has selected an interaction that is not prioritized. Since interaction
$1$ has the largest index, it is
directly selected when the value of the selector corresponds
to an interaction that is not prioritized ($\lnot ip[selector]$).

The wires $component.port.e$ and $component.port.s$ represent the 
port enabled and selected wires, respectively. 
For example, the timer port 
in the timer component ($timer.timer.e$) is enabled when the 
component is in control location $s_0$ and the guard $timer.t < n$ is
enabled. Additionally, this port is selected when the interaction $0$ 
in which it is involved has been selected ($timer.timer.s = is[0]$).

Lines 21 to 38 show the next state functions for each of the variables
and control variables of the components of the system. 
In the first execution cycle ($cycle == 0$), 
the engine selects and executes interactions.
In the second execution cycle ($cycle == 1$), 
it executes internal transitions. 
When the interaction $1$ is selected ($is[1]$) and the engine is 
in its first cycle, the value of the variable $light.m$ is transfered
to the variable $timer.n$. Otherwise, $timer.n$ retains its current
value.

In the internal transition execution cycle, the next state value of
the variable $light.m$ is dictated by the control location at which the component is currently at, and whether the port $light.done$
is selected (equivalently the interaction $1$ has been selected).
For example, if the port $light.done$ is selected and the component 
is at the yellow light location ($Y$), the value of
$light.m$ will be 10. 

Finally, the control location variables $timer.\ell$ and 
$light.\ell$ are also updated according to the internal transitions of the 
atomic components. For example when the $light.done$ port is selected
and the light component is in the green control location ($G$), 
 $light.\ell$ moves from $G$ to $Y$. 

\begin{lstlisting}[caption=Sample of \biptool{} generated code,label=sample:code:bip,float=bt]
/** Wire definitions **/
$ie[0] = true \land timer.timer.e$
$ie[1] = true \land timer.done.e \land light.done.e$

$ip[0] = ie[0] \land \lnot ie[1]$
$ip[1] = ie[1]$

$is[0] = ip[0] \land ((selector == 0) \lor (\lnot ip[selector] \land \lnot ip[1]))$
$is[1] = ip[1] \land ((selector == 1) \lor (\lnot ip[selector]))$

$timer.timer.e = (timer.\ell == s_0) \land (t < n)$
$timer.done.e = (timer.\ell == s_0) \land (t \geq n)$
$light.done.e = true \land ((light.\ell == G)$ 
             $||(light.\ell == R)$
             $||(light.\ell == Y))$
             
$timer.timer.s = is[0]$
$timer.done.s = is[1]$
$light.done.s = is[1]$

/** Next state functions: Interactions **/
$timer.n = (cycle == 0)? (is[1]? light.m : timer.n) : timer.n$

/** Next state functions: Transitions **/
$timer.t = (cycle == 1)?$ 
   $((timer.timer.s)? (timer.t + 1) :$
   $((timer.done.s)? 0 : timer.t)) : timer.t$;
$light.m = (cycle == 1)?$
   $((light.done.s \land light.\ell == G)? 5 :$
   $((light.done.s \land light.\ell == Y)? 10 :$
   $((light.done.s \land light.\ell == R)? 3 : light.m))) : light.m$
$timer.\ell = (cycle == 1)?$
   $((timer.timer.s)? s_0:$
   $((timer.done.s)? s_0:timer.\ell)):timer.\ell$
$light.\ell = (cycle == 1)?$
   $((light.done.s \land light.\ell == G)? Y :$
   $((light.done.s \land light.\ell == Y)? R :$
   $((light.done.s \land light.\ell == R)? G : light.\ell))) : light.\ell$
\end{lstlisting}
%%%%%%%%%%%%%%%%%%%%%%%%%%%%%%%%%%%%%%%%%%%%%%%%%%%%%%%%%%%%%%%%%%%%%%%%%%%%%%%%%%%%%%%%%%
%%%%%%%%%%%%%%%%%%%%%%%%%%%%%%%%%%%%%%%%%%%%%%%%%%%%%%%%%%%%%%%%%%%%%%%%%%%%%%%%%%%%%%%%%%
%%%%%%%%%%%%%%%%%%%%%%%%%%%%%%%%%%%%%%%%%%%%%%%%%%%%%%%%%%%%%%%%%%%%%%%%%%%%%%%%%%%%%%%%%%
%
%%
%%This section presents how we translate a BIP system $(B,\mathit{Init}, v)$ to its corresponding sequential circuit  $\big( (V, E),G, O\big)$. 
