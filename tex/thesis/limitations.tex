

Recent advances in SAT enabled tools like the 
Alloy Analyzer~\cite{jackson2002alloy} and CBMC to check designs of real systems.
However,
these designs often need to be partial, leaving out 
important functionality aspects of the systems, 
to enable the analysis to complete.  
Moreover, the analysis is typically bound to
relatively small limits, e.g., fewer than 7 nodes in 
a tree structure with the Alloy Analyzer.

There are three limiting aspects of translating high-level 
programs to SAT.  
\begin{description}

\item[Disadvantage 1]
The translation to CNF depends on
the bounds; a small increase in the bound on variable
ranges can cause a large increase in the size of the
translated CNF formula due to unwinding loop and recursion
structures in programs, or eliminating quantifiers 
%and unrolling transitive closure 
in declarative first order logic.
%for
%example, for an undirected seven-node tree the translation
%from Alloy to CNF generates a formula with over one million
%variables and five million clauses.  

\item[Disadvantage 2] 
The SAT solver is restricted to
using optimizations, such as symmetry
breaking~\cite{Aloul02SymSAT} and observability don't
cares (ODC)~\cite{FuYuMalik2005}, that apply at the level
of CNF formulas.  
However these optimizations usually aim
at increasing the speed of the solver and often result in
larger formulas as they add literals and clauses to the
CNF formula to encode symmetry and ODC
optimizations~\cite{ZhuKu06SATSweepODC}.  
Often times when
the analyzer successfully generates a large CNF formula,
the underlying solver requires intractable resources.

\item[Disadvantage 3] 
Often times the CNF formula
generated needs to be regenerated with higher bounds in
case the unwinding bounds were not large enough for the
loops to complete as is the case with CBMC. 
Note that multiple bounds exist and they need not be all
increased during one iteration.
\end{description}

To extend the applicability of static analysis to a wider
class of programs as well as to check more
sophisticated specifications 
and gain more confidence in the
results, we need to scale the analysis to significantly
larger bounds; limits on the range of design and program
variables.