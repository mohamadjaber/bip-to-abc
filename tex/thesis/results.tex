We evaluate \mytool{} against three different program verification benchmarks. 
We first present a case study of linear and binary search and compare 
\mytool's results with those obtained from CBMC.
The second
set of benchmarks 
contains standard functions such as searching, sorting, linked list operations
and array partitioning. 
The third set contains benchmarks
from the second competition on Software Verification
(SV-COMP 2013)~\cite{svcomp}, a competition aimed at the thorough evaluation of 
automatic program verification tools. We ran all experiments on an
Intel Core i7 machine with 8 GB memory. 

\section{Searching: \mytool{} and CBMC}
\label{chap:res:casestudies}
%% searching studies with CBMC
The programs $\Pm_1$ and $\Pm_2$ in Figures~\ref{fig:linearsearch} and~\ref{fig:binarysearch}
show the linear search and binary search algorithms annotated with appropriate
preconditions and postconditions, respectively. The precondition in $\Pm_1$ ensures that 
the provided size $n$ is valid, \ie positive and smaller than the largest allowed array size. 
The precondition in $Pm_2$ checks that the provided size $n$ is valid and that the 
array $a$ is sorted in ascending order. 
The postcondition in both $\Pm_1$ and $\Pm_2$ 
checks that when the returned index $rv$ is between 0 and the size $n$, then 
array element at the index, $a[rv]$, is equal to the element $e$ to search for. It also 
checks that when $rv$ is invalid, the element $e$ is not present in the 
array $a$. 

We use \mytool{} and CBMC to verify that the programs $\Pm_1$ and $\Pm_2$ satisfy the 
precondition-postcondition pairs provided. We make use of the assume statement \cci{\_\_CPROVER\_assume}
in CBMC to implement the preconditions, and we encode the quantifiers as while loops. 
In \mytool{}, we use the dedicate property grammar and the FOL support to write the specifications. 
We implement the check $precondition \implies postcondition$ as an assertion statement
in CBMC.

\begin{figure}[bt]
\begin{Verbatim}[fontsize=\relsize{-1.0}, numbersep=4pt,numbers=left]
int ls (int [] a, int e, int n) {
  @pre ls {0 <= n && n <= MAX_ARRAY_SIZE}
  int i = 0;
  while (i < n) {
    if (a[i] == e)
      return i;
    i = i + 1;  
  }
  return -1;
  
  @post ls { (rv >= 0 && rv < n) -> (a[rv] == e) &&
    (rv == -1) -> (forall(int i:[0..n-1]) {a[i] != e})
  }
}
\end{Verbatim}
\caption{The linear search algorithm with pre and post conditions}
\label{fig:linearsearch}
\end{figure}

\begin{figure}[bt]
\begin{Verbatim}[fontsize=\relsize{-1.0}, numbersep=4pt,numbers=left]
int bs (int [] a, int e, int n) {
  @pre bs {0 <= n && n <= MAX_ARRAY_SIZE && isSorted(a,n)}
  int low = 0, high = n -1;
  while (low < high) {
    int mid = (low + high) >> 1;
    if (a[mid] < e) {
      low = mid + 1;
    } else {
      high = mid;
    }
  }
  if ((low == high) && (a[low] == el)) {
    return low;
  } else {
    return -1;
  }
  
  @post bs { (rv >= 0 && rv < n) -> (a[rv] == e) &&
    (rv == -1) -> (forall(int i:[0..n-1]) {a[i] != e})
  }
}

isSorted(int [] a, int n) = forall(int i:[0..n-2]) {
  a[i] <= a[i+1]
}
\end{Verbatim}
\caption{The binary search algorithm with pre and post conditions}
\label{fig:binarysearch}
\end{figure}

We compare the results obtained from \mytool{} and CBMC in terms of the problem size and
verification time. The problem size is defined as the number of variables and clauses 
in the generated CNF formulae in CBMC, and the number of latches, AND gates and logic levels
in the generated AIGs in \mytool{}. We report on the size of the generated formulae and AIGs
after performing optimization and reduction steps in both CBMC and \mytool{}. We compare 
the time taken by both tools to perform all of the required steps to return a decision, 
including both optimization and verification.
Since CBMC supports only $16$, $32$ and $64$ bits scalars, we use the size of the input
array $a$ as the variable to change between different settings of the programs $\Pm_1$
and $\Pm_2$. We set a time-out limit of $30$ minutes and do not set a limit on the 
amount of memory the program can use (up to the machine's physical limit).

Note that in the addition to the above, CBMC requires users to provide an {\em unwinding
limit}. This limit is used by CBMC to unroll loops in the input programs $\Pm_1$ and $\Pm_2$.
We use an unwinding limit of $2\times array\_size$ for our experiments and we enable 
the CBMC unwinding assertions. \mytool{} does not require any unwinding limit since
it does not perform unrolling, it uses the program counter to execute the loops infinitely 
many times using the same AIG. 

Table~\ref{tb:cbmc_compare} shows the results of using \mytool{} and CBMC to verify the 
programs $\Pm_1$ and $\Pm_2$ for different sizes of the input array $a$, shown in the 
column {\em size}. The columns {\em Vars} and {\em Clauses} report on the size of the
CNF formula generated by CBMC in terms of the number of variables and clauses, respectively. 
The columns {\em lat}, {\em and} and {\em lev} show the size of the generated AIG by \mytool{}
in terms of the number of latches, and gates and logic levels, respectively. The last two columns
show the time taken by \mytool{} and CBMC to perform both reduction and verification. 
The {\em Time-out} entry indicates that the tool reached the time out limit before 
returning a conclusive result about the verification problem.

\begin{table}[htbp]
\caption{\mytool{} and CBMC comparison}
\begin{tabular}{|c|c|c|c|c|c|c|c|c|}
\cline{3-9}
\multicolumn{1}{l}{} & \multicolumn{1}{c}{} &  \multicolumn{2}{|c|}{CBMC formula size} & \multicolumn{3}{l|}{\mytool{} AIG size} & \multicolumn{2}{c|}{Time (s)} \\ \hline
\Pm & size  & Vars  & Clauses & lat & and & lev & \mytool & CBMC \\ \hline
$\Pm_1$ & 3 & 2416 & 6784 & 41 & 313 & 15 & 4.36 & 0.016 \\ \hline
$\Pm_1$ & 7 & 4612 & 15008 & 68 & 568 & 19 & 12.4 & 722.4 \\ \hline
$\Pm_1$ & 15 & 9112 & 34496 & 119 & 1116 & 21 & 16.87 & \color{red}{Time-out} \\ \hline
$\Pm_1$ & 31 & 18332 & 84928 & 226 & 2346 & 24 & 33.67 & \color{red}{Time-out} \\ \hline
$\Pm_1$ & 63 & 37216 & 230208 & 461 & 5100 & 26 & 99.64 & \color{red}{Time-out} \\ \hline
$\Pm_1$ & 127 & 75876 & 695616 & 984 & 11315 & 28 & 396.98 & \color{red}{Time-out} \\ \hline
\hline
$\Pm_2$ & 3 & 6503 & 24533 & 56 & 55 & 19 & 1.04 & 0.085 \\ \hline
$\Pm_2$ & 7 & 16172 & 68130 & 83 & 850 & 17 & 1.47 & 1.91 \\ \hline
$\Pm_2$ & 15 & 42461 & 197223 & 143 & 1943 & 20 & 27.69 & 38.493 \\ \hline
$\Pm_2$ & 63 & 390623 & 2133649 & 529 & 9052 & 25 & 1152.22 & \color{red}{Time-out} \\ \hline
\end{tabular}
\label{tb:cbmc_compare}
\end{table}


The results show that \mytool{} is able to verify the linear search program $\Pm_1$ 
for sizes much higher than those provided by CBMC. For array size of $15$ and above, \mytool{}
is able to efficiently generate the AIG, and call ABC to reduce and verify it, while CBMC
reached the time out limit without giving any decision. Also, the size of the generated AIG 
by \mytool{} is always smaller than the size of the CNF formula generated by CBMC.
For example, for an array size of $15$, CBMC's CNF formula contains $9112$ variables
and $34496$ clauses, while \mytool{}'s AIG has $119$ bit registers and $1116$ AND gates.
This clearly shows the advantage that the program counter encoding provides over 
loop unrolling. 

For the binary search program ($\Pm_2$), both tools show similar performance for sizes of $3$, $7$
and $15$. \mytool{} outperforms CBMC for a bound of $63$ since CBMC reached the time out limit
while \mytool{} was able to verify the program in $1152$ seconds. Similarly to $\Pm_1$, 
the size of the generated AIGs by \mytool{} is orders of magnitudes smaller than the size of the 
generated CNF formulae by CBMC. For example, for a size of $63$, the number of variables  
in the generated AIG is $99\%$ smaller than the number of variables in the generated CNF formula.

\section{Standard benchmarks}
\label{chap:res:standardbench}
\begin{table}[bt]
\caption{Results of standard benchmarks}
\begin{tabular}{|c|c|c|c|c|c|c|c|c|c|c|}
\cline{3-10}
\multicolumn{1}{c}{} & \multicolumn{1}{c}{} & \multicolumn{3}{|c|}{Before reduction} & \multicolumn{3}{c|}{After reduction} & \multicolumn{2}{c|}{Time (s)} & \multicolumn{1}{c}{}  \\ \hline
\Pm & $b$ & lat & and & lev & lat & and & lev & Ver. & Total & Check \\ \hline
ls & 2 & 86 & 719 & 24 & 41 & 313 & 15 & 0.33 & 4.36 & \checkmark \\ \hline
ls & 3 & 118 & 1064 & 27 & 68 & 568 & 19 & 3.89 & 12.4 & \checkmark \\ \hline
ls & 4 & 174 & 1781 & 30 & 119 & 1116 & 21 & 2.41 & 16.87 & \checkmark \\ \hline
ls & 5 & 286 & 3362 & 45 & 226 & 2346 & 24 & 1.43 & 33.67 & \checkmark \\ \hline
ls & 6 & 526 & 6895 & 78 & 461 & 5100 & 26 & 4.57 & 99.64 & \checkmark \\ \hline
ls & 7 & 1054 & 14780 & 143 & 984 & 11315 & 28 & 21.32 & 396.981 & \checkmark \\ \hline
ls & 8 & 4798 & 70742 & 529 & 4718 & 55364 & 33 & 682.11 & 8022.11 & \checkmark \\ \hline
\hline
bsort & 2 & 114 & 1198 & 29 & 44 & 393 & 16 & 0.29 & 5.79 & \checkmark \\ \hline
bsort & 3 & 169 & 2218 & 35 & 68 & 885 & 20 & 17.1 & 31.09 & \checkmark \\ \hline
bsort & 4 & 276 & 5607 & 47 & 117 & 2106 & 22 & 1390.25 & 1426.98 & N/A \\ \hline
\hline
ss & 2 & 112 & 1208 & 27 & 43 & 427 & 15 & 0.53 & 5.81 & \checkmark \\ \hline
ss & 3 & 167 & 2239 & 35 & 69 & 949 & 19 & 209.37 & 223.54 & \checkmark \\ \hline
ss & 4 & 280 & 5676 & 47 & 125 & 2236 & 22 & 1800 & 1852.76 & N/A \\ \hline
\hline
a-p & 2 & 110 & 1896 & 33 & 57 & 689 & 19 & 0.87 & 2.79 & \checkmark \\ \hline
a-p & 3 & 147 & 2509 & 34 & 87 & 1174 & 24 & 93.47 & 97.56 & \checkmark \\ \hline
a-p & 4 & 208 & 3829 & 38 & 141 & 2419 & 29 & 2127 & 2135.7 & N/A \\ \hline
\hline
lli & 2 & 237 & 4310 & 38 & 98 & 871 & 19 & 109.63 & 118.89 & \checkmark \\ \hline
lli & 3 & 344 & 6117 & 41 & 179 & 1693 & 26 & 1800 & 1811 & N/A \\ \hline
\hline
llr & 2 & 197 & 2906 & 33 & 84 & 722 & 21 & 47.72 & 71.3829 & \checkmark \\ \hline
llr & 3 & 293 & 4454 & 39 & 157 & 1387 & 25 & 1800.15 & 1830.21 & N/A \\ \hline
\hline
bs & 3 & 94 & 879 & 30 & 56 & 555 & 19 & 0.11 & 1.04 & \checkmark \\ \hline
bs & 4 & 151 & 1832 & 42 & 83 & 850 & 17 & 0.54 & 1.47 & \checkmark \\ \hline
bs & 5 & 268 & 5185 & 62 & 143 & 1943 & 20 & 25.42 & 27.69 & \checkmark \\ \hline
\end{tabular}
\label{tb:psq:standard}
\end{table}

Table~\ref{tb:psq:standard} shows the results of applying 
\mytool{} on a set of standard program functions. The first column shows
the program \Pm to verify. \cci{ls} stands for {\em linear search}, 
\cci{bs} stands for {\em binary search}, \cci{bsort} stands for 
{\em bubble sort}, \cci{ss} stands for {\em selection sort},
\cci{a-p} stands for {\em array partitioning}, \cci{lli} stands for 
{\em linked list insert} and \cci{llr} stands for {\em linked list remove}. 
The column $b$ shows the chosen bit width for the variables in \Pm. 
Table~\ref{tb:psq:properties:standard} lists the properties verified for each of the 
standard benchmarks. 

\begin{table}[bt]
\caption{The properties checked for the standard benchmarks}
\begin{tabular}{|p{0.2\textwidth}|p{0.7\textwidth}|}
\hline
\multicolumn{1}{|c|}{\Pm} & Property \\ \hline
Linear-search & The element is actually in the array if the return index is valid, and is not present in the array if the return index is invalid \\ \hline
Binary-search & The element is actually in the array if the return index is valid, and is not present in the array if the return index is invalid \\ \hline
Bubble-sort & The array is actually in sorted order \\ \hline
Selection-sort & The array is actually in sorted order \\ \hline
Array-partition & The array is partitioned around the element at 0 \\ \hline
Linked-list & The list is consistent and the insertion (removal) actually took place \\ \hline
\end{tabular}
\label{tb:psq:properties:standard}
\end{table}


To evaluate the size of the generated AIGs, we report on the 
number of latches (\cci{lat}), the number of AND  gates (\cci{and})
and the number of logic levels (\cci{lev}) as recorded by the 
ABC tool. We show the size of the AIGs before and after applying 
synthesis and reduction algorithms. We use common synthesis 
algorithms such structural sweeping ({\em ssweep}), 
retiming ({\em retime}), refactoring ({\em refactore})
and several other combinations of algorithms provided by ABC. 

For performance evaluation, the column {\em Ver.} shows the 
time taken by the verification algorithm and the column {\em Total} 
shows the total time taken by ABC to perform both synthesis (reduction)
and verification. 
The set of provided programs are all correct (\ie
contain no bugs), the {\em Check} column shows the result of the 
proof algorithm applied by ABC. A conclusive check ($\checkmark$) 
indicates that the solver was able to assert that the program
satisfies its specifications. A non conclusive check (N/A) 
indicates that the solver hit the time limit before returning 
a validity answer. We set a time-out of $1800$ seconds on induction based
proof algorithm and a $1000000$ {\em Binary Decision Diagrams} 
(BDD) size limit for algorithms based on 
BDD reachability. The amount of memory that \mytool{} 
is allowed to use is only limited by the machine's physical limit, \ie $8$ GB.

The results in Table~\ref{tb:psq:standard} clearly show the advantage that
using the sequential circuit encoding provides. The ABC synthesis engine is always
able to rewrite the generated AIG in a way to greatly reduce the number of AND gates
and logic levels. For example, for the linear search algorithm with a bit bound of 
8, ABC achieved a $93\%$ reduction in the number of logic levels (from 529 to 33), 
and a $21\%$ reduction in the number of AND gates. For more complex designs,
such as the linked list insertion, the reduction algorithms achieved $50\%$
reduction in the number of logic levels, 
$80\%$ reduction in the number of AND gates 
and $58\%$ reduction in the number of latches. 

On average, the ABC synthesis algorithms achieved $43\%$ reduction in the 
number of latches, $53\%$ reduction in the number of AND gates and 
$47\%$ reduction in the number of logic levels. Therefore we can conclude
that \mytool{} can effectively make use of ABC's synthesis algorithm to 
reduce the problem by half and thus help the proof algorithms. Furthermore, 
we note that even in the cases where the ABC solver was not able to 
provide a conclusive result about the properties to verify, \mytool{} was
still able to generate and efficiently reduce the AIG circuits. This in fact
allows us to try the validity checks using higher time-out limits and BDD size limits, 
or  try new proof algorithms in the future.
%The reduction rate that the ABC synthesis algorithms can achieve in terms
%of the number of latches can be sometime limited by the nature of the 
%program \Pm, and the number of variables in it. Since we wrote our algorithms
%in a way to minimize unused variables in the code
%Since the number of latches in the AIG is highly dependent on the 
%number of variables in the program \Pm and the provided bit width, 
%the reduction rate that the ABC synthesis algorithms can be sometimes 
%limited 

\section{SV-COMP 2013 benchmarks}
\label{chap:res:svcompbench}
%% svcompetition benchmarks
We evaluated \mytool{} against a select set of benchmarks obtained from the
second competition on Software Verification SVCOMP'13~\cite{svcomp}.
We selected the benchmarks from the {\em ControlFlowInteger} and the {\em Loops}
and compare the execution time obtained from running \mytool{} with the 
tools ranked first, second and third in each of the two categories. 

\begin{table}[tb]
\caption{SVCOMP'13 results}
\centering
\begin{tabular}{|c|c|c|c|c|c|}
\hline
\Pm & status & time (s) & gold  & silver & bronze \\ \hline
locks\_5\_safe & \color{green}{safe} & 0.28 & 0.4 & 1.2 & 1.3 \\ \hline
locks\_6\_safe & \color{green}{safe} & 0.32 & 0.43 & 1.3 & 1.6 \\ \hline
locks\_7\_safe & \color{green}{safe} & 0.52 & 0.4 & 1.3 & 1.9 \\ \hline
locks\_8\_safe & \color{green}{safe} & 0.62 & 0.4 & 1.3 & 2.3 \\ \hline
locks\_9\_safe & \color{green}{safe} & 1.05 & 0.39 & 1.3 & 3.5 \\ \hline
locks\_10\_safe & \color{green}{safe} & 0.88 & 0.4 & 1.3 & 6.4 \\ \hline
locks\_11\_safe & \color{green}{safe} & 3.42 & 0.39 & 1.4 & 24 \\ \hline
locks\_12\_safe & \color{green}{safe} & 4.21 & 0.4 & 1.4 & 110 \\ \hline
locks\_13\_safe & \color{green}{safe} & 4.18 & 0.43 & 1.4 & 100 \\ \hline
locks\_14\_safe & \color{green}{safe} & 5.9 & 0.4 & 1.4 & 100 \\ \hline
locks\_15\_safe & \color{green}{safe} & 6.83 & 0.4 & 1.4 & 100 \\ \hline
locks\_14\_unsafe & \color{green}{unsafe} & 0.99 & 3.2 & 1.6 & 1.8 \\ \hline
locks\_15\_unsafe & \color{green}{unsafe} & 0.91 & 4.2 & 1.6 & 1.8 \\ \hline
\hline
Average & N/A & 2.316 & 0.911 & 1.377 & 34.969 \\ \hline
\hline
array\_safe & \color{green}{safe} & 0.116 & 0.05 & 0.15 & 0.48 \\ \hline
array\_unsafe & \color{green}{unsafe} & 0.091 & 0.09 & 0.47 & 0.46 \\ \hline
count\_up\_down\_s & \color{green}{safe} & 0.137 & 0.28 & {\bf \color{red}{450}} & 0.39 \\ \hline
count\_up\_down\_u & \color{green}{unsafe} & 0.095 & 0.04 & 0.15 & 0.43 \\ \hline
invert\_string\_s & \color{green}{safe} & 1.808 & 0.03 & 14 & {\bf \color{red}{0.56}} \\ \hline
invert\_string\_u & \color{green}{unsafe} & 0.266 & 0.11 & 10 & 0.68 \\ \hline
\hline
Average & N/A & 0.419 & 0.100 & 79.128 & 0.500 \\ \hline
\end{tabular}
\label{tb:svcomp}
\end{table}


Table~\ref{tb:svcomp} summarizes the results obtained from running \mytool{} on the
select set of SVCOMP'13 benchmarks. The first column shows the name of the 
benchmark as presented in the competition. The {\em status} column shows the decision 
that \mytool{} returned, it is colored in green to indicate that the results are accurate, \ie
\mytool{} produced no false negative nor false positives. 
We report on the total execution time taken by \mytool{}, the winner on the category ({\em gold}),
the first runner ({\em silver}) and the second runner ({\em bronze}) for each of the selected
benchmarks. Additionally, we report on the average execution time taken by each of the four 
tools for each of the two categories from which we selected the benchmarks. 
The benchmarks labeled \cci{locks\_*} belong to the {\em ControlFlowInteger} category,
while the remaining benchmarks belong to the {\em Loops} category. 

In the {\em ControlFlowInteger} category, \mytool{} outperformed the bronze tool 
on all of the benchmarks. It was also able to outperform the silver tool
on the first six safe benchmarks and outperform the golden tool on the 
first two. For the unsafe benchmarks, \mytool{} surpassed the other tools 
and was able to find a counterexample $1.7$ times faster than the fastest tool.
On average, \mytool{} topped the bronze tool and was ranked very closely behind
the silver one. 

In the {\em Loops} category, \mytool{} surpassed both of the silver and the 
bronze tools on all of the benchmarks and came very close the gold tool.
In fact, the silver tool produced a false counterexample on the \cci{count\_up\_down\_s}
benchmark and the bronze tool produced a false counterexample on the \cci{invert\_string\_s}
benchmark. \mytool{} was able to accurately verify and disprove all of the benchmarks. 
On average, \mytool{} ranked second outperforming both the silver and the bronze tools, and
came very closely behind the gold tool. 