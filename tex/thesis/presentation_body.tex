%% presentation body

%% Introduction
\section{Introduction}

\begin{frame}{Heartbleed}
%\resizebox{1.0\textwidth}{!}{
\begin{figure}
 \centering
 \includegraphics[scale=0.4]{figures/heartbleed}
\end{figure}
%}
\end{frame}

\begin{frame}{Program defects }
  \begin{itemize}
  \item
    %Safety critical systems rely on software and hardware programs to provide accurate services. 
    Correct and reliable software and hardware systems are desirable
  \begin{itemize}
   \item Safety critical systems (medical, navigation, and military)
  \end{itemize}
  \item
	Defects may lead to significant losses: possible loss of life
  \begin{itemize}
   \item 
    The Ariane 5 missile, the Therac-25 machine~\cite{ammann2008introduction}
   \item
    The buggy Intel Pentium processors  (\$475 million)~\cite{kropf1999introduction}
  \end{itemize}
%  \item 
%    e.g. The Ariane 5 missile, The Therac-25 machine, \dots~\cite{ammann2008introduction}
  \end{itemize}
\end{frame}

\begin{frame}{Proposed solutions: dynamic analysis}
	 \begin{itemize}
	 	\item 
		 Given a program $\mathcal{P}$
		 \begin{itemize}
		   \item A set of inputs $\mathcal{I}$, and outputs $\mathcal{O}$
		   \item A set of input and expected output pairs $\mathcal{T} \subset \mathcal{I} \times \mathcal{O}$
		 \end{itemize}
		\item 
		 Check $\forall (i,o) \in \mathcal{T},~\mathcal{P}(i) \stackrel{?}{=} o$
		\item 
		 Exhaustive testing is infeasible
	 \end{itemize}
	 \begin{block}{Program Testing}
	 Testing can show the existence of bugs, not their absence~\cite{ammann2008introduction}
	 \end{block}
\end{frame}


\begin{frame}{Proposed solutions: static analysis}
  % - Difference between validation and proof : Validation is an argument whose conclusion is always true, 
  %   whenever its premises are true. 
  % - Proof is a process designed to establish or discover a fact or a truth.
  \begin{itemize}
	 	\item 
		 Accurately specify the specifications of SW/HW/Em systems
		\item 
		 Prove that the implementation satisfies the specifications 
      \begin{itemize}
        \item Theorem proving: limited  to decidable theories~\cite{ouimet2007formal}
        \item Symbolic model checking: State space explosion~\cite{baier2008principles}
        \item {\em Bounded model checking over a finite domain}~\cite{biere2003bounded}
	    \end{itemize}
		\item
	     Establish correctness by construction 
      \begin{itemize}
        \item {\em Component based systems (CBS)}, compositional approaches, automated logic synthesis
	    \end{itemize}
  \end{itemize}
\end{frame}

%%%%%%	 \begin{itemize}
%%%%%%	  \item Theorem proving~\cite{ouimet2007formal}: 
%%%%%%	   \begin{itemize}
%%%%%%	    \item
%%%%%%	    Given axioms, assumptions and deduction rules
%%%%%%	    \item 
%%%%%%	    Search for theorem in all possible logical sequences 
%%%%%%	    \item 
%%%%%%	    All possible logical sequences are infinite
%%%%%%	   \end{itemize}
%%%%%%	  \item Model checking~\cite{baier2008principles}:
%%%%%%	   \begin{itemize}
%%%%%%	   \item
%%%%%%	   Symbolically compute reachable states of a transition system
%%%%%%	   \item 
%%%%%%	   Suffers from the state space explosion problem
%%%%%%	  \end{itemize}
%%%%%%	  \end{itemize}

%%\begin{frame}{Bounded Model Checking (BMC)}
%%\begin{itemize}
%%	  \item 
%%	  Bounded model checking~\cite{biere2003bounded}:
%%	  \begin{itemize}
%%	   \item
%%	   Validate that the specifications hold for a finite domain bounded by
%%	   the number of objects or transitions, the execution time, \ldots
%%	   \item
%%	   Proof as good as the provided bound
%%	   \item 
%%	   Not always possible to generate formula for SAT checking
%%	  \end{itemize}
%%\end{itemize}
%%\end{frame}

\begin{frame}{Bounded model checking (BMC) tools (1) }
  \framesubtitle{C-BMC: CNF SAT solver for C bounded model checking} 
%% Alloy Analyzer, is a solver that takes the constraints of a model 
%% and finds structures that satisfy them
%% Alloy entity, relation : For example for the file system, there are entities 
%% and relations between them (e.g. root folder, parent, files, children folders, ...)
  \begin{itemize}
  	\item
	 CBMC~\cite{clarke2004tool} 
	\begin{itemize}
	 \item
	 (C code, bound) $\rightarrow$ Conjunctive Normal Form (CNF) $\rightarrow$ Satisfiability (SAT)
	 \item
	  The size of the generated CNF grows with the program size and the bound
	 \item
	  Can generate false negatives due to unwinding assertions 
  \item
    The unwinding bound needed to prove the specifications depends on the runtime of the program (if it terminates)
	\end{itemize}
\end{itemize}
\end{frame}

\begin{frame}{Bounded model checking (BMC) tools (2) }
  \framesubtitle{Alloy: CNF SAT solver for FOL with TC bounded model checking} 
  \begin{itemize}
	\item 
 Alloy~\cite{jackson2002alloy}
	 \begin{itemize}
  \item 
	  (First Order Logic (FOL) + Transitive Closure (TC), bound) $\rightarrow$ CNF $\rightarrow$ SAT
	  \item The generated CNF formulae can be very large 
	  \item Analysis is limited to small bounds, e.g. 10 entities in a file system~\cite{fz2007relational}
	 \end{itemize}
	 \item 
  CBMC, Alloy, and other existing CNF based solutions do not scale well to large bounds
 \end{itemize}
\end{frame}

%% ABC 
\begin{frame}{ABC~\cite{brayton2010abc}}
 \begin{itemize}
  \item 
   Industrial strength sequential synthesis and verification tool
  \item 
   Takes as input an And-Inverter-Graph (AIG) representation of a sequential circuit
 \begin{itemize}
  \item Sequential circuit is a Boolean formula with memory elements
 \end{itemize}
  \item Reduces AIG with synthesis algorithms 
  \item Checks correctness of reduced AIG with proof algorithms
 \end{itemize}
\end{frame}

%% CBS design 
\begin{frame}{CBS design}
  %% Talk about cbs design and bip and what then use
  %% in terms of NuSMV and DFinder
  \begin{itemize}
   \item {\em Behavior-Interaction-Priority} (BIP) is a framework for CBS design 
   \item It allows for creating complex systems by coordinating the behavior of 
   a set of components
   \item 
    It aims at generating correct by construction code for embedded systems
%%   \item It supports three layers of modeling
%%    \begin{enumerate}
%%     \item Behavior: described by {\em Labeled Transition Systems} (LTS) 
%%     with data and C functions
%%     \item Interaction: synchronization and data transfer between components
%%     \item Priority: express scheduling policies between interactions
%%    \end{enumerate}
  \end{itemize}
\end{frame}

%% BIP verification 
\begin{frame}{BIP verification}
  %% BIP verification: NuSMV and DFinder
%  \begin{itemize}
%   \item Existing techniques for verifying BIP modes include
    \begin{itemize}
     \item DFinder~\cite{dfinder}: a tool for checking invariants and deadlock freedom of BIP systems
      \begin{itemize}
%       \item Computes invariants of the system in a compositional way
%       \item Checks for the SAT of the generated formula
       \item Does not handle data transfer between components
      \end{itemize}
     \item NuSMV~\cite{nusmv}: translate BIP systems to NuSMV models and perform model checking
      \begin{itemize}
%       \item Uses SAT and {\em Binary Decision Diagrams} (BDD) based model checking techniques
%       \item Supports {\em Computational Tree Logic} (CTL) and {\em Linear Temporal Logic} (LTL)
%       specifications
       \item Does not always scale well due the state space explosion problem 
      \end{itemize}
    \end{itemize}
%  \end{itemize}
\end{frame}


%% our work presentation starts here
\begin{frame}{Verification of C and BIP using ABC }
\begin{itemize}
 \item 
  We present two techniques with supporting tools to address the C and BIP verification problems
\begin{itemize} \item \mytool{} \item \biptool{}
 \end{itemize}
\end{itemize}
\end{frame}



\begin{frame}{\mytool }
\begin{itemize}
  \item Input
    \begin{itemize}
  \item An imperative program \Pm
  \item A precondition \Pre{} 
  \item A postcondition \Post{} written in (FOL)
  \item A bound $b$ on the data width of \Pm variables 
    \end{itemize}
  \item Translates $\Pm \models \pair{\Pre}{\Post}|_b$ into an equisatisfiable AIG
    \begin{itemize}
      \item Using program counter semantics
    \end{itemize}
  \item Uses ABC reduction and proof algorithms to reduce and check the generated AIG
\end{itemize}
\end{frame}

\begin{frame}{\mytool -- continued  }
\begin{itemize}
  \item In case ABC($\Pm \models \pair{\Pre}{\Post}|_b$) was not conclusive
    \begin{itemize}
  \item Guess or use heuristics to compute a temporal bound $T$
  \item Use BMC within ABC to check whether $\Pm \models \pair{\Pre}{\Post}|_b$ within $T$
  \item Use BMC within ABC to check whether $\Pm$ terminates within $T$
    \end{itemize}
\end{itemize}
\end{frame}


\begin{frame}{\biptool}
\begin{itemize}
 \item Input
 \begin{itemize}
  \item An input BIP system $B$
  \item An invariant $\mathcal{I}$ in FOL
  \item A bound $b$ on the width of data variables
  \end{itemize}
  \item Translates the problem $\left( B \models \mathcal{I}|_b \right)$ into an equisatisfiable AIG
  \item Uses ABC reduction and proof algorithms to reduce and check the generated AIG
\end{itemize}
\end{frame}

\section{Demo}
\begin{frame}{Binary search}
\begin{itemize}
 \item {\bf Inputs.}
 \begin{itemize}
  \item $a$: array of scalars
  \item $n$: number of elements in $a$
  \item $e$: element to search for in $a$
 \end{itemize}
 \item {\bf Output.}
  \begin{itemize}
   \item $idx$: index of $e$ in $a$ if found, $-1$ otherwise
  \end{itemize}
 \item {\bf Specification.}
  \begin{itemize}
   \item 
    $\Pre = n \geq 0 \land \forall{(i \in 0\ldots(n-2))}.~(a[i] \leq a[i+1])$
   \item 
    $\Post = (idx = -1) \implies \forall{(i \in 0\ldots n)}.~(a[i] \neq e) ~\land$ \\
    $\quad\quad(idx \neq -1) \implies (a[idx] = e)$
  \end{itemize}
\end{itemize}
\end{frame}

\begin{frame}[containsverbatim]
\frametitle{Binary search code}
\begin{Verbatim}[fontsize=\relsize{-2.5}, numbersep=4pt,numbers=left]
int binarysearch(int a [], int n, int e) {
 // @pre bs {(n >= 0) && forall(int i:[0..n-2]) {a[i] <= a[i+1]}}
   int low = 0, high = n - 1;
   while (low < high) {
     mid = (low + high) >> 1;
     if (a[mid] < el) {
       low = mid + 1;
     } else {
       high = mid;
     }
   }  
   if ((low == high) && (a[low] == el)) {
     return low;
   } 
   return -1;
   
   // @post bs {((rv == -1) -> (forall(int i:[0..n-1]) {a[i] != e}) &&
   //           ((rv != -1) -> (a[rv] == e)}
}
\end{Verbatim}
\end{frame}


\begin{frame}[c]
 \frametitle{\mytool{} demo}
 \begin{itemize}
  \item 
  \href{run:/home/mohamad/Desktop/Demo/BinarySearch.avi}{\mytool{} in action!} 
  \item 
  \href{run:/home/mohamad/Desktop/Demo/CounterEx.avi}{Check counter example!}
 \end{itemize} 
\end{frame}



%\begin{frame}{We propose \mytool}
%  \begin{itemize}
%  	\item 
%	 Given :
%	 \begin{enumerate}
%	 	\item
%		 $S$ : An imperative program implementing a functional specification.
%		\item 
%		 $P, Q$: pre/post conditions in temporal and first order logic specifying the functionality of $S$.
%		\item 
%		 Semantics: $P$ holds, then $S$ executes and completes, then $Q$ holds after the completion of $S$
%	 \end{enumerate}
%	\item 
%	 \mytool:
%	 \begin{enumerate}
%	 	\item
%		 Assumes a sequential program counter execution model
%	 	\item
%		 Generates $C$ an sequential circuit equivalent to $(P,S,Q)$. 
%		\item 
%		 Uses the synthesis and verification ABC framework~\cite{brayton2010abc} to validate $C$. 
%	 \end{enumerate}
%  \end{itemize}
%\end{frame}
%
%\begin{frame}{The program $S$}
%  \begin{itemize}
%  	\item
%	 \mytool~supports:
%	 \begin{enumerate}
%	 	\item
%		 Data: scalars, arrays, assignments, and arithmetic, and Boolean operations.
%	 	\item
%		 Control: branch, loops, and function calls including recursion.
%	 \end{enumerate}
%	\item
%	 The program counter execution model eliminates the need to unwind loops.
%  \end{itemize}
%\end{frame}
%
%\begin{frame}{The conditions $P$ and $Q$}
%  \begin{itemize}
%  	\item 
%	 The pre/post conditions are expressed in:
%	 \begin{enumerate}
%	 	\item 
%		 First Order Logic (FOL) 
%		 \begin{itemize}
%		 	\item 
%			Boolean formulas over Boolean and arithmetic operations.
%		 	\item 
%			Boolean formulas over existential and universal quantification of scalar variables.
%	     \end{itemize}
%		\item
%		 Temporal Logic : 
%		 \begin{itemize}
%		 	\item
%			 Use Sequential Extended Regular Expressions (SERE), a subset of the property specification language (PSL).
%		 \end{itemize}
%	 \end{enumerate}
%  \end{itemize}
%\end{frame}


\section{Preliminaries}
% First Order Logic
\begin{frame}{First Order Logic}
  \begin{itemize}
    \item Terms: \begin{itemize} \item constants (0,1), \item scalar variables (x,y), \item arithmetic and functional operations over terms (x+y, f(x,y))
      \end{itemize}
    \item Atomic propositions: Boolean variables (b), predicate operations over terms (f(x,y) <y-1)
    \item Propositions: 
    \begin{itemize}\item Atomic propositions, 
        \item Boolean operations over Propositions (x< y and b),
        \item Universal and existential quantifiers over scalar and Boolean variables ($\forall x,y,z. (x\le y \wedge y\le z) \implies (x \le z)$)
   \end{itemize}
  \end{itemize}  
%%  
%%  \begin{itemize}
%%  	\item
%%	 Scalars.
%%	\item 
%%	 Boolean formulas over:
%%	 \begin{enumerate}
%%	  \item 
%%	   Binary and unary arithmetic and boolean operators.
%%	  \item
%%	   Universal and Existential quantifiers of scalar variables~\cite{bradley2007calculus}.
%%	 \end{enumerate}
%%	\item e.g. $\forall{i} \in [0\dots10], i < n \imp a[i] == e$
%%  \end{itemize}
\end{frame}

\begin{frame}{AIG and sequential circuits}
\begin{definition}
  A {\em sequential circuit } is a tuple $C = \left((V,E),G,O)\right)$ where
\begin{itemize}
 \item $(V,E)$ is a directed acyclic graph 
 \item $G: V \mapsto types$ maps vertices to their types 
 \begin{itemize} \item {\em primary inputs, registers} and {\em logic gates}\end{itemize}
 \item $O \subseteq V$ primary outputs of $C$
 \item An AIG is sequential circuit such as
 \begin{itemize}
  \item $types = \set{primary~input, register, 2~bit~AND~gate, inverter}$
 \end{itemize}
\end{itemize}
\end{definition}
\end{frame}

\begin{frame}{AIG registers and gates}
\begin{itemize}
 \item An AIG register node has two fan-ins
 \begin{enumerate}
  \item initial value function (can be $true$, $false$, dont-care)
  \item a next state function
 \end{enumerate}
 \item An AIG AND gate has two input fan-ins
 \item An AIG Inverter has one input fan-in
\end{itemize}
\end{frame}

%%\begin{frame}{AIG}
%% \begin{definition}
%%  An AIG is a sequential circuit $C = \left((V,E),G,O)\right)$ such that
%% \end{definition}
%% \begin{itemize}
%%  \item $(V,E)$ is acyclic
%%  \item $types = \set{primary~input, register, 2~bit~AND~gate, inverter}$
%%  \item For simplicity, inverters are labels on the edges 
%% \end{itemize}
%%\end{frame}

\begin{frame}{Advantages of AIG encoding}
\begin{enumerate}
  \item More succinct than CNF
  \item State-holding while CNF is state free
  \begin{itemize}
   \item preserves the structure of the program
   \item removes the necessity to unroll loops
  \end{itemize}
  \item Allows the reuse of subcomponents of the AIG 
  \item There is a wide range of automatic analysis techniques for AIG not applicable to CNF
 \end{enumerate}
\end{frame}

\begin{frame}[containsverbatim]
\frametitle{Software encoding}
\begin{itemize}
 \item \thislanguage{} is a software interpretation of an AIG
\end{itemize}
\begin{Verbatim}[frame=single,fontsize=\relsize{-2.5}]
declarations of variables //registers
declarations of wires //named intermediate results, outputs

definition of wires 
  //assignment statements binding wire values to expressions

initial value definitions 
 //assignment statements binding variables to their initial value gates

while (true) {
    next state function definitions 
      //assignment statements binding variables to their next state 
      //value fucntions
}
\end{Verbatim}
\end{frame}

\begin{frame}[containsverbatim]
\frametitle{Example}
\begin{itemize}
 \item Consider the following example of a two bits counter
\end{itemize}
\begin{Verbatim}[frame=single,fontsize=\relsize{-2.5}]
// Block 1:
int x; // this represents two registers

// Block 2: skipped since no wires

// Block 3: initial value for registers
x = 0; 

// Block 4: next state computation 
while (true) {
   // this statement signals that everything inside it is done
   // at the same clock cycle
   @dotogether {
      x = (x == 0)? 1 :
          (x == 1)? 2 :
          (x == 2)? 3 :
          0;
   }
}
\end{Verbatim}
\end{frame}


\section{\mytool{}}
\subsection*{Methodology}
%% presentation for the psq tool
\begin{frame}{\mytool{}}
 \begin{itemize}
  \item \mytool{} accepts input programs and specifications written in an imperative language
  \psqlanguage
  \item Given a program \Pm and a precondition postcondition pair \pair{\Pre}{\Post},  \mytool{}:
  \begin{enumerate}
   \item resolves function calls and quantifiers
   \item translates $\Pm \models \pair{\Pre}{\Post}|_b $ into an equivalent program \Pm' in \thislanguage
   \item synthesizes \aigcircuit an equisatisfiable AIG for \Pm'
   \item calls ABC and performs synthesis and verification on \aigcircuit
  \end{enumerate}
 \end{itemize}
\end{frame}

\begin{frame}[containsverbatim]
\frametitle{\psqlanguage}
\begin{itemize}
 \item A program written in \psqlanguage~is defined as follows
\end{itemize}
\begin{Verbatim}[fontsize=\relsize{-2.0}]
program := (statement | declaration)+
declaration := variable_declaration | property_declaration 
              | function_declaration
statement := assignment_statement | selection_statement 
              | loop_statement | sync_statement

property_declaration := '@pre' precondition | '@post' postcondition

sync_statement := '@dotogether' { statement+ }

... 
\end{Verbatim}
\end{frame}

\begin{frame}[containsverbatim]
\frametitle{Array search}
  \small {
  \begin{tabular}{p{2.2in}|p{2.4in}}
\begin{Verbatim}[fontsize=\relsize{-3.5}, 
frame=topline, framesep=4mm, label=\fbox{\Pm in \psqlanguage}]
int ArraySearch(int [] a, int d, 
                   int s, int e, int n) {
@pre as { 0 <= s && s <= e && 
              e < n && n <= MAXSIZE; }
 int i = s;          // pc = pc+1
 while(i <= e) {     // pc = (i<=e)? 5:10;
   if (a[i] == d) {  // pc = (a[i]==d)? 6:8;
     break;}         // pc = 10;
   else {
     i = i+1;}       // pc = 4;
 }
 return i;           // pc = 11; rv = i;
@post as {
  ((rv >= s && rv <= e) -> a[rv] == d) xor 
  (rv == -1 -> forall(int i:[s .. e]) 
                           { a[i] != d })
} }
\end{Verbatim}
& 
%\begin{Verbatim}[fontsize=\relsize{-2}], numbersep=4pt,numbers=left]
\begin{Verbatim} [fontsize=\relsize{-3.5},
frame=topline, framesep=4mm, label=\fbox{\Pm' in \thislanguage}]
dotogether {
  preas = 0 <= s && s <= e 
            && e < n && n <= MAXSIZE;
  //initial values
  pc = 0; notdone = true; postas = true; }
// next state functions
while (true) { dotogether {  
 i = (pc == 3) ? s : (pc == 8) ? i+1 : i;
 notdone = (pc == 11) ? false : true;
 rv = (pc == 10) ? i : rv;
 pc = (pc == 0) ? 3: (pc == 3) ? 4 : 
   (pc == 4) ? ( (i <= e) ? 5 : 10 ) :
   (pc == 5) ? (a[i] == d) ? 6 : 8 ) :
   (pc == 6) ? 10 :
   (pc == 8) ? 4  : 
   (pc == 10) ? 11 : pc; }
 postas = ((rv>=s && rv<=e) -> a[rv]==d) xor
   (rv == -1 -> forall(int i:[s .. e]) 
                          { a[i]!=d }); }
\end{Verbatim}
\end{tabular}
}
\end{frame}

\begin{frame}[containsverbatim]
\frametitle{Code transformation}
\begin{itemize}
 \item \mytool{} preprocesses the program with $\rho$ 
 \item $\rho$ resolves function calls
 \begin{itemize} \item assignment statement marshalling arguments, return values, and save program counter 
   \end{itemize}
 \item $\rho$ resolves recursion
 \begin{itemize} \item increases dimensions of local variables, and simulate stack 
   \end{itemize}
 \item $\rho$ resolves quantifiers
 \begin{itemize} \item unrolls quantifiers with finite constant bounds
   \end{itemize}
 \item $\rho$ injects new variables and statements  as needed
 %that simplify the transformation
 %from \psqlanguage{} to \thislanguage
\end{itemize}
\end{frame}

\begin{frame}[containsverbatim]
\frametitle{Handling quantifiers and properties}
%\small {
  \begin{tabular}{p{2.2in}|p{2.4in}}
\begin{Verbatim}[fontsize=\relsize{-3.5}, 
frame=topline, framesep=4mm, label=\fbox{\Pm}]
int a [N], b [N], i;

@pre inv {
  // Q1:
  forall(int i:[0..N-2]) {a[i] <= a[i+1]}; }

i = 0;
while (i < N) {
 @dotogether {
  b[i] = a[(N-1)-i];
  i = i + 1; } }
  
@post inv {
  // Q2:
  forall(int i:[0..N-2]) {b[i] >= b[i+1]} }

\end{Verbatim}
& 
%\begin{Verbatim}[fontsize=\relsize{-2}], numbersep=4pt,numbers=left]
\begin{Verbatim} [fontsize=\relsize{-3.5},
frame=topline, framesep=4mm, label=\fbox{$\rho(\Pm)$}]
int a[N], b[N], i, Q1::i, Q2::i;
bool Q1::rv, Q2::rv, preinv, postinv;

@dotogether {
 Q1::i = 0; Q2::i = 0;
 Q1::rv = true; Q2::rev = true;
}
while (Q1::i <= N-2) {
 @dotogether { 
  Q1::rv = Q1::rv && (a[i] <= a[i+1])
  Q1::i = Q1::i + 1; }}
preinv = Q1::rv;

i = 0;
while (i < N) {
 @dotogether {
  b[i] = a[(N-1)-i];
  i = i + 1; } }
  
while (Q2::i <= N-2) {
 @dotogether { 
  Q2::rv = Q2::rv && (b[i] >= b[i+1])
  Q2::i = Q2::i + 1; }}
postinv = Q2::rv;
assert (preinv -> postinv);

\end{Verbatim}
\end{tabular}
%}
\end{frame}

\begin{frame}{Handling recursion}
\begin{itemize}
 \item $\rho$ simulates recursion using arrays
 \item It increases the dimension of the recursive function's local variables
 \item It injects a stack pointer variable \CodeIn{sp} to address the variables
 \item Stack pointer is incremented before recursive call and decremented 
 after it returns
\end{itemize}
\end{frame}

\begin{frame}[containsverbatim]
\frametitle{Function calls and recursion}
  \begin{tabular}{p{2.2in}|p{2.4in}}
\begin{Verbatim}[fontsize=\relsize{-3.5}, 
frame=topline, framesep=4mm, label=\fbox{\Pm}]
int mult (x, y) {
  int result;
  if (y == 1) {
    result x;
  } else {
    result = mult(x, y-1);
  }
  return result;
}

int a, b;

a = 2; b = 3;
a = mult(a, b);
\end{Verbatim}
& 
%\begin{Verbatim}[fontsize=\relsize{-2}], numbersep=4pt,numbers=left]
\begin{Verbatim} [fontsize=\relsize{-3.5},
frame=topline, framesep=4mm, label=\fbox{$\rho(\Pm)$}]
int sp;
sp = 0;

int mult(int x [], int y []) {
  int result [];
  if (y[sp] == 1) {
    result[sp] = x[sp]; 
  } else {
    x[sp+1] = x[sp];
    y[sp+1] = y[sp] - 1;
    sp = sp + 1;
    call_func(mult);
    sp = sp - 1;
    result[sp] = mult::rv[sp+1];
  }
  mult::rv[sp] = result[sp];
}

int a, b;
a = 2; b = 3;
x[0] = 2;
y[0] = 3;
call_func(mult);
a = mult::rv[0];
\end{Verbatim}
\end{tabular}
\end{frame}

\begin{frame}{Transformation to \thislanguage}
 \begin{itemize}
  \item \mytool{} uses {\em program counter} semantics to 
  synthesize AIGs from \psqlanguage{} programs
  \item Each statement $s$ is assigned a unique label ($label(s)$)
  \item Assignment statements alter the data flow of the program 
  \item Selection, loop and function call statements alter the control
  flow (program counter) of the program
  \item \mytool{} adds a program counter variable (\cci{pc}) to the code
  and uses it for control flow
 \end{itemize}
\end{frame}

\begin{frame}{Transformation algorithm}
\begin{itemize}
 \item To generate \thislanguage{} program \Pm' given a \psqlanguage{} program \Pm
 \begin{enumerate}
  \item The wire and variable declaration block in \Pm' is directly mapped from the 
  declarations in \Pm
  \item The wire definitions block in \Pm' is directly mapped from the definitions 
  in \Pm
  \item The variable initialization block in \Pm' is automatically generated
  for variables used before assignment (i.e. free variables)
 \end{enumerate}
\end{itemize}
\end{frame}

\begin{frame}[containsverbatim]
\frametitle{Data Transformation algorithm}
\begin{lstlisting}
foreach target x // variable (v) or array access (a[exp])
  $next\_list$ += $``x :="$ 
  foreach assignment statement $s = x := exp$
   $next\_list$ += $``(pc == label(s))? exp :"$ 
  endfor
  $next\_list$ += $``x;"$
endfor
\end{lstlisting}
\end{frame}

\begin{frame}[containsverbatim]
\frametitle{Control Transformation}
\begin{lstlisting}
$n_{pc} = ``pc =" $
foreach statement $s$
  if (is_conditional_statement ($s$))
    $n_{pc}$ += $``(pc==label(s))?~ (condition(s)?~then(s) : else(s)) :"$
  elsif (is_loop_statement ($s$))
    $n_{pc}$ += $``(pc==label(s))?~ (condition(s)?~body(s):next(s)) :"$ 
    $n_{pc}$ += $``(pc==last\_statement\_body(s))?~label(s) :"$
  elsif (is_function_call($s$))
    $f_d :=$ function_declaration_of $($s$)$
    $n_{pc}$ += $``(pc==label($s$))?~ (body(f_d)) :"$
  elsif (is_return_statement ($s$))
    $f_d :=$ function_declaration_of $($s$)$
    $n_{pc}$ += $``(pc==label(s))?~(return(call(f_d))) :"$
  else 
    $n_{pc}$ += $``(pc==label(s))?~(next(s)) :"$
  endif
endfor
\end{lstlisting}
\end{frame}

\begin{frame}[containsverbatim]
\frametitle{Post processing}
\begin{itemize}
 \item Post processing step to eliminate non-constant array access terms
\end{itemize}
  \begin{tabular}{p{1.2in}|p{3.4in}}
\begin{Verbatim}[fontsize=\relsize{-3.5}, 
frame=topline, framesep=4mm, label=\fbox{\Pm}]
 ....
 l1: a[i] = x + 1;
 ...
 
 l2: x = a[i] + 2;
\end{Verbatim}
& 
%\begin{Verbatim}[fontsize=\relsize{-2}], numbersep=4pt,numbers=left]
\begin{Verbatim} [fontsize=\relsize{-3.5},
frame=topline, framesep=4mm, label=\fbox{\Pm'}]
a[0] = ((pc == l1) && (i == 0))? (x+1) : a[0];
a[1] = ((pc == l1) && (i == 1))? (x+1) : a[1];
a[2] = ((pc == l1) && (i == 2))? (x+1) : a[2];
...
a[N-1] = ((pc == l1) && (i == N-1))? (x+1) : a[N-1];

x = (pc == l2)? (i == 0)? a[0] :
                (i == 1)? a[1] :
                (i == 2)? a[2] :
                ...
                (i == N-2)? a[N-2] : a[N-1];
\end{Verbatim}
\end{tabular}
\end{frame}

\begin{frame}{Finally }
\begin{itemize}
 \item Translation of variables and operations into bit-vector registers and operations
   \begin{itemize}
     \item \cci{initial value definitions} map to register initial value gates
     \item \cci{next state function definitions} map to register next state gates
\end{itemize}
\end{itemize}
\end{frame}

\subsection{Results}
\begin{frame}{Comparison with CBMC}
\begin{itemize}
 \item $\Pm_1$: linear search algorithm, $\Pm_2$: binary search algorithm
\end{itemize}
\begin{table}
\centering
\footnotesize{
\begin{tabular}{|c|c|c|c|c|c|c|c|c|}
\cline{3-9}
\multicolumn{1}{l}{} & \multicolumn{1}{c}{} &  \multicolumn{2}{|c|}{CBMC formula size} & \multicolumn{3}{l|}{\mytool{} AIG size} & \multicolumn{2}{c|}{Time (s)} \\ \hline
\Pm & size  & Vars  & Clauses & lat & and & lev & \mytool & CBMC \\ \hline
$\Pm_1$ & 3 & 2416 & 6784 & 41 & 313 & 15 & 4.36 & 0.016 \\ \hline
$\Pm_1$ & 7 & 4612 & 15008 & 68 & 568 & 19 & 12.4 & 722.4 \\ \hline
$\Pm_1$ & 15 & 9112 & 34496 & 119 & 1116 & 21 & 16.87 & {\color{red} Time-out} \\ \hline
$\Pm_1$ & 31 & 18332 & 84928 & 226 & 2346 & 24 & 33.67 & {\color{red} Time-out} \\ \hline
$\Pm_1$ & 63 & 37216 & 230208 & 461 & 5100 & 26 & 99.64 & {\color{red} Time-out} \\ \hline
$\Pm_1$ & 127 & 75876 & 695616 & 984 & 11315 & 28 & 396.98 & {\color{red} Time-out} \\ \hline
\hline
$\Pm_2$ & 3 & 6503 & 24533 & 56 & 55 & 19 & 1.04 & 0.085 \\ \hline
$\Pm_2$ & 7 & 16172 & 68130 & 83 & 850 & 17 & 1.47 & 1.91 \\ \hline
$\Pm_2$ & 15 & 42461 & 197223 & 143 & 1943 & 20 & 27.69 & 38.493 \\ \hline
$\Pm_2$ & 63 & 390623 & 2133649 & 529 & 9052 & 25 & 1152.22 & {\color{red} Time-out} \\ \hline
\end{tabular}}
\end{table}
\end{frame}

\begin{frame}{Array algorithms}
\begin{table}
\tiny{
\centering
\begin{tabular}{|c|c|c|c|c|c|c|c|c|c|c|}
\cline{3-10}
\multicolumn{1}{c}{} & \multicolumn{1}{c}{} & \multicolumn{3}{|c|}{Before reduction} & \multicolumn{3}{c|}{After reduction} & \multicolumn{2}{c|}{Time (s)} & \multicolumn{1}{c}{}  \\ \hline
\Pm & $b$ & lat & and & lev & lat & and & lev & Ver. & Total & Check \\ \hline
ls & 2 & 86 & 719 & 24 & 41 & 313 & 15 & 0.33 & 4.36 & \checkmark \\ \hline
ls & 3 & 118 & 1064 & 27 & 68 & 568 & 19 & 3.89 & 12.4 & \checkmark \\ \hline
ls & 4 & 174 & 1781 & 30 & 119 & 1116 & 21 & 2.41 & 16.87 & \checkmark \\ \hline
ls & 5 & 286 & 3362 & 45 & 226 & 2346 & 24 & 1.43 & 33.67 & \checkmark \\ \hline
ls & 6 & 526 & 6895 & 78 & 461 & 5100 & 26 & 4.57 & 99.64 & \checkmark \\ \hline
ls & 7 & 1054 & 14780 & 143 & 984 & 11315 & 28 & 21.32 & 396.981 & \checkmark \\ \hline
ls & 8 & 4798 & 70742 & 529 & 4718 & 55364 & 33 & 682.11 & 8022.11 & \checkmark \\ \hline
\hline
bs & 3 & 94 & 879 & 30 & 56 & 555 & 19 & 0.11 & 1.04 & \checkmark \\ \hline
bs & 4 & 151 & 1832 & 42 & 83 & 850 & 17 & 0.54 & 1.47 & \checkmark \\ \hline
bs & 5 & 268 & 5185 & 62 & 143 & 1943 & 20 & 25.42 & 27.69 & \checkmark \\ \hline
\hline
bsort & 2 & 114 & 1198 & 29 & 44 & 393 & 16 & 0.29 & 5.79 & \checkmark \\ \hline
bsort & 3 & 169 & 2218 & 35 & 68 & 885 & 20 & 17.1 & 31.09 & \checkmark \\ \hline
bsort & 4 & 276 & 5607 & 47 & 117 & 2106 & 22 & 1390.25 & 1426.98 & N/A \\ \hline
\hline
ss & 2 & 112 & 1208 & 27 & 43 & 427 & 15 & 0.53 & 5.81 & \checkmark \\ \hline
ss & 3 & 167 & 2239 & 35 & 69 & 949 & 19 & 209.37 & 223.54 & \checkmark \\ \hline
ss & 4 & 280 & 5676 & 47 & 125 & 2236 & 22 & 1800 & 1852.76 & N/A \\ \hline
\hline
a-p & 2 & 110 & 1896 & 33 & 57 & 689 & 19 & 0.87 & 2.79 & \checkmark \\ \hline
a-p & 3 & 147 & 2509 & 34 & 87 & 1174 & 24 & 93.47 & 97.56 & \checkmark \\ \hline
a-p & 4 & 208 & 3829 & 38 & 141 & 2419 & 29 & 2127 & 2135.7 & N/A \\ \hline
\end{tabular}
}
\end{table}
\end{frame}

\begin{frame}{List algorithms}
\begin{table}
\footnotesize{
\centering
\begin{tabular}{|c|c|c|c|c|c|c|c|c|c|c|}
\cline{3-10}
\multicolumn{1}{c}{} & \multicolumn{1}{c}{} & \multicolumn{3}{|c|}{Before reduction} & \multicolumn{3}{c|}{After reduction} & \multicolumn{2}{c|}{Time (s)} & \multicolumn{1}{c}{}  \\ \hline
\Pm & $b$ & lat & and & lev & lat & and & lev & Ver. & Total & Check \\ \hline
lli & 2 & 237 & 4310 & 38 & 98 & 871 & 19 & 109.63 & 118.89 & \checkmark \\ \hline
lli & 3 & 344 & 6117 & 41 & 179 & 1693 & 26 & 1800 & 1811 & N/A \\ \hline
\hline
llr & 2 & 197 & 2906 & 33 & 84 & 722 & 21 & 47.72 & 71.3829 & \checkmark \\ \hline
llr & 3 & 293 & 4454 & 39 & 157 & 1387 & 25 & 1800.15 & 1830.21 & N/A \\ \hline
\end{tabular}
}
\end{table}
\begin{itemize}
 \item On average $43\%$ reduction in number of latches, $53\%$ in number of AND gates,
 $47\%$ reduction in number of logic levels
\end{itemize}
\end{frame}

\begin{frame}{SVCompetition'13 (ControlFlowInteger)}
\begin{itemize}
 \item Ranked closely behind silver tool
\end{itemize}
\begin{table}
 \centering
 \footnotesize{
\begin{tabular}{|c|c|c|c|c|c|}
\hline
\Pm & status & time (s) & gold  & silver & bronze \\ \hline
locks\_5\_safe & safe & 0.28 & 0.4 & 1.2 & 1.3 \\ \hline
locks\_6\_safe & safe & 0.32 & 0.43 & 1.3 & 1.6 \\ \hline
locks\_7\_safe & safe & 0.52 & 0.4 & 1.3 & 1.9 \\ \hline
locks\_8\_safe & safe & 0.62 & 0.4 & 1.3 & 2.3 \\ \hline
locks\_9\_safe & safe & 1.05 & 0.39 & 1.3 & 3.5 \\ \hline
locks\_10\_safe & safe & 0.88 & 0.4 & 1.3 & 6.4 \\ \hline
locks\_11\_safe & safe & 3.42 & 0.39 & 1.4 & 24 \\ \hline
locks\_12\_safe & safe & 4.21 & 0.4 & 1.4 & 110 \\ \hline
locks\_13\_safe & safe & 4.18 & 0.43 & 1.4 & 100 \\ \hline
locks\_14\_safe & safe & 5.9 & 0.4 & 1.4 & 100 \\ \hline
locks\_15\_safe & safe & 6.83 & 0.4 & 1.4 & 100 \\ \hline
locks\_14\_unsafe & unsafe & 0.99 & 3.2 & 1.6 & 1.8 \\ \hline
locks\_15\_unsafe & unsafe & 0.91 & 4.2 & 1.6 & 1.8 \\ \hline
\hline
Average & N/A & 2.316 & 0.911 & 1.377 & 34.969 \\ \hline
\end{tabular}}
\end{table}
\end{frame}

\begin{frame}{SVCompetition'13 (Loops)}
\begin{itemize}
 \item Solved problems correctly that silver and bronze got wrong
 \item Ranked closely behind gold tool
\end{itemize}
\begin{table}
 \centering
 \footnotesize{
\begin{tabular}{|c|c|c|c|c|c|}
\hline
\Pm & status & time (s) & gold  & silver & bronze \\ \hline
array\_safe & safe & 0.116 & 0.05 & 0.15 & 0.48 \\ \hline
array\_unsafe & unsafe & 0.091 & 0.09 & 0.47 & 0.46 \\ \hline
count\_up\_down\_s & safe & 0.137 & 0.28 & {\bf \color{red}{450}} & 0.39 \\ \hline
count\_up\_down\_u & unsafe & 0.095 & 0.04 & 0.15 & 0.43 \\ \hline
invert\_string\_s & safe & 1.808 & 0.03 & 14 & {\bf \color{red}{0.56}} \\ \hline
invert\_string\_u & unsafe & 0.266 & 0.11 & 10 & 0.68 \\ \hline
\hline
Average & N/A & 0.419 & 0.100 & 79.128 & 0.500 \\ \hline
\end{tabular}}
\end{table}
\end{frame}


\section{\biptool{}}
%% bip tool presentation
\subsection*{Methodology}

\begin{frame}{\biptool}
\begin{itemize}
 \item Given a BIP system \Pm
 \begin{itemize}
  \item an invariant $\mathcal{I}$
  \item a bound $b$ on the data width of variables 
 \end{itemize}
 \item \biptool{} translates $\Pm \models \mathcal{I}|_b$ into an  AIG problem
 \item It uses ABC to reduce and check the AIG problem
 \item Translation is done using an intermediate \thislanguage program \aigcircuit
\end{itemize}
\end{frame}

\begin{frame}{BIP by example}
\begin{figure}
\centering
\resizebox{0.8\textwidth}{!}{
  \input{figures/traffic.pdftex_t}
}
\end{figure}
\end{frame}

\begin{frame}{BIP in short}
 \begin{itemize}
   \item A BIP design is a collection of components connected with interactions
   \item An interaction is a connection between component ports, 
   \begin{itemize} \item labeled with guards, and data transfer actions (assignment statements)\end{itemize}
   \item Each component has a finite state machine 
   \begin{itemize} \item transitions are labeled with ports, guards, and actions \end{itemize} 
 \end{itemize}
\end{frame}

\begin{frame}{BIP Execution model}
   The progress of the BIP model is divided into two steps:
    \begin{enumerate}
      \item First step \begin{itemize}
          \item computes the enabled interactions, \item selects non-conflicting active interactions, 
          and \item executes interaction data transfer actions\end{itemize}
	 \item Second step 
  \begin{itemize}  \item executes the atomic component transitions involved in the selected interactions. \end{itemize}
	\end{enumerate} 
\end{frame}

\begin{frame}{Example}
\begin{itemize}
\item
 Assume that we are in the following state:
 $$(t = 9, n = 10, timer\_location = s_0) \land (light\_location = G, m = 3)$$
 \end{itemize}
 \begin{columns}[c]
  \column{0.5\textwidth}
  \begin{tabular}{|c|c|c|c|c|c|}
	  \hline
	  Step & $t$ & $n$ & $tl$ & $ll$ & $m$ \\ 
	  \hline
	     $s$ & $9$ & $10$ & $s_0$ & $G$ & $3$ \\
	     \hline
	     $s+1$ & $10$ & $10$ & $s_0$ & $G$ & $3$ \\
	  \hline
	     $s+2$ & $10$ & $3$ & $s_0$ & $G$ & $3$ \\
	  \hline
	  $s+3$ & $0$ & $3$ & $s_0$ & $Y$ & $5$ \\
	  \hline
  \end{tabular}
   \column{0.5\textwidth}
   \resizebox{0.8\textwidth}{!}{
  \input{figures/traffic.pdftex_t}
}
 \end{columns}
\end{frame}

\begin{frame}{BIP components}
\begin{itemize}
\item An atomic component $B$ is a tuple $\left(X,P,L,T\right)$ where 
\begin{itemize}
 \item $X$: set of data variables, e.g. \{$t$, $n$, \ldots\}
 \item $P$: set of ports, e.g. \{done, timer\}
 \begin{itemize}
  \item Each port can have a set of variables $X_p \subseteq X$
 \end{itemize}
 \item $L$: set of control locations, e.g \{$s_0, G, R, Y$\}
 \item $T \subset L \times P \times \mathbb{B}[X] \times Exp[X] \times L$: transition relation
 \begin{itemize}
  \item $\mathbb{B}[X]$: set of Boolean guards over $X$, e.g. $x \geq n$
  \item $f_\tau \in Exp[X]$: partial mapping $x \in X \mapsto expression~ f_\tau^{x}(X)$, e.g.
  $t = 0$
 \end{itemize}
\end{itemize}
\item Using dot notation: $B_i.P$ the set of ports of atomic component $B_i$
\end{itemize}
\end{frame}

\begin{frame}{BIP interaction}
\begin{itemize}
 \item An interaction $a$ is a tuple $(P_a, G_a, F_a)$ where
 \begin{itemize}
  \item $P_a \subseteq \cup_{i=1}^{N} B_i.P$: non empty set of ports 
  \begin{itemize}
   \item $\forall{i}:1 \leq i \leq N	: |B_i.P \cap P_a| \leq 1$
  \end{itemize}
  \item $G_a$: Boolean guard over variables $X_a = \bigcup_{p \in P_a} X_p$
  \item $F_a$: Update function $X_a \mapsto X_a$
 \end{itemize}
\end{itemize}
\end{frame}

\begin{frame}[containsverbatim]{BIP to \thislanguage: Declarations}
  \begin{lstlisting}[basicstyle=\scriptsize\ttfamily]
// |J|: number of interactions, |I|: number of atomic components
wire bool ie[|J|]; // interaction enablement 
wire bool ip[|J|]; // interaction priority
wire bool is[|J|]; // interaction selected

wire int selector; // non-deterministic priority selector
bool cycle; 

foreach $i \in [1..|I|]$
  int $B_i.\ell$;
  foreach $j \in [1..|B_i.P|]$ 
    wire bool $B_i.p_j.e$; // port enablement
    wire bool $B_i.p_j.s$; // port selected
  endfor 
    
  foreach $j \in [1..|B_i.X|]$ 
    int $B_i.x_j$; // variable registers
  endfor
endfor
  \end{lstlisting}
% \end{columns}
\end{frame}

\begin{frame}[containsverbatim]{BIP to \thislanguage: Interactions}
\begin{itemize}
 \item {\small Assume interaction0 = $(true,\set{timer},\emptyset)$
 and interaction1 = $(true, \set{B_0.done, B_1.done}, B_0.done.n = B_1.done.m)$}
\end{itemize}
\begin{columns}
 \column{0.5\textwidth}
 \begin{lstlisting}
ie[0] = $true \land B_0.timer.e$
ie[1] = $true \land B_0.done.e \land B_1.done.e$

ip[0] = $ie[0] \land \lnot ie[1]$
ip[1] = $ie[1]$

is[0] = $ip[0] \land ((selector == 0) \lor \lnot ip[1])$
is[1] = $ip[1] \land ((selector == 1) \lor true)$
 \end{lstlisting}
 \column{0.5\textwidth}
    \resizebox{0.9\textwidth}{!}{
  \input{figures/traffic.pdftex_t}
}
\end{columns}
\end{frame}

\begin{frame}[containsverbatim]{BIP to \thislanguage: Ports}
\begin{columns}
 \column{0.5\textwidth}
 \begin{lstlisting}
$B_0.timer.e = (B_0.\ell == s_0) \land (t < n)$
$B_0.done.e = (B_0.\ell == s_0) \land (t \geq n)$
$B_1.done.e = true \land ((B_1.\ell == G)$ 
             $||(B_1.\ell == R)$
             $||(B_1.\ell == Y))$
             
$B_0.timer.s = is[0]$
$B_0.done.s = is[1]$
$B_0.done.s = is[1]$
 \end{lstlisting}
 \column{0.5\textwidth}
    \resizebox{0.9\textwidth}{!}{
  \input{figures/traffic.pdftex_t}
}
\end{columns}
\end{frame}

\begin{frame}[containsverbatim]{BIP to \thislanguage: Initialization}
\begin{columns}
 \column{0.5\textwidth}
 \begin{lstlisting}
$B_0.\ell = s_1$
$t = 0$

$B_1.\ell = G$
$n = 10$
$m = 3$
 \end{lstlisting}
 \column{0.5\textwidth}
    \resizebox{0.9\textwidth}{!}{
  \input{figures/traffic.pdftex_t}
}
\end{columns}
\end{frame}

\begin{frame}[containsverbatim]{BIP to \thislanguage: Interaction execution}
\begin{columns}
 \column{0.5\textwidth}
 \begin{lstlisting}
$n = (cycle == 0)? (is[1]? m : n) : n$
 \end{lstlisting}
 \column{0.5\textwidth}
    \resizebox{0.9\textwidth}{!}{
  \input{figures/traffic.pdftex_t}
}
\end{columns}
\end{frame}

\begin{frame}[containsverbatim]{BIP to \thislanguage: Transition execution}
\begin{columns}
 \column{0.5\textwidth}
 \begin{lstlisting}
$t = (cycle == 1)?$ 
   $((B_0.timer.s)? (t + 1) :$
   $((B_0.done.s)? 0 : t)) : t$;
$m = (cycle == 1)?$
   $((B_1.done.s \land B_1.\ell == G)? 5 :$
   $((B_1.done.s \land B_1.\ell == Y)? 10 :$
   $((B_1.done.s \land B_1.\ell == R)? 3 : m))) : m$
$B_0.\ell = (cycle == 1)?$
   $((B_0.timer.s)? s_0:$
   $((B_0.done.s)? s_0:B_0.\ell)):B_0.\ell$
$B_1.\ell = (cycle == 1)?$
   $((B_1.done.s \land B_1.\ell == G)? Y :$
   $((B_1.done.s \land B_1.\ell == Y)? R :$
   $((B_1.done.s \land B_1.\ell == R)? G : B_1.\ell))) : B_1.\ell$
 \end{lstlisting}
 \column{0.5\textwidth}
    \resizebox{0.9\textwidth}{!}{
  \input{figures/traffic.pdftex_t}
}
\end{columns}
\end{frame}

%%%%%%%%%%%%%%%%%%%%%%%%%%%%%%%%%%%%%%%%%%%%%%%%%%%%%%%%%%%%%%%%%%%%%%%%%%%%%%%%%%%%%%%%%%%%%%%%%%
%%%%%%%%%%%%%%%%%%%%%%%%%%%%%%%%%%%%%%%%%%%%%%%%%%%%%%%%%%%%%%%%%%%%%%%%%%%%%%%%%%%%%%%%%%%%%%%%%%
%%%%%%%%%%%%%%%%%%%%%%%%%%%%%%%%%%%%%%%%%%%%%%%%%%%%%%%%%%%%%%%%%%%%%%%%%%%%%%%%%%%%%%%%%%%%%%%%%%
%%%%%%%%%%%%%%%%%%%%%%%%%%%%%%%%%%%%%%%%%%%%%%%%%%%%%%%%%%%%%%%%%%%%%%%%%%%%%%%%%%%%%%%%%%%%%%%%%%
\subsection{Results}
\begin{frame}{Evaluation}
\begin{itemize}
 \item We evaluated \biptool{} against two benchmarks
 \begin{enumerate}
  \item An Automatic Teller Machine (ATM) system~\cite{atm}
  \item The Quorum consensus protocol~\cite{guerraoui2012speculative}
 \end{enumerate}
\end{itemize}
\end{frame}

\begin{frame}{ATM system}
\begin{itemize}
 \item Composed of 3 components: User, ATM and Bank
 \item User enter card and secret code before timer expires
 \item The ATM 
 \begin{enumerate}
  \item checks for correctness of the entered code and ejects card if wrong
  \item waits for user to enter amount of money 
  \item checks with the bank if transaction is valid and transfers money
  \item eject card on completed or failed transaction
  \end{enumerate}
 \item We assume the presence of single bank and multiple ATMs and users
 \item We need to check the deadlock freedom of such a system 	
\end{itemize}
\end{frame}

\begin{frame}{ATM results}
\begin{itemize}
 \item \biptool{} outperforms NuSMV as the number of users and ATMS increases
\end{itemize}
\begin{table}
\centering
\small{
\begin{tabular}{|c|c|c|c|c|c|c|c|c|c|}
\cline {2-10}
\multicolumn{1}{c|}{} &  \multicolumn{3}{c|}{Original} & \multicolumn{3}{c|}{After reduction} &  \multicolumn{3}{c|}{Time(s)} \\ \hline
ATMs & lat & and & lev & lat & and & lev & Ver. & Total& NuSMV \\ \hline
2 & 78 & 2308 & 125 & 37 & 552 & 25 & 21.83 & 26.1 & 1.4\\ \hline
3 & 102 & 3689 & 197 & 50 & 804 & 29 & 32.65 & 38.87 & 142.6 \\ \hline
4 & 146 & 5669 & 234 & 63 & 1036 & 29 & 590  & 597 & 3361 \\ \hline
\end{tabular}}
\end{table}
\end{frame}

\begin{frame}{The Quorum protocol}
\begin{itemize}
 \item Consensus protocols between multiple clients to decide on a value
 \item System is composed of several communicating clients and servers
 \item Clients suggest values and wait for acceptance notifications from servers
 \item All clients converge to a single decided value under perfect channel conditions
\end{itemize}
\end{frame}

\begin{frame}{Quorum results}
\begin{itemize}
 \item \biptool{} verified the system satisfies its invariant
 for $2$ servers and $4$ clients while NuSMV failed
 \item \biptool{} found defects in the system when NuSMV failed
\end{itemize}
\begin{table}
\centering
\small{
\begin{tabular}{|c|c|c|c|c|c|c|c|c|c|}
\cline{2-9}
\multicolumn{1}{c|}{} & \multicolumn{ 3}{c|}{Original} & \multicolumn{ 3}{c|}{After reduction} & \multicolumn{ 2}{c|}{Time (s)} & \multicolumn{1}{l}{} \\ \hline
Design & lat & and & lev & lat & and & lev & Ver. & Tot. & Decision \\ \hline
2-2-v & 264 & 3614 & 105 & 66 & 641 & 29 & 240.6 & 245 & $\checkmark$\\ \hline
2-2-d & 264 & 3508 & 101 & 65 & 923 & 51 & 0.11 & 0.78 & $\chi$\\ \hline
4-2-v & 390 & 6453 & 151 & 117 & 1170 & 30 & \multicolumn{2}{c|}{58 hours}& $\checkmark$\\ \hline
4-2-d & 390 & 6305 & 145 & 117 & 1129 & 50 & 0.24 & 0.31 & $\chi$ \\ \hline
\end{tabular}}
\end{table}
\end{frame}


\section{Conclusion}
%% conclusion
\begin{frame}{Conclusion}
\begin{itemize}
\item Presented \mytool{} and \biptool{}
\item \mytool{}
\begin{itemize}
 \item takes an imperative program \Pm, with FOL precondition \Pre{} and postcondition \Post{}
 \item translates them into an equisatisfiable AIG \aigcircuit
 \item uses the ABC AIG solver to check that \Pm satisfies \pair{\Pre}{\Post} under a bound $b$ on
 the data width of variables
 \item proved able to reach higher bounds than CBMC and ranked amongst top
 three tool in SVComp'13
\end{itemize}
\end{itemize}
\end{frame}

\begin{frame}{Conclusion}
\begin{itemize}
\item \biptool{}
\begin{itemize}
 \item takes a BIP system \Pm with an invariant $\mathcal{I}$
 \item translates them into an equisatisfiable AIG \aigcircuit
 \item uses ABC to check that \Pm satisfies $\mathcal{I}$ under a bound $b$ on the
 data width of variables
 \item outperforms NuSMV on the verification of an ATM system and the Quorum protocol 
\end{itemize}
\end{itemize}
\end{frame}

\begin{frame}{Future work}
 \begin{itemize}
  \item Add support for temporal properties expressed in
  Linear Temporal Logic (LTL)
  \item Optimize the BIP execution engine by executing interactions and 
  component transitions in parallel
 \end{itemize}
\end{frame}

\begin{frame}{Thank you!}
\begin{figure}
 \centering
 \resizebox{0.81\textwidth}{!}{
   \includegraphics{figures/bye} 
 }
\end{figure}
\end{frame}


%% bibliography
\section*{References}
\begin{frame}[allowframebreaks]{References}
\bibliographystyle{alpha}
\bibliography{references/refs}
\end{frame}


%%%% Backup slides
\section*{Backup slides}
\begin{frame}[c]
  \LARGE{BACKUP SLIDES}
\end{frame}

\begin{frame}{AIG Example}
 \begin{exampleblock}{Or gate}
  Consider the formula $F := (a \lor b)$ where $a$ and $b$ are primary inputs
 \end{exampleblock}
 \begin{itemize}
   \item We can write $F$ as $\lnot ( \lnot a \land \lnot b )$
 \end{itemize}
 \begin{center}
 	\resizebox{0.2\textwidth}{!}{
		 \input{figures/aig_ex.pdftex_t}
 	}
 \end{center}
\end{frame}

\begin{frame}{ABC reduction}
\scriptsize{
\begin{tabular} {|p{2.5cm}|p{6.5cm}|l|}
\hline
\centering{{\bf Technique}} & {\bf Description} & {\bf ABC command} \\
\hline
Balancing & Balancing reduces the number of AIG levels by applying associativity 
transformations~\cite{brayton2010abc} & \cci{balance} \\
\hline
Structural Register Sweep (SRS) & SRS reduces the number of registers in the circuit
by eliminating stuck-at-constant registers~\cite{mishchenko2008scalable} & \cci{scl -l} \\
\hline
Signal Correspondence (Scorr) & Scorr computes a set of classes of sequentially-equivalent
nodes using $k$-step induction~\cite{mishchenko2008scalable} & \cci{ssweep} \\
\hline
Rewriting & AIG rewriting iteratively selects and replaces 
rooted AIG subgraphs with smaller pre-computed subgraphs in order to reduce the size of 
the AIG~\cite{bjesse2004dag} & \cci{rewrite} \\
\hline
Refactoring & AIG refactoring is a variation of AIG rewriting that uses a heuristic
algorithm to compute one large cut for each AIG node, then replaces the structure
of these cuts with a factored form if an improvement is observable~\cite{mishchenko2006dag} & \cci{refactor}  \\
\hline
Retiming & Retiming aims at manipulating registers over 
combinational nodes in a given logic network, while maintaining the output 
functionality and logic structure~\cite{hurst2007fast} & \cci{retime}\\
\hline
\end{tabular}}
\end{frame}

\begin{frame}
\scriptsize{
\begin{tabular} {|p{2.5cm}|p{6.5cm}|l|}
\hline
\centering{{\bf Technique}} & {\bf Description} & {\bf ABC command} \\
\hline
%-- Verification -- %
Temporal Induction & Temporal induction uses SAT solvers to carry simple and k-step induction proofs
over the time steps of the sequential circuit~\cite{een2003temporal} & \cci{ind} \\
\hline
Interpolation & Interpolation-based algorithms aim at finding interpolants 
in order to derive an over-approximation of the reachable states of the
AIG network with respect to the property~\cite{amla2005analysis} & \cci{int} \\
\hline
Property Directed Reachability (Pdr) & Pdr tries to prove that 
there is no transition from an initial state of the AIG to a bad state~\cite{een2011efficient} & \cci{pdr} \\
\hline
\end{tabular}}
\end{frame}

