\mytool~targets the verification of imperative programs annotated with 
precondition and postcondition specifications. An imperative program 
is a sequence of instructions that describes in full details the 
steps that the execution unit must take to accurately 
implement the required functionality. In order to verify 
that a program accurately implements its functionality, the developer
provides a set of formal specifications in the form of 
a precondition-postcondition pair. A precondition is a FOL formula over the 
the program's inputs that specifies which combination of inputs are acceptable; 
i.e. under which inputs the program is expected to work. 
A postcondition is a FOL formula over the program's inputs 
and outputs that defines the program's expected output. The
postcondition relates the program's outputs to its inputs~\cite{bradley2007calculus}. 

Given a program \Pm, 
a precondition and postcondition pair \pair{\Pre}{\Post}, 
and a bound $b$ on the domain of \Pm~and its variables,
\mytool~checks whether \Pm satisfies its specifications 
($\Pm \models \pair{\Pre}{\Post}|_{b}$); i.e. when the bounded inputs 
of \Pm satisfy \Pre, the outputs of \Pm must necessarily satisfy \Post.
\mytool~accepts programs written an imperative language, \psqlanguage, 
a subset of C++ extended with support for FOL and a block synchronization 
construct, the \cci{do\_together} block.
The tool then translates the problem ($\Pm \models \pair{\Pre}{\Post}|_{b}$)
into an equisatisfiable AIG using a program counter encoding. 
The generated AIG has a single output $o$ that is set to $1$ iff 
\Pm violates its specifications. The tool then uses the ABC
sequential AIG model checker~\cite{brayton2010abc} to check that $o$ is never set to $1$. 
If the check is successful, \Pm satisfies its specifications. Otherwise, \mytool~
returns the violating trace (i.e. the counterexample) to the user for debugging 
of \Pm.

%% illustrative example 
%\section{Illustrative Example} \label{chap:c2aig:sec:illustrative}
\begin{figure}[tb]
  \centering
  \small {
  \begin{tabular}{p{2.5in}|p{.01in}p{2.9in}}
\begin{Verbatim}[fontsize=\relsize{-2.5}, numbersep=4pt,numbers=left,
frame=topline, framesep=4mm, label=\fbox{(a) Array search program}]
int ArraySearch(int [] a, int d, int s, int e, int n) {
@pre as { 0 <= s && s <= e && e < n && n <= MAXSIZE; }
  int i = s;          // pc = pc+1
  while(i <= e) {     // pc = (i <= e) ? 5 : 10;
    if (a[i] == d) {  // pc = (a[i] == d) ? 6 : 8;
      break;}         // pc = 10;
    else {
      i = i+1;}       // pc = 4;
  }
  return i;           // pc = 11; rv = i;
@post as {
  ((rv >= s && rv <= e) -> a[rv] == d) xor 
  (rv == -1 -> forall(int i:[s .. e]) { a[i] != d })
} }
\end{Verbatim}
& &
%\begin{Verbatim}[fontsize=\relsize{-2}], numbersep=4pt,numbers=left]
\begin{Verbatim} [fontsize=\relsize{-2.5}, numbersep=4pt,numbers=left,
frame=topline, framesep=4mm, label=\fbox{(b) Program with program counter}]
dotogether {
  preas = 0 <= s && s <= e && e < n && n <= MAXSIZE;
  //initial values
  pc = 0; notdone = true; postas = true; }
while (notdone) { dotogether {   // next state functions
  i = (pc == 3) ? s : (pc == 8) ? i+1 : i;
  notdone = (pc == 11) ? false : true;
  rv = (pc == 10) ? i : rv;
  pc = (pc == 0) ? 3: (pc == 3) ? 4 : 
    (pc == 4) ? ( (i <= e) ? 5 : 10 ) :
    (pc == 5) ? (a[i] == d) ? 6 : 8 ) :
    (pc == 6) ? 10 :
    (pc == 8) ? 4  : 
    (pc == 10) ? 11 : pc; }
  postas = ((rv >= s && rv <= e) -> a[rv] == d) xor
    (rv == -1 -> forall(int i:[s .. e]) { a[i] != d }); }
\end{Verbatim}
\end{tabular}
}
\caption{(a) Array search program, and (b) equivalent array search program with program
counter}
\label{f:arraysearch}
\end{figure}

The Array search program in Figure~\ref{f:arraysearch}(a) takes as input an array $a$, 
a start index $s$, an end index $e$, a data value $d$, and the number of 
elements in the array $n$.
It is annotated with a specification consisting of a precondition and a postcondition. 
The precondition states that the start $s$ and end $e$ indices are within array
bounds and that the array size $n$ is within the bound on array sizes. 
The postcondition states that if $rv$ is valid between $s$ and $e$ inclusive, 
then $a[rv]$ must be equal to $d$, 
otherwise, $rv$ must be invalid (-1) and all entries in $a$ between $s$ 
and $e$ inclusive are not equal to $d$. 

Figure~\ref{f:arraysearch}(b) shows an equivalent encoding of the array search 
program using a program counter execution model. 
The equivalent program introduces Boolean variables \cci{preas}, \cci{postas}, and 
\cci{notdone} to encode the preconditon, the postcondition, and the running
state of the program, respectively. 
The equivalent program also introduces a program counter variable
\cci{pc} which encodes the control flow of the program as indicated in 
the comments of Figure~\ref{f:arraysearch}(a). 
The \cci{rv} variable denotes the return value of the original program. 

The \cci{notdone} variable is initialized to $true$, and \cci{pc} 
program counter is initialized to the first executable line of the program $3$. 
Once \cci{pc} reaches the last executable line of the program $13$, 
the program terminates and thus \cci{notdone} becomes $false$. 
Assignment statements are grouped by target variables, and encoded
into conditional assignment statements that depend on the value of \cci{pc}. 
For example, the iterator $i$ is assigned to $s$ when \cci{pc} is $3$, incremented
when \cci{pc} is $8$, and remains the same otherwise. 

The program in Figure~\ref{f:arraysearch}(b) is semantically equivalent to the original
program in Figure~\ref{f:arraysearch}(a). 
Furthermore, the assignment statements on Lines 1 and 3 assign initial values to the 
target variables. 
The assignment statements inside the \cci{while} loop (Lines 5 to 13) compute 
the next state value of each of the target variables of the program. 

Our method translates the program in Figure~\ref{f:arraysearch}(b) to a sequential 
circuit where an iteration of the \cci{while} loop
is equivalent to a single time step in the sequential circuit. 
The method represents each Boolean variable with one register, and each scalar 
variable with a finite vector (bit-vector) of registers 
with initial value and next state functions. 
The initial state functions of the vector of registers corresponding to
a variable are connected to a vector of gates that represents the right hand 
side initial value assignment statement of the variable. 
For example, 
\cci{pc} ranges from $0$ to $11$ and can be encoded using $4$ registers.
%\todo{Add a figure for the sequential circuits}.
%The sequential circuit in Figure~\ref{f:pccircuit} shows the registers,
%their initial value functions set to $0$, and their next state functions
%set to gates representing multiplexers that choose the next value of the
%\cci{pc} registers based on the current state of the cuicuit. 


Program variables that are not initialized in the code are considered
input variables and the methods connects their initial value functions 
to free primary inputs.
The method connects the next state functions of register vectors corresponding to 
program variables to gates that represent
the right hand side of the next state assignment.
The conditional, arithmetic, and Boolean operations in the right hand side
expressions are encoded as combinational logic circuits in the usual manner. 

Our method takes the resulting sequential circuit, designates a gate therein that 
represents $preas \wedge done \rightarrow postas$ as the output gate, passes
the circuit to ABC, and checks for the validity of the designated gate.
The ABC solver returns a counterexample $a=[0 0 0],s=0,e=1,n=3,d=1,rv=2$ 
where $d$ is not in $a$, and the return value is $e+1$, while the postcondition
requires an invalid index (-1). 

The provided counter example can be used to fix the program. 
A possible fix is to replace Line 6 with \cci{return i;}, and Line 10 with
\cci{return -1;}.
Our method takes the fixed program, transforms it into a sequential circuit,
and passes it to ABC which validates the correctness of the program modulo the 
finite size of the variable vectors using symbolic model checking. 




%\section{Comparison with recent work}
\section{Limitations of translation to CNF}
\label{s:intro:limitations}
Existing tools such as CBMC~\cite{clarke2004tool}
check for pointer safety, within bound
array access and user defined assertions in C programs.
Given a C program and a bound on the range of variables, CBMC unwinds the program's 
loops and recursive functions, and unfolds the program 
into a Boolean (CNF) formula that asserts the specified properties. 
It then uses SAT methods and tools such as MiniSat~\cite{sorensson2005minisat}
to check the CNF formula for counter examples. 
%As the given bound increases, and the program becomes more complexe, the size
%of the generated CNF formula rapidly grows and SAT problem becomes practically
%impossible to solve. Moreover, programs containing unbounded loops might 
%even render the process of generating the CNF formula infeasible. 



Recent advances in SAT enabled tools like the 
Alloy Analyzer~\cite{jackson2002alloy} and CBMC to check designs of real systems.
However,
these designs often need to be partial, leaving out 
important functionality aspects of the systems, 
to enable the analysis to complete.  
Moreover, the analysis is typically bound to
relatively small limits, e.g., fewer than 7 nodes in 
a tree structure with the Alloy Analyzer.

There are three limiting aspects of translating high-level 
programs to SAT.  
\begin{description}

\item[Disadvantage 1]
The translation to CNF depends on
the bounds; a small increase in the bound on variable
ranges can cause a large increase in the size of the
translated CNF formula due to unwinding loop and recursion
structures in programs, or eliminating quantifiers 
%and unrolling transitive closure 
in declarative first order logic.
%for
%example, for an undirected seven-node tree the translation
%from Alloy to CNF generates a formula with over one million
%variables and five million clauses.  

\item[Disadvantage 2] 
The SAT solver is restricted to
using optimizations, such as symmetry
breaking~\cite{Aloul02SymSAT} and observability don't
cares (ODC)~\cite{FuYuMalik2005}, that apply at the level
of CNF formulas.  
However these optimizations usually aim
at increasing the speed of the solver and often result in
larger formulas as they add literals and clauses to the
CNF formula to encode symmetry and ODC
optimizations~\cite{ZhuKu06SATSweepODC}.  
Often times when
the analyzer successfully generates a large CNF formula,
the underlying solver requires intractable resources.

\item[Disadvantage 3] 
Often times the CNF formula
generated needs to be regenerated with higher bounds in
case the unwinding bounds were not large enough for the
loops to complete as is the case with CBMC. 
Note that multiple bounds exist and they need not be all
increased during one iteration.
\end{description}

To extend the applicability of static analysis to a wider
class of programs as well as to check more
sophisticated specifications 
and gain more confidence in the
results, we need to scale the analysis to significantly
larger bounds; limits on the range of design and program
variables.


The work in\cite{seraICSE07} takes a declarative 
formula $\phi$ in first order logic
(FOL) with transitive closure and a bound on the 
universe of discourse and
translates it to a %n equisatisfiable 
sequential circuit expressed in VHDL. 
It then passes the sequential circuit to a sequential 
circuit solver and decides
the validity of $\phi$ within the bound. 
It scales to bounds larger than what is possible with 
Kodkod~\cite{kodkodTJ2007}
which translates $\phi$ into a propositional Boolean 
formula in conjunctive
normal form (CNF) and checks its validity with a 
Boolean satisfiability solver. 

The work in\cite{sebacASE07} translates an imperative 
C program, with an assertion
statement therein, and a bound on the input size, 
into a %n equisatisfiable
sequential circuit expressed in VHDL. 
It then passes the sequential circuit to a 
sequential circuit solver and decides
the validity of the assertion within the bound. 
It scales to bounds larger than what is possible 
with CBMC\cite{cbmcDAC03} which
translates the program with a bound on the input 
size and the number of loop
iterations into a propositional Boolean formula 
in conjunctive normal form (CNF)
and checks for correctness using a Boolean 
satisfiability solver. 

Our method extends the work in \cite{seraICSE07,sebacASE07} in that
\be
\i it supports function calls including recursion, and requires a bound on
recursion depth only if the recursive function uses local variables, 
\i it enables a termination guarantee check within a bound on execution
time, it then uses the execution time bound with bounded model checking to
decide correctness,
\i it directly translates the program into bit level representation using {\em
and inverted graphs} (AIG) instead of the VHDL representation that requires a
VHDL compiler to be translated into bit level,
\i it uses ABC~\cite{brayton2010abc}, an open source sequential circuit solver, instead of
SixthSense~\cite{mony2004scalable} an IBM internal sequential circuit solver, 
\i and it is an open source tool available online
~\footnote{\label{fn:online}\url{
http://webfea.fea.aub.edu.lb/fadi/dkwk/doku.php?id=sa}}.
\ee

%CBMC's check, if valid, is only as good as the bound for which 
%it made the check, any bug that occurs beyond this bound is not detected.

%% formalization
\section{The \psqlanguage~ imperative language} \label{chap:c2aig:sec:formalization}
The grammar in Figure~\ref{fig:grammar} describes \psqlanguage, \mytool's
imperative input language. It is composed of a subset of the C++
programming language, extended with first order logic support 
and some special constructs. A \cci{program} is a list of 
declarations and statements. Variables can be declared to be of two 
kinds: (1) {\em wires} and (2) {\em registers}. Wires are non-memory 
elements used to monitor the values of different variables or terms
in the program, at every instance in the program's execution. Once assigned
to a term, wires will reflect the value of the term at 
every point in the program. On the contrary, register variables 
are memory elements that only change value once they are the target
of an assignment statement executed at a specific program point. 
The value of a register variable is memorized between two different
assignment statements. In what follows, we refer by wires to 
wire variables, and by variables to register variables. 

A variable can either be single or an array. Arrays can be declared to 
have a constant predefined size, or can be left free to have the maximum
number of elements to be determined by \mytool's runtime engine.
Similarly, wires can either be singular or arrays. 
\mytool~currently support one dimensional and two dimensional arrays. 
Given a two dimensional array $a$ of size $n$ by $m$ where $n$ and $m$ are constants, 
the tool transforms $a$ into a single dimensional array $a'$ of size 
$n \times m$, and translates all array accesses $a[i][j]$ into accesses
of the form $a'[i \times (n \times m) + j]$. 


{\em Statements} can be assignment statements, control statements or 
synchronization statements. Assignment statements modify the state of
program by assigning new values to select program variables. Control 
statements are used to modify the control state of the program by 
selecting the next program location, possibly begin dictated by the 
value of a certain term. \mytool~supports the 
\cci{if-then-else} selection control statement and the \cci{while} loop
control statement. 
Synchronization statements start with the \cci{dotogether}  
modifier, and are used to enforce the execution of a list of data-independent
assignment or selection statements at the same program point
(equivalently, at the same clock cycle).


Additionally, \mytool~ provides support for {\em function definitions} and 
{\em function calls}. Defining and calling a function is done in the same manner 
as in a regular C++ program, with the exception that \cci{return}
statements must always be present at the end of a function's body.
\mytool~allows also allows the declaration of recursive functions.
Expressions in \psqlanguage~extend the definition of terms presented in 
Definition~\ref{def:term} of Chapter~\ref{chap:back} with the addition 
of allowing terms to also be function calls, as depicted by the notation
\cci{term\_with\_function\_call} on line 31 of Figure~\ref{fig:grammar}. 

%% FOL properties
\mytool~extends the subset of C++ above with support for FOL specifications,
written in the form of {\em pre-condition}, {\em post-condition}
pairs. FOL expressions are Boolean expressions that can be Boolean 
terms or function calls, or {\em quantified expressions}. A quantified expression 
is either universally (\cci{forall}) or existentially  (\cci{exists}) quantified.


\begin{figure}[h!]
\centering
\begin{Verbatim}[fontsize=\relsize{-1.0}, numbersep=4pt,numbers=left]
program: block+
block: (declaration | statement)
statement: assignment_statement | conditional_statement | 
                        loop_statement | sync_statement		

// declarations
declaration: variable_declaration | function_declaration | property_declaration
variable_declaration: modifier? type id (`[' num `]' (`[' num `]')?)?
                      (`=' term)?
function_declaration: type id `(' argument_list? `)'
                      `{' block* return_statement `}'

// statements
assignment_statement: target `=' expression `;'
sync_statement: `@dotogether' `{' (assignment_statement | conditional_statement)+ `}'
conditional_statement: `if' `(' expresssion `)' `{' block `}'
                       `else' `{' block+ `}'
loop_statement: `while' `(' expression `)' 
                `{' block+ `}'
return_statement: `return' expression `;'

// properties
property_declaration: precondition | postcondition
precondition: `@pre' id `{' property+ `}'
postcondition: `@post' id `{' property+ `}'
property: expression | quantified_property
quantified_property: (`forall'|`exists') `(' range `)'
                     `{' property+ `;' `}'   

// expressions
expression: term_with_function_call
function call: id `(' call_arguments? `)'
call_argument: id (`,' id)*

argument_list: variable_declaration (`,' variable_declaration)*
modifier: `wire' | `const'
specifier: `int' | `bool'            
target: id | target `[' expression `]'
range: id `[' expression `...' expression `]' 
\end{Verbatim}
\caption{The \psqlanguage~imperative language}
\label{fig:grammar}
\end{figure}

\section{Translation to AIG}
\label{chap:c2aig:sec:translation}
Given an imperative program \Pm written in the \psqlanguage~language, 
and annotated with a FOL precondition postcondition 
specification pair \pair{\Pre}{\Post}, 
\mytool{} transforms the tuple (\Pm, \pair{\Pre}{\Post}) into an 
\thislanguage~program \Pm'. The tool then synthesizes \Pm'
into an equisatisfiable AIG $C$ having a single output
representing the formula $\lnot \left( \Pre \land \Pm \implies \Post \right)$. 
The tool then uses the ABC AIG solver to verify the validity 
of $\lnot o$ and thus prove that \Pm satisfies its
specification pair \pair{\Pre}{\Post} if successful. If the validity check fails, 
the tool returns a counterexample to be used by the developer for debugging.

%% Step 1 : some code transformations
\subsection{Preprocessing} \label{chap:c2aig:subsec:preproc}
\mytool~first starts by transforming 
\Pm into an intermediate program \Pm' $= \rho ($\Pm$)$ where
$\rho$ is a code transformation function that simplifies function calls, 
recursive functions and properties. 
%In what follows, we assume that we are given an imperative
%program \Pm with a set of variables $V$.
%
\paragraph{Function calls.} 
$\rho$ does not inline functions; instead, it uses a program counter mechanism to
avoid inlining and reuse the same code and thus the same AIG after synthesis. 
Key to that is the transformation of function class as follows. 
%Details of this mechanism are presented in Section~\ref{chap:c2aig:sec:transformation_et}.
Let $f(args_f)$ by a call to function declaration $f_d(args_{f_d}$ where $args_f$ is a list 
of expressions passed as arguments to $f$, and $args_{f_d}$ is the list of 
arguments declared in $f_d$. Let $ret(f_d)$ be the return variable of $f_d$, 
and $args_f(i)$ and $args_{f_d}(i)$ be the i'th 
argument passed to $f$ and declared in $f_d$, respectively, for $0 \leq i \leq |args_f|$.

Consider the assignment statement $s = (target := f(args_f))$ where $target$ is the 
target of the assignment as defined in line 38 of Figure~\ref{fig:grammar}.
The transformation function $\rho(s)$ is defined as follows:
\begin{equation}
\rho(s) = 
\begin{cases}
 & \left(  ;_{i = 0}^{|args_f|} \left( args_{f_d}(i) := args_f(i) \right) \right) ; \\
 & call\_func(f_d) ~;~\left( target := ret(f_d) \right) 
\end{cases}
\end{equation}
The \cci{;} operator represents an ordering of the statements, where $s_1;s_2$
means that $s_1$ executes before (or possibly at the same time as) $s_2$. 
Intuitively, $\rho$ copies the arguments passed to the call $f$ onto
the arguments declared in the function declaration $f_d$. It then 
adds the \cci{call\_func($f_d$)} statement, a special statement
that redirects the control of the program to the starting point of $f_d$'s body. 
It finally assigns that $target$ of $s$ to the return variable of the function $f_d$. 


%Given a function declaration $f_{dec}$, 
%we denote by $args(f_{dec}) \subset V$ the list of arguments of $f_{dec}$,
%by $ret(f_{dec}) \in V$ the return variable of $f_{dec}$, and by $body(f_{dec})$ 
%the list of statements in the function $f_{dec}$. 
%Given a function call $f$, we denote by $f_{dec} = dec(f)$ the function 
%declaration corresponding to the call $f$, and by $args(f) \subset V$ 
%the list of arguments passed to $f$. Additionally, 
%we denote by $args(f)[i]$ and $args(f_{dec})[i]$ the i'th argument passed to $f$
%and declared in $f_{dec}$, respectively. 
%
%\noindent Given an assignment of the form $x := f_c$ where $x \in V$, 
%let $f_d = dec(f_c)$, $a_{f_c} = args(f_c)$ and $a_{f_d} = args(f_d)$,
%the transformation function $\rho(x := f)$ is defined as:
%\begin{equation}
%\rho(x := f_c) = \begin{cases}
%\left( \bigcirc_{i = 0}^{|a_{f_c}| - 1} a_{f_d}(i) := a_{f_c}(i) \right) \circ \\
%\rho(body(f_d)) \circ \left( x := ret(f_d) \right)
%\end{cases}
%\end{equation}
%%\set{args(dec(f))[i] := args(f)[i]~|~i \in 0\ldots|args(f)|} \\
%where $\circ$ is the concatenation operator. 
%\mytool~assigns all argument variables in $dec(f)$ to all variables passed
%in the argument list of $f$, it then executes the list of statements in the 
%body of the function, and replaces references to the function call $f$ 
%by references to the return variable of $dec(f)$. Figure~\ref{fig:trans:function_call}
%shows an example of applying the transformation function $\rho$ onto a program containing
%a function call \cci{foo(x,y)}. The statement \cci{body(foo)} on line 5 of $\rho ( $ \Pm $)$
%corresponds the actual execution of the body of the function \cci{foo}. 


\begin{figure}[bt]
\centering
\begin{tabular}{p{0.33\textwidth}|p{0.01in}p{0.33\textwidth}}
\begin{Verbatim}[fontsize=\relsize{-2.0}, numbersep=4pt,numbers=left,
frame=topline, framesep=4mm, label=\fbox{Program \Pm}]
int fact (int n) {
   int result;
   if (n == 0) {
      result = 1;
   } else {
      result = n * fact(n-1);
   } 
   return result;
}

y = fact (x);
\end{Verbatim}
& &
\begin{Verbatim} [fontsize=\relsize{-2.0}, numbersep=4pt,numbers=left,
frame=topline, framesep=4mm, label=\fbox{Program $\rho ( $ \Pm $)$}]
int fact::result [];
int fact (int [] n) {
    if (n[fact::sp] == 0) {
        result[fact::sp] = 1;    
    } else {
        n[fact::sp + 1] = n[fact::sp] - 1;
        fact::sp = fact::sp + 1;
        call_func(fact);
        fact::sp = fact::sp - 1;
        result[fact::sp] = n[fact::sp]
                   * fact::ret[fact::sp+1];    
    }
    return result[fact::sp];
}

fact::n[0] = x;
call_func(fact);
y = fact::ret[0];
\end{Verbatim}
\end{tabular}
\caption{Transformation function $\rho$ on function calls}
\label{fig:trans:function_call}
\end{figure}
%
\paragraph{Recursive functions.} 
Let $f_d (args_{f_d})$ be a recursive function with arguments $args_{f_d}$,
and let $f(args_f)$ be a {\em recursive call} to $f_d$, i.e. a call to $f_d$
from inside the body of $f_d$. Using $\rho$,
\mytool~emulates recursion by (1) adding a stack pointer variable
$sp_{f_d}$ that maintains the recursive depth of the current function call, 
and (2) increasing the dimensionality of all arguments and local variable of $f_d$
by 1. In other words, variables become arrays and arrays become two dimensional 
arrays. Subsequently, all assignments and references to arguments or local variables
of $f_d$ are replaced by array access terms indexed by the current value 
of $sp_{f_d}$.
The recursion depth (i.e. $sp_{f_d}$) is incremented before
each {\em recursive call} $f$ and decremented once it returns. 

Consider the assignment statement $s = (target := f(args_f))$ where $target$ is the 
target of the assignment as defined in line 38 of Figure~\ref{fig:grammar}, and 
$f$ is a recursive call to $f_d$.
The transformation function $\rho(s)$ is then defined as follows:
\begin{equation}
\rho(s) = 
\begin{cases}
  & \left(  ;_{i = 0}^{|args_f|} \left( args_{f_d}(i)[sp_{f_d}+1] := args_f(i) \right) \right) ; \\
 & \left( sp_{f_d} := sp_{f_d} + 1 \right) ; call\_func(f_d) ; \left( sp_{f_d} := sp_{f_d} - 1 \right) \\
 & \left( target := ret(f_d)[sp_{f_d}+1] \right) ; 
\end{cases}
\end{equation}
Note that arguments of $f_d$ to be assigned are indexed by the future 
value of the recursion stack depth pointer (i.e. $sp_{f_d}+1$) before the
the pointer is incremented. The same is applied to the return variable 
of $f_d$ after the pointer has been decremented. 

Additionally, consider the assignment statement $s = (target := f(args_f))$ where
$f$ is a non-recursive call to $f_d$, $sp_{f_d}$ is guaranteed to have a value 
of $0$ and the transformation function $\rho(s)$ is defined as:
\begin{equation}
\rho(s) = 
\begin{cases}
  & \left(  ;_{i = 0}^{|args_f|} \left( args_{f_d}(i)[0] := args_f(i) \right) \right) ; \\
 &call\_func(f_d) ; \left( target := ret(f_d)[0] \right) ; 
\end{cases}
\end{equation}

Figure~\ref{fig:trans:function_call}
shows an example of applying the transformation function $\rho$ onto a program containing
a recursive function call \cci{fact(n)} that computes the factorial 
of an integer \cci{n}.
The argument \cci{n} of \cci{fact} and the local variable \cci{result} are transformed into the
arrays \cci{fact::n} and \cci{fact::result}. \cci{fact::sp} is the recursive stack pointer
variable added by $\rho$ for the function \cci{fact}.
Local references to \cci{n} are \cci{result} are replaced 
by the sequence access terms \cci{fact::n[fact::sp]} and \cci{result[fact::sp]}.

Lines 6-11 of $\rho(\Pm)$ in Figure~\ref{fig:trans:function_call} show the 
result of applying $rho$ on the recursive function call \cci{fact(n-1)}
on line 6 of the program \Pm. The next value of the argument $n$ is assigned
to the current value of $n$ decremented by $1$, as shown in the statement
\cci{fact::n[fact::sp + 1] = fact::n[fact::sp] - 1}. The stack pointer variable
is then incremented before adding the function call statement \cci{call\_func(fact)}
that gives the control to the body of the function \cci{fact}. Once the function
call returns, the stack pointer variable is decremented, and any reference
to the return variable of \cci{fact} is replaced by an array access term to the 
next value of the stack pointer (\cci{fact::sp + 1}).
Lines 16-18 of \Pm' shows the result of applying $\rho$ to a non-recursive call
to \cci{fact}, in which the arguments and return variable of \cci{fact} are 
replaced by array access terms indexed by $0$.
%Given a recursive function declaration $f$,
%the transformation function $\rho$ increases the dimension of all 
%arguments and local variables of $f$ by $1$; i.e. variables are transformed 
%into arrays and arrays into matrices. All assignments and references to an 
%argument or a local variable are replaced by array or matrix access operators. 
%These new array accesses are indexed by a special added variable, the 
%{\em function stack pointer} \cci{sp}. This variable tracks the number of recursive
%calls to $f$, and is incremented before each call to $f$, and decremented once 
%the function call returns. 
%
%\noindent Given an assignment of the form $x := f_c$ where $x \in V$ and $f_c$
%is a call to a recursive function, 
%let $f_d = dec(f_c)$, $a_{f_c} = args(f_c)$, $a_{f_d} = args(f_d)$,
%and $sp_f$ be the function stack pointer of $f_d$,
%the transformation function $\rho(x := f)$ is defined as:
%\begin{equation}
%\rho(x := f_c) = \begin{cases}
%\left( sp_f = sp_f + 1 \right) \circ
%\left( \bigcirc_{i = 0}^{|a_{f_c}| - 1} a_{f_d}(i)[sp_f] := a_{f_c}(i)[sp_f] \right) \circ \\
%\rho(body(f_d)) \circ \left( x := ret(f_d)[sp_f]  \right) \circ \left( sp_f = sp_f - 1 \right)
%\end{cases}
%\end{equation}
%
\paragraph{Quantifiers.} 
Consider the assignment statement $s = \left(target := Q (i:[t_1 \ldots t_2]) \set{\mathcal{B}} \right)$
where $target$ is the target variable of the assignment, $Q$ is either `$forall$' or `$exists$',
$i$ is a quantified variable, $t_1$ and $t_2$ are terms representing the range of $i$, and 
$\mathcal{B}$ is a Boolean FOL formula. $\rho(s)$ is defined as follows:
\begin{equation}
\rho(s) = 
\begin{cases}
\left( Q_r := true \right) ; \left( i := t_1 \right) ; \\
while(i <= t_2) \{ \rho \left( Q_r := Q_r ~(\&\& ~or~ ||)~ \mathcal{B} \right) ; 
\left( i := i + 1 \right) \}
; \\ 
\left( target := Q_r \right)   
\end{cases}
\end{equation}
For the quantified expression $\left( Q (i:[t_1 \ldots t_2]) \set{\mathcal{B}} \right)$,
$\rho$ adds a Boolean return variable $Q_r$ that is initialized to $true$ and holds
the value of the expression. Then $\rho$ transforms the expression into a while loop
that iterates over all possible values of $i$, and assigns the return variable $Q_r$
to its current value conjuncted or disjuncted with the value of $\mathcal{B})$.
The type of the operation performed is determined by the nature of the quantifier; 
conjunction for the universal quantifier (`$forall$') and disjunction for the 
existensial quantifier (`$exists$'). $\rho$ finally adds the actual assignment 
statement of $target$ as $target := Q_r$. Note that $\rho$ is also applied 
to the update assignment to the variable $Q_r$ in order to handle the cases
where $\mathcal{B}$ is also quantified or it contains a function call. 


Figure~\ref{fig:trans:forall} shows an example of applying the transformation 
function $\rho$ on a program \Pm containing an assignment of the variable \cci{y}
to a universal quantifier. 
Note that we added the expression \cci{forall::ret} to the condition of the 
\cci{while} loop in order to allow for early exit of the loop. 
Additionally, in the case the bounds on the quantified variable are constants, 
we optimize the transformation by unrolling the loop into a single large
Boolean expression.

\begin{figure}[bt]
\centering
\begin{tabular}{p{0.33\textwidth}|p{0.01in}p{0.33\textwidth}}
\begin{Verbatim}[fontsize=\relsize{-2.5}, numbersep=4pt,numbers=left,
frame=topline, framesep=4mm, label=\fbox{Program \Pm}]
y = forall(int i:[0 .. N]) 
               { a[i] != e };
\end{Verbatim}
& &
\begin{Verbatim} [fontsize=\relsize{-2.5}, numbersep=4pt,numbers=left,
frame=topline, framesep=4mm, label=\fbox{Program $\rho ( $ \Pm $)$}]
forall::i = 0;
forall::ret = true;

while (forall::i <= N && forall::ret) {
    forall::ret = forall::ret 
                  && (a[i] != e);
    forall::i = forall::i + 1;
}

y = forall::ret;
\end{Verbatim}
\end{tabular}
\caption{Transformation function $\rho$ on universal quantifier}
\label{fig:trans:forall}
\end{figure}
%
\paragraph{Pre/Post conditions.} 
Consider the \psqlanguage~
precondition and postcondition declarations $@pre~ \Pre ~\{ \mathcal{B} \}$
and $@post ~\Post ~\{ \mathcal{F} \}$ where $\mathcal{B}$ and $\mathcal{F}$ are 
Boolean expressions. For brevity, we refer by $\Pre$ and $\Post$ to the precondition
and postcondition delcarations, respectively.  
The transformation function $\rho$ is defined as follows:
%\begin{equation}
\begin{eqnarray}
& \rho(\Pre) = \rho \left( x_\Pre := \mathcal{B} \right) \\
& \rho(\Post) = \rho \left( x_\Post := \mathcal{F} \right) ; assert(x_\Pre \implies x_\Post)
\end{eqnarray}
%\end{equation}
$\rho$ creates for each declaration a Boolean variable that is used to hold its value
across the program's execution. $\rho$ then replaces the precondition declaration \Pre~
by a statement that assign its created variable $x_\Pre$ to its declared FOL 
formula $\mathcal{B}$. Similarly, $\rho$ replaces the postcondition declaration 
\Post~by the statement $x_\Post := \mathcal{F}$. In order to resolve quantification
or function calls in $\mathcal{B}$ and $\mathcal{F}$, we also apply $\rho$ to the created
statements. 

Additionally, $\rho$ adds an assertion statement 
$assert(x_\Pre \implies x_\Post)$ after the postcondition assignment statement. 
This assertion statement will not 
be evaluated, instead it used the help \mytool~ determine the outputs to pass
to the ABC AIG solver for verification. 
%%
%%For each pre/post condition declaration, \mytool~
%%creates a Boolean variable that is assigned to the value of the FOL formula representing
%%the property. The property is evaluated and assigned to the declared variable at
%%the same location where the pre/post condition is declared. 
%%
%%Additionally, for the postcondition declaration, \mytool~ adds an assertion statement 
%%of the form $precondition \implies postcondition$. This assertion statement will not 
%%be evaluated, instead it used the help the synthesizer described in
%%Section~\ref{chap:c2aig:sec:transformation_et} determine the primary outputs to pass
%%to the ABC sequential solver for verification. 


All statements and expressions in \Pm~ that are not mentioned in 
this Section are kept unchanged by the transformation function $\rho$. 

\section{Tranformation to \thislanguage} \label{chap:c2aig:sec:transformation_et}
%% C to ETCircuit transformation
After preprocessing, \mytool~transforms the program $\Pm' = \rho(\Pm)$ into 
an equisatisfiable \thislanguage~program $\mathcal{C}$. Synthesizing 
an AIG from $\mathcal{C}$ is then a direct translation of variables into bit registers
and building their next state function according to the \cci{next} block
definition in $\mathcal{C}$. 
\mytool~uses {\em program counter semantics} to translate \Pm'
into $\mathcal{C}$. The program counter is used to ensure the
correct sequencing of assignments in the program; i.e. to provide the concept of time.

Our method assigns a unique {\em label} for each statement in the
program \Pm'. Let $s$ be any statement in \Pm', 
and let $s_a, s_i, s_w$, and $f_d$ be an assignment statement, 
a conditional statement, a loop statement and a function declaration
in \Pm', respectively. Additionally, let $e$ be an array access expression,
and let $call(f_d)$ be an added \cci{call\_func} statement
that calls the function declaration $f_d$. 
Our method defines the following functions:
\begin{itemize}
 \item \cci{label($s$)}: The unique label identifying the statement $s$
 \item \cci{next($s$)}: The label of the statement that directly follows $s$ in the program order.
 \item \cci{condition($s_i$)}: The Boolean condition of the selection statement $s_i$.
 \item \cci{then($s_i$)}: The label of the first statement in the {\em then} 
 code block of $s_i$.
 \item \cci{else($s_i$)}: The label of the first statement in the {\em else}
 code block of $s_i$.
 \item \cci{condition($s_w$)}: The Boolean condition of the loo statement $s_w$.
 \item \cci{body($s_w$)}: The label of of the first statement in the body of the $s_w$.
 \item \cci{last($s_w$)}: The last statement in the body of the loop $s_w$.
 \item \cci{target($s_a$)}: The target variable of the assignment $s_a$.
 \item \cci{expression($s_a$}: The expression to be assigned to the target of $s_a$.
 \item \cci{body($f_d$)}: The label of the first statement in the body of the 
 function $f_d$.
 \item \cci{return($call(f_d)$)}: The label of the statement to which the function call 
 statement should return after calling $f_d$.
 \item \cci{base($e$)}: The array which the expression $e$ is indexing.
 \item \cci{index($e$)}: The index at which the the base of the expression $e$ is indexed.
\end{itemize}


\begin{algorithm}[tb]
\begin{algorithmic}[1]
\State {\bf Input:} \psqlanguage~ program \Pm', $entry\_statement$
\State {\bf Output:} \thislanguage~ program $\mathcal{C}$
\State 
\State // build the initialization list
\State init\_list $[ ]$
\ForAll { $v \in V$}
 \If{used\_before\_assigned($v$)}
  \State init\_list.insert ($``v = free\_inputs"$)
 \EndIf
\EndFor
\State init\_list.insert ($``pc = label(entry\_statement)"$)
\State
\State // build the next state list for variables
\State next\_state\_list $[ ]$
\ForAll {$v \in V$}
 \State $next_v = ``v = "$
 \ForAll {Assignment statement $s$ such that $target(s) = v$}
  \State $next_v += ``(pc==label(s))? ~expression(s) : "$
 \EndFor
 \State $next_v += ``v"$
 \State next\_state\_list.insert $(next_v)$
\EndFor
\State 
\State // build the next state for the program counter
\State $n_{pc} = ``pc = "$
\ForAll {statement $s$}
 \If {is\_conditional\_statement ($s$)}
   \State $n_{pc} += ``(pc==label(s))?~ (condition(s)?~then(s) : else(s)) :"$
 \ElsIf {is\_loop\_statement ($s$)}
   \State $n_{pc} += ``(pc==label(s))?~ (condition(s)?~body(s):next(s)) :"$ 
 \ElsIf {is\_function\_call(s)}
   \State $f_d :=$ function\_declaration\_of $(s)$
   \State $n_{pc} += ``(pc==label(s))?~ (body(f_d)) :"$
 \ElsIf {is\_return\_statement (s)}
   \State $f_d :=$ function\_declaration\_of $(s)$
   \State $n_{pc} += ``(pc==label(s))?~(return(call(f_d))) :"$
 \Else 
   \State $n_{pc} += ``(pc==label(s))?~(next(s)) :"$
 \EndIf
\EndFor
\State next\_state\_list.insert $(n_{pc})$
\end{algorithmic}
\caption{\psqlanguage~to \thislanguage~transformation algorithm}
\label{algo:psqtoettrans}
\end{algorithm}

Algorithm~\ref{algo:psqtoettrans} shows the procedure used 
in \mytool~to translate a preprocessed \psqlanguage~program \Pm'
with a set of variables $V$ and an initial entry statement, into an \thislanguage~program 
$\mathcal{C}$, to be used for the synthesis of the equisatisfiable 
AIG. Intuitively, the wire declarations, variable definitions and wire definitions are directly moved from 
 \Pm' to their corresponding blocks in \aigcircuit.
Additionally, in order to model time, \mytool~adds a new
scalar variable to the program \Pm', the {\em program
counter} \cci{pc}. This variable is key to avoid 
inlining functions and unrolling loops. It is used to 
ensure the proper sequencing of the statements of \Pm';
the current value of the \cci{pc} variable defines which 
statement is to be execute. For example, reassigning the 
value of \cci{pc} to the starting point of a loop's body
allows \mytool~to make use of the same code block (thus
the same AIG) to execute a single loop infinitely many 
times. 

The algorithm starts by building the initialization list
of \aigcircuit. It assigns the program counter variable
\cci{pc} to the label of the entry statement of \Pm'.
Then for each variable $v \in V$ that is used before
being assigned (i.e. used with an undefined value), it 
assigns $v$ to a set of free inputs (AIG primary inputs)
who's values are to be set by the ABC AIG solver. 
Note that since variables can be initialized at different
points in the program (and not necessarily at the starting
point of the algorithm), we delay their
initialization step 
into the \cci{next} block, where these variables
will get their initial value at the appropriate \cci{pc}
value.

Subsequently, the algorithm moves to building the
\cci{next} state block of \aigcircuit. For each 
variable $v \in V$, and for each assignment statement 
$s$ such that $target(s)=v$, the next state of $v$ will
be $expression(s)$ iff $pc == label(s)$. If at a given 
statement label, $v$ is not being assigned, it retains
its current value. For example, consider the 
following assignment
statements for $v$:
\begin{Verbatim}[fontsize=\relsize{-1.0}, numbersep=4pt,numbers=left]
(l1): x = 1;
(l2): x = 3;
(l3): x = 5;
\end{Verbatim}
The algorithm builds the next state function of $x$ as 
the following statement
\begin{Verbatim}[fontsize=\relsize{-1.0}, numbersep=4pt,numbers=left]
x = (pc == l1)? 1 : (pc == l2)? 3 : (pc == l3)? 5 : x;
\end{Verbatim}

After traversing all of \Pm's variables, the algorithm
then builds the next state assignment for the program counter.
At a given statement $s$, the default behavior of the \cci{pc} is to move from the current label $label (s)$
to the label of the statement that directly follows
$s$, i.e. $next(s)$. 
Only control statements such as conditional statements, 
loop statements, function call statements and return 
statements are allowed to alter the default behavior of 
\cci{pc}.

Conditional statements alter the value of the 
program counter based on the evaluation of their
conditional expression. Consider a conditional statement
$s$ and let $e = condition(s)$. When \cci{pc} is at 
$label(s)$, its next value is either the label 
of the first statement in then \cci{then} branch of $s$
if $e$ evaluates to $true$, or to the first
statement in the \cci{else} branch of $s$ otherwise. 
Therefore at $label(s)$, the next state of the \cci{pc}
is defined as $(e)?~ then(s) : else(s)$.

Similarly, loop statements move the program counter
into their body when their conditional expression evaluates
to $true$, and to their next statement once it is $false$.
Consider a loop statement $s$ and let $e = condition(s)$.
At $label(s)$, the next state of \cci{pc} is defined
as $(e)?~ body(s):next(s)$. Note that according to the 
\psqlanguage, the loop statement block is considered 
as a single statement, and thus the next statement 
to follow $s$ would be the first statement outside its 
body, i.e. its exit point. Additionally, for the last
statement in the body of the loop, its next statement
is defined as the loop statement itself. This ensure that
once the \cci{pc} reaches that last statement in the 
loop, it once again evaluates its condition in order to
determine whether to exit or stay in the loop.  

Finally, function call statements and return statements 
redirect the program counter into the body of the function 
and to the first statement after the call, respectively. 
Note that for a non-recursive function, there is at most
one active function call at a time, therefore it is easy
to track the label at which the program counter 
should return to.
For a recursive function, we maintain a stack of function calls and redirect the control to statement after the 
call that is on top of the stack. 

\subsubsection*{Post-processing}
After constructing the \thislanguage~program \aigcircuit,
\mytool~employs a last post-processing step to resolve 
array accessing. Given an array access expression 
$a[i]$ where $a$ is an array and $i$ is an index, resolving
this access expression is to transform it into an array
access where the index is constant. An array access 
expression with a constant index is handled as a regular 
register variable in \aigcircuit. 

The first part of algorithm~\ref{alg:resolv_target_exp} 
shows the post
processing of a reference to an array access expression 
$e$. The algorithm pushes the variable index $i$ outside
of the array access by creating a set of checks for the 
value of $i$ at each possible index of the array $a$, and
returning the appropriate array element accordingly.

The second part of algorithm~\ref{alg:resolv_target_exp} 
shows the 
post processing of statement $s$ where $target(s)$
is an array access of the form $a[i]$. Since this 
assignment statement can be to any of the elements of the 
array $a$, as dictated by the value of $i$, the algorithm
creates an assignment statement for each possible 
element of $a$. This ensure that only the correct element
get assigned to the value of $expression(e)$ by creating
a check for $i$ at each possible index of the array.  

Figure~\ref{fig:trans:arrays} shows an \thislanguage~program 
\aigcircuit~that contains array access assignments and 
references. Post-processing transforms the assignment
statement on line 14 of the program \aigcircuit into 
assignment statements over all the elements of the
array $a$, with added checks on the value of the index
$i$. Also post-processing replaces the expression $b[k]$ 
in the right hand side of the assignment statements 
with the expression $(k==0)?~b[0] : b[1]$. This expression
checks on the value of $k$ for all possible indices to the 
array $b$ and returns the corresponding
element. 

\begin{figure}[bt]
\centering
\begin{tabular}{p{0.33\textwidth}|p{0.01in}p{0.33\textwidth}}
\begin{Verbatim}[fontsize=\relsize{-2.5}, numbersep=4pt,numbers=left,
frame=topline, framesep=4mm, label=\fbox{Program \aigcircuit}]
int a [2];
int b [2];
int i, k;
@dotogether {
  a[0] = free_inputs;
  a[1] = free_inputs;
  i = free_inputs;
  k = free_inputs;
  b[0] = free_inputs;
  b[1] = free_inputs;
}
 
@dotogether {
  a[i] = (pc == l1)? b[k] : a[i];
}
\end{Verbatim}
& &
\begin{Verbatim} [fontsize=\relsize{-2.5}, numbersep=4pt,numbers=left,
frame=topline, framesep=4mm, label=\fbox{\aigcircuit~after
post-processing}]
int a [2];
int b [2];
int i, k;
@dotogether {
  a[0] = free_inputs;
  a[1] = free_inputs;
  i = free_inputs;
  k = free_inputs;
  b[0] = free_inputs;
  b[1] = free_inputs;
}
 
@dotogether {
  a[0] = (i == 0)? 
           ((pc == l1)? ((k == 0)? b[0] : b[1]) : a[0]) 
           : a[0];
  a[1] = (i == 1)? 
           ((pc == l1)? ((k == 0)? b[0] : b[1]) : a[1]) 
           : a[1];
}
\end{Verbatim}
\end{tabular}
\caption{Post processing of a \thislanguage~\aigcircuit}
\label{fig:trans:arrays}
\end{figure}


\begin{algorithm}[h!]
 \begin{algorithmic}[1]
  \State // array access on right hand side
 \State {\bf Input:} Array access expression $e$
  \State {\bf Output:} Array access expressions $e'$ with 
  constant indexing
  \State $a := base(e)$
  \State $i := index(e)$
  \State $N := size(a) - 1$
  \For{$j \in 0 \ldots N-1$}
   \State $e' += (i == j)?~ a[j] :$
  \EndFor
  \State $e' += a[N]$
  \State
  \State // array access on left hand side
  \State {\bf Input:} Statement of the form $a[i] = e$, 
  $e$ is an expression
  \State {\bf Output:} A list of statements where all array accesses have a constant index 
  \State List\_of\_statements $[ ]$
  \State $N := size(a) - 1$
  \For{$j \in 0 \ldots N-1$}
   \State List\_of\_statements.insert $(a[j] = (i == j)?~ e : a[j])$
  \EndFor
 \end{algorithmic}
 \caption{Array access resolving algorithm}
 \label{alg:resolv_target_exp}
\end{algorithm}



\section{\thislanguage~to AIG}
\label{chap:c2aig:sec:etcirc_to_aig}
Given a bound $b$ on the bit width of variables,
\mytool~synthesizes an AIG by a direct translation of a \thislanguage~program \aigcircuit~into
a sequential circuit. The synthesis proceeds as follows.  
Scalar register variables and array elements are translated into vectors 
of $b$ bits registers, while Boolean variables and array elements are directly mapped
onto one bit registers. The \cci{init} block of \aigcircuit~is used to determine the 
initial value of the instantiated registers. 

The \cci{next} block of \aigcircuit~infers the next state functions for all 
registers in the AIG, including the program counter. For each variable, 
we translate the expression of its next state function into a hierarchy of 
multiplexers that reflect the ternary choice expressions in \aigcircuit's 
assignment expressions.

Finally, \mytool~builds the primary output of the AIG 
as the negation of the assertion statement introduced in the 
pre-processing step. Given a precondition \Pre~and 
a postcondition  \Post, the primary output $o$ will
be $\lnot (x_\Pre \implies x_\Post)$ where $x_\Pre$ and
$x_\Post$ are the precondition and postcondition variables
introduced in the pre-processing step. 