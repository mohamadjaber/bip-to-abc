
%% the algorithm
\section{BIP to \thislanguage}
\label{chap3:bip2aig:bip2etc}
Given a BIP system $\Pm = (B, Init, v)$, \biptool~translates \Pm into 
a \thislanguage~program \aigcircuit that includes its own customized execution engine. 

Listing~\ref{alg:bip2aig} shows the algorithm used by \biptool~to implement 
this translation. The algorithm is divided into 4 steps: (1) wire and variable
declarations, (2) wire definitions, (3) \cci{init} block and (4) \cci{next} block construction. 
The lists \cci{decl-list}, \cci{wiredef-list}, \cci{init-list} and \cci{next-list}
represent the list of declarations and lists of statements used in the 
declaration, wire definition, \cci{init} and \cci{next} blocks of \aigcircuit, respectively. 



\begin{lstlisting}
decl-list

wiredef-list

do-together {
  init-list
}

while(true) {
  do-together {
    next-list
  }
}
\end{lstlisting}

\begin{lstlisting}
/*@\textbf{BIP-to-CAIG}@*/(B, Init, v)
  generateDeclartionList()
  generateWireDefList()
  generateInitList()
  generateNextList()
\end{lstlisting}

\begin{lstlisting}
/*@\textbf{generateDeclartionList()}@*/
  // interaction enablement wires
  append $wire~bool~ie[|J|]$ to decl-list
  // interaction priority wires
  append $wire~bool~ip [ |J| ]$ to decl-list 
  // interaction selected wires
  append $wire~bool~is[|J|]$ to decl-list 
  // non-deterministic priority selector wire
  append $wire~int~selector$ to decl-list 
  // cycle denotes transition or interaction mode
  append $bool~cycle$ to decl-list  

  foreach $i \in [1..|I|]$
    foreach $j \in [1..|B_i.P|]$ 
      // port enablement
      append $wire~bool~B_i.p_j.e$ to  decl-list 
      // port selected
      append $wire~bool~B_i.p_j.s$ to  decl-list 
    endfor

    // location registers
    append $int~B_i.\ell$ to decl-list
    
    foreach $j \in [1..|B_i.X|]$ 
      // variable registers
      append $int~B_i.x_j$ to  decl-list 
    endfor
  endfor
\end{lstlisting}


\begin{lstlisting}
/*@\textbf{generateDeclartionList()}@*/
  // interaction enablement wires
  append $wire~bool~ie[|J|]$ to decl-list
  // interaction priority wires
  append $wire~bool~ip [ |J| ]$ to decl-list 
  // interaction selected wires
  append $wire~bool~is[|J|]$ to decl-list 
  // non-deterministic priority selector wire
  append $wire~int~selector$ to decl-list 
  // cycle denotes transition or interaction mode
  append $bool~cycle$ to decl-list  

  foreach $i \in [1..|I|]$
    foreach $j \in [1..|B_i.P|]$ 
      // port enablement
      append $wire~bool~B_i.p_j.e$ to  decl-list 
      // port selected
      append $wire~bool~B_i.p_j.s$ to  decl-list 
    endfor

    // location registers
    append $int~B_i.\ell$ to decl-list
    
    foreach $j \in [1..|B_i.X|]$ 
      // variable registers
      append $int~B_i.x_j$ to  decl-list 
    endfor
  endfor
\end{lstlisting}


\begin{lstlisting}
/*@\textbf{generateWireDefList()}@*/
  // iterate over interactions
  foreach $j \in [1..|J|]$ 
    append $ie[j] := a_j.guard \wedge \bigwedge_{p\in a_i.P} component(p).p.e$ to  wiredef-list 
    append $ip[j] := ie[j] \wedge \,(\forall k \neq j: ie[k] \Rightarrow a_k < a_j)$ to  wiredef-list 
    append $is[j] := ip[j] \wedge \,(selector = j \vee (\lnot ip[selector] \land \forall k > j: \neg ip[k])$ to  wiredef-list 
  endfor
  // iterate over components
  foreach $i \in [1..|I|]$ 
    // iterate over component ports
    foreach $j \in [1..|B_i.P|]$ 
      append $B_i.p_j.e := \bigvee_{\tau \in transitions(B_i.p)} \tau.guard \land B_i.\ell = \tau.src$ to  wiredef-list 
      append $B_i.p_j.s := \bigvee_{a_k \in interactions(B_i.p_j)} is[k]$ to  wiredef-list 
    endfor
  endfor
\end{lstlisting}

\begin{lstlisting}
/*@\textbf{generateInitList()}@*/
  // initialize to interaction mode
  append $cycle := 0$ to init-list 
  foreach $i \in [1..|I|]$
    append $B_i.\ell := Init.B_i$ to  init-list 
    foreach $j \in [1..|B_i.X|]$
      // v is the initial valuation
      append $B_i.x_j := v(B_i.x_j)$ to init-list  
    endfor
  endfor
\end{lstlisting}

\begin{lstlisting}
/*@\textbf{generateNextList()}@*/ 
  // iterate over components
  foreach $i \in [1..|I|]$ 
    // iterate over variables, where $\textcolor{darkgreen}{B_i.X = \{x_1, \ldots, l_{|B_i.X|}\}}$ 
    foreach $j \in [1..|B_i.X|]$ 
      append $B_i.x_j := cycle = 0?$ to st
      // iterate over interactions
      foreach $k \in [1..|J|]$ 
        // iterate over interaction assignments
        foreach $\sigma \in a_k.action$
          if($B_i.x_j = \sigma.term$)
            append $is[k]?\, \sigma.expr:$ to st
          endif
        endfor
      endfor
      // interaction mode and no data transfer for $\textcolor{darkgreen}{B_i.x_j}$
      append $B_i.x_j$: to st 
      // iterate over component transitions
      foreach $\tau \in B_i.T$ 
        // iterate over transition assignments
        foreach $\sigma \in \tau.action$
          if($B_i.x_j = \sigma.term$)
            append $(B_i.port(\tau).s \wedge \tau.source = B_i.\ell)?\, \sigma.expr:$ to st 
          endif
        endfor
      endfor  
      
      append $B_i.x_j$ to  st 
      append $st$ to  next-list 
    endfor
    // switch cycle
    append $cycle := \neg cycle$ to  next-list 
  endfor
\end{lstlisting}


Given a port $p$ from the system \Pm, we define the function
$interaction(p) = \set{a \in \gamma~|~p \in a.P}$ where $\gamma$
is the set of interactions in the composite component $B$ of \Pm.
This function returns the set of interactions in which the port 
$p$ takes part. Additionally, the function 
$component(p) = B_i$ such that $p \in B_i.P$ for $i \in I$ defines
the component to which the port $p$ belongs. Furthermore the function
$transitions(p) = \set{\tau \in component(p)~|~\tau \in B.T \land \tau.port = p}$ defines the 
set of transitions $\tau$ in the component $B = component(p)$ such that $\tau$ 
is labeled by $p$. 

Intuitively, for each component $B_i$ in $B$, \biptool~defined a scalar variable 
$B_i.l$ that represents the control location $l \in B_i.L$ at which the 
component $B_i$ is currently at. Additionally, it declares a scalar variable
$B_i.x_j$ for each variable $x_j \in B_i.X$. 
The algorithm defines the initial values for control locations and data variables 
based on the $Init$ relation of the system \Pm. It thus builds the initialization 
list by adding the assignments $``B_i.\ell := Init.B_i"$ and $``B_i.x_j := v(B_i.x_j)"$
for $j \in [1\ldots|B_i.X|]$ and $i \in [1\ldots|I|]$, where $Init.B_i$ is 
the initial location from which $B_i$ starts and $v(B_i.x_j)$ is the initial valuation of the
variable $x_j \in B_i.X$.


For each atomic component $B_i$ in $B$, and for every port $p \in B_i.P$, 
\biptool~defines two Boolean wires. 
(1) The {\em port enablement} wire ($B_i.p_j.e$) that is set to \true~iff the port $p_j \in B_i$ is 
{\em enabled}. The port $p_j$ is enabled iff there exists a transition $\tau \in transitions(p_j)$
such that the guard of $\tau$ evaluates to \true~and $B_i$ is in the control location $l = \tau.src$.
(2) The {\em port selected} wire ($B_i.p_j.s$) that is set to \true{} iff the port $p_j \in B_i$ is
{\em selected}. The port $p_j$ is selected iff there exists an interaction $a \in interactions(p_j)$
that is selected for execution by the BIP engine.

For each interaction $a_k \in \gamma$, \biptool~defines three Boolean wires. 
(1) The {\em interaction enablement} wire, $ie[k]$, that is set to \true{} iff the interaction
$a_k$ is enabled. $a_k$ is enabled iff the guard of the interaction $a_k.guard$ evaluates
to \true{} and all ports $p \in a_k.P$ are enabled. 
(2) The {\em interaction priority} wire, $ip[k]$, that is set to \true{} iff the interaction 
$a_k$ is enabled and there does not exist any interaction $a_j \in \gamma$ such that 
$a_j$ is enabled and has a higher priority than $a_k$.
(3) The {\em interaction selected} wire, $is[k]$, that is set to \true{} iff the
interaction $a_k$ is selected for execution. Given a set of enabled interactions 
with equal priority, the BIP engine synthesized by \biptool{}
selects interactions for execution based on the following procedure. It makes use of a
non-deterministic selector (the \cci{selector} scalar wire defined on line 10 
of Listing~\ref{alg:bip2aig}) and selects the active interaction $a_j$ having 
an index $j$ equal to the non-deterministic value of the selector wire.
If no such interaction exists, the engine selects the interaction with the largest 
index. 
This is depicted by the assignment $``is[j] = ip[j] \land \left( (selector = j) 
\land (\lnot (ip[selector] \land \forall{k > j}\,:\lnot ip[k])) \right)"$ on line 32 of Listing~\ref{alg:bip2aig}.

%% the bip engine we use
\biptool{} synthesizes a BIP execution engine that is based on a two clock cycles 
approach. In the first cycle, the engine selects an interaction $a \in \gamma$ 
and executes its actions. In the second cycle, the engine checks for possible
transitions $\tau \in B_i$ such that $\tau.port$ is selected, for $i \in [1\ldots|I|]$,
and executes their actions $\tau.actions$. Contrarily to the BIP engine in~\cite{BasuBBCJNS11},
\biptool's engine executes the actions of all possible transitions simultaneously.
Additionally, the engine updates the locations of the atomic components $B_i$ of $B$
in accordance with the executed transitions; \ie{} executing a transition $\tau \in B_i.L$ 
transitions $B_i$ from control location $l_1 = \tau.src$ to location $l_2 = \tau.dest$.

\biptool{} updates the values of the data variables in accordance with the 
actions of the executed interactions and transitions. An assignment statement 
$\sigma \in a_k.actions$ where $a_k \in \gamma$ is an interaction executes 
and updates the value of its target data variable iff the interaction $a_k$ 
is selected for execution, \ie when $is[k]$ evaluate to \true{} and the engine
is in its first cycle. Similarly $\sigma \in \tau.actions$ where $\tau$ is a transition
executes and update the value of its target data variable iff the transitions $\tau$
is allowed to execute, \ie when $\tau.port$ is selected and the engine is in its
second cycle. Updates to data variable are shown on lines 61 and 69 of the 
algorithm in Listing~\ref{alg:bip2aig}.

\biptool{} uses the same AIG synthesis engine used in \mytool{} in order to 
synthesize an AIG from the generated \thislanguage{} program \aigcircuit. 
If the developer wishes the check for deadlock freedom, \biptool{} adds 
an assertion statement to \aigcircuit{} to check that at each cycle, 
there is at least one interaction enabled for execution. Additionally, 
developers can also add custom assertions defined as FOL properties. These assertions
are handled in the same manner that \mytool{} handles them, and \biptool{}
uses the ABC AIG solver to check for the validity of the assertions. 

\section{Illustrative Example}
Table~\ref{tb:bip:exec} shows a sample execution trace of \mytool's BIP engine
on the traffic light example presented in Figure~\ref{fig:traffic:bip}. The variables 
$timer.t$, $timer.n$, $timer.\ell$, $light.\ell$ and $light.m$ represent $t$, $n$
and the control location variable in the timer component, and the control location variable
and $m$ in the light component, respectively. 
We start at cycle $c$ where that state of the BIP system is $$\left(timer.t = 9, timer.n = 10,
timer.\ell = s_0, light.\ell = G, light.m = 3\right)$$
and we assume that the execution engine has picked the interaction involving the 
$timer$ port of the timer component has been selected and executed. 

At cycle $c+1$, the engine executes the internal transition in the timer component
that corresponds to the $timer.timer$ port being selected and the component is in location $s_0$. 
Therefore the variable $timer.t$ is incremented and the component 
remains in the same control location $s_0$. Since the light component
had no ports involved in the executed interaction, it will not change 
state. 

At cycle $c+2$, the guard $t \geq n$ is enabled and thus the $timer.done$ and $light.done$ ports are enabled. 
The engine selects the interaction connecting the two components
for execution. It executes the actions associated with this interaction 
and transfers the value of the variable $done.m$ to the variable $done.n$.

At cycle $c+3$, the engine executes the interaction transitions that 
correspond to the interaction executed at cycle $c+2$. In the timer 
component, the variable $timer.t$ is reset $0$ and the control location 
remains in location $s_0$. In the light component, the variable 
$light.m$ takes the value $5$ and the control location moves to location 
$Y$ corresponding to the yellow light being displayed. 

\begin{table}[bt]
\centering
\caption{Sample of \biptool{} execution}
\begin{tabular}{|c|c|c|c|c|c|}
\hline
Cycle & $timer.t$ & $timer.n$ & $timer.\ell$ & $light.\ell$ & $light.m$ \\ 
\hline
$c$ & $9$ & $10$ & $s_0$ & $G$ & $3$ \\
		\hline
		$c+1$ & $10$ & $10$ & $s_0$ & $G$ & $3$ \\
			\hline
			$c+2$ & $10$ & $3$ & $s_0$ & $G$ & $3$ \\
				\hline
	 $c+3$ & $0$ & $3$ & $s_0$ & $Y$ & $5$ \\
	 \hline
\end{tabular}
\label{tb:bip:exec}
\end{table}

Listing~\ref{sample:code:bip} shows a code snippet from the code that \biptool{} generates
when applied to the traffic light controller BIP system shown in Figure~\ref{fig:traffic:bip}.
Lines 2 to 19 show the wire definitions for the system. The interaction 
labeled by $0$ corresponds to the interaction involving the $timer.timer$
port, while the interaction labeled $1$ correspond to the interaction 
involving the ports $timer.done$ and $light.done$. 
Interaction $0$ is enabled ($ie[0]$) when its guard is enabled ($true$ 
in this case) and the port $timer.timer$ that it involves is enabled, 
\ie{} the wire $timer.timer.e$ is set to $true$. 
Similarly, interaction $1$ is enabled ($ie[1]$) when its guard is
enabled and the ports $timer.done$ and $light.done$ are enabled 
($timer.done.e \land light.done.e$).

The wires $ip[0]$ and $ip[1]$ are asserted when each of the two interactions 
is prioritized. We assume in our case that interaction $1$ has a higher
priority, and thus interaction $0$ is only prioritized when it is enabled
and interaction $1$ is not. In any other case, it is interaction $1$ that
is prioritized by the engine. 

Furthermore, the wires $is[0]$ and $is[1]$ represent the execution of the 
engine's interaction selection procedure. The wire $selector$ is
a non-deterministic wire used by the engine to make non-deterministic
choices when several interactions are prioritized. 
The interaction $0$ is selected
when it is prioritized and either the non-deterministic selector 
selects it ($selector == 0$)
or the interaction $1$ is not prioritized. 
The interaction $1$ is selected when it is prioritized and either 
the selector selects it ($selector == 1$) or the selector wire
has selected an interaction that is not prioritized. Since interaction
$1$ has the largest index, it is
directly selected when the value of the selector corresponds
to an interaction that is not prioritized ($\lnot ip[selector]$).

The wires $component.port.e$ and $component.port.s$ represent the 
port enabled and selected wires, respectively. 
For example, the timer port 
in the timer component ($timer.timer.e$) is enabled when the 
component is in control location $s_0$ and the guard $timer.t < n$ is
enabled. Additionally, this port is selected when the interaction $0$ 
in which it is involved has been selected ($timer.timer.s = is[0]$).

Lines 21 to 38 show the next state functions for each of the variables
and control variables of the components of the system. 
In the first execution cycle ($cycle == 0$), 
the engine selects and executes interactions.
In the second execution cycle ($cycle == 1$), 
it executes internal transitions. 
When the interaction $1$ is selected ($is[1]$) and the engine is 
in its first cycle, the value of the variable $light.m$ is transfered
to the variable $timer.n$. Otherwise, $timer.n$ retains its current
value.

In the internal transition execution cycle, the next state value of
the variable $light.m$ is dictated by the control location at which the component is currently at, and whether the port $light.done$
is selected (equivalently the interaction $1$ has been selected).
For example, if the port $light.done$ is selected and the component 
is at the yellow light location ($Y$), the value of
$light.m$ will be 10. 

Finally, the control location variables $timer.\ell$ and 
$light.\ell$ are also updated according to the internal transitions of the 
atomic components. For example when the $light.done$ port is selected
and the light component is in the green control location ($G$), 
 $light.\ell$ moves from $G$ to $Y$. 

\begin{lstlisting}[caption=Sample of \biptool{} generated code,label=sample:code:bip,float=bt]
/** Wire definitions **/
$ie[0] = true \land timer.timer.e$
$ie[1] = true \land timer.done.e \land light.done.e$

$ip[0] = ie[0] \land \lnot ie[1]$
$ip[1] = ie[1]$

$is[0] = ip[0] \land ((selector == 0) \lor (\lnot ip[selector] \land \lnot ip[1]))$
$is[1] = ip[1] \land ((selector == 1) \lor (\lnot ip[selector]))$

$timer.timer.e = (timer.\ell == s_0) \land (t < n)$
$timer.done.e = (timer.\ell == s_0) \land (t \geq n)$
$light.done.e = true \land ((light.\ell == G)$ 
    $||(light.\ell == R)$
    $||(light.\ell == Y))$
    
$timer.timer.s = is[0]$
$timer.done.s = is[1]$
$light.done.s = is[1]$

/** Next state functions: Interactions **/
$timer.n = (cycle == 0)? (is[1]? light.m : timer.n) : timer.n$

/** Next state functions: Transitions **/
$timer.t = (cycle == 1)?$ 
 $((timer.timer.s)? (timer.t + 1) :$
 $((timer.done.s)? 0 : timer.t)) : timer.t$;
$light.m = (cycle == 1)?$
 $((light.done.s \land light.\ell == G)? 5 :$
 $((light.done.s \land light.\ell == Y)? 10 :$
 $((light.done.s \land light.\ell == R)? 3 : light.m))) : light.m$
$timer.\ell = (cycle == 1)?$
 $((timer.timer.s)? s_0:$
 $((timer.done.s)? s_0:timer.\ell)):timer.\ell$
$light.\ell = (cycle == 1)?$
 $((light.done.s \land light.\ell == G)? Y :$
 $((light.done.s \land light.\ell == Y)? R :$
 $((light.done.s \land light.\ell == R)? G : light.\ell))) : light.\ell$
\end{lstlisting}
%%%%%%%%%%%%%%%%%%%%%%%%%%%%%%%%%%%%%%%%%%%%%%%%%%%%%%%%%%%%%%%%%%%%%%%%%%%%%%%%%%%%%%%%%%
%%%%%%%%%%%%%%%%%%%%%%%%%%%%%%%%%%%%%%%%%%%%%%%%%%%%%%%%%%%%%%%%%%%%%%%%%%%%%%%%%%%%%%%%%%
%%%%%%%%%%%%%%%%%%%%%%%%%%%%%%%%%%%%%%%%%%%%%%%%%%%%%%%%%%%%%%%%%%%%%%%%%%%%%%%%%%%%%%%%%%
%
%%
%%This section presents how we translate a BIP system $(B,\mathit{Init}, v)$ to its corresponding sequential circuit $\big( (V, E),G, O\big)$. 
