\section{One loop program (\caig) Syntax and Semantics}
\label{sec:this}
A {\em one loop program} (\caig) uses Boolean, 
integer and array variables. 
A variable can be either a register denoting a memory element or a {\em wire} denoting a functional macro.
An \caig starts with variable declarations followed by the wire variable definitions. 
Then, memory variables are initialized simultaneously using the \cci{do-together} construct. 
After initialization, an infinite loop keeps updating the value of the memory variables simultaneously. The listings in Figure \ref{fig:gr} shows the syntax of an \caig. 


The wires are defined by a list of assignments, named \cci{wiredef-list}. 
Each wire can be the target of at most one assignment statement. 

If a wire is not assigned, then it is said to be a non-deterministic {\em primary input}. 
It takes a new non-deterministic value at each iteration of the loop.

The list of statements \cci{init-list} assigns initial values to 
the register variables. All the assignment statements within \cci{init-list} execute
simultaneously as indicated with the \cci{do-together} construct.

Similarly, the \cci{next-list} list of statements updates the values 
of the register variables. 
%All the assignment statements within \cci{next-list} execute simultaneously as indicated with the \cci{do-together} keyword.

Each assignment statement has a left-hand side target term 
which is either a variable or an access operator to an 
array element. 
The right-hand side of an assignment is a combinational expression over the program variables,  
Boolean and arithmetic operators, and a ternary choice 
operator. 
A combinational expression depends on its input variables only and does not require additional 
memory elements to be executed. 
The ternary choice \cci{(a? b : c)} returns $b$ if $a$ 
is \true and $c$ otherwise. 

\begin{figure}
\begin{tabular}{p{3cm}p{0.5cm}p{12cm}}
\begin{lstlisting}
decl-list

wiredef-list

do-together {
  init-list 
}

while(true) {
  do-together {
    next-list
  } 
}
\end{lstlisting}
&
&
\begin{lstlisting}
type: bool | int | bool [NUM] | int [NUM]; 
declaration: wire type id; | type id;

expr: term | uop expr| expr bop expr | expr ? expr : expr;
term: id | id[expr]; 

decl-list: declaration+
assignment: term = expr
wiredef-list: (assignment)*

init-list: (assignment)* 
next-list: (assignment)* 
\end{lstlisting}
\end{tabular}
\caption{\caig Syntax}
\label{fig:gr}
\end{figure}


\begin{definition}[\caig Trace]
A trace of an \caig program consists of the sequence of valuations of all non-wire variables. The semantics of an \caig program is given by its set of possible traces.  
\end{definition}
