\section{\this components}
\label{s:this}
The grammar in Table~\ref{t:gr} describes \caig grammar. 
A \caig component starts with a list of declarations of 
wire,  register variables, and arrays of them. 

Then wires are defined in a list of assignment statements. 
Each wire can be the target of at most on assignment statement. 
If a wire is not assigned then it is a {\em primary input}. 

The \cci{init} list of statements assign initial values for 
the register variables. 
All the assignment statements within \cci{init} execute
simultaneously as indicated with the \cci{do\_together} keyword.

Similarly, the \cci{next} list of statements update the values 
of the register variables. 
All the assignment statements within \cci{next} execute 
simultaneously as indicated with the \cci{do\_together} keyword.

Each assignment statement has a left hand side target term 
which is either a variable or an access operator to an 
array element. 
The right hand side of an assignment is a combinational expression
with Boolean and arithmetic operators and a ternary choice 
operator. The ternary choice \cci{(a? b : c)} returns $b$ if $a$ 
is \true and $c$ otherwise. 

The translation of an \caig component into a sequential circuit 
proceeds as follows. 
For each register variable in the component instantiate a 
vector of bit registers. 

\begin{table}
\begin{tabular}{p{6cm}p{6cm}}
\begin{Verbatim}
component: decl wiredef init while(true) { next } 
type: bool | int | bool [NUM] | int [NUM]; 
declaration: wire type id; | type id;

decl : declaration+
wiredef: (term = expr;)*
term : id | id[expr]; 

init: do_together { (term = expr;)* }
next: do_together { (term = expr;)* }
expr : term | uop expr| expr bop expr | expr ? expr : expr;
\end{Verbatim}
&
\begin{lstlisting}
decl-list

wiredef-list

do-together {
  init-list
}

while(true) {
  do-together {
    next-list
  }
}
\end{lstlisting}
\end{tabular}
\caption{\caig component grammar}
\label{t:gr}
\end{table}
