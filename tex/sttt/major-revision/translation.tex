\section{BIP to \caig}
\label{sec:bip2aig}
%%%%%%%%%%%%%%%%%%%%%%%%%%%%%%%%%%%%
%%%%%%%%%%%%%%%%%%%%%%%%%%%%%%%%%%%%
%
Given a BIP system $\Pm = (B, Init, v)$, \biptool~
calls function \cci{BIP-to-OLP} (see \figref{fig:bip-to-abc-function}) to translate \Pm into 
an \caig~program \redbox{including an encoding of the semantics of interactions and priorities}. 
It calls four functions that fill \cci{decl-list}, \cci{wiredef-list}, \cci{init-list}, \cci{next-list}. 
All these functions use the \cci{append} call to add code fragments to lists. 
%
\begin{figure}
\begin{lstlisting}
/*@\textbf{BIP-to-OLP}@*/(B, Init, v)
  generateDeclarationList()
  generateWireDefList()
  generateInitList()
  generateNextList()
\end{lstlisting}
\caption{Translation of BIP system into an \caig~program.}
\label{fig:bip-to-abc-function}
\end{figure}
%

\begin{enumerate}
\item Function \cci{generateDeclarationList()} (see \figref{fig:generate-decl-list}) fills \cci{decl-list} as follows. 
It creates three arrays of wires to denote interaction semantics. 
The elements of array $\mathit{ie}$ denote whether all logical constraints except priority rules are met for a given interaction. 
The elements of array $\mathit{ip}$ denote whether a given interaction is enabled after applying priority rules. 
The elements of array $\mathit{is}$ denote whether an enabled interaction is selected for execution. 
Currently, one interaction is selected to avoid executing conflicting interactions. 
Two interactions are conflicting if they involve the same components. 
\redbox{The centralized Engine allows only to execute one interaction at a time. 
BIP also allows the generation of distributed implementations~\cite{BonakdarpourBJQS12} where non-conflicting interactions can be executed in parallel. However, an additional layer is added to resolve conflict, which may introduce a huge overhead due to the communication between the layers. Additionally, when the interactions do not contain big computations, which is the case in general (only data transfer), executing non-conflicting interactions may not be useful at the cost of the communication overhead.
Nonetheless, in our case, it is possible to simply add a circuit to select all non-conflicting interactions, which is ongoing work.  
}

Wire $\mathit{selector}$ is a non-deterministic primary input used to select one of the enabled interactions.
Boolean register $\mathit{cycle}$ is used to denote whether the system is executing actions corresponding to either interaction or transition. 
Function \cci{generateDeclarationList()} also declares two wires ($B_i.p_j.e$ and $B_i.p_j.s$) for each port $p_j$. 
For a port $p_j$, wire $B_i.p_j.e$ indicates whether the port is enabled and wire $B_i.p_j.s$ indicates whether the port is selected by the interaction for execution. 
Moreover, for each component $B_i$ the function declares a register variable $B_i.\ell$ denoting the current location of $B_i$. 
Similarly, the function declares a variable register $B_i.x_j$ for each variable $x_j$ in component $B_i$.  

\begin{figure}
\begin{lstlisting}
/*@\textbf{generateDeclarationList()}@*/
  // interaction enablement wires
  append $wire~bool~ie[|J|]$ to decl-list
  // interaction priority wires
  append $wire~bool~ip [ |J| ]$ to decl-list 
  // interaction selected wires
  append $wire~bool~is[|J|]$ to decl-list 
  append $bool~b[|J|]$ to decl-list 
  // non-deterministic priority selector wire
  append $wire~int~selector$ to decl-list 
  // cycle denotes transition or interaction mode
  append $bool~cycle$ to decl-list  

  foreach $i \in [1..|I|]$
    foreach $j \in [1..|B_i.P|]$ 
      // port enablement
      append $wire~bool~B_i.p_j.e$ to  decl-list 
      // port selected
      append $wire~bool~B_i.p_j.s$ to  decl-list 
    endfor

    // location registers
    append $int~B_i.\ell$ to decl-list
    
    foreach $j \in [1..|B_i.X|]$ 
      // variable registers
      append $int~B_i.x_j$ to  decl-list 
    endfor
  endfor
\end{lstlisting}
\caption{\cci{generateDeclarationList()} function.}
\label{fig:generate-decl-list}
\end{figure}
%
\item Function \cci{generateWireDefList()} (see \figref{fig:generate-wire-list}) fills \cci{wiredef-list} with functional macro definitions as follows.
The enable wire $B_i.p_j.e$ is \true when there exists a transition $\tau$ labeled with port $p$, its source ($\tau.\mathit{src}$) is the current location ($B_i.\ell$), and its guard is \true. 

Array element $\mathit{ie}[j]$ corresponding to interaction $a_j$ is evaluated to \true when the guard of $a_j$ is \true and all its ports are enabled.
Array element $ip[j]$ is evaluated to \true when $ie[j]$ is \true and $a_j$ has higher priority than other enabled interactions. 
\redbox{
Array element $is[j]$ is evaluated to \true when $ip[j]$ is true and either (1) $a_j$ is selected ($selector$ equals to $j$), or (2) the selected interaction is not enabled and all interactions with index greater than $j$ are not enabled. 
}
The Boolean bit-vector $b$ is redundant with wire $is$ and is declared to simplify the proof of Theorem \ref{theorem:correct}. 
The use of a non-deterministic selector is added for fairness. 
The selected wire $B_i.p_j.s$ is \true when there exists a selected interaction $a_k$ (i.e., $is[k]$ is \true) involving $B_i.p_j$.
%
\begin{figure}
\begin{lstlisting}
/*@\textbf{generateWireDefList()}@*/
  // iterate over components
  foreach $i \in [1..|I|]$ 
    // iterate over component ports
    foreach $j \in [1..|B_i.P|]$ 
      append $B_i.p_j.e := \bigvee_{\tau \in transitions(B_i.p_j)}$ $\tau.guard \land B_i.\ell = \tau.src$ to  wiredef-list 
    endfor
  endfor
  
  // iterate over interactions
  foreach $j \in [1..|J|]$ 
    append $ie[j] := a_j.guard \wedge \bigwedge_{p\in a_i.P} component(p).p.e$ to  wiredef-list 
    append $ip[j] := ie[j] \wedge \,(\forall k \neq j: ie[k] \Rightarrow a_k < a_j)$ to  wiredef-list 
    append $is[j] := ip[j] \wedge \,\big(selector = j \, \vee$ $(\lnot ip[selector] \land \forall k > j: \neg ip[k])\big)$ to  wiredef-list 
  endfor
  
  // iterate over components
  foreach $i \in [1..|I|]$ 
    // iterate over component ports
    foreach $j \in [1..|B_i.P|]$ 
      append $B_i.p_j.s := \bigvee_{a_k \in interactions(B_i.p_j)} is[k]$ to  wiredef-list 
    endfor
  endfor
\end{lstlisting}
\caption{\cci{generateWireDefList()} function.}
\label{fig:generate-wire-list}
\end{figure}

\item Function \cci{generateInitList()} (see \figref{fig:generate-init-list}) fills \cci{init-list} with initial value definitions taken from $\mathit{Init}$ for location variables ($B_i.\ell)$ and $v$ for component variables ($B_i.x_j)$.
Register variable $\mathit{cycle}$ is initialized to $\false$ to denote an interaction execution mode. 

\begin{figure}
\begin{lstlisting}
/*@\textbf{generateInitList()}@*/
  // initialize to interaction mode
  append $cycle := 0$ to init-list 
  foreach $i \in [1..|I|]$
    append $B_i.\ell := Init.B_i$ to  init-list 
    foreach $j \in [1..|B_i.X|]$
      // v is the initial valuation
      append $B_i.x_j := v(B_i.x_j)$ to init-list  
    endfor
  endfor
  // iterate over interactions
  foreach $j \in [1..|J|]$ 
    append $b[j] = is[j]$ to  init-list 
  endfor
\end{lstlisting}
\caption{\cci{generateInitList()} function.}
\label{fig:generate-init-list}
\end{figure}

\item Function \cci{generateNextList()} (see \figref{fig:generate-next-list}) fills \cci{next-list} with the next state value definitions of register variables.
Each component variable can be modified either in an interaction action or in a transition action.
The value of variable $\mathit{cycle}$ makes this distinction. 

In the interaction mode (when $\mathit{cycle}$ is equal to $\false$), the function considers each assignment statement $\sigma$ from the action of interaction $a_j$.
The function appends a conditional clause requiring $a_k$ to be selected for execution so that the target variable $B_i.x_j$ of $\sigma$ is assigned to the expression of $\sigma$ ($\sigma.\mathit{expr}$). 
The sequence of conditional clauses forms a nested ternary conditional expression where the last expression retains the previous value of the variable. 

Similarly, in the transition mode ($\mathit{cycle}$ equals to $\true$), the function considers each assignment statement $\sigma$ from the action of transition $\tau$. The function appends a conditional clause requiring the port of the transition $\tau$ to be selected for execution and the location of the component to be equal to the source of the transition. The target variable $B_i.x_j$ of $\sigma$ is assigned to the expression of $\sigma$ ($\sigma.expr$). 

In the transition mode, the function considers the current location of each component $B_i.\ell$ and appends a conditional clause requiring the transition source to be equal to the current location and the port of the transition to be selected. The expression corresponding to the conditional clause updates the current location to be the destination of the transition ($\tau.dest$).  In the interaction mode, the location retains its value.
Finally, variable $\mathit{cycle}$ is toggled. 

% The sequence of conditional clauses form a nested ternary conditional expressions where the last expression retains the previous value of the variable.
\begin{figure}
\begin{lstlisting}
/*@\textbf{generateNextList()}@*/ 
  // iterate over interactions
  foreach $j \in [1..|J|]$ 
    append $b[j] = is[j]$ to  next-list 
  endfor

  // iterate over components - interaction-mode
  foreach $i \in [1..|I|]$ 
    // iterate over variables, where 
    // $\textcolor{darkgreen}{B_i.X = \{x_1, \ldots, l_{|B_i.X|}\}}$ 
    foreach $j \in [1..|B_i.X|]$ 
      // interaction mode
      append $B_i.x_j := cycle = 0?$ to var-st
      // iterate over interactions
      foreach $k \in [1..|J|]$ 
        // iterate over interaction assignments
        foreach $\sigma \in a_k.action$
          if($B_i.x_j = \sigma.term$)
            append $is[k]?\, \sigma.expr:$ to var-st
          endif
        endfor
      endfor
      // interaction mode and no data transfer for $\textcolor{darkgreen}{B_i.x_j}$
      append $B_i.x_j$: to var-st 
      
      // iterate over component transitions - 
      // transition-mode
      append $B_i.\ell := cycle = 0?\, B_i.\ell :$ to loc-st
      foreach $\tau \in B_i.T$ 
        // iterate over transition assignments
        foreach $\sigma \in \tau.action$
          if($B_i.x_j = \sigma.term$)
            append $(B_i.port(\tau).s \wedge \tau.src = B_i.\ell)?\, \sigma.expr:$ to var-st 
          endif
        endfor
        append $(B_i.port(\tau).s \wedge \tau.src = B_i.\ell)?\, \tau.dest:$ to loc-st 
      endfor  
      
      append $B_i.x_j$ to var-st 
      append var-st to  next-list 
      
      append $B_i.\ell$ to loc-st 
      append loc-st to  next-list 

    endfor
    // switch cycle
    append $cycle := \neg cycle$ to  next-list 
  endfor
\end{lstlisting}
\caption{\cci{generateNextList()} function.}
\label{fig:generate-next-list}
\end{figure}
%
\end{enumerate}
%%%%%%%%%%%%%%%%%%%%%%%%%%%%%%%%%%%
\subsection{Correctness}
%%%%%%%%%%%%%%%%%%%%%%%%%%%%%%%%%%%
%
Given a BIP system $S$ and its corresponding \caig program $P = \cci{BIP-to-OLP(S)}$. 
Let $Tr_s$ be the set of traces of $S$ and let $Tr_p$ be the set of traces of $P$. 
Consider $T'$ the projection of $Tr_p$ constrained by omitting the states where \cci{cycle} is equal to $\false$. 
Formally, $T'  = \{t' \,\mid\, t'[i] = t[2\times i] \wedge t \in Tr_P \wedge i \in \mathbb{N}\}$. 
Intuitively, $T'$ represents the semantics of the original BIP model regardless of the built-in scheduler details 
(i.e., the enable exchange, interaction selection, data transfer details). 
%
\begin{theorem}[BIP \caig Equivalence]
\label{theorem:correct}
The BIP system $S$ is semantically equivalent to $P$: $Tr_s=T'$. 
\end{theorem}
\begin{proof}
\label{proof:equi-correct}
%
The proof is done by induction on the length of traces and 
on the structure of $S$ and $P$.

{\bf Left case: $Tr_s \subseteq T'$.~~}
Consider $t \in Tr_s$, there exists $t' \in T'$ and $t = t'$. 
\begin{itemize}
\item
\textit{Induction basis.}
Consider the initial state: $t[0]$ and $t'[0]$. 
\cci{generateInitList()} sets \cci{cycle} to $\true$ and 
assigns each location to \textit{Init} and each variable 
to its initial value. Thus, $t[0]$ is equal to $t'[0]$. 
\item
Let $t[0..k]$ be the prefix of $t$ of length $k+1$ where 
$k\ge 0$, using the induction hypothesis, 
there exists at least one trace 
$t'[0..k] \in T'$ and $t[0..k]=t'[0..k]$. 

Consider valuations $t[k]$ and $t'[k]$ 
that correspond to the firing interactions.
Since they are equal, the next state of locations
and data variables will be the same at $t[k+1]$ and 
$t[k+1]$ as enforced by \cci{generateNextList()}. 

In case no interactions are enabled at step $k+1$, 
both $P$ and $S$ will preserve the same state at step $k+1$ 
and thus $t[0..k+1] =t'[0..k+1]$. 
Otherwise, let $a_j$ be \redbox{an} interaction that is enabled 
according to state $t[k+1]$, 
then according to BIP semantics, interaction $a_j$ must be 
also enabled and of the highest priority. 
This means that $a_j$ is also enabled and of the highest 
priority under $t'[k]$. 
The primary input \cci{selector} wire variable is a 
nondeterministic variable and can assume all indexes on 
interactions including the value 
$j$, thus setting the enabled interaction $a_j$ in $t'[k+1]$. 
Therefore, there exists a setting for \cci{selector} at 
step $k+1$ such that $P$ executes $a_j$. 

Without loss of generality, 
let that nondeterministic setting be the one in $t'[k+1]$. 
Therefore, $t[k+1] = t'[k+1]$. 
\end{itemize}
%

{\bf Right case: $T' \subseteq Tr_s$.~~}
Consider $t' \in T'$ there exists $t \in Tr_s$ and $t = t'$.
%
\begin{itemize}
\item
\textit{Base case.}
The base case is similar as before and $t[0]$ is equal $t'[0]$.
\item
\textit{Induction case.}
Let $t'[0..k]$ be the prefix of $t'$ of length $k+1$ 
where $k\ge 0$, using the induction hypothesis, 
there exists a trace 
$t[0..k] \in Tr_s$ and $t[0..k]=t'[0..k]$. 
As for the previous proof, \cci{generateNextList()} 
guarantees that the state of the locations and the data 
variables are equal in the next states $t[k+1]$ and 
$t'[k+1]$ of $S$ and $P$, respectively. 

Furthermore, the two states are equivalent if no interaction 
was enabled in $t'[k+1]$. 

In case an interaction $a_j$ is selected to execute in $P$ 
at step $k+1$ as set in state $t'[k+1]$, 
then it must be enabled and of high priority.
Thus, it must also be enabled and of high priority in $S$ 
according to state $t[k+1]$
as guaranteed by \cci{generateNextList()}. 
The semantics of BIP allows the nondeterministic selection of
one of the enabled and high priority interactions, 
and without loss of generality let that selection 
be $a_j$ in $t[k+1]$. 
Therefore $t[k+1]=t'[k+1]$. 
\end{itemize}
%
Consequently, the claim holds as expected. 
\end{proof}
%
%%%%%%%%%%%%%%%%%%%%%%%%%%%%
\subsection{One Cycle Optimization}
%%%%%%%%%%%%%%%%%%%%%%%%%%%%
%
Recall that an interaction specifies a strong synchronization among its involved components.
Data transfer can take place during such synchronization.
The operational semantics of BIP requires to (1) first execute the data transfer of the selected interaction, and then to (2) execute the functions of the corresponding transitions of atomic components.
For this purpose, in the above translation, we used $\mathit{cycle}$ Boolean register to denote whether the system is executing actions corresponding to either interaction or transition.
However, in some cases, data transfers of all interactions modify some variables that are not assigned in the corresponding transitions of those interactions.
This can be detected by doing a static data dependency between interactions and their transitions. 
This may drastically improve the performance of the system since data transfers as well as functions of transitions may be executed in one cycle. Note that, our implementation supports this optimization. 
\redbox{Moreover, it is possible to do source-to-source transformations to compose the effect of data transfer, and hence one cycle-based implementations could be always generated.}  
%
