% !TEX root = answers.tex
\section{Reviewer 2 - Report 101-R-407}
%
%%%%%%%%%%%%%%%%%%%%%%%%%%%%%%%%%%%%%%%%
%%%%%%%%%%%%%%%%%%%%%%%%%%%%%%%%%%%%%%%%
%
The paper present a method to verify BIP models. This is accomplished by
translating the BIP specification into a synchronous circuit, and using ABC to
perform the verification. The method is rather straightforward. Since BIP has
essentially a state machine semantics, all is needed is to create a list of
expressions that update the state variables at each clock cycle. Then, this
representation is translated into an AIG circuit and fed to ABC for
verification.

Overall the paper is interesting from a tutorial point of view, however the
method employed is not particularly innovative. Specifically, this kind of
translations have already been developed for several high level models, such
as the synchronous languages. For instance, Esterel can be compiled to Verilog
or BLIF (see the Columbia Esterel Compiler). How does your method differ?

\answer{TODO Fadi}

I think the interesting part is the generation of the scheduler and the
detection of the properties, which I think should be expanded at the expense
of the details of the translations. For instance, it is rather obvious how one
would go about translating the OLP into the AIG. Instead, it is not obvious
how to generate the output that says whether the system is deadlock free: do
you use DFinder to generate the corresponding invariant? Also, it is not clear
how you define the invariants. You show them in the benchmarks. But are these
part of the BIP specification, or are they defined separately?

\answer{TODO Jaber - user defined invariant or predefined patterns such as deadlock (no interaction is enabled)}


Regarding the scheduler, you describe the procedure in Section 5. Here, it
would be easier for the reader if you explained the method by which you
activate the interactions, rather than dividing the procedure into the various
steps. Essentially you need to select an interaction, and you build a circuit
to do so. Describe the circuit, and perhaps put a picture showing how the
different arrays are related.

\answer{TODO Fadi and Mohamad} 


I was in particular struck by this statement:

"Currently, one interaction is selected to avoid executing conflicting
interactions."

Does the BIP framework do the same, or is it able to execute several
interactions at a time, if they are not conflicting? If so, does this mean
that your semantics is slightly different? To my understanding, wire
"selector" makes a non-deterministic choice, which is evaluated under all
possible values during verification. Is this right? I.e., does the
verification account for all possible selections? But still, not for
concurrent selections?


\answer{Jaber TODO}


It would be extremely interesting to compare the OLP to the code that is
generated for NuSMV. In my understanding, NuSMV uses a very similar notation:
the transition relation is defined using next-state expressions. In fact, I
would argue that you could simply start from the generated NuSMV to derive the
circuit, instead of introducing the OLP. Why didn't you follow this path? Are
there differences in the way for instance the schedulers are generated?

Understanding these differences could also shed light into the evaluation:
right now, it is difficult to tell whether the improvement in performance is
because ABC is a lot better than NuSMV, or whether the code that you generate
for ABC is better than the one you generate for NuSMV. What if, for instance,
you were to generate NuSMV code out of the OLP? How would NuSMV perform
compared to the original code? Also, you mention you perform certain
reductions outside ABC. Do you include these reductions also for NuSMV?

\answer{TODO, extend the related work, 
as we indicated in the related work section our technique could benefit from the translation from BIP to NuSMV however, we have already done a tool that does translation from olp to aig...}

A few minor things:

\begin{itemize}
\item The text of the second paragraph of step 2 (section 5, page 9) must be
revised ("is true either"? "interaction greater with an index greater"?).
\answer{todo}

\item On page 3, I don't understand the meaning of: "The interaction bit vector is
evaluated in real-time". Does the traditional BIP main loop not evaluate the
interactions in real time?
\answer{todo}

\item There is confusion when you says that you denote by p.X the set of variables
assigned to the port p. X was previously (two lines above) used to denote
the set of all variables (I assume), and now you use it to denote $x_p$, which
was defined as a subset of X. Please change notation.
\answer{todo}
\end{itemize}



