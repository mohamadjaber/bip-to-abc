\section{Reviewer 1 - Report 101-R-441}
The paper deals with (formal) verification and synthesis of embedded systems.
The approach suggested by the authors is to start with a high-level 
specification of the target system using the BIP framework, from which both a 
C implementation and a sequential circuit are generated.
The former is used for simulation and runtime verification, whereas the later 
is the input of existing verification tool (ABC).
The main contributions of the paper are the definition and implementation of 
automatic transformations from BIP to one loop programs (OLPs) and 
corresponding C implementations, and from OLP to And-Inverted-Graphs (AIGs) 
which are representations of sequential circuits considered by the ABC 
verification framework.
The proposed tool-chain is also capable to map back counter-examples found by 
ABC to original BIP models, and to visualize them through human readable wave 
forms.
Such tool-chain is evaluated on two examples.

The paper contributes to existing work in the domain of verification of BIP 
models.

It does not provide a direct comparison with the verification tool DFinder, 
but as explained by the authors, since the current implementation of DFinder 
does not consider transfer of data between components it is only applicable 
to a restricted class of systems.


\answer{Exactly. DFinder does not handle system models with data transfer. 
This limitation hampers the practical application of DFinder and of the BIP framework, since data transfer is necessary and common in the design of real-life systems~\cite{QiangB15}. We have modified the text to elaborate more on this.}


As shown by the experimental results, the verification with ABC on the AIGs 
generated from BIP scaled better to large systems than existing NuSMV-based 
model-checking approaches for BIP.

However, such results are mainly due existing reduction algorithms 
implemented in ABC, the contribution of the paper regarding verification of 
BIP models being only the translations from BIP to OLP, and from OLP to AIG 
(which is the input format of ABC).
Moreover, these transformations are the core results of the paper, but their 
presentation should be improved as they lack too many details (apart from 
Section 2).

\answer{
%
The transformation from BIP to OLP is discussed in 
Section 6 (BIP to OLP) 
with detailed source to source algorithms that are ready 
to be implemented. 
For example, algorithm generateDeclarationList in Figure 9 
takes a BIP system and generates the declaration list. 
Similarly, algorithms generateWireDefList, generateInitList, 
and generateNextList in Figures 10, 11 and 12, respectively 
generate the other components of the OLP program. 
They are all called from the transformation algorithm in 
Figure 8. 
%
The transformation from OLP to AIG is also detailed in 
source to source algorithms. Algorithm {\bf variable} 
creates that primary inputs, the registers and the wires. 
Algorithm {\bf traverse} recursively traverses the expressions
and builds corresponding circuits for them based on a 
{\bf library}
of gates that correspond to the primitive unary 
and binary operations. 
Algorithm {\bf olp-to-aig}  uses the {\bf variables} and 
{\bf traverse} algorithms to instantiate the registers
and primary inputs and then connect the initial state and 
next state functions to their corresponding registers. 
%
We added details with formal definitions of OLP 
syntax and semantics to further address this point. 
%
In case the reviewer is referring to the ABC algorithms, 
we provide now a description of some of them in the Appendix.
}

For instance, 
Section 3 is too short and I would expect it to include a 
proper (formal) definition of the semantics of one loop programs.
My main concern regards proofs of both theorems: they are established 
informally and without the necessary details, which make them not very 
convincing and strongly limits their interest.
For these reasons I think the paper cannot be accepted in its present form, 
and can only be considered for publication if (i) better definitions of the 
semantics of OLP and AIG are provided, and if (ii) proofs are updated w.r.t. 
these semantics and properly formalized.

\answer{
  We restructured the section to separate OLP syntax
  and semantics and provided formal definitions for OLP 
  constructs and semantics. 
  We also modified the definition of AIG traces. 
  We added a definition for a {\em full AIG trace} that is 
  a sequence of valuation of all gates. 
  We defined an {\em AIG trace} to be a sequence of AIG states. 
  We modified the AIG OLP equivalence theorem to claim 
  trace equivalence between the OLP program and the corresponding
  AIG. The proof is now by induction on the length of the traces. 
  The BIP OLP equivalence proof is still based on full semantic 
  equivalence. We modified it slightly to highlight both proof
  directions: traces of OLP include traces of BIP and
  traces of BIP include traces of OLP. 
}

My other comments and suggestions are provided below.

\begin{itemize}
\item p.1: "Runtime verification [...] exhibit expensive [...]." $\rightarrow$ "Runtime 
verification [...] exhibitS expensive [...]"
\done
\item p.2: "However, DFinder [...] only the verification of deadlock-freedom.": 
This is not correct, the method is applicable to the verification of any 
safety property, although being based on abstractions it performs better for 
a sub-class of safety properties, e.g. deadlock-freedom.

\answer{We agree with the reviewer's comment. The internal implementation of DFinder is based on the computation of invariants, and the method is application to the verification of any safety property. However, the available tool (online) does only support deadlock freedom detection. 
In order to verify safety property, we need to manually instrument the system so that it enters a deadlock state when the property is violated. We modified the text to reflect this. 
Moreover, as we mentioned before, DFinder does not handle system models with data transfer. We have built several simple examples with data transfer and an obvious deadlock where DFinder failed to detect them.}

\item p.3: "BIP is based on the generation [...] and the so-called BIP engine, 
which simulates BIP semantics [...]" $\rightarrow$ I would prefer "BIP is based on the 
generation [...] and the so-called BIP engine, which interprets the BIP 
semantics [...]"
\done

\item p.3: "1. Each atomic component sends [...] its current location.": 
Technically in engine-based execution of BIP models are shared-memory 
execution, i.e. states of components are not sent nor duplicated but simply 
accessed (read) by the engine. In multi-thread execution, components only 
send notification massages (without data) when they complete a transition 
execution.

\answer{We agree with the reviewer's comment. However, from the semantics point of view atomic components send their states. 
The send is simply read (by the engine) in case of the single-thread (shared-memory) implementation, however, in case of distributed implementation the send will be implemented as an actual send. For this, we preferred (and to be consistent with the BIP related work) to use "send".}


\item p.3: "We recall the necessary concepts [...] desin:[...]" $\rightarrow$ "We recall the 
necessary concepts [...] design:[...]"
\done

\item p.4, §2: $"x := f^x(X) \in f_{\tau}" \rightarrow "(x,f^x(X)) \in f_{\tau}"$
\done

\item p.4: "The semantics [...] is an LTS over configurations [...]": At this 
stage of the paper configurations are not defined. It would be good to either 
give an informal definition of them here, or to move their formal definition 
(the one after Def. 3) here.
\done
\answer{We moved the definition so that the "configurations" are well defined.}


\item p.4, Def.3: This presentation of the semantics using $v_p$ is useful for 
further composition of components including transfer of data, but it should 
be commented otherwise it is very hard to understand. It would be good to 
state that valuations $v_p$ are parameters that are further instantiated when 
composing components w.r.t. data transfer functions considered for 
connectors, and to cite existing papers which have the same presentation of 
the BIP semantics. Moreover, in Definition 3 the notation $v/v_p$ is not 
defined which brings additional fuzziness.

\done
\answer{
We have added that valuation of the variables attached to a port will be further instantiated when 
composing components, and added some references. Moreover, we have added a preliminary section where we introduced preliminary concepts and notations such as $v/v_p$
}


\item p.4, § after Def.3: Explanations provided here do not provide any 
clarification to Def.3. I would expect more here, as explained in the 
previous point.

\done

\item p.4: "2.1.2 Creating composite components" $\rightarrow$ "2.1.2 From Atomic to 
Composite Components"
\done

\item p.4: "Assuming [...] how to connect the components in the set ${B_i}_{i \in 
I} with I \subseteq [1,n] [...]"$ $\rightarrow$ "Assuming [...] how to connect a subset 
${B_i}_{i \in I}, I \subseteq [1,n]$, of the components [...]"
\done

\item p.4, Def.4: To make sense, when you should compose components with disjoint 
sets of ports and variables.
\item p.4: "The meaning of the above rule [...]" $\rightarrow$ "The meaning of the rule of 
Figure 2 [...]"
\item p.4, Def.6: "[...] is a state of $B_i$." $\rightarrow$ "[...] is a configuration of 
$B_i$.". In general avoid using different words for the same thing, especially 
if you properly defined one of them in your context.
\done
\answer{We updated the paper to only use state.}


\item p.5, Fig.2: According to Def.6, variables $v_i$ correspond valuations 
associated to configurations $q_i = (l_i, v_i)$. Use another symbol to denote 
the result of the transfer of data $F_a({ v_{p_i} }_{i \in I})$!
\done

\item p.5, Def.10: "$\exists a_i$ [...]" $\rightarrow$ "$\exists a_i$ [...] and $b_i(index(a_i))$ is 
true"
\done

\item p.5 "[...] execute simultaneously [...]": I would prefer "[...] can execute 
simultaneously [...]" since parallel execution is a matter of implementation.
\done
\answer{We have restructured this section.}

\item p.6, Fig.4: This syntax is ambiguous, which is especially annoying for 
ternary choices (a? b : c). Please add parenthesis in expressions of Fig.5 
otherwise there is no way we can guess how to interpret them.
\done

\item p.6, Fig.5: "timer.timer.e = is[0]" $\rightarrow$ "timer.timer.s = is[0]"
\done

\item p.6, Fig.5: "timer.done.e = is[1]" $\rightarrow$ "timer.done.s = is[0]"
\done

\item p.6, Fig.5: "light.done.e = is[1]" $\rightarrow$ "light.done.s = is[0]"
\done

\item p.6, Fig.5: In the next-list, replace also ".e" by ".s".
\done

\item p.6 "Function G : V $\mapsto$ Types [...]" $\rightarrow$ "Function G : V $\to$ Types 
[...]"
\done

\item p.7: "[...] vertices in R or I [...]" $\rightarrow$ "[...] vertices which are either 
inputs or registers [...]"
\done

\item p.7, Def.15: This definition is not complete and should mention Figure 6. 
Moreover, the next paragraph is not useful.
\done

\item p.7: "node" $\rightarrow$ "vertex": Again, avoid using different words for the same 
thing.

\done

\item p.7: "fanouts": There is no proper definition of fanouts whereas there is 
one for fanins. Please add one (definitions of fanins and fanouts could be 
factorized using "resp.").
\done
\answer{We modified the definition of fanins to include fanouts
as well.}

\item p.8, Th.1: "[...] the set of traces of P is equal to the set of traces of 
A.": Technically, traces of P and traces of A are different mathematical 
objects, please establish correspondence between the two.
\done
\answer{The trace equivalence proof now considers single
states in both traces. }

\item p.8, Sec.5: "[...] with its own customized execution engine." $\rightarrow$ "[...] 
including an encoding of the semantics of interactions and priorities."
\done 
\item p.9, 1.: "Currently, [...] to avoid executing conflicting interactions. Two 
interactions [...].": This is misleading: the parallel execution of 
interactions is not only restricted by conflicts when priorities are 
considered. This is a topic in itself addressed by existing papers, please 
cite some of them here.
\done

\item p.9, 2.: "Array element is[j] [...] when ip[j] is true either [...]" $\rightarrow$ 
"Array element is[j] [...] when ip[j] is true and either [...]"
\done
\item p.9, 2.: "[...] and j is the first enabled interaction greater with an 
index greater than j": This does not make sense, please reformulate.
\done
\item p.9, 4.: "[...] when cycle is equal to zero [...]" $\rightarrow$ "[...] when cycle is 
equal to true [...]"
\done
\item p.9, 4.: "[...] when cycle is equal to one [...]" $\rightarrow$ "[...] when cycle is 
equal to false [...]"
\done
\item p.10, Sec 5.1: "[...] where cycle is equal to zero [...]" $\rightarrow$ "[...] where 
cycle is equal to false [...]"
\done
\item p.10: "[...] sets cycle to zero [...]" $\rightarrow$ "[...] sets cycle to true [...]"
\done
\item p.11: "Otherwise, let $a_j$ be the interaction [...]" $\rightarrow$ "Otherwise, let $a_j$ 
be an interaction [...]"
\done
\item p.11, Sec.5.2: "The operational semantic [...]" $\rightarrow$ "The operational 
semantics [...]"
\done
\item p.11, Sec.5.2: Why can't you simply compose the effect of data transfer 
transition execution to have one-cycle implementations in any case?
\answer{
Yes, it is possible to do source-to-source transformations to compose the effect of data transfer, and hence one cycle-based implementations could be always generated. This is pure technical and we are working on integrating it on our tool.
}
\item p.11, Sec.6.1: "The module [...] the generate AIG [...]" $\rightarrow$ "The module 
[...] the generated AIG [...]"
\done
\end{itemize}
