%%%%%%%%%%%%%%%%%%%%%%%%%%%%%%%%%%
%%%%%%%%%%%%%%%%%%%%%%%%%%%%%%%%%%
%
\subsection{Semantics of one loop programs.}
%

\redbox{
A variable can be either a register
denoting a memory element, 
or a wire denoting a functional macro. 
Memory variables are initialized simultaneously using the 
\cci{do-together} construct. 
After initialization, an infinite loop keeps updating the 
value of memory variables simultaneously. 
The listings in Figure \ref{fig:gr} shows the syntax of 
an \caig. 
}

\redbox{
If a wire is not assigned, then it is said to be a 
non-deterministic {\em primary input}. 
It takes a new non-deterministic value at each iteration of 
the loop.
The list of statements \cci{init-list} assigns initial values to 
the register variables.
Similarly, the \cci{next-list} list of statements updates 
the values of the register variables. 
The semantics of \caig expressions are defined by the typical 
valuation rules of the corresponding unary and binary operators. 
The ternary choice \cci{(a? b : c)} returns $b$ if $a$ 
is \true and $c$ otherwise. 
}

\redbox{
Formal semantics for \caig semantics that are given 
in terms of \caig state and trace follow. 
}

\redbox{
\begin{definition}[\caig state]
The state of an \caig program $P$ ranging over a set of 
non-wire variables $V=\set{v_1,v_2\ldots,v_n}$ is defined as 
the valuation 
$\sigma: V \mapsto D$. 
The valuation $\sigma$ maps variables in 
$V$ to $D=D_1\times D_2 \times \ldots \times D_n$. 
Domain $D_i$ is $\mathbb{B}$, $Data$,
$\mathbb{B}^k$ and $Data^k$ 
when 
$\mathit{type}(v_i)$ is \cci{Boolean}, \cci{int}, 
\cci{Boolean[k]}, and \cci{int[k]}, respectively 
where $1 \le i \le n$ and $k >0$.
\end{definition}
}

\redbox{
\begin{definition}[\cci{do-together} semantics]
All the assignment statements within \cci{init-list} and 
\cci{next-list} execute
simultaneously as indicated with the \cci{do-together} 
construct.
\end{definition}
}


\redbox{
\begin{definition}[\caig trace]
A trace $\pi$ of length $\ell$ of an \caig with variables 
$V=\set{v_1,v_2,\ldots,v_n}$ is a sequence 
of \caig states $\sigma_0,\sigma_1,\ldots,\sigma_{\ell -1}$. 
State $\sigma_0$ is defined as the valuations given by 
the $\mathit{init-state}(v_i)$ functions, $1\le i \le n$. 
State $\sigma_{k+1}$ corresponds to the valuations 
given by functions $\mathit{next-state}(v_i)$ 
where references to variables $v_j\in V$ are substituted by 
the corresponding valuations from $\sigma_{k}$, 
$0\le k \le \ell$. 
\end{definition}
}
