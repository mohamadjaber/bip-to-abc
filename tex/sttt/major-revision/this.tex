\section{One Loop Programs (\caig) - Syntax and Semantics}
\label{sec:this}
%%%%%%%%%%%%%%%%%%%%%%%%%%%%%%%%%%
%%%%%%%%%%%%%%%%%%%%%%%%%%%%%%%%%%
%
\paragraph{Syntax of one loop programs.}
%
A {\em one loop program} (\caig) uses Boolean, 
integer and array variables. 
A variable can be either a register denoting a memory element or a {\em wire} denoting a functional macro.
An \caig starts with variable declarations followed by wire variable definitions. 
Memory variables are initialized simultaneously using the \cci{do-together} construct. 
After initialization, an infinite loop keeps updating the value of memory variables simultaneously. The listings in Figure \ref{fig:gr} shows the syntax of an \caig. 

Wires are defined by a list of assignments, named \cci{wiredef-list}. 
Each wire can be the target of at most one assignment statement. 

If a wire is not assigned, then it is said to be a non-deterministic {\em primary input}. 
It takes a new non-deterministic value at each iteration of the loop.

The list of statements \cci{init-list} assigns initial values to 
the register variables.
All the assignment statements within \cci{init-list} execute
simultaneously as indicated with the \cci{do-together} construct.

Similarly, the \cci{next-list} list of statements updates the values 
of the register variables. 
%All the assignment statements within \cci{next-list} execute simultaneously as indicated with the \cci{do-together} keyword.

Each assignment statement has a left-hand side target term 
which is either a variable or an access operator to an 
array element. 
The right-hand side of an assignment is a combinational expression over the program variables,  
Boolean and arithmetic operators, and a ternary choice 
operator. 
A combinational expression depends only on its input variables and does not require additional  memory elements to be executed. 
The ternary choice \cci{(a? b : c)} returns $b$ if $a$ 
is \true and $c$ otherwise. 

\begin{figure*}
\begin{tabular}{p{3cm}p{0.5cm}p{10cm}}
\begin{lstlisting}
decl-list

wiredef-list

do-together {
  init-list 
}

while(true) {
  do-together {
    next-list
  } 
}
\end{lstlisting}
&
&
\begin{lstlisting}
type: bool | int | bool [NUM] | int [NUM]; 
declaration: wire type id; | type id;

expr: term | uop expr| expr bop expr | expr ? expr : expr;
term: id | id[expr]; 

decl-list: declaration+
assignment: term = expr
wiredef-list: (assignment)*

init-list: (assignment)* 
next-list: (assignment)* 
\end{lstlisting}
\end{tabular}
\caption{\caig Syntax}
\label{fig:gr}
\end{figure*}
%
\begin{example}
Figure~\ref{fig:caigtraffic} shows an \caig that corresponds to the BIP system for the traffic-light controller shown in \figref{fig:traffic:bip}. 

\begin{figure}
\begin{tabular}{p{4.2cm}p{.2cm}p{12.5cm}}
\begin{lstlisting}
/*** decl-List ***/
int timer.t;
int timer.n;
int light.m;
int timer.$\ell$;
int light.$\ell$;
bool cycle;

wire int selector;
wire bool timer.timer.e;
wire bool timer.timer.s;
wire bool timer.done.e;
wire bool timer.done.s;
wire bool light.done.e;
wire bool light.done.s;
wire bool ie[2];
wire bool ip[2];
wire bool is[2];
\end{lstlisting}
& & 
\begin{lstlisting}
/*** wiredef-list ***/
timer.timer.e = (0 == timer.$\ell$) && (timer.t < timer.n);

timer.done.e  = (0 == timer.$\ell$) && (timer.t == timer.n);
light.done.e  = (0 == light.$\ell$)  || (1 == light.$\ell$) || (2 == light.$\ell$);

ie[0] = timer.timer.e;
ie[1] = (light.done.e && timer.done.e);

ip[0] = ie[0];
ip[1] = ie[1];

is[0] = (ip[0] && ( selector == 0 || (!ip[selector]  && !ip[1]);
is[1] = (ip[1] && ( selector == 1 || (!ip[selector]);

timer.timer.e = is[0];
timer.done.e  = is[1];
light.done.e  = is[1] ;
\end{lstlisting}
\\
\vspace{-2em}
\begin{lstlisting}
do-together {
  /*** init-list ***/
  timer.t = 0; 
  timer.n = 10; 
  timer.$\ell$ = 0;

  light.m = 5; 
  light.$\ell$ = 0;

  cycle = true; 
}/* end do-together */
\end{lstlisting}
& & 
\vspace{-2em}
\begin{lstlisting}
while(true) {
  do-together {
    /*** next-list ***/    
    timer.n = cycle? is[1]? light.m : timer.n : timer.n; 
    
    timer.$\ell$ = cycle? timer.$\ell$: timer.timer.e && timer.$\ell$ == 0? 0 : timer.timer.e && timer.$\ell$ == 0? 0 : timer.$\ell$;
    
    timer.t = cycle? timer.t : timer.$\ell$ == 0 && timer.timer.e? (timer.t + 1) : timer.$\ell$ == 0 && timer.done.e? 0 : 
              timer.t; 
              
    light.$\ell$ = cycle? light.$\ell$ : light.$\ell$ == 2 && light.done.e? 0: light.$\ell$ == 1 && light.done.e? 0 : light.$\ell$ == 0 
              && light.done.e? 1 : light.$\ell$; 
              
    light.m = cycle? light.m : light.$\ell$ == 0 && light.done.e? 3: light.$\ell$ == 1 && light.done.e? 10: light.$\ell$ == 2 
              && light.done.e? 5 : light.m; 
    
    cycle = !cycle; 
  } /*end do-together*/ 
} /*end while(true)*/
\end{lstlisting}
\end{tabular}
\vspace{-2em}
\caption{Sample of $\caig$ generated code of traffic light system}
\label{fig:caigtraffic}
\end{figure}

\end{example}
%
In \secref{sec:bip2aig}, we shall see how to automatically translate a BIP system into \caig.
%
%\paragraph{Semantics of one loop programs.}
%
%The semantics of an \caig is defined in terms of its traces.
%
\begin{definition}[\caig Trace]
A trace of an \caig is the sequence of valuations of all non-wire variables.
\end{definition}
%
The semantics of an \caig is given by its set of possible traces.  
%
