% !TEX root = main.tex
%
\section{Preliminaries and Notation}
\label{sec:prelim}
%%%%%%%%%%%%%%%%%%%%%%%%%%%%%%%
%%%%%%%%%%%%%%%%%%%%%%%%%%%%%%%
%

We introduce some preliminary concepts and notations.
%
\paragraph{Functions.} For two functions $v\in [ X \rightarrow Y]$ and $v'\in [X' \rightarrow Y']$, the substitution function noted $v/v'$, where $v/v' \in [X \cup X' \rightarrow Y \cup Y']$, is defined as: $v/v'(x) = v'(x)$ if $x \in X'$ and $v(x)$ otherwise.
%



\paragraph{Transition Systems.}
%
Labeled Transition System (LTS) are used to define the semantics of BIP systems.
An LTS defined over an alphabet $\Sigma$ is a 3-tuple $\tuple{\mathrm{Lab},\mathrm{Sta},\mathrm{Trans}}$ where $\mathrm{Lab}$ is a set of labels, $\mathrm{Sta}$ is a non-empty set of states and $\mathrm{Trans}\subseteq \mathrm{Sta}\times \mathrm{Lab} \times \mathrm{Sta}$ is the transition relation.
A transition $\tuple{s,e,s'}\in\mathrm{Trans}$ means that the LTS can move from state $s$ to state $s'$ by consuming label $e$.
We abbreviate $\tuple{s,e,s'}\in\mathrm{Trans}$ by $s \stackrel{e}{\rightarrow}_\mathrm{Trans} s'$ or by $s\stackrel{e}{\rightarrow} s'$ when clear from context.
Moreover, $s \stackrel{e}{\rightarrow}$ is a short for $\exists s'\in\mathrm{Sta}: s\stackrel{e}{\rightarrow} s'$.
%
%The \emph{traces} of LTS $L=\tuple{\mathrm{Lab},\mathrm{Sta},\mathrm{Trans}}$, noted $\traces(L)$, are the finite sequences over $\mathrm{Lab} \cup \mathrm{Sta}$ of the form
%\[
%\mathit{s_0} \cdot \mathit{e_0} \cdot \mathit{s_1} \cdots \mathit{s_n},
%\]
%for some $n\in\mathbb{N}$, where $s_i \in \mathrm{Sta}$ and $e_i \in \mathrm{Lab}$.
%A trace starts from the initial state $\mathit{s_0}$, and follows  transition relation $\mathrm{Trans}$.
%%
%A state $s \in \mathrm{Sta}$ is \emph{reachable} in $L$ if $s$ occurs in one of the traces of $L$.
%
%
%


\section{BIP - Behavior Interaction Priority}
\label{sec:bip}
%%%%%%%%%%%%%%%%%%%%%%%%%%%%%%%%%%%%%%%%%%%%%%%%%%%%%%%%%%%%%%%%%%%%%%%%%%%%%%%%%%%%%%%%%%%%%%%
%%%%%%%%%%%%%%%%%%%%%%%%%%%%%%%%%%%%%%%%%%%%%%%%%%%%%%%%%%%%%%%%%%%%%%%%%%%%%%%%%%%%%%%%%%%%%%%
%
We recall the necessary concepts of the BIP framework~\cite{bip11}.
BIP allows to construct systems by superposing three layers of design: Behavior, Interaction, and Priority.
The \emph{behavior} layer consists of a set of atomic components represented by transition systems. 
The \emph{interaction} layer provides the collaboration between components. 
Interactions are described using sets of ports. 
The \emph{priority} layer is used to specify scheduling policies applied to the interaction layer, given by a strict partial order on interactions.
%
% \vspace{-1em}
%%%%%%%%%%%%%%%%%%%%%%%%%%%%%%%%%%%%%%%%%%%%%%%%%%%%%%%%%%%%%%%%%%%%%%%%
%%%%%%%%%%%%%%%%%%%%%%%%%%%%%%%%%%%%%%%%%%%%%%%%%%%%%%%%%%%%%%%%%%%%%%%%
\subsection{Component-based Construction}
%%%%%%%%%%%%%%%%%%%%%%%%%%%%%%%%%%%%%%%%%%%%%%%%%%%%%%%%%%%%%%%%%%%%%%%%
%%%%%%%%%%%%%%%%%%%%%%%%%%%%%%%%%%%%%%%%%%%%%%%%%%%%%%%%%%%%%%%%%%%%%%%%
%
BIP offers primitives and constructs for designing and composing complex behaviors from atomic components. Atomic components are Labeled Transition Systems (LTS) extended with C functions and data. Transitions are labeled with sets of communication ports. 
Composite components are obtained from atomic components by specifying interactions and priorities.
%
%
\subsubsection{Atomic Components.}
%
An atomic component is endowed with a finite set of local variables $X$ taking values in a domain $\mathit{\Data}$. Atomic components synchronize and exchange data with each others through \emph{ports}.
%
\begin{definition}[Port]
A port $p[x_p]$, where $x_p\subseteq X$, is defined by a port identifier $p$ and some data variables in a set $x_p$ (referred to as the support set). %We denote by $p.X$ the set of variables assigned to the port $p$, that is, $x_p$.
\end{definition}
%
%
\begin{definition}[Atomic component]
An atomic component $B$ is defined as a tuple $(P,L,$ $T,X,\{g_\tau\}_{\tau \in T}, \{f_{ \tau}\}_{\tau \in T})$, 
where:
\begin{itemize}
\item $(P,L,T)$ is an LTS over a set of ports $P$. $L\ignore{=\{l_1,l_2,\ldots,l_k\}}$ is a set of control locations and $T \subseteq L \times P \times L$ is a set of transitions.
\item $X$ is a set of variables.
\item For each transition $\tau \in T$: 
\begin{itemize}
\item $g_{\tau}$ is a Boolean condition over $X$: the guard of $\tau$,
\item $f_\tau = \{ (x, f^x(X)) \mid x\in X\}$ where $(x,f^x(X)) \in f_\tau$ expresses the assignment statement $x := f^x(X)$ updating $x$ with the value of the expression $f^x(X)$. 
%\item $f_\tau\in \{x := f^x(X)\mid x\in X\}^*$: the computation step of $\tau$, a sequence of assignments.
%\item $f_\tau$ is a function that updates the set of variables $X$: the computation step of $\tau$.
\end{itemize}
\end{itemize}
\end{definition}
%
For $\tau = (l,p,l')\in T$ a transition of the internal LTS, $l$ (resp. $l'$) is referred to as the source (resp.
destination) location and $p$ is a port through which an interaction with another component can take place. Moreover, a transition $\tau = (l,p,l')\in T$ in the internal LTS involves a transition in the atomic component of the form $(l,p,g_\tau,f_\tau,l')$ which can be executed only if the guard $g_\tau$ evaluates to $\true$, and $f_\tau$ is a computation step: a set of assignments to local variables in $X$.

In the sequel we use the dot notation.
Given a transition $\tau = (l,p,g_\tau,f_\tau,l')$, $\tau.\source$, $\tau.\port$, $\tau.\guard$, $\tau.\func$, and $\tau.dest$ denote $l$, $p$, $g_\tau$, $f_\tau$, and $l'$, respectively.
Also, the set of variables used in a transition is defined as $\var(f_\tau) = \{x \in X \mid (x,f^x(X)) \in f_\tau\}$.
Given an atomic component $B$, $B.P$ denotes the set of ports of the atomic component $B$, $B.L$ denotes its set of locations, etc.

Given a set $X$ of variables, we denote by $\valu{X}$ the set of valuations
defined on $X$. Formally, $\valu{X} = \{ \sigma : X \rightarrow \mathit{\Data}\}$, where
$\mathit{\Data}$ is the set of all values possibly taken by variables in $X$.

%
%\paragraph{Semantics of Atomic Components.}
%The semantics of an atomic component is an LTS over configurations and ports, formally defined as follows:
%
\begin{definition}[Semantics of Atomic Components]
\label{def:semantic-atom}
The semantics of the atomic component $B = (P,L, T, X, \{g_{\tau}\}_{\tau \in T}, \{f_{\tau}\}_{\tau \in T})$ is defined as the labeled transition system  $S_B = (Q_B, P_B, T_B)$, where: 
\begin{itemize}
\item $Q_B = L\times \valu{X}$, where $\valu{X}$ denotes the set of valuations on $X$,
\item $P_B = P \times \valu{X}$ denotes the set of labels, that is, ports augmented with valuations of variables, 
\item $T_B$ is the set of transitions defined as follows. $T_B= \{ ((l,v),p(v_{p}), (l',v'))\in Q_B\times P_B\times Q_B \mid \exists \tau = (l, p[x_{p}], l') \in T: g_{\tau}(v) \wedge v'=f_{\tau}(v/v_{p})\}$, where $v_{p}$ is a valuation of the variables of $p$.  
\end{itemize}
\end{definition}
%
A state is a pair $(l,v)\in Q_B$ where $l \in L$ is a control location, $v \in \valu{X}$ is a valuation of the variables in $X$. The evolution of states $(l, v)\stackrel{p(v_{p})}{\rightarrow} (l', v')$, where $v_{p}$ is a valuation of the variables attached to port $p$, is possible if there exists a transition $(l, p[x_p], g_\tau, f_\tau, l')$, such that $g_\tau(v)=\true$. In this case, we say that $p$ is \emph{enabled} in state $(l, v)$. Execution of port $p$ results in updated the valuation of $v$ to $v'=f_\tau(v/v_p)$.

Note that valuation of the variables attached to port $p$ are further instantiated when composing components with respect to data transfer functions of interactions~\cite{jaberthesis,FalconeJNBB15}.

 
%As a result, the valuation $v'$ of variables is modified to $v=f_\tau(v'/v_p)$.
%


\subsubsection{Composing Atomic Components.} 
Assuming some available atomic components $B_1,$ $\ldots,B_n$, we show how to connect a subset $\{B_i\}_{i\in I}, I\subseteq [1,n]$, of the components using an \emph{interaction}. An interaction $a$ is used to specify the sets of ports that have to be jointly executed.


\begin{definition}[Interaction]
\label{def:connector}
An interaction $a$ is a tuple $(P_a, G_a,F_a)$, where:
\begin{itemize}
\item $P_a \subseteq \cup_{i = 1} ^ { n } B_i.P$ is a nonempty set of ports that contains
at most one port of every component, that is, $\forall i: 1 \leq i \leq n: |B_i.P \cap P_a| \leq 1 $. We denote by $X_a = \cup_{p \in P_a} x_p$ the set of variables available to $a$,
\item $G_a : \valu{X_a} \rightarrow \{\true, \false\}$ is a guard, 
\item $F_a : \valu{X_a} \rightarrow \valu{X_a}$ is an update function. 
\end{itemize}
\end{definition}
%
$P_a$ is the set of connected ports called the support set of $a$. For each $i\in I$, $x_i$ is a set of variables associated with the port $p_i$.


%
\begin{definition}[Composite Component]
  A composite component is defined from a set of available atomic components $\{B_i\}_{i\in I}$ and a set of interactions $\gamma=\{a_j\}_{j\in J}$.
The connection of the components in $\{B_i\}_{i\in I}$ using the set $\gamma$ of interactions is denoted by $\gamma(\{B_{i}\}_{i\in I})$.
\end{definition}

\begin{figure*}
\begin{mathpar}
\inferrule*
{
    a = (\{p_i\}_{i \in I},G_a,F_a) \in \gamma \and G_a(\{v_{p_i}\}_{i \in I}) \hva\\
    \forall i\in I:\ q_i \goesto[p_i(\mu_i)]_i q'_i \wedge \mu_i = F^i_{a}(\{v_{p_i}\}_{i \in I}) \and
    \forall i\not\in I:\ q_i = q'_i
}
{
    (q_1,\dots,q_n) \goesto[a] (q'_1,\dots,q'_n)
}
\end{mathpar}
\caption{Semantics Rule of Composite Component}
\label{fig:bip-rule}
\end{figure*}

%
\begin{definition}[Semantics of Composite Components]
\label{def-runtimesemanticscomposite}
A state $q$ of a composite component $\gamma(\{B_1, \ldots, B_n\})$, where $\gamma$ connects the $B_i$'s for $i\in [1,n]$, is an $n$-tuple $q=(q_1,\ldots,q_n)$ where $q_i=(l_i,v_i)$ is a state of $B_i$. Thus, the semantics of $\gamma(\{B_1, \ldots, B_n\})$ is precisely defined as the labeled  transition system $S = (Q,\gamma,\goesto)$, where:
\begin{itemize}
\item $Q= B_1.Q\times \ldots\times B_n.Q$, 
\item $\goesto$ is the least set of transitions satisfying the rule defined in Figure \ref{fig:bip-rule}.
%
In this rule, $v_{p_i}$ denotes the valuation of the variables attached to the port $p_i$ and $F^i_{a}$ is the partial function derived from $F_a$ restricted to the variables associated with $p_i$. $\mu_i$ denotes the valuation of the variables attached to port $p_i$ after executing function $F_{a}$ of interaction $a$.
\end{itemize}
\end{definition}
%
The meaning of the rule defined in Figure \ref{fig:bip-rule} is the following: if there exists an interaction $a$ such that all its ports are enabled in the current state and its guard evaluates to \true, then the interaction can be fired. When $a$ is fired, all involved components evolve according to the interaction and uninvolved components remain in the same state. 

Notice that several distinct interactions can be enabled at the same time, thus introducing non-determinism in the product behavior.
One can add priorities to reduce non-determinism. In this case, one of the interactions with the highest priority is chosen non-deterministically.\footnote{The BIP engine implementing this semantics chooses one interaction at random, when faced with several enabled interactions.}
%
\begin{definition}[Priority]
  \label{defn:priority}
  Let $S = (Q,\gamma,\goesto)$ be the behavior of the composite component $\gamma(\{B_1, \ldots, B_n\})$.  A {\em priority model} $\pi$ is a
  strict partial order on the set of interactions $A$. Given a priority model $\pi$, we
  abbreviate $(a,a')\in \pi$ by $a \prec_\pi a'$ or $a \prec a'$ when clear from the context. Adding the priority model $\pi$ over $\gamma(\{B_1, \ldots, B_n\})$ defines a new composite component 
  $\pi\big(\gamma(\{B_1, \ldots, B_n\})\big)$ 
  noted $\pi(S)$ and whose behavior is defined by $(Q, \gamma, \goesto_\pi)$, where $\goesto_\pi$ is the least set of transitions satisfying the following rule:
\begin{mathpar}
\inferrule*
	{
      q \goesto[a] q' \quad
      \neg\big(\exists a'\in A,\exists q''\in Q: a \prec a' \wedge q \goesto[a'] q'' \big)
    }
    {
      q \goesto[a]_\pi q'
    }
\end{mathpar}
\end{definition}
%
An interaction $a$ is enabled in $\pi(S)$ whenever $a$ is enabled in $S$ and $a$ is maximal according to $\pi$ among the active interactions in $S$.


Finally, we consider systems defined as a parallel composition of components together with an initial state.
%
\begin{definition}[System]
\label{def:system}
A BIP system ${\cal S}$ is a tuple $(B,\mathit{Init}, v)$ where $B$ is a composite component,  $\mathit{Init}\in B_1.L\times \ldots\times B_n.L$ is the initial state of $B$, and $v \in \valu{X^{Init}}$ where $X^{Init} \subseteq \cup_{i = 1} ^ { n } B_i.X$.
\end{definition}

Given a port $p$ from the system \Pm, we denote by (1) $interaction(p)$ to be the set of interactions that are connected to $p$; (2) $component(p)$ to be the component to which the port $p$ belongs; (3) $transitions(p)$ to be the set of transitions labeled by $p$. 

We define the function $\mathtt{index}$ that assigns for each interaction $a \in \gamma$ a positive integer in $\big[0,|\gamma| - 1\big]$, i.e., $\mathtt{index}: \gamma \rightarrow \big[0,|\gamma| - 1\big]$. 

%\begin{definition}[Global State]
%A state of a BIP system ${\cal S}$ is a tuple $(q,b)$ where $q\in Q$ and $b$ is a Boolean bit-vector of size $|\gamma|$ 
%that indicates which interaction is selected to execute in the next step. 
%Formally, let $a \in \gamma$ be the interaction governing the transition $q_i \goesto[a]_\pi q_{i+1}$, 
%$b[i]$ is $\true$ when $i = \mathtt{index}(a)$ and is $\false$ for all $i \in [1, |\gamma|], i\not=\mathtt{index}(a)$. 
%\end{definition}

\begin{definition}[Trace]
  A trace $t$ of length $\ell$ of a system $(B,\mathit{Init}, v)$ is the sequence of global states  
  $q_0 \cdot q_1 \cdots q_{\ell-1}$ 
  such that: $q_0$ = $(\mathit{Init}, v)$, and 
  $\forall i\in [0,l-1]: q_i\in Q \wedge \exists a_i\in A: q_i \goesto[a_i]_\pi q_{i+1}$.
%
%  and: 
%
%\[ b_i[k] = \left\{ 
%  \begin{array}{l l}
%    \true & \quad k = \mathtt{index}(a_i)\\
%    \false & \quad \text{otherwise}
  %\end{array} \right.\]
That is, $a_i$ is an interaction enabled on $q_i$ and its execution results in state $q_{i+1}$. 
We denote by $t[i]$ the $i^{\text{th}}$ state in the trace, i.e., state $q_i$.
\end{definition}

\begin{example}
Figure \ref{fig:traffic:bip} shows a traffic light controller system modeled in BIP. 
It is composed of two atomic components, \cci{timer} and \cci{light}. The timer counts
the amount of time for which the light must stay in a specific state (i.e. a specific

color of the light). The light component determines the color of the traffic light. Additionally, it
informs the timer about the amount of time to spend in each location through a data transfer on the interaction between the two components. 

\begin{figure}
 \centering
 \resizebox{0.5\textwidth}{!}{
   \input{figures/traffic.pdf_t}
 }
 \caption{Traffic light in BIP}
 \label{fig:traffic:bip}
\end{figure}
\end{example}

