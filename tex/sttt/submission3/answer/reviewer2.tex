% !TEX root = answers.tex
\section{Reviewer 2 - Report 101-RV-1-R-486}
%
%%%%%%%%%%%%%%%%%%%%%%%%%%%%%%%%%%%%%%%%
%%%%%%%%%%%%%%%%%%%%%%%%%%%%%%%%%%%%%%%%
%
The paper has certainly improved since the first revision, however there are
still aspects that I think should be addressed. In particular, the following
two:

\begin{itemize}
\item I appreciate the author's discussion of the differences between BIP and the
Esterel flow in the answer to reviewers. However, apparently this discussion
did not make it into the paper. I think it would be useful to add a paragraph
in the related work explaining your point of view.

\answer{
We included a comparison in the Related Work Section.
}

\item The relationship between the NuSMV code generated from BIP and the OLP code
is still unclear. In the answer, you mention this is an interesting direction
for future work. However, I think it is very important to discuss some of the
aspects in *this* paper, otherwise the introduction of the OLP appears
unmotivated. 
The cycle-based semantics of OLP is apparently the same as that
of NuSMV, and I don't see any differences in how the transition relation is
generated. Differences may lie in the way you account for interactions. This
has to be explained, to give the reader a sense of what may cause the
different performance between the two tools. 

\answer{
  We address this concern by clarifying two main differences in the
  translations. 
}
    
\item If instead there are no
differences in how you account for interactions, then you have to explain why
you need the OLP, since a lot of the paper is concerned with describing how it
works (even more now, as another reviewer asked you add more details!).

\answer{
  The OLP program is a direct concurrent C implementation of the design. 
  It can be directly executed on CPUs. 
  Furthermore, in case the design was partitioned and parts needed to be run on CPUs, 
  the OLP is readily available to do that. 
  We already discuss this in the Introduction and in the Implementation and Evaluation Sections. 
  We address the concern of the reviewer by modifying the paragraph in the Introduction Section to 
  clarify the utility of the OLP.
  We also added a comparison between the OLP and the NuSMV translation in the Related Work Section.
}

\item Then, there are some other minor corrections that could improve readability.
\begin{itemize}
\item  Page 5. "The wire modifier keyword is used the variable is a wire"
Rephrase this sentence.

\done

\item  Explain earlier the difference between a wire and a register. Is it the same
as in Verilog? This comes only later, in Section 4.2. Plus, there is some
confusion in the added text. For instance, initially, bottom of Page 5, you
say that init-list is a list of assignments where the target terms are wire
variables ("The init-list and the next-list are lists of assignment
statements where the target terms are wire and register variables,
respectively."). Later on, on Page 6, you say "The list of statements
init-list assigns initial values to the register variables". This seems like
a contradiction. Is it wires or registers that get assigned in the
init-list?
    \answer{xxxxxxx}

\item Is "non-wire" the same as "register"?

  \answer{yes. we clarify this in the text.} 

\item What is a do-together construct, and what does a while(true) construct do?
  \answer{ The semantic of the do-together construct is defined in Definition 15. We added two clarifying sentences to address this concern}

\item On page 9, you do not define what it means for two traces to be "equivalent".

    \answer{xxxxxxx}

\item On page 10, you mention the "centralized" engine, but you never discussed
what the centralized engine is (or, at least, you didn't name it that way, I
understand it must be the BIP default simulation engine).
    \answer{xxxxxxx}

\item I understand the file property.txt represents the conditions to be verified.
You introduce this by starting a sentence with the name of the file. This is
not so easy to read. First explain that you need to express these
verification conditions, then give examples of these conditions, then say
that these conditions are stored in a file called property.txt. Perhaps
give examples.

The paragraph is followed by a picture with the invocation of the tool, with
no explanation.
    \answer{xxxxxxx}

\item This sentence is an overstatement: "Using ABC’s synthesis and reduction
algorithms, BipSV was able to reduce the size of the generated AIGs for all
designs by a factor larger than 50\%". It isn't BipSV that was able to reduce
the size of the generated AIGs, but rather ABC, unless you have driven the
ABC engine is some non-obvious way. In that case, I would expect the paper
to discuss the proper way to drive the ABC reduction techniques in order to
obtain better results (and perhaps a comparison of those ways in terms of
reduction to see what the real improvement is). The same in the following
sentences: you should compare ABC to NuSMV.

\end{itemize}
\end{itemize}

















