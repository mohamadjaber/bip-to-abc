\section{Reviewer 1 - Report 101-RV-1-R-492}
The paper has been slightly modified with respect to its original version, so I
will only comment the modified parts.
Please refer to my initial review for the rest of the paper, when applicable.
Most of my remarks has been taken into account by the authors, but I still have
the following observations that should be considered for the final version
before its publication.

\begin{itemize}
\item p.3, Sec.3: "[...] of design : Behavior, Interaction [...]" $\rightarrow$ "[...] of
design: Behavior, Interaction [...]" (in English there is no white space before ":")
\answer{
The space was an artifact of the color highlighting for the changes in the document. It is fixed now. 
}

\item p.4, Def.6: "Where $v_{p_i}$ denotes [...] interaction a." $\rightarrow$ "In this rule,
$v_{p_i}$ denotes [...] interaction a."

\done

\item p.4, after Def.6: "The meaning [...] is the following : [...]" $\rightarrow$ "The meaning
[...] is the following: [...]" (again, no white space before ":")

\done


\item p.6, Def.13\&16: "init - state(v)" and "next - state(v)": It is not a good idea
to use "-" in function names here, since it could be interpreted as a minus
operation. I think it is better to use "\_" instead of "-".

\answer{ We used another style with the dash such that it does not look like a minus sign. }

\item p.7, Def.20: Definition 20 is disconnected from Definitions 19 and 21, and
there are redundancies between all these definitions.

    \answer{ We added a sentence to definition 20 that relates to definitions 19 and 21. Note that 
    definition 20 was added in the first review to tend to a reviewer's request.}     

\item Moreover, its formulation should be improved: "Term $u_j$ denotes the j-th
incoming edge [...], implying that $(u_j,v) \in E$." $\rightarrow$ "Term $u_j$ denotes the j-th
incoming edge [...], that is, $(u_j,v) \in E$.".
\done

\item 
"The mapping [...] in the following sense." This sentence is incorrectly
placed, the notion of consistency being explained in the next paragraph.

\answer{The definition is rearranged. We moved part of the paragraph inside the figure. }

\item Also, please explain how rules of Fig.6 can be used in practice, since in
principle there might be ordering issues related to them (this works only
because you assume well-formed circuits, but you need to explain precisely why).

\answer{ We related the well-formed constraint that results in no combinational 
    cycles to the valuation directly after the definitions where
we discuss Figure 6.}

\item p.12: "This can be detected by doing a static data dependency [...]" $\rightarrow$ "This
can be detected by doing a static data dependency analysis [...]"

\done

\item p.13: "property.txt [...]" $\rightarrow$ "The file property.txt [...]"

\done

\end{itemize}




% \done
% \answer{}
