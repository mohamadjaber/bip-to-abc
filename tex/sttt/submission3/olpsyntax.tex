\subsection{Syntax of One Loop Programs}
%

Figure~\ref{fig:gr} illustrates the syntax of a 
{\em one loop program} (\caig).
\redbox{
An \caig starts with 
a list of variable declarations \cci{decl-list}. 
\caig declarations allow Boolean, integer, array of Boolean 
and array of integer types.
The \cci{wire} modifier keyword is used to denote that 
the variable is a wire. 
Otherwise, the variable is a register variable. 
A register variable represents a data storage/memory element. A wire represents a functional macro, which is used 
to connect different elements (wires or registers).}


\begin{definition}[\caig variables]
The set of variables $V$ of an \caig is defined to be 
the set of all non-wire, i.e., register variables declared in \cci{decl-list}.
The function $\mathit{type}: V \mapsto \set{int, Boolean, int[1], Boolean[1], int[2], Boolean[2], \ldots}$ maps a variable
$v \in V$ to its declared type. 
\end{definition}


The \cci{wiredef-list} follows \cci{decl-list} and is a list 
of assignment statements where the target term is a wire 
variable. 
An assignment has a left-hand side \cci{term} and 
a right-hand side expression \cci{expr}. 
The term is either an identifier \cci{id} or an array 
access expression \cci{id[expr]} where \cci{id} is the name
of the array and \cci{expr} is an expression. 
\caig expressions are built with terms, expressions with unary
operators (\cci{-,!}), expressions with binary operators (\cci{+,-,*,/,<,>,<=,>=,==,\&\&,||}), 
or expressions with a ternary choice operator 
(\cci{~?~:~}). 
Let $s$ be a \cci{wiredef-list} assignment with target $t$ and 
expression $e$; expression $e$ must not refer to target $t$. 



The \cci{init-list} and the \cci{next-list} are lists 
of assignment statements where the target terms are register variables.
\redbox{
The \cci{init-list} is embodied in a \cci{do-together} 
construct, which implies concurrent execution of all its statements. }
Expressions in \cci{init-list} assignments 
must not refer to non-wire
variables. 
The \cci{next-list} is embodied in a \cci{do-together} construct
which is in turn embodied in a \cci{while(true)} loop construct. 
\redbox{The loop makes sure that the design runs indefinitely.}



\begin{definition}[Well formed \caig]
An \caig is {\em well formed} when both 
the \cci{init-list} and the \cci{next-list}
contains one assignment per non-wire variable and
the \cci{wiredef-list} contains at most 
one assignment per wire. 
\end{definition}
%
Hereafter, we consider only well-formed \caig.


\begin{definition}[Non-deterministic wires]
We define the set of {\em non-deterministic wires} of 
an \caig to be the set of wire variables that are not 
targets of assignment statements in \cci{wiredef-list}.
\end{definition}



\begin{definition}[init and next state functions]
Consider $v\in V$ and consider $s_{init}$ and $s_{next}$ 
the assignment statements where $v$ is the target term in 
\cci{init-list} and \cci{next-list}, respectively. 
We define functions \cci{init-state(v)} and \cci{next-state(v)} 
to be the functions corresponding to the right-hand side 
expressions of $s_{init}$ and $s_{next}$, respectively. 
\end{definition}

\begin{example}[Well formed \caig]
Figure~\ref{fig:caigtraffic} shows an \caig that corresponds to the BIP system for the traffic-light controller shown in \figref{fig:traffic:bip}. 

\begin{figure}
\begin{tabular}{p{4.2cm}p{.2cm}p{12.5cm}}
\begin{lstlisting}
/*** decl-List ***/
int timer.t;
int timer.n;
int light.m;
int timer.$\ell$;
int light.$\ell$;
bool cycle;

wire int selector;
wire bool timer.timer.e;
wire bool timer.timer.s;
wire bool timer.done.e;
wire bool timer.done.s;
wire bool light.done.e;
wire bool light.done.s;
wire bool ie[2];
wire bool ip[2];
wire bool is[2];
\end{lstlisting}
& & 
\begin{lstlisting}
/*** wiredef-list ***/
timer.timer.e = (0 == timer.$\ell$) && (timer.t < timer.n);

timer.done.e  = (0 == timer.$\ell$) && (timer.t == timer.n);
light.done.e  = (0 == light.$\ell$)  || (1 == light.$\ell$) || (2 == light.$\ell$);

ie[0] = timer.timer.e;
ie[1] = (light.done.e && timer.done.e);

ip[0] = ie[0];
ip[1] = ie[1];

is[0] = (ip[0] && ( selector == 0 || (!ip[selector]  && !ip[1]);
is[1] = (ip[1] && ( selector == 1 || (!ip[selector]);

timer.timer.e = is[0];
timer.done.e  = is[1];
light.done.e  = is[1] ;
\end{lstlisting}
\\
\vspace{-2em}
\begin{lstlisting}
do-together {
  /*** init-list ***/
  timer.t = 0; 
  timer.n = 10; 
  timer.$\ell$ = 0;

  light.m = 5; 
  light.$\ell$ = 0;

  cycle = true; 
}/* end do-together */
\end{lstlisting}
& & 
\vspace{-2em}
\begin{lstlisting}
while(true) {
  do-together {
    /*** next-list ***/    
    timer.n = cycle? is[1]? light.m : timer.n : timer.n; 
    
    timer.$\ell$ = cycle? timer.$\ell$: timer.timer.e && timer.$\ell$ == 0? 0 : timer.timer.e && timer.$\ell$ == 0? 0 : timer.$\ell$;
    
    timer.t = cycle? timer.t : timer.$\ell$ == 0 && timer.timer.e? (timer.t + 1) : timer.$\ell$ == 0 && timer.done.e? 0 : 
              timer.t; 
              
    light.$\ell$ = cycle? light.$\ell$ : light.$\ell$ == 2 && light.done.e? 0: light.$\ell$ == 1 && light.done.e? 0 : light.$\ell$ == 0 
              && light.done.e? 1 : light.$\ell$; 
              
    light.m = cycle? light.m : light.$\ell$ == 0 && light.done.e? 3: light.$\ell$ == 1 && light.done.e? 10: light.$\ell$ == 2 
              && light.done.e? 5 : light.m; 
    
    cycle = !cycle; 
  } /*end do-together*/ 
} /*end while(true)*/
\end{lstlisting}
\end{tabular}
\vspace{-2em}
\caption{Sample of $\caig$ generated code of traffic light system}
\label{fig:caigtraffic}
\end{figure}

\end{example}

