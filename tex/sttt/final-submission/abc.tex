\section{ABC reduction and verification techniques}
\label{app:abc}

The ABC framework provides a set of algorithms that can 
be applied iteratively to (1) reduce the AIG into 
an equivalent AIG and (2) verify that a designated 
output of an AIG is always true. 
In what follows we provide brief descriptions of several 
reduction and verification ABC algorithms. 

% -- Synthesis techniques -- %
\subsection{Structural Register Sweep (SRS)}
SRS detects registers that are stuck-at-constant and eliminates 
them from a given sequential AIG circuit. The technique starts by zeroing up all 
initial values of registers in the circuit. It then uses the ternary simulation
algorithm in order to detect stuck-at-constant registers. The algorithm starts from 
the initial values of the registers and simulates the circuit using x values for the
circuit's primary inputs. The simulation algorithm stops when a new ternary state is 
equal to a previously computed ternary state. In this case, any register having the 
same constant value at each reachable ternary state will be declared to be 
stuck-at-constant and thus eliminated. The structural sweeping algorithm stop when 
no further reduction in the number of registers is possible~\cite{mishchenko2008scalable}. 

\subsection{Signal Correspondence (Scorr)} 
Scorr uses $k$-step induction in order to detect and merge sets of classes of 
sequentially-equivalent nodes~\cite{mishchenko2008scalable}. The base case for this algorithm is that the equivalence
between the classes holds for the first $k$ frames, and the inductive case is that 
given the base case, starting from any state, the equivalence holds in the 
$(k+1)^{st}$ state. Key to the signal correspondence algorithm is the way the candidate
equivalences are assumed for the base case. Abc implements speculative reduction, 
originally presented in~\cite{mony2005exploiting}, which merges, but does not remove, any node of an equivalence 
class onto its representative, in each of the first $k$ time frames. Instead of removing the 
merged node, a constraint is added to assert that the node and its representative are equal. 
This technique is claimed to decrease the number of constraints added to the SAT solved for 
induction. 

\subsection{Rewriting}
Rewriting aims at finding nodes in a Directed Acyclic Graph (DAG) where by replacing subgraphs rooted 
at these nodes by pre-computed subgraphs can introduce important reductions in the DAG size, while 
keeping the functionality of these nodes intact. The algorithm traverses the DAG in depth-first post-order
and gives a score for each root node. The score represents the number of nodes that would result
from performing a rewrite at this node. If a rewrite exists such that the size of the DAG is decreased, such 
a rewrite is performed and scores are recomputed accordingly.  
Rewriting has been proposed initially in~\cite{bjesse2004dag}, targeted for Reduced Boolean Circuits (RBC); 
it was later implemented and improved for ABC in~\cite{mishchenko2006dag}. 

%% Retiming
\subsection{Retiming}
Retiming a sequential circuit is a standard technique used in sequential synthesis, 
aiming at the relocation of the registers in the circuit in order to optimize 
some of the circuit characteristics. Retiming can either targets the minimization of the delay 
in the circuit, or the minimization of the number of registers given a delay constraint, 
or the unconstrained minimization of the number of registers in the circuit. It 
does so while keeping the output functionality of the circuit intact~\cite{hurst2007fast}

% -- Verification techniques -- %
\subsection{Property Directed Reachability (Pdr)}
The Pdr algorithm aims at proving that no 
violating state is reachable from the initial state of a given AIG network. 
It maintains a trace representing a list of over-approximations of the states
reachable from the initial state, along with a set of {\em proof-obligations}, 
which can be a set of bad states or a set of states from which a bad state is
reachable. Given the trace and the set of obligations, the Pdr algorithm manipulates 
them and keeps on adding facts to the trace until either an inductive invariant 
is reached and the property is proved, or a counter example is found (a bad is state
is proven to be reachable). The algorithm was originally developed by Aaron Bradley 
in~\cite{bradley2011sat,bradley2007checking} and was later improved by Een et. al in~\cite{een2011efficient}.

\subsection{Temporal Induction}
Temporal induction carries an inductive proof of the property 
over the time steps of a sequential circuit.
Similar to a standard inductive proof, it consists of a base
case and an inductive hypothesis. These steps are typically 
expressed as SAT problems to be solved by traditional SAT solvers.  
$k$-step induction strengthens simple temporal inductive proofs 
by assuming that the property holds for the first $k$ time steps (states), 
i.e. a longer base case needs to be proven~\cite{een2003temporal}. Since the target is
to prove unsatisfiability (proving that the negation of the property 
is unsatisfiable), if the base case is satisfiable, a counter-example 
is returned. Otherwise, the induction step is checked by assuming that
the property holds for all the states except the last one (the $(k+1)$'th 
state)~\cite{biere2009handbook}.   

\subsection{Interpolation}
Given an unsatisfiable formula $A \land B$, an interpolant $I$ is
a formula such that $A \implies I$, $I \land B$ is unsatisfiable and
$I$ contains only common variables to $A$ and $B$. 
Given a system $M$, a property $p$ and a bound $k$, interpolation
based verification starts by attempting bounded model checking (BMC) with the bound $k$. 
If a counter-example is found, the algorithm returns. Otherwise, it
partitions the problem into a prefix $pre$ and a suffix $suf$, such that the 
problem is the conjunction of the two. 
Then the interpolant $I$ of $pre$ and $suf$ is computed, it represents
an over-approximation of the set of states reachable in one step from the initial state
of the algorithm. If $I$ contains no new states, a fixpoint is reached 
and the property is proved. Otherwise, the algorithm reiterates and replaces
the initial states with new states added by $I$~\cite{amla2005analysis}. 

