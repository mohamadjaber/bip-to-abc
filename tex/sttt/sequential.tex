\section{Sequential Circuit}
\label{sec:sequential}
%%%%%%%%%%%%%%%%%%%%%%%%%%%%%%%%%%
%%%%%%%%%%%%%%%%%%%%%%%%%%%%%%%%%%
%

%
\begin{definition}[Sequential circuit]
A {\em sequential circuit} is a tuple $\left( (V, E),G,
O\right)$.
Pair $(V,E)$ represents a directed graph on
vertices $V$ and edges $E \subseteq V\times V$, where $E$
is a total order.  Function $G: V \mapsto
\mathit{Types}$ maps vertices to $\mathit{Types}$.
There are three disjoint types: {\em primary inputs}, {\em
bit-registers} (which we often simply refer to as {\em
registers}), and logical {\em gates}.  Registers have designated
{\em initial values}, as well as {\em next-state
functions}.  Gates describe logical functions such as
the conjunction or disjunction of other vertices. 
A subset $O$ of $V$ is specified as the {\em
primary outputs} of $V$.  
We denote the set of primary input variables by $I$,
and the set of bit-register variables by $R$.  
\label{def:back:seq_circuit}
\end{definition}
%
\begin{definition}[Fanins]
The direct \emph{fanin}s of a gate $u$ is defined as
$\{v \mid (v,u)\in E\}$, i.e., the set of source vertices connected
to $u$ in $E$.  
The {\em support} of $u$ 
is $\mathit{Fanins}(u) \cap (I \cup R)$, i.e., the set of all source vertices in $R$ or $I$ that are connected to $u$.
%with $E^+$, the transitive closure of $E$.
\label{def:back:fanins} 
\end{definition}
%
%We restrict gates to have 2~fanins, and
%compute the NAND function; since NAND is functionally
%complete, this is not a limitation.  
%
For a sequential
circuit to be syntactically well-formed, vertices in $I$
should have no fanins, vertices in $R$ should have
2~fanins (the next-state function and the initial-value
function of that register), %gates should have two fanins,
and every cycle in the sequential circuit should contain
at least one vertex from $R$.  The initial-value functions
of $R$ shall have no register in their support.  
In the following, 
we consider only well-formed sequential circuits which can be 
verified by a structural check linear in the size of the 
sequential circuit.
%This is not a limitation since NAND is {\em functionally complete} as it implements
%negated conjunction. 
%


The ABC synthesis and model checker framework reasons about the And-Inverted-Graph (AIG) representation of a sequential circuit which are sequential circuits with only NAND gates with exactly two fanins.~\cite{brayton2010abc}. 


%%%%%%%%%%%%%%%%%%%%%%%%%%%%%%%%%%%
\subsection{Semantics of Sequential Circuits}
\label{s:back:crct_semantics}
%%%%%%%%%%%%%%%%%%%%%%%%%%%%%%%%%%%
%
The semantics of a sequential circuit is defined in terms of its states and traces.
%
\begin{definition}[State]
A \emph{state} $\sigma: R \rightarrow \mathbb{B}$ is a Boolean valuation of vertices in $R$. 
%A {\em concrete input} is a Boolean valuation to vertices in $I$.
\end{definition}
%
\begin{definition}[Trace]
A {\em trace} is a mapping $t: V \times \mathbb{N} \rightarrow
\mathbb{B}$ that gives a value to vertices in
$V$ across time {\em steps} denoted as indexes from
$\mathbb{N}$.
The mapping must be consistent with $E$ and
$G$ in the following sense.
Term $u_{j}$ denotes the source vertex of
the $j$-th incoming edge to $v$, implying that
$(u_{j},v)\in E$.
The value of gate $v$ at time $i$ in
trace $t$ is denoted by $t(v,i)$.
\end{definition}
%
\begin{figure*}
\[
t(v,i)=
   \begin{cases}
      s^i_{v}            &:v \in I \ \text{with sampled value $s_{v}^i$}\\
      t(u_1, 0)       &:v \in R,i=0,u_1:=\ \text{initial-state of $v$}\\
      t(u_2, i-1)        &:v \in R,i>0,u_2:=\ \text{next-state of $v$}\\
      G_v\big(t(u_{1},i),...,t(u_{n},i)\big) &: v \ \text{is a combinational gate with function $G_v$}
   \end{cases} \newline
\]
\caption{Semantics of sequential circuits given in terms of traces. $t(v,i)$ denotes the valuation of gate $v$ at step $i$ in trace $t$}
\label{fig:gate-value}
\end{figure*}
%


Given a sequence of input valuations and an initial state, 
\figref{fig:gate-value} defines the resulting trace as a sequence of Boolean valuations to all vertices in $V$
which is consistent with the Boolean functions of the
gates. 

We will refer to the transition from one valuation
to the next one as a {\em step}.  A node in the circuit is
said to be justifiable if there is an input sequence which, when
applied to an initial state, will result in that node
taking value $\true$.  A node in the circuit is
valid if its negation is not justifiable.  We will refer
to targets and invariants in the circuit; these are simply
vertices in the circuit whose justifiability and validity
is of interest respectively.
A sequential circuit can naturally
be associated with a finite-state machine (FSM),
which is a graph on the reachable states.  However, the 
circuit is very different from its FSM; among
other differences, it is exponentially more succinct in
almost all cases of interest~\cite{BuClMcDiHw92}. 
%
%Figure~\ref{f:back:cctfsm}(a) shows a 2-bit write enabled 
%buffer which uses a multiplexer enabled by $w_{en}$ to 
%update its register bits $r$ with values in inputs $i$ 
%when $w_{en}$ is $1$ and retains its state otherwise.
%It also shows the corresponding FSM with no labels on
%transitions for clarity of exposition purposes.
%Figure~\ref{f:back:cctfsm}(b) shows a 3-bit version of 
%the same circuit with the corresponding FSM. 
%Only the arcs corresponding to state $000$ are shown for 
%clarity of exposition.
%
%%%%%%%%%%%%%%%%%%%%%%%%%%%%%
\subsection{Translation from \caig to AIG Circuits}
%%%%%%%%%%%%%%%%%%%%%%%%%%%%%
%
The translation from \caig to AIG circuits is done as follows and is illustrated in the example of \figref{fig:circuittraffic}.
\begin{enumerate}
\item The translation of an \caig into an AIG circuit
first constructs registers, wires, and primary input variables for each 
\caig variable. 
%
\item It then recursively traverses the right-hand side expressions 
of the assignment statements in \cci{wiredef-list}, \cci{init-list},
and \cci{next-list} to build the corresponding combinational circuits.
%
\item It connects the resulting vertices of the combinational 
circuits of the right-hand side expressions to the fanins of the registers corresponding to the left-hand side target variables. 

\item The vertices corresponding to the \cci{init-list} right-hand side expressions
are connected to the  initial value fanins of the registers. Similarly, those of the \cci{next-list} are connected to the next state value fanins. 


%
\item Finally, it connects the vertices of the combinational circuits built 
for the \cci{wiredef-list} expressions to the corresponding 
wires referring to the 
variables declared as wire variables in \cci{decl-list}.
\end{enumerate}

\paragraph{Variables.} 
%
We consider each variable not declared as a wire in \cci{decl-list}.
We instantiate a corresponding 
vector of AIG registers with an adequate bit width. 
The width of the bit vector can be selected by the user, 
or can be set to match the default width of the declared type. 
Typically the default values for the bit width are 
32 bits for an integer, one bit for a Boolean, and a 
finite two dimensional bit vector for an array. 
In our case, and for \caig programs generated from BIP systems, 
we will not have arrays of register variables
and we will only have fixed size arrays of Boolean wires as 
discussed in \secref{sec:bip2aig}.
%
We say that a variable declared as wire in \cci{decl-list}
is non-deterministic when it does not have a corresponding assignment statement in \cci{wiredef-list}. 
For each non-deterministic variable, we instantiate a corresponding
vector of primary inputs with an adequate bit-width. 
%
We consider variables declared as wires in \cci{decl-list} with 
a corresponding assignment statement in \cci{wiredef-list} as functional macros. 
For each functional macro we 
instantiate a vector of identity gates (a sequence of two negation gates) 
where the fanouts correspond to the wire variable and the fanins correspond to
the expression defining the wire variable in \cci{wiredef-list}. 
%
We denote the gates corresponding to each variable $v$ by the function \cci{vargates}$(v)$. 

\begin{lstlisting}
/*@\textbf{variables}@*/(decl-list)
  foreach variable $v$ in decl-list
    if ($v$ is not a wire) 
      vargates($v$) = instantiate-registers($v$,type($v$))
    elseif ($v$ is not assigned in wiredef-list) 
      // non-deterministic input
      vargates($v$) = instantiate-primary-inputs(
                      $v$,type($v$))
    endif
  endfor
\end{lstlisting}
%
\paragraph{Assignment statements.}
We consider each assignment statement in \cci{wiredef-list}, \cci{init-list},
and \cci{next-list} and traverse the right-hand side expressions of
each assignment with the recursive \cci{traverse} routine. 
%
If the expression refers to a variable $v$ (base case), 
then the traversal returns \cci{vargates(v)}. 
%
If the expression is a logical, conditional, or arithmetic expression, then
the \cci{library} routine finds an equivalent circuit for it with the adequate bit
width in a complete table of circuits.
For example, if the expression is a ternary conditional statement of the
form $b?~e_1:e_2$, then routine \cci{library} instantiates a multiplexer, 
connects its two data fanins to the nodes corresponding to $e_1$ and $e_2$, connects its control fanins to the nodes corresponding to $b$,
and returns its fanouts. 
%
\begin{lstlisting}
/*@\textbf{traverse}@*/($exp$)
  if ($exp$ is a variable) 
    return vargates(exp)
  endif

  foreach $i [1~..~ exp.operands.size()]$ 
    $wirevec[i]$ = traverse($exp.operands[i]$) 
  endfor

  return library($exp.operation$, $wirevec$)
\end{lstlisting}
%
\paragraph{Connections.}
%
We connect the nodes corresponding to the right-hand side expressions 
of the assignment statements in the \cci{init-list} and \cci{next-list}
expressions 
to the initial value and next value fanins of the corresponding register gates, 
respectively. 
%
We connect the nodes corresponding to the right-hand side expressions
of the assignment statements in the \cci{wiredef-list} expressions to the 
fanins of the corresponding wire identity gates.
Below we define \texttt{olp-to-aig} method that takes an \caig program as input and generates its corresponding AIG circuit. 

\begin{lstlisting}
// P is an /*@\textcolor{darkgreen}{\caig}@*/ program
/*@\textbf{olp-to-aig}@*/($P$) 
  // instantiate aig variables and construct
  // vargates
  variables($P$.decl-list); 
  
  // s is of the form term = expr 
  foreach assignment $s  \in$ init-list 
    next-state($s$.term) = traverse($s$.expr);  
  endfor
  
  foreach assignment $s  \in$ wiredef-list
    vargates($s$.term) = traverse($s$.expr);  
  endfor
  
  foreach assignment $s  \in$ next-list
    next-state($s$.term) = traverse($s$.expr);  
  endfor
\end{lstlisting}


\paragraph{Invariants.}
%
A special variable in the \caig program denotes the conjunction of all the invariants of the system in addition to the deadlock freedom property. This variable is the designated output of the resulting 
AIG circuit. ABC verifies that the designated output is always $\true$. 
%
\begin{theorem}[AIG \caig equivalence]
Let $P$ be an \caig and $A$ be the AIG circuit generated from it (i.e., $A$ = \texttt{olp-to-aig($P$)}). We have
the set of traces of $P$ is equal to the set of traces of $A$.
\end{theorem}
%
The proof can be done by structural induction on $P$ and $A$. 
The base cases of the induction are 
(1) the variables of $P$ and their corresponding registers of $A$;
(2) non-deterministic input of $P$ and its corresponding primary inputs of $A$; and 
(3) the arithmetic and logical operations of $P$ and their corresponding primary gates provided by \texttt{library} of $A$. 
The structure is preserved by \texttt{vargates}. 
Timing is preserved by the semantics of AIG that infinitely repeats the execution of the 
next state functions similar to the lone \texttt{while} loop in \caig.
%
\begin{example}[Generated AIG circuit]
Figure~\ref{fig:circuittraffic} shows a sample circuit generated 
by traversing the right-hand side expressions of the initial 
value and next state function assignment statement corresponding to variable
$timer.\ell$. 
The sample circuit shows AND, $=$, $<$ and multiplexer gates for simplicity; 
all those gates can be readily implemented using NAND gates. 
A multiplexer takes a Boolean control input, and uses its value to 
choose one of its two data inputs. 

The next state function depends on variables $cycle$, $timer.t$, and $timer.n$ and on 
the wire variable $timer.timer.e$. 
The registers of the variables are connected directly to the circuit. 
The circuit for the wire variable $timer.timer.e$ is constructed by traversing 
the right-hand side expression of its 
assignment statement in the wire definition list. 
Then, the constructed circuit is connected to the input of the corresponding AND gate. 

Note that (1) the initial value of $timer.\ell$ registers is $0$, and  (2) the next state fanins of $timer.\ell$ are connected to a multiplexer whose data inputs are equal to $0$ or $timer.\ell$. Thus the constant propagation algorithm replaces $timer.\ell$ with $0$ and propagates that effect. 

\begin{figure}
\resizebox{.9\columnwidth}{!}{
  \input{figures/trafficlightcircuit.pdf_t}
}
\caption{Sample circuit for the $timer.\ell$ registers; the $<$, $=$, 
  and multiplexer gates can be easily implemented using NAND gates. }
\label{fig:circuittraffic}
\end{figure}

\end{example}