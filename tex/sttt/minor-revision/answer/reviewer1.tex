\section{Reviewer 1 - Report 101-RV-2-R-528}
I have only the following minor remarks:

\begin{itemize}
\item p.10: ``In order to avoid concurrently-executing conflicting interactions [...] can be simultaneously 
executed.'': This part is a bit misleading with respect to the state of the art of code generation for BIP 
models. There exist 3 different implementations: a single-threaded, a multi-threaded and a distributed 
one. The multi-threaded version executes components and [non conflicting] interactions in parallel, 
without any specific overhead (except classical overheads due to threads synchronisation).

\item Moreover, I think the degree of parallelism offered by the multi-threaded version is greater than the 
one obtained by merging cycles as proposed in the paper. In the case of merging of cycles in OLPs a 
cycle has to be fully completed before executing a new one (e.g. all non conflicting interactions 
enabled initially has to be completed before starting new interactions) whereas the multi-thread 
implementation of BIP can start new interactions at any type (e.g. after the completion of only a 
subset of the initially non conflicting interactions).

\item p.15: ``The OLP program is a concurrent C implementation of the BIP system with a minor 
modification involving replacing the do-together directives with OpenMP API directives.'': I think this 
could be reformulated into ``OLP programs generated from BIP systems can be straightforwardly 
translated into concurrent C implementations with a minor modifications (e.g. replacing the do-
together directives with OpenMP API directives).''
\end{itemize}
