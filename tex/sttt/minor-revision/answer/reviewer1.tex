\section{Reviewer 1 - Report 101-RV-2-R-528}
I have only the following minor remarks:

\begin{itemize}
\item p.10: ``In order to avoid concurrently-executing conflicting interactions [...] can be simultaneously 
executed.'': This part is a bit misleading with respect to the state of the art of code generation for BIP 
models. There exist 3 different implementations: a single-threaded, a multi-threaded and a distributed 
one. The multi-threaded version executes components and [non conflicting] interactions in parallel, 
without any specific overhead (except classical overheads due to threads synchronisation).

\answer{
Thank you for referring us to the multi-threaded code generation for BIP models. 
We added the following discussion in the paper. 
%
In the multi-threaded implementation~\cite{BasuBBS08}, 
the involved components in the non-conflicting interactions 
execute simultaneously with no overhead
except the classical thread synchronization overhead. 
However, while each component executes in a separate thread, the engine executes in a single engine thread.
The single engine thread is responsible for sequentially 
(1) selecting an interaction for execution, 
(2) executing the corresponding action, and
(3) signaling the static threads associated with the involved components for execution. 
}

\item Moreover, I think the degree of parallelism offered by the multi-threaded version is greater than the 
one obtained by merging cycles as proposed in the paper. In the case of merging of cycles in OLPs a 
cycle has to be fully completed before executing a new one (e.g. all non conflicting interactions 
enabled initially has to be completed before starting new interactions) whereas the multi-thread 
implementation of BIP can start new interactions at any type (e.g. after the completion of only a 
subset of the initially non conflicting interactions).

\answer{
First, we would like to note that the selection of non-conflicting interactions can happen 
simultaneously 
in \biptool as opposed to sequentially in ~\cite{BasuBBS08}. 
%
Second, the reviewer is correct in the sense that merging cycles allows only the execution 
of non-conflicting interactions and other 
interactions need to wait until the cycle ends. 
%
When generating code for FPGAs, the cycle synchronization is a necessary constraint to abide with. 
% 
This may be relaxed by allowing longer interactions to span multiple cycles and introducing a busy state in the involved components. 
%
As for software execution, our current model of computation for the one loop is that it iterates once all the assignment statements 
inside it are done. 
We can relax that to have the loop iterate when an interaction is done and guard the interactions that are in progress with the 
necessary logic.
We reflect this in the paper, and we thank the reviewer for this feedback as we plan to include it in future work. 
}

\item p.15: ``The OLP program is a concurrent C implementation of the BIP system with a minor 
modification involving replacing the do-together directives with OpenMP API directives.'': I think this 
could be reformulated into ``OLP programs generated from BIP systems can be straightforwardly 
translated into concurrent C implementations with a minor modifications (e.g. replacing the do-
together directives with OpenMP API directives).''

\done 

\end{itemize}
