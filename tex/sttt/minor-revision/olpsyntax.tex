\subsection{Syntax of one loop programs.}
%
\redbox{
Figure~\ref{fig:gr} illustrates the syntax of a 
{\em one loop program} (\caig).
An \caig starts with 
a list of variable declaration statements \cci{decl-list}. 
\caig declarations allow Boolean, integer, array of Boolean 
and array of integer types.
The \cci{wire} modifier keyword is used
the variable is a wire. 
Otherwise, the variable is a register variable. 
}

\redbox{
\begin{definition}[\caig variables]
The set of \caig variables $V$ of an \caig is defined to be 
the set 
of all non-wire variables declared in \cci{decl-list}.
The function $\mathit{type}: V \mapsto \set{int, Boolean, int[1], Boolean[1], int[2], Boolean[2], \ldots}$ maps a variable
$v \in V$ to its declared type. 
\end{definition}
}

\redbox{
The \cci{wiredef-list} follows \cci{decl-list} and is a list 
of assignment statements where the target term is a wire 
variable. 
An assignment statement has a left hand side \cci{term} and 
a right hand side expression \cci{expr}. 
The term is either an identifier \cci{id} or an array 
access expression \cci{id[expr]} where \cci{id} is the name
of the array and \cci{expr} is an expression. 
\caig expressions can be either terms, expressions with unary
operators (\cci{-,!}), expressions with binary operators (\cci{+,-,*,/,<,>,<=,>=,==,\&\&,||}), 
or expressions with a ternary choice operator 
(\cci{~?~:~}). 
Let $s$ be a \cci{wiredef-list} assignment with target $t$ and 
expression $e$; expression $e$ must not refer to target $t$. 
}

\redbox{
The \cci{init-list} and the \cci{next-list} are lists 
of assignment statements where the target terms are wire and
register variables, respectively. 
The \cci{init-list} is embodied in a \cci{do-together} 
construct. 
Expressions in \cci{init-list} assignments 
must not refer to non-wire
variables. 
The \cci{next-list} is embodied in a \cci{do-together} construct
which is in turn embodied in a \cci{while(true)} construct. 
}

\redbox{
\begin{definition}[Well formed \caig]
An \caig is {\em well formed} in case each of 
the \cci{init-list} and the \cci{next-list}
contains one assignment per non-wire variable and
the \cci{wiredef-list} contains at most 
one assignment per wire. 
For brevity, \caig hereafter denotes well 
formed \caig. 
\end{definition}
}

\redbox{
\begin{definition}[Non-deterministic wires]
We define the set of {\em non-deterministic wires} of 
an \caig to be the set of wire variables who are not 
targets of assignment statements in \cci{wiredef-list}.
\end{definition}
}

\redbox{
\begin{definition}[init and next state functions]
Consider $v\in V$ and consider $s_{init}$ and $s_{next}$ 
the assignment statements where $v$ is the target term in 
\cci{init-list} and \cci{next-list}, respectively. 
We defined 
functions $\mathit{init-state}(v)$ and 
$\mathit{next-state}(v)$ 
to be the functions corresponding to the right hand side 
expressions of $s_{init}$ and $s_{next}$, respectively. 
\end{definition}
}
