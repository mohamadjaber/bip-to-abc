\section{Conclusion and Future Work}
\label{sec:conclusion}

In this paper we present a method for embedded system synthesis, runtime verification,
and model checking with supporting tools for the BIP framework. 
The method takes a BIP system and generates a concurrent C program with a system 
specific scheduler embedded therein. 
The concurrent C program serves as a software runtime verification simulator for the 
BIP system.
The method then takes the concurrent C program and generates an AIG circuit which is an
FPGA implementation of the BIP system. 
The method applies synthesis reduction techniques using the ABC framework 
to simplify and reduce the AIG circuit
into a smaller and a less complex circuit that can be readily implemented with an 
FPGA. 
The method passes the reduced AIG circuit with a designated output that is \true
when the BIP system invariants are \true to ABC proof and model checking 
algorithms. In case ABC finds a counterexample, the methods maps the values from 
the counterexample to the original ABC system and provides the user with a debug
visualization tool. 
We successfully used the system to verify and debug several case studies. 

%Currently, the system specific scheduler makes conservative decisions to avoid 
%data transfer conflicts. In the future, we plan to extend support for conflict and 
%dependency analysis so that the system specific scheduler allows for more concurrent
%behaviors. 

%For future work, we are considering several research directions. 
Currently, the system-specific scheduler makes conservative decisions to avoid interaction conflicts. Two interactions conflict if they share a port or they use conflicting ports of the same component.
An important extension is to allow parallel execution of non-conflicting interactions using techniques presented in \cite{BonakdarpourBJQS12}. Another interesting direction is to generate correct and efficient sequential circuit given real-time software (i.e., with real-time constraints) modeled using the real-time version of BIP~\cite{AbdellatifCS13}. 
%