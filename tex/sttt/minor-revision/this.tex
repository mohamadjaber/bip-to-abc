\section{One Loop Programs (\caig) - syntax and semantics}
\label{sec:this}
%
%%%%%%%%%%%%%%%%%%%%%%%%%%%%%%%%%%%%%%%%%%%%%%%%%%%%%
%%%%%%%%%%%%%%%%%%%%%%%%%%%%%%%%%%%%%%%%%%%%%%%%%%%%%
%
This section introduces the syntax and semantics of one loop programs.
%
\subsection{Syntax of one loop programs.}
%
\redbox{
Figure~\ref{fig:gr} illustrates the syntax of a 
{\em one loop program} (\caig).
An \caig starts with 
a list of variable declaration statements \cci{decl-list}. 
\caig declarations allow Boolean, integer, array of Boolean 
and array of integer types.
The \cci{wire} modifier keyword is used
the variable is a wire. 
Otherwise, the variable is a register variable. 
}

\redbox{
\begin{definition}[\caig variables]
The set of \caig variables $V$ of an \caig is defined to be 
the set 
of all non-wire variables declared in \cci{decl-list}.
The function $\mathit{type}: V \mapsto \set{int, Boolean, int[1], Boolean[1], int[2], Boolean[2], \ldots}$ maps a variable
$v \in V$ to its declared type. 
\end{definition}
}

\redbox{
The \cci{wiredef-list} follows \cci{decl-list} and is a list 
of assignment statements where the target term is a wire 
variable. 
An assignment statement has a left hand side \cci{term} and 
a right hand side expression \cci{expr}. 
The term is either an identifier \cci{id} or an array 
access expression \cci{id[expr]} where \cci{id} is the name
of the array and \cci{expr} is an expression. 
\caig expressions can be either terms, expressions with unary
operators (\cci{-,!}), expressions with binary operators (\cci{+,-,*,\/,<,>,<=,>=,==,\&\&,||}), 
or expressions with a ternary choice operator 
(\cci{~?~:~}). 
Let $s$ be a \cci{wiredef-list} assignment with target $t$ and 
expression $e$; expression $e$ must not refer to target $t$. 
}

\redbox{
The \cci{init-list} and the \cci{next-list} are lists 
of assignment statements where the target terms are wire and
register variables, respectively. 
The \cci{init-list} is embodied in a \cci{do-together} 
construct. 
Expressions in \cci{init-list} assignments 
must not refer to non-wire
variables. 
The \cci{next-list} is embodied in a \cci{do-together} construct
which is in turn embodied in a \cci{while(true)} construct. 
}

\redbox{
\begin{definition}[Well formed \caig]
An \caig is {\em well formed} in case each of 
the \cci{init-list} and the \cci{next-list}
contains one assignment per non-wire variable and
the \cci{wiredef-list} contains at most 
one assignment per wire. 
For brevity, \caig hereafter denotes well 
formed \caig. 
\end{definition}
}

\redbox{
\begin{definition}[Non-deterministic wires]
We define the set of {\em non-deterministic wires} of 
an \caig to be the set of wire variables who are not 
targets of assignment statements in \cci{wiredef-list}.
\end{definition}
}

\redbox{
\begin{definition}[init and next state functions]
Consider $v\in V$ and consider $s_{init}$ and $s_{next}$ 
the assignment statements where $v$ is the target term in 
\cci{init-list} and \cci{next-list}, respectively. 
We defined 
functions $\mathit{init-state}(v)$ and 
$\mathit{next-state}(v)$ 
to be the functions corresponding to the right hand side 
expressions of $s_{init}$ and $s_{next}$, respectively. 
\end{definition}
}

%%%%%%%%%%%%%%%%%%%%%%%%%%%%%%%%%%
%%%%%%%%%%%%%%%%%%%%%%%%%%%%%%%%%%
%
\subsection{Semantics of one loop programs}
%


Recall that a variable can be either a register
denoting a memory element, 
or a wire denoting a functional macro. 
Memory variables are initialized simultaneously using the 
\cci{do-together} construct. 
After initialization, an infinite loop keeps updating the 
value of memory variables simultaneously. 
The listings in Figure \ref{fig:gr} shows the syntax of 
an \caig. 



If a wire is not assigned, then it is said to be a 
non-deterministic {\em primary input}. 
It takes a new non-deterministic value at each iteration of 
the loop.
The list of statements \cci{init-list} assigns initial values to 
the register variables.
Similarly, the \cci{next-list} list of statements updates 
the values of the register variables. 
The semantics of \caig expressions are defined by the typical 
valuation rules of the corresponding unary and binary operators. 
The ternary choice \cci{(a? b : c)} returns $b$ if $a$ 
is \true and $c$ otherwise. 



Formal semantics of \caig are given in terms of \caig state and trace as follows. 
%
For this purpose, we consider an \caig $P$ ranging over a set of 
non-wire variables $V=\set{v_1,v_2\ldots,v_n}$.


\begin{definition}[\caig state]
The state of $P$ is defined as 
the valuation 
$\sigma: V \rightarrow D$. 
The valuation $\sigma$ maps variables in 
$V$ to $ D = \mathbb{B} \cup \mathit{Data} \cup \mathbb{B}^k \cup \mathit{Data}^k$ 
such that $\sigma(v_i) \in \mathbb{B}$ (resp. $Data$, $\mathbb{B}^k$, $Data^k$) when $\mathit{type}(v_i)$ is \cci{Boolean} (resp. \cci{int}, 
\cci{Boolean[k]}, and \cci{int[k]}), where $1 \le i \le n$ and $k >0$.
\end{definition}



\begin{definition}[\cci{do-together} semantics]
All the assignment statements \cci{init-list} and 
\cci{next-list} can execute
simultaneously as indicated with the \cci{do-together} 
construct.
\end{definition}



\begin{definition}[\caig trace]
A trace $\pi$ of length $\ell$ of $P$ is a sequence 
of \caig states $\sigma_0,\sigma_1,\ldots,\sigma_{\ell -1}$. 
State $\sigma_0$ is defined as the valuation given by 
%the $\mathit{init-state}(v_i)$ functions, $1\le i \le n$. 
the \cci{init-state($v_i$)} functions, with $1 \le i \le n$. 
State $\sigma_{k+1}$ corresponds to the valuations 
%given by functions $\mathit{next-state}(v_i)$ 
given by functions \cci{next-state($v_i$)} 
where references to variables $v_j\in V$ are substituted by 
the corresponding valuations from $\sigma_{k}$, 
$0\le k \le \ell$. 
\end{definition}




\begin{figure*}
\begin{tabular}{p{3cm}p{0.2cm}p{10cm}}
\begin{lstlisting}
decl-list

wiredef-list

do-together {
  init-list 
}

while(true) {
  do-together {
    next-list
  } 
}
\end{lstlisting}
&
&
\begin{lstlisting}
type: bool | int | bool [NUM] | int [NUM]; 
declaration: wire type id; | type id;

expr: term | uop expr| expr bop expr | expr ? expr : expr;
term: id | id[expr]; 

decl-list: declaration+
assignment: term = expr
wiredef-list: (assignment)*

init-list: (assignment)* 
next-list: (assignment)* 
\end{lstlisting}
\end{tabular}
\vspace{-2em}
\caption{\caig Syntax}
\label{fig:gr}
\end{figure*}
%

%
In \secref{sec:bip2aig}, we shall see how to automatically translate a BIP system into \caig.
